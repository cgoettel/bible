\section{Tabernacles and temples}\label{app:tabernacle}
Numbers 10:11 contains the phrase \Hebrew{הָעֵדֻת מִשְׁכַּן} which is difficult to translate. The word \Hebrew{מִשְׁכַּן} is classically rendered \textit{Tabernacle}, but Koehler-Baumgartner says the following:
\begin{quote}
    \textbf{dwelling-place, home} of Y.\footnote{meaning ``the \textsc{Lord}''}
\end{quote}
It can also mean tomb, sanctuary (especially the central sanctuary of Israel while in the desert), or Tabernacle. The word \Hebrew{הָעֵדֻת} is classically rendered \textit{testimony}, but Koehler-Baumgartner says this about it:
\begin{quote}
    \textbf{warning signs, reminders, urgings}
\end{quote}
Making an idiomatic rendering of this proves difficult because the sense of ``the home of the \textsc{Lord}'' is important, but also stating that it is a home that is to serve as a reminder or an urging (most likely to be righteous).

In Numbers 12, the term \textit{tent} is interchangeable with \textit{Tabernacle}. It is also this way in most of the Pentateuch.

\subsection{Temple rituals in antiquity}\label{app:sins-transgressions}
In Leviticus, there are gradations of wrongdoing:
\begin{itemize}
    \item \Hebrew{פֶּשַׁע} intentional wrongdoing
    \item \Hebrew{חֵטְא} unintentional wrongdoing
\end{itemize}

The tabernacle complex also has gradations of purification. If the sin is unintentionally, or unknowingly committed, the sinner need only up until the entrance of the holy place (meaning they must purify at the laver and at the altar). In cases where the community has unintentionally sinned, the sacrifice must be made in the Holy Place. Finally, intentional sins (whether by an individual or the community) can only be redeemed by the high priest on the Day of Atonement in the Holy of Holies.
