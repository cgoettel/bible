\chapter{Names of the Lord}\label{app:names-of-the-lord}
\section{The Tetragrammaton}
\subsection{Etymology}
To help prevent people from speaking the ineffable name of God, the Tetragrammaton (\Hebrew{יהוה}) was sometimes voweled with the same vowel points as \Hebrew{אֲדֹנָי} (Adonai). More commonly it is written \Hebrew{יְהוָה}.

The English transliteration is Jehovah, as follows:
\begin{center}
    \Hebrew{יהוה} > hwhy > yhwh > jhwh > jhvh > Jehovah
\end{center}
The \Hebrew{י} was substituted for a \emph{j} because \emph{y} is a weak phoneme. The \Hebrew{ו}, anciently pronounced as \emph{w} was changed to the stronger \emph{v}. Thus did these changes make Jehovah out of the original Yahweh.

\subsection{Sanctity}
Ezra is one of the founders of modern Judaism~--- an ethnic religion practiced by a monotheistic people, the Judahites. It is the only surviving ethnic religion in the world besides Hinduism. This religion observed the sanctity of the Tetragrammaton by not pronouncing it. This practice became so strong that individuals who pronounced it would be put to death. The Romans allowed the Jews to stone those who uttered the ineffable name. Christ may very well have pronounced this when He declared, ``Before Abraham was \textsc{I am}'' (John 8:58).

This practice continued through the 5$^\text{th}$ and 6$^\text{th}$ centuries \textsc{ad}. It is still considered sacrilegious by very observant Jews, so much so that even \Hebrew{אֱלוֺהִים} is pronounced by Hasidic Jews as \Hebrew{אֱלוֺקִים}. In writing, ``G-d'' or ``G\_{}d'' is substituted for ``God.'' However, this practice is not peculiar to Judaism as Parley P.\ Pratt would write ``G-d'' when quoting blasphemers.

When reading, observant Jews will say ``Adonai'' (master, master of the universe, master of an individual) or ``Hashem'' (lit., the name) in place of Yahweh.

Some Orthodox Jews will not write \Hebrew{יהוה} because things containing this cannot be burned, erased, or destroyed. For this reason there is a place in the synagogue known as the \Hebrew{גְּנִיזָה} (genizah, pl.\ genizot) which is used for the disposal of sacred writings. Unfortunately, many \Hebrew{גְּנִיזָוֺת} were destroyed in the Middle Ages by fire (either arson or accidental). Fortunately, the \Hebrew{גְּנִיזָה} in the ben-Asher synagogue in Cairo, Egypt (built in the 10$^\text{th}$ century \textsc{ad}) has never burned. The Damascus scroll, likely part of the Dead Sea Scrolls, was found there because some of the documents belonging to the Qumran community were left there.

Sometimes, especially in the Dead Sea Scrolls, the Tetragrammaton is rendered in the Paleo-Hebrew: \textphnc{hwhy}.

\section{Ahman}
Not much is known on this name. What follows is only preliminary work\footnote{From Professor Stephen D.\ Ricks in Heb~432R (Biblical Hebrew Syntax), Summer term 2013.} and is in no way meant to be authoritative.

Ahman, possibly related to \Hebrew{אֹמֵן}, meaning \emph{to be true}, \emph{to be faithful}, or \emph{to be realized}. May also refer to \emph{faith} or \emph{veracity}.

Could possibly be related to the Book of Mormon name \Hebrew{אֹמְנִי} (Omni) which means either \emph{faith in me} or \emph{my faith}, \Hebrew{אֹמְנִי} being either a subjective or objective genitive.
