\heading{15}{Only those who are pure and clean my enter the temple~--- qualifications for cleanliness enumerated}

\begin{inparaenum}
  \noindent\verse{15:1} A psalm of David.\smallskip%%
  
  \pc O Lord, who shall dwell in Your tent?\pa Who will reside in Your holy mountain?%%
  
  \pb \verse{15:2} Those who walk uprightly,%
  \alt{follows sincerely/honestly}%
  \ed{Although \Hebrew{תָּמִים} can mean ``perfectly,'' it is more modest to here render it as ``uprightly.''}%
  \pa work righteousness~---%%
  
  \pc who speaks truth from his heart;\pa \verse{15:3} who doesn't slander with his tongue,%%
  
  \pc do evil to his companion,%
  \halot{רֵעָה}{(female) \textbf{companion} (of maidens)}%
  \pa nor lift up a reproach on his neighbor.%%
  
  \pb \verse{15:4} The rejected%
  \ca{crrp}{corruption}%
  \ed{This makes enough sense as is that the corruption doesn't seem to make a noticeable difference.}
  are despised%
  \ca{2 Mss `\Hebrew{וְנ}, \septuagint\ \Greek{πονηρευόμενος}, \peshitta\ \textit{mrgzn'} irritator}{two Hebrew manuscript codices have ``and he honors,'' the Septuagint has ``doing evil,'' the Peshitta has ``he who irritates''}
  in his eyes;\pa he honors those who revere the \lord.%%
  
  \pc If he's sworn to his own inconvenience,\alt{hurt}%
  \ca{\septuagint(\peshitta) \Greek{τῷ πλησίον αὐτοῦ}, \symmachus\ \Greek{ἐταῖρος εἶναι}}{Septuagint (and Peshitta) ``to his neighbor,'' Symmachus' Greek translation of the Old Testament ``fellow companion''}%
  \pa he doesn't retract.\alt{deviate.}%%
  
  \pb \verse{15:5} He doesn't put his money to usury,\pa doesn't take bribes\alt{rewards, gifts} against\alt{from} the innocent.%%
  
  \pc He who does these things\pa shall never be shaken.\alt{moved.}%%
\end{inparaenum}
