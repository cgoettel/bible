% xelatex (use Makefile)
\documentclass[twoside]{book}

% Page layout options
% Note: the "showframe" option is useful for debugging.
\usepackage[paperwidth=5.25in, paperheight=8in, top=0.5in, bottom=0.25in, outer=1in, inner=0.44in, marginparwidth=0.5in, headsep=0.125in, footskip=0.25in]{geometry}
% \setlength{\topmargin}{-0.7in} % Relative to 1" default % This is for line spacing or something. See Bringhurst.

% Font spacing options
% \renewcommand\baselinestretch{12pt} % Kills compilation. % For line spacing.
\textwidth=23pc
\parindent=2em

% Packages that don't necessarily belong elsewhere.
\usepackage{mathtools}
\usepackage{nicefrac}
\usepackage{textcomp} % For miscellaneous symbols, viz.: copyright, degree.

% Hyphen penalty (higher numbers discourage hyphenation, up to 10000).
% 4000 seems to work fine because it only allows hyphenation in extreme conditions (and doesn't require having to use ~ all the time). 8000 was previously being used but anything with a badness over 5000 is ugly to the naked eye, so 4000 is now being used (5000 wasn't enough because it allowed things in the range 4800--5200).
% Never use 10000.
\sloppy\hyphenpenalty=4000

% Widow and orphan penalty.
\widowpenalty=10000
\clubpenalty=10000

% %%%%%%%%%%%%%%%%%%
%     APPENDICES
% %%%%%%%%%%%%%%%%%%
\usepackage[toc,page]{appendix}

% %%%%%%%%%%%%%%%%%%%%%%%%%%%
%     CHAPTER and SECTION
% %%%%%%%%%%%%%%%%%%%%%%%%%%%
% The "explicit" option requires the title to be given, supplied as #1.
\usepackage[explicit]{titlesec}

% Note: options for reconfiguring titles are found in the document.

% Defining two new commands which will be dynamically changed on each call of \book
\newcommand\englishchaptertitle{}
\newcommand\hebrewchaptertitle{}

%% Shorter way of declaring a chapter with both a Hebrew and an English name.
\newcommand\book[2]{%
  \chapter[#1\hfill#2~~~]{#1}% ~ added for spacing in ToC.
  \renewcommand\englishchaptertitle{#1}%
  \renewcommand\hebrewchaptertitle{#2}%
  \thispagestyle{fancy}% Forces no page number in cfoot.
  \markboth{1\thinspace:\thinspace1}{1\thinspace:\thinspace1}%
}

% %%%%%%%%%%%%%%%%%%%%%%%%
%     CHAPTER HEADINGS
% %%%%%%%%%%%%%%%%%%%%%%%%
\newcommand\heading[2]{%
  \medskip%
  \markboth{#1:1}{#1:1}%
  \marginpar{%
    \vspace{-1em}%
    \color{Gray}%
    {%
      \Huge%
      \textbf{%
        \hspace*{0.1em}%
        \begin{center}%
          #1%
        \end{center}%
      }%
    }%
  }%
  \noindent%
  \small%
  \emph{%
    \newline%
    #2.%
  }%
  \smallskip%
}

% %%%%%%%%%%%%%%%%%%%%%%%%%%%
%     ENUMERATE and VERSE
% %%%%%%%%%%%%%%%%%%%%%%%%%%%
% paralist allows for separate paragraphs in the same enumeration environment.
\usepackage{paralist}

% Change verse number to red superscript.
\usepackage[usenames,dvipsnames,svgnames]{xcolor}
\renewcommand\labelenumi{\hspace*{0.5em}\color{Red}{$^{\arabic{enumi}}\hspace*{-0.1em}$}}

% Patch enumi counter to always start with 1.
\usepackage{etoolbox}
\newcounter{myc}[enumi]
\patchcmd{\enumi}{}{\setcounter{myc}{1}}{}{}

\renewcommand\verse[1]{%
  \markboth{#1}{#1}% Sets up header. Stores #1 as both \leftmark and \rightmark.
  \item% Print verse number.
}

% Allows for more control of enumerations and lists.
\usepackage{enumitem}

% %%%%%%%%%%%%%%%%%%%%%%%%%%%%%%%
%     FONTS and ABBREVIATIONS
% %%%%%%%%%%%%%%%%%%%%%%%%%%%%%%%
\usepackage{fontspec}
\usepackage{metalogo}

% Egyptian hieroglyphs
\usepackage{graphicx}

% Greek
\newfontfamily\sblg[Script=Greek,Scale=MatchUppercase]{SBL Greek}
\newcommand\Greek[1]{{\sblg #1}}

% Hebrew
\newfontfamily\sblh[Script=Hebrew,Scale=MatchUppercase]{SBL Hebrew}
\newcommand\Hebrew[1]{{\sblh #1}}

% Phoenician
\usepackage{phoenician}

% Schwa
\renewcommand\schwa{e}

% Syriac
\newfontfamily\sbls[Script=Syriac]{Estrangelo (V1.1)}
\newcommand\Syriac[1]{{\large\sbls #1}}

% ABBREVIATIONS
\newfontfamily\abbrev[Scale=MatchUppercase]{Apparatus SIL}

\newcommand\caref[3]{#1~#2,#3}
\newcommand\haref[3]{#1~#2$_\text{#3}$}
\newcommand\vref[3]{#1~#2\thinspace:\thinspace#3}

\newcommand\aquila{\Greek{α´}}
\newcommand\arabica{{\abbrev Å}}
\newcommand\bomberg{{\abbrev 𝔅}}
\newcommand\fragheb{{\abbrev ℭ}}
\newcommand\latina{{\abbrev 𝔏}}
\newcommand\masoretic{{\abbrev 𝔐}}
\newcommand\missing{{\scriptsize>}}
\newcommand\peshitta{{\abbrev 𝔖}}
\newcommand\qumran{{\abbrev 𝔔}}
\newcommand\sampen{{\abbrev\large ⅏}}
\newcommand\septuagint{{\abbrev 𝔊}\thinspace}
\renewcommand\super[1]{\textsuperscript{#1}}
\newcommand\superit[1]{\textsuperscript{\textit{#1}}}
\newcommand\symmachus{\Greek{σ´}}
\newcommand\targum{{\abbrev 𝔗}}
\newcommand\tetragrammaton{\Hebrew{יהוה}}
\newcommand\theodotion{\Greek{θ´}}
\newcommand\vulgate{{\abbrev 𝔙}}
% Others: http://scripts.sil.org/cms/scripts/render_download.php?format=file&media_id=ApparatusSIL_ViewGlyph&filename=ApparatusSIL_ViewGlyph.pdf

% %%%%%%%%%%%%%%%%%
%     FOOTNOTES
% %%%%%%%%%%%%%%%%%
\usepackage[perpage,multiple,hang,flushmargin]{footmisc}
% perpage: Resets footnote numbering for each page.
% multiple: Allows footnotes to be separated by a comma.
% hang and flushmargin: Removes the indentation.

% Makes footnotes lowercase, italic alpha instead of Arabic numerals.
% \renewcommand{\thefootnote}{\textit\alph{footnote}}

% Force footnotes to stay on the same page and not bleed over. Long footnotes should be placed in the appendix.
% xxxx: this isn't perfect in preventing footnotes from appearing on the next page.
\interfootnotelinepenalty=10000

% Remove line above footnotes.
% \renewcommand{\footnoterule}{\hrule height 0pt}

% To indicate when a word used in English rendering is understood from the Hebrew, but not explicitly stated. The KJV simply italicizes all such instances, but this is cleaner and more in keeping with the BHS.
\usepackage{morefloats} % Allows more floats to be placed on a page. The \understood command generates a hefty amount of floats and this helps alleviate some problems with using this command so often. Note that this is not a panacea.
\usepackage[mathscr]{euscript} % Allows for \mathscr command.
\newcommand\understood{\textdegree\marginpar{{\tiny\hfil$\mathscr{U}$\hfil}}}

% Abbreviated footnote commands
\newcommand\alt[1]{\footnote{\textsc{alt} #1}}
\newcommand\ca[2]{\footnote{\textsc{ca} #1 (#2)}}
\newcommand\cf[1]{\footnote{cf.\ #1}}
\newcommand\ed[1]{\footnote{\textsc{ed} #1}}
\newcommand\halot[2]{\footnote{\textsc{halot} \Hebrew{#1}: #2}}
\newcommand\halotu[2]{\footnote{\textsc{halotu} \Hebrew{#1}: #2}}
\newcommand\ie[1]{\footnote{\textsc{ie} #1}}
\newcommand\lit[1]{\footnote{\textsc{lit} #1}}
\newcommand\ylt[1]{\footnote{\textsc{ylt} #1}}

% Abbreviated footnote commands for \footnotetext command
\newcommand\fntalt[1]{\footnotetext{\textsc{alt} #1}}
\newcommand\fntca[2]{\footnotetext{\textsc{ca} #1 (#2)}}
\newcommand\fntcf[1]{\footnote{cf.\ #1}}
\newcommand\fnted[1]{\footnotetext{\textsc{ed} #1}}
\newcommand\fnthalot[2]{\footnotetext{\textsc{halot} \Hebrew{#1}: #2}}
\newcommand\fnthalotu[2]{\footnotetext{\textsc{halotu} \Hebrew{#1}: #2}}
\newcommand\fntie[1]{\footnotetext{\textsc{ie} #1}}
\newcommand\fntlit[1]{\footnotetext{\textsc{lit} #1}}
\newcommand\fntylt[1]{\footnotetext{\textsc{ylt} #1}}

% %%%%%%%%%%%%%%%%%%%%%%%%%
%     HEADER and FOOTER
% %%%%%%%%%%%%%%%%%%%%%%%%%
% Header options for dictionary style headers.
\usepackage{fancyhdr}

\renewcommand{\headrulewidth}{0pt} % Removes horizontal line.

\usepackage{fix2col} % Fixes potential problems with marks on two column pages.
\usepackage{substr}

% Extracts book name and chapters from rightmark and leftmark.
\fancyhead[LE]{\rightmark}
\fancyhead[RO]{\leftmark}
\fancyhead[LO,RE]{\thepage}
\fancyhead[CE]{\englishchaptertitle} % Puts book name (e.g., Genesis, Exodus) in center header on even pages.
\fancyhead[CO]{\hebrewchaptertitle} % Puts Hebrew book name in center header on odd pages.
\cfoot{} % Removes page number from center foot.

% %%%%%%%%%%%%%%
%     POETRY
% %%%%%%%%%%%%%%
% Use `\p[a-e]` for beginning of line spacing.
% Use `\pa` for interline separations.
\newcommand\pa{\noindent\hspace{1.5em}}
\newcommand\pb{\noindent\hspace{2em}}
\newcommand\pc{\noindent\hspace{2.5em}}
\newcommand\pd{\noindent\hspace{3em}}
\newcommand\pe{\noindent\hspace{3.5em}}

% %%%%%%%%%%%%%%%%%%%%%%%%%
%     TABLE OF CONTENTS
% %%%%%%%%%%%%%%%%%%%%%%%%%
\AtBeginDocument{\addtocontents{toc}{\protect\thispagestyle{empty}}} % Removes page numbering from ToC.
\renewcommand\contentsname{CONTENTS} % Changes title of page.
\setcounter{tocdepth}{0} % Forces ToC to only displays books.
\def\numberline#1{} % Removes numbering from ToC but not from chapters.

% %%%%%%%%%%%%
%     YEAR
% %%%%%%%%%%%%
\usepackage{datetime}
\newdateformat{justtheyear}{\THEYEAR}

% %%%%%%%%%%%%%%%%
%     DOCUMENT
% %%%%%%%%%%%%%%%%
\begin{document}
  \frontmatter
  \pagestyle{empty}
  \titleformat{\chapter}{\large}{}{0em}{\uppercase{#1}}
  \titlespacing*{\chapter}{0pt}{0pt}{12pt}
  \titleformat{\section}{\scshape}{}{0em}{#1}
  \titleformat{\subsection}{}{}{0em}{\textit{#1}}
  \input{half-title}
  \clearpage
  \input{title-page}
  \clearpage
  \input{license}
  \clearpage
  \input{dedication}
  \clearpage
  \chapter{Preface}\thispagestyle{empty}
\section{The Bible}
The Bible is the word of God so far as it is translated correctly. Personally, the belief that the Bible is infallible shows a clear lack of understanding and education since the Bible is rife with poor translations, mistranslations, and even typos. However, having the Bible in as good of condition as we have it today is a miracle~--- one for which I thank the Lord.

\section{Translation philosophy}
Growing up with the King James Version of the Holy Bible was a two-edged sword: on one hand, it's a beautifully written and well-accepted version; on the other hand, it's a poetic translation. Personally, non-idiomatic translations show a lack of understanding on the translator's part as to how language works. Poetic translations are difficult to render, but read beautifully; however, they are non-intuitive and not properly suited for most audiences. This translation is a rather idiomatic translation. Only a few liberties were taken, all of which are marked in the footnotes.

This project is also partly a work of scholarship. There are theological points discussed as well as matters of textual criticism. None of this is meant to prove anything, merely to lend credence to beliefs. As Professor Steven D.\ Ricks said, ``Scholarship isn't improving a point: it's enhancing probabilities'' (26 January 2015).

In cases where no correct translation can be given, the Hebrew (or Aramaic) is given instead.

\section{The Tetragrammaton}
The Tetragrammaton\lit{a word having four letters} is the holy name of God, written \Hebrew{יהוה}. In Orthodox Hebrew culture it is unlawful for this word to be uttered by man but once a year by the High Priest on the Day of Atonement in the Holy of Holies. Traditionally, the Tetragrammaton is rendered ``the \textsc{Lord}'' or \textsc{God} (in small caps). This tradition has been adhered to in this edition except in the case of \Hebrew{יְהֹוָה אֲדֹנָי}\footnote{Ketiv. Qere ``adonai elohim.''} where it is usually rendered as ``the Lord \textsc{God}.''\footnote{To avoid rendering it as ``the Lord \textsc{Lord}.''} See further in Appendix~\ref{app:names-of-the-lord}.

\section{Textual basis}
This text was translated from the \textit{Biblia Hebraica Stuttgartensia}. Inspiration for this translation was taken from the \textit{Darby English Bible}, \textit{Louis Segond}, and \textit{Young's Literal Translation}. The lexicons used were \textit{The Brown-Driver-Briggs Hebrew and English Lexicon}, Holladay's \textit{A Concise Hebrew and Aramaic Lexicon of the Old Testament}, and Koehler and Baumgartner's \textit{Hebrew and Aramaic Lexicon of the Old Testament}.

Some of these passages very closely resemble other renderings. Often, there are few ways of translating a passage and many translations share commonalities: this translation is no different.

\section{Footnotes and appendix}
Footnotes are used to show alternate renderings and to provide historical, symbolical, and other, expository notes. An appendix appears in the back of the book and contains notes too long for inclusion in footnotes.

\section{Abbreviations}
\subsection{Books of the Bible}
\begin{table}[!h]
    \centering
    \setlength\tabcolsep{1.75em}
    \begin{tabular}{llll}
        Gen   & 1~Ki  & Eccl  & Obad \\
        Ex    & 2~Ki  & Songs & Jon  \\
        Lev   & 1~Chr & Is    & Mic  \\
        Num   & 2~Chr & Jer   & Nah  \\
        Deut  & Ezra  & Lam   & Hab  \\
        Josh  & Neh   & Ez    & Zeph \\
        Judg  & Est   & Dan   & Hag  \\
        Ruth  & Job   & Hos   & Zech \\
        1~Sam & Ps    & Joel  & Mal  \\
        2~Sam & Prov  & Amos  & ~
    \end{tabular}
\end{table}

\subsection{Other abbreviations}
\begin{description}[labelsep=3em, font=\normalfont, itemsep=-0.25em]
    \item[\textsc{alt}] alternatively
    \item[Aram.] Aramaic
    \item[\textsc{bh}] Biblical Hebrew
    \item[\textsc{bhs}] \textit{Biblia Hebraica Stuttgartensia}
    \item[Davidson] \textit{The Analytical Hebrew and Chaldee Lexicon} by Benjamin Davidson
    \item[\textsc{ed}] editorial note
    \item[Fr.] French
    \item[\textsc{halot}] Koehler and Baumgartner's \textit{Hebrew and Aramaic Lexicon of the Old Testament}
    \item[\textsc{kjv}] King James Version
    \item[L] the Leningrad codex
    \item[\textsc{lit}] literally
    \item[\textsc{lsg}] \textit{Louis Segond} edition of the Bible
    \item[pl.] plural
    \item[Sp.] Spanish
    \item [W\&O] Waltke and O'Connor's \textit{An Introduction to Biblical Hebrew Syntax}
    \item[\textdegree\dots$\mathscr{U}$] the English word is understood from the Hebrew, but not explicitly said. Much like italics in the \textsc{kjv}.
    \item[\fragheb] fragmentum codicis Hebraici in geniza Cairensi repertum
    \item[\latina] vetus versio Latina
    \item[\masoretic] Masoretic text
    \item[\peshitta] Peshitta
    \item[\qumran] the books of the Hebrew manuscripts recently discovered in Qumran near Chirbet, \textit{Discoveries in the Judaean Desert I}
    \item[\sampen] Samaritan Pentateuch
    \item[\septuagint] Septuagint
    \item[\targum] Targum
    \item[\vulgate] Vulgate
    \item[$\dagger$] Used in \textsc{halot} to mean that all undisputed forms have been enumerated. Meaning that the definition given for a particular verse is the correct definition for that verse and nobody is arguing about it.
\end{description}

\section{Bibliography}
\begin{enumerate}
  \item Farrar, Frederic W. \textit{The Life of Christ}. Salt Lake City, UT: Bookcraft, 1994.
  \item Ricks, Steven D., and Donald W.\ Parry, eds. \textit{The Temple: Ancient and Restored}. Vol.\ 3.\ Temple on Mount Zion. Orem, UT: The Interpreter Foundation, 2016.
  \item Taylor, John. \textit{The Gospel Kingdom}. Edited by G.\ Homer Durham. Second Collector's ed. Salt Lake City, UT: Bookcraft, 1990.
  \item Tov, Emanuel. \textit{Textual criticism of the Hebrew Bible}. Minneapolis: Fortress Press, 2012.
\end{enumerate}

  {\cleardoublepage}
  \tableofcontents
  \clearpage
  \mainmatter
  
  % openrightfalse forces TeX to not generate a blank page between chapters.
  \makeatletter
  \@openrightfalse
  \makeatother
  
  \pagestyle{fancy}
  \book{Genesis}{\Hebrew{בראשית}}
  \heading{1}{God creates the world~--- the various acts of the creation enumerated~--- man and woman created in God's image~--- dominion of the earth given to man}

\begin{inparaenum}
  \verse{1:1} In the beginning, God\ed{It is not ``the Gods'' because every verb is conjugated for the third masculine singular, not plural. \Hebrew{אֱלוֺהִים} is the plural of majesty for God. Theologically, Christ created the Universe under the direction of the Father, and, although He had help throughout the planning and construction phases, the honor and glory go to Him and the Father solely, not the rest of the gods that assisted.} created\ed{This verb, \Hebrew{בּרא}, means ``to create.'' It does not carry with it the notion of \textit{ex nihilo} creation, but rather of organization. \textsc{halot} notes that this verb is ``only of God.''} the Heavens and Earth.%%
  \verse{1:2} The earth was formless and void~--- darkness moved upon the face of the deep, and the Spirit of God moved upon the face of the waters.%%
  \verse{1:3} God said, ``Let there be light!'' And there was light.%%
  \verse{1:4} And God saw the light that it was good, so God divided the light from the darkness.%%
  \verse{1:5} And God called the light Day; and the darkness Night. And there was an evening and a morning: the first day.%%
  
  \verse{1:6} And God said, ``Let there be an expanse in the midst of the waters: let it separate the waters.''\lit{the waters from the waters.}%%
  \verse{1:7} So God made the expanse. And it separated between the waters which are under the expanse and the waters which are above the expanse~--- and thus it was.%%
  \verse{1:8} And God called the expanse Heaven. And there was an evening and a morning: the second day.%%
  
  \verse{1:9} God said, ``Collect the waters under Heaven unto one place, and let the dry land appear\alt{be seen}~--- and thus it was.''%%
  \verse{1:10} And God called the dry land, Earth; and the collection of waters He called, Seas.%%
  \verse{1:11} God said, ``Let Earth yield tender grass, seed producing herbs, and fruit trees yielding fruit after their kind (the seed of which is in them) on Earth:'' and thus it was.%%
  \verse{1:12} So Earth brought forth grass, seed producing herbs after its kind, and trees yielding fruit (the seed being\lit{which is} in them) after their kind~--- and God saw that it was good.%%
  \verse{1:13} And there was an evening and a morning: the third day.%%
  
  \verse{1:14} God said, ``Let there be lights in the expanse of Heaven to separate\alt{divide} the day from the night. Let them be for signs and for seasons, for days and for years;%%
  \verse{1:15} let them be for lights in the expanse of Heaven to illuminate\lit{give light to} Earth.'' And thus it was.%%
  \verse{1:16} So God made the two great lights: the greater light to rule the day and the lesser\lit{small} light and the stars to rule the night;%%
  \verse{1:17} and God placed them in the expanse of Heaven to illuminate Earth,%%
  \verse{1:18} to rule during the day and night, and to separate the light from the darkness~--- and God saw that it was good.%%
  \verse{1:19} And there was an evening and a morning: the fourth day.%%
  
  \verse{1:20} God said, ``Let the waters teem with life\lit{teeming, living creatures} and let fowls fly on the earth and before the Heavens.''%%
  \verse{1:21} God created the great sea monsters and every living, teeming creature\lit{soul} which are innumerable in the waters after their kind and all the winged birds after their kind. And God saw that it was good.%%
  \verse{1:22} God blessed them, saying, ``Be fruitful and multiply. Fill the waters in the sea and let the birds multiply in the earth.''%%
  \verse{1:23} And there was an evening and a morning: the fifth day.%%
  
  \verse{1:24} God said, ``Let living souls come forth from the earth\lit{Let the earth bring forth living souls} after their kind, wild animals, reptiles,\alt{small animals, creeping things} and the wild, untamed animals\halot{xxxx}{\textbf{animals}, \Hebrew{חַיּוֺת} or collective singular, rarely a single animal \haref{Gn}{37}{20}: --- 1.\ \textbf{animals} of all kinds, mostly untamed\dots\ land animals\dots\ beasts of burden\dots\ water animals\dots\ --- 2.\ \textbf{wild, predatory animals}} of the earth, after their kind.'' And thus it was.%%
  \verse{1:25} God made the wild, untamed animals of the earth after their kind, the wild animals after their kind, the ground reptiles after their kind. And God saw that it was good.%%
  \verse{1:26} God said, ``Let Us make Adam\alt{man, Man, mankind}\ed{God is not here making the physical bodies for all of mankind~--- that task is soon delegated to Adam and Eve and their posterity. God is here making the body of Adam (not even Eve yet) so it would be incorrect to say ``Let Us make Man in Our image.''} in Our image and according to Our likeness. Give them dominion over the fish of the sea, the birds in the sky, the wild animals~--- over the whole earth. And give them dominion over all the reptiles which creep upon the earth.''%%
  \verse{1:27} So God created Adam in His image. In the image of God created He him. Male and female created He them.%%
  \verse{1:28} God blessed them and He\lit{God} said to them, ``Be fruitful. Multiply. Replenish the earth. Subdue\alt{Subjugate} it. Have dominion over the fish of the sea, the birds in the sky, and on all life that moves on the earth.''%%
  \verse{1:29} And God said, ``Look, I have given you every seed-bearing herb in the whole world and every tree which has tree-producing seeds. These shall be your food.''\lit{These shall be food to you.}%%
  \verse{1:30} Every wild animal on the earth, every bird in the sky, all the reptiles in the world~--- in which is a living soul~--- and every green herb: these shall be for food.\lit{for food} And thus it was.%%
  \verse{1:31} God saw everything that He had made and it was good, very good. And there was an evening and a morning: the sixth day.%%
\end{inparaenum}

  \heading{2}{God completes His work and sanctifies the seventh day~--- the rivers of Eden are described~--- God places Adam in the Garden and commands him to not eat of the tree of the knowledge of good and evil~--- Eve is created, given to Adam, and they are married}

\begin{inparaenum}
  \verse{2:1} The heavens, the earth, and their hosts were completed.%%
  \verse{2:2} God completed His work\ed{Might be closer to ``life's work,'' although that doesn't really work in the eternal sense.}\alt{mission, business, occupation [unlikely], labor, enjoyment, craft, job, task} that he'd done by the seventh day and, on the seventh day, stopped\ed{from \Hebrew{שׁבת}} all His work that he'd been doing.%%
  \verse{2:3} God blessed the seventh day~--- He sanctified it~--- because on it he stopped all\ed{What are the theological implications of \textit{all}?} His work that God had created to do.%%
  
  \verse{2:4} These are the origins of the heavens and the earth, when they were created, in the day the \textsc{Lord} God made the earth and the heavens,%%
  
  \verse{2:5} every bush\alt{shrub} of the field before it was on the earth, and every green plant\halot{xxxx}{collectively, \textbf{green plants}: weeds, grass, vegetables, cereals, growing during rainy season, not perennials} of the field before they sprouted (because the \textsc{Lord} God had not yet\understood\ let it rain on the earth and there was no man to till and cultivate\ed{Both of these are possible definitions of \Hebrew{עבד}, but I feel both are needed to properly articulate what is being said.} the ground.%%
  \verse{2:6} The subterranean fresh-water stream\ed{Straight from \textsc{halot}. Sounds like existing ground water, possibly flowing. In \vref{Job}{36}{27} it refers to the ``heavenly stream.''} rose up from the earth and watered the entire surface\lit{face} of the ground.%%
  \verse{2:7} The \textsc{Lord} God created\halot{xxxx}{older, concrete synonym of \textit{b\=ar\=a'}} Adam\alt{man, mankind; however, ``mankind'' would not be theologically correct.}\ed{All later instances of \textit{Adam} in this chapter come from \Hebrew{הָאָדָם} and could easily be rendered as ``man'' or ``mankind.'' However, we know from context that this is Adam and it is justifiable to identify him by name.}~--- loose soil\halot{xxxx}{dry, fine particles of dirt\dots\ dust} from the ground~--- and breathed the breath of life into his nostrils: and man became a living soul.%%
  \verse{2:8} The \textsc{Lord} God planted, in the east, a garden in Eden; He there put the man whom He had created.%%
  \verse{2:9} The \textsc{Lord} God made every tree that is excellent\alt{desirable, precious, beloved} to behold and good for food to\ed{Changed to infinitive to read more idiomatically.} sprout from the ground~--- the tree of life to sprout\understood\ in the midst of the Garden, and the tree of the knowledge of good and evil.%%
  \verse{2:10} A river went out of Eden to water the Garden. It there forked\alt{branched, divided} and became four river branches.%%
  \verse{2:11} The name of one is Pison. That one surrounds the entire land of Havilah, where there is gold.\lit{that has gold there.}%%
  \verse{2:12} The gold in that land is good. There is also bdellium\halot{xxxx}{the fragrant, transparent, yellowing gum-resin of a South Arabian tree.} and onyx\halot{xxxx}{a precious stone, traditionally onyx; suggested to be carnelian or \textit{lapis lazuli}} stone.%%
  \verse{2:13} The name of the second river is Gihon. That one surrounds the entire land of Cush.%%
  \verse{2:14} The name of the third river is Hiddekel. That one flows east toward Asshur.\ie{Assyria} The fourth river is the Euphrates.%%
  \verse{2:15} The \textsc{Lord} God took Adam and settled\alt{put, set} him in the Garden of Eden to work in\understood\ it and to guard it.%%
  \verse{2:16} The \textsc{Lord} God gave an order to Adam, saying, ``From every tree of the Garden you can definitely eat;%%
  \verse{2:17} however, from the tree of the knowledge of good and evil you shall not eat\lit{from it} because in the day you eat from it you will certainly die.''%%
  \verse{2:18} The \textsc{Lord} God said, ``It is not good that Adam\alt{man, this man, the man; an interesting theological implication here is that of God saying that it is not good for man to be alone. Does this refer specifically to Adam or to all men in general?} should be alone. I will create a helper for him, to be his counterpart.''\lit{I will make/create for him a helper corresponding to him.}%%
  \verse{2:19} From the ground, the \textsc{Lord} God formed every field animal and every bird of the sky, and brought them to Adam to see what he would call them. And everything that Adam called each of the living creatures, so was its name.%%
  \verse{2:20} Adam named all the animals and the birds of the sky and every field animal. But regarding Adam, there was not found a helper to be his counterpart.%%
  \verse{2:21} So the \textsc{Lord} God caused a deep sleep to fall on Adam, and he fell sleep. And God\understood\ took one of his ribs, and He closed up the skin in its place.%%
  \verse{2:22} The \textsc{Lord} God built the rib that He had taken from Adam into a woman, and he brought her to Adam.%%
  \verse{2:23} Adam said:%%
  
  \pvcc{``At last!\footnotemark}{Bone of my bones}{and flesh of my flesh.}%%
  \fntlit{This time!}%%
  
  \pvcc{She\footnotemark\ shall be named `woman'}{because from a man}{she has been taken.}%%
  \fntlit{This}%%
  
  \verse{2:24} Therefore, a man shall leave\alt{abandon, forsake, give up} his father and mother behind, and he shall fasten himself to\alt{cling, cleave, stick with, hold onto, pursue, join} his wife and they shall become one flesh.''\halot{xxxx}{qualifies this as referring to a relationship.}%%
  \verse{2:25} They were both naked, Adam and his wife, and they were not ashamed.%%
\end{inparaenum}

  \heading{3}{The snake convinces Eve to eat the forbidden fruit~--- she does so and then convinces Adam to do likewise~--- they are reproved by the Lord~--- the snake is cursed~--- curses are given to Adam and Eve~--- they are driven out of the Garden of Eden~--- the Lord has cherubim protect the tree of life}

\begin{inparaenum}
  \verse{3:1} The snake was subtler\alt{cleverer, shrewder} than all the other field animals that the \textsc{Lord} God had created. He\ed{Theological implication here is that the snake, representing the adversary, is male; otherwise, ``she'' or ``it'' would be acceptable pronouns.} said to the woman, ``Has God really said that you shall not eat from every tree of the Garden?''%%
  \verse{3:2} The woman replied to the snake, ``We may eat the fruit of the trees\lit{From the fruit of the trees} of the garden,%%
  \verse{3:3} `but from the fruit of the tree that's in the middle of the garden,' God said, `don't eat it and don't touch it or else you'll die.'\thinspace''%%
  \verse{3:4} The snake said to the woman, ``You won't certainly die%%
  \verse{3:5} because God knows that when\lit{in the day} you\ed{plural} eat from it, your eyes will be opened and you'll be like God, knowing good and evil.''%%
  \verse{3:6} The woman saw that the tree was good for food, that it is wanting\alt{craving, longing, pleasant, beautiful, lovely} to the eyes~--- that the tree is pleasurable to make one wise. So she took some of its fruit and ate. She also gave some of it\understood\ to her husband who was\understood\ with her, and he ate.%%
  \verse{3:7} Both of their eyes were opened and they knew that they were naked. They sewed together fig leaves and made loincloths\alt{aprons, girdles} for themselves.%%
  \verse{3:8} They heard the voice of the \textsc{Lord} God, walking in the breeze of day.\ed{It is unclear if this is the evening or morning wind. The Hebrew is \Hebrew{לְרוּחַ הַיּוֺם}, literally meaning ``in/during the wind/breeze of day.''} Adam and his wife were hiding themselves from the face of the \textsc{Lord} God who\understood\ was in the midst of the Garden.%%
  \verse{3:9} The \textsc{Lord} God called to Adam and said to him, ``Where are you?''%%
  \verse{3:10} And he replied, ``I heard your voice in the Garden and I was afraid because I'm naked~--- that's why\understood\ I hid.''\lit{hid myself.}%%
  \verse{3:11} He said, ``Who told you that you're naked? Have you eaten from the tree that I commanded you to not eat from?''\lit{From the tree that I commanded you to not eat from, have you eaten?}%%
  \verse{3:12} Adam said, ``The woman, whom you gave to me, she gave me fruit\understood\ from the tree and I ate.''%%
  \verse{3:13} The \textsc{Lord} God said to the woman, ``What is this that you've done?'' And the woman said, ``The snake tricked\alt{deceived. The niphal meaning is to ``give oneself false hopes.''} me and I ate.''%%
  \verse{3:14} The \textsc{Lord} God said to the snake,\smallskip%%
  
  \pvbd{``Because you've done this,}{you shall be cursed}{more than every animal,\footnotemark}{and more than every field animal.}%%
  \fntlit{from all the animals}%%
  
  \pvbc{You shall go about on your belly}{and you shall eat loose soil\footnotemark}{all the days of your life.}%%
  \fnthalot{xxxx}{loose \textbf{soil}, crumbling to \textbf{dust}.}%%
  
  \pvad{\vn{3:15} I will put enmity\footnotemark}{between you and the woman~---}{between your seed}{and her seed.}%%
  \fntalt{hostile intention}%%
  
  \pvbb{He will crush your head,}{you will snap at\footnotemark\ his heel.''}\bigskip%%
  \fntalt{snatch at}%%
  
  {\noindent\verse{3:16} He said to the woman:}%%
  
  \pvbc{``I will greatly multiply}{your hardships\footnotemark\ and pregnancy.\footnotemark}{You will bear children in pain.\footnotemark}%%
  \stepcounter{footnote}%%
  \fntalt{pain}%%
  \stepcounter{footnote}%%
  \fnthalot{xxxx}{traditionally \textbf{pregnancy}; other: \textbf{sensory pleasure}}%%
  \stepcounter{footnote}%%
  \fnted{This raises an interesting theological question: Could Eve previously bear (had previously born?) children without pain? Or is this an anagogical sense that teaches that child birth in Heaven is painless?}%%
  
  \pvbb{You shall long\footnotemark\ for your husband}{and he shall govern you.''}\bigskip%%
  \stepcounter{footnote}%%
  \fnthalot{xxxx}{\Hebrew{שׁוק}: by-form of Heb.\ II \Hebrew{שׁקק}; Arb.\ \textit{\v s\=aqa(w)} I and II to fill with longing, desire, craving\dots\ Eth.\ \textit{\d{s}\v eh\v eqa} to wish, desire}%%
  
  \verse{3:17} He said to Adam, ``Because you have listened to what your wife has said\lit{the voice of your wife} and have eaten from the tree that I commanded you, saying, `Don't eat from it,'%%
  
  \pvba{The ground will be cursed on your account.}%%
  
  \pvbb{You shall eat in hardship and pain\footnotemark}{all the days of your life.}%%
  \stepcounter{footnote}%%
  \fnted{\Hebrew{עֶצֶב} means both of these}%%
  
  \pvab{\vn{3:18} Thorn bushes and thistles\footnotemark\footnotemark\ shall sprout up before you.\footnotemark}{You shall eat the green plants\footnotemark\ of the field.}%%
  \stepcounter{footnote}%%
  \fnthalot{xxxx}{Subspecies of \textit{Centaurea pallescens}: a sort of \textbf{thistle}}%%
  \stepcounter{footnote}%%
  \fnted{Both of these are singular, but better understood as plural.}%%
  \stepcounter{footnote}%%
  \fntlit{to you}%%
  \stepcounter{footnote}%%
  \fnthalot{xxxx}{coll.\ \textbf{green plants}: weeds, grass, vegetables, cereals, growing during rainy season, not perennials}%%
  
  \pvab{\vn{3:19} By the sweat of your face}{shall you eat bread}%%
  
  \pvbb{until you return to the ground}{because you've been taken from out of it.}%%
  
  \pvbb{Because you are dirt,}{and to dirt shall you return.''}%%
  
  \verse{3:20} Adam called his wife\lit{called the name of his wife} Eve because she is the mother of every living thing.%%
  \verse{3:21} The \textsc{Lord} God made animal skin shirts\halot{xxxx}{long \textbf{shirt-like (under-)garment}} for Adam and his wife, and He clothed them.%%
  
  \verse{3:22} The \textsc{Lord} God said, ``Surely Adam has become like one of us,\ed{This is an interesting theological point (``one of us'') because it assumes more than one God. However, as noted in \vref{Gen}{1}{1}, \Hebrew{אֱלֹהִים} is always conjugated in the 3ms.} understanding good and evil. So now, lest he reaches\lit{stretches} his hand out to also take from the tree of life, eats, and lives forever~---''%%
  \verse{3:23} The \textsc{Lord} God drove him out of the Garden of Eden to till the ground where he'd been brought.%%
  \verse{3:24} He banished\alt{drove out} Adam. He had cherubim and the flame\halot{xxxx}{Metaphorical for ``blade.''} of a sword sojourn to the east of the Garden of Eden, turning this way and that to guard the path to\understood\ the tree of life.%%
\end{inparaenum}

  \heading{4}{Adam and Eve have two sons, Cain and Abel~--- Cain raises crops, Abel tends flocks~--- Cain and Abel both present sacrifices to the Lord, but only Abel's sacrifice is accepted~--- Cain kills Abel~--- the Lord curses Cain, marks him, and banishes him from society~--- Cain's son, Lamech, kills a man and enters into a secret combination with his wives~--- Adam and Eve have Seth and the people begin to call on God's name}

\begin{inparaenum}
    \verse{4:1} Adam had intercourse with his wife Eve and she became pregnant and gave birth to Cain. She said, ``I have produced a man with the \textsc{Lord}.''\ed{understood: ``with His help.''}%%
    \verse{4:2} She continued to birth his brother, Abel. Abel was shepherding his sheep and small animals, and Cain was a tiller of the ground.%%
    \verse{4:3} Sometime later, Cain brought in a sacrifice\halot{xxxx}{older passages: offering or sacrifice of homage, allegiance (of either meat or cereal)} to the \textsc{Lord} from the fruit of the ground.%%
    \verse{4:4} Also, Abel brought in the firstborn\halot{xxxx}{of cattle}\ed{\textsc{ylt} is incorrect in stating this as ``female firstlings.'' The noun is feminine, but that does not mean that they are female cattle, merely that the noun is feminine.} of his sheep and goats, and of their fatty pieces.\halot{xxxx}{pieces of fat} The \textsc{Lord} looked with favor on Abel and his sacrifice.%%
    \verse{4:5} But He did not look with favor on Cain and his sacrifice, and Cain became very much indeed\ed{Word for word from \textsc{halot}.} angry and he was downcast.\alt{his face sagged.}%%
    \verse{4:6} The \textsc{Lord} said to Cain, ``Why are you angry? Why are you downcast?%%
    \verse{4:7} If you do well, shall you not be exalted?\halot{xxxx}{\textbf{exaltation} (?) refer to commentaries.}\halotu{xxxx}{\textbf{raising, lifting up} a phrase completed with \Hebrew{פָּנִים}, the raising of the face, countenance \haref{Gn}{4}{7}.}\ed{It's important to here note that this does not refer to lifting oneself up at the last days, which, beyond being theologically impossible for all but the Savior, is definitionally impossible because the object is \Hebrew{פָּנִים}. However, the first definition (which \textsc{halot} doesn't explicitly say refers to \vref{Gen}{4}{7}) gives: ``\textbf{elevation, exaltation} (to which one raises oneself), ascent sbj.\ \Hebrew{לִוְיָתָן} \haref{Jb}{41}{17}.''} If you don't do well, sin lurks at the entryway\lit{entrance}~--- its desire will be towards you and it will gain dominion over you.''%%
    \verse{4:8} Cain said to his brother Abel, ``Let's go into the field.''\ca{mlt Mss Edd hic interv; frt ins c \sampen\septuagint\peshitta\vulgate\ \Hebrew{נֵלְכָה הַשָּׂדֶה} cf \targum\super{J\thinspace J\thinspace I\thinspace I}}{many editions of manuscripts this interval [xxxx what?] [this is an exasperated way of saying that many manuscripts contain an omission in \masoretic\ which states ``Let's go into the field.'' See further in Tov 2012 p.\ 221]} So they were in the field and Cain rose up against his brother Abel and killed him.%%
    \verse{4:9} The \textsc{Lord} said to Cain, ``Where is your brother Abel?'' He replied, ``I don't know. Do I have to keep watch of my brother?''%%
    \verse{4:10} He said, ``What have you done? The sound of your brother's blood cries out to me from the ground!%%
    \verse{4:11} And now, you shall be cursed from the ground which has opened its mouth to take your brother's blood from your hand.%%
    \verse{4:12} When you till the ground, it will never again\lit{not again} yield its strength to you. You shall be a wanderer, and homeless, in the earth.''%%
    \verse{4:13} Cain said to the \textsc{Lord}, ``My punishment\halot{xxxx}{for guilt} is too great for me to bear.%%
    \verse{4:14} Look, today you've banished me\alt{driven me out} from\lit{from off} the face of the land. From your face I am hid. I will be a wanderer, and homeless, in the earth, and it shall be that all those who find me will kill me.''%%
    \verse{4:15} The \textsc{Lord} said to him, ``Indeed, anyone who kills Cain shall suffer vengeance sevenfold.''\ed{understood: ``of his punishment.''} The \textsc{Lord} placed a distinguishing mark on Cain so that anyone who found him would not attack\alt{hit, beat, smite} him.%%
    \verse{4:16} Cain went out from the presence of the \textsc{Lord} and lived in the land of Nod, eastward of Eden.%%
    
    \verse{4:17} Cain had intercourse with his wife and she became pregnant and gave birth to Enoch.\alt{Hanoch} He\ie{Cain} built a city and named the city after the name of his son: Enoch.%%
    \verse{4:18} Enoch had Irad,\lit{Irad was born to Enoch; this pattern continues throughout the generations in this chapter.} Irad had Mehujael, Mehujael had Methusael, Methusael had Lamech.%%
    \verse{4:19} Lamech took two wives, the first was named Adah, the second was named Zillah.%%
    \verse{4:20} Adah gave birth to Jabal, who\lit{he} is the forefather of the tent dwellers and ranchers.\lit{those who own/tend cattle.}%%
    \verse{4:21} His brother's name was Jubal, who\lit{he} is the forefather of the lyrists\ie{one who plays the lyre}\halot{xxxx}{\textbf{lyre} (stringed instrument with sounding-board or -chest)} and organists.\halot{xxxx}{\textbf{(vertical) flute}}\lit{those who have to handle the lyre and the organ.}%%
    \verse{4:22} Zillah, as well, gave birth to Tubal-cain, smith\ylt{instructor} of every copper and iron artisan.\alt{craftsman} Tubal-cain's sister was Naamah.%%
    \verse{4:23} Lamech said to his wives:\smallskip%%
    
    \pvbd{``Adah and Zillah,}{hear my voice!}{You wives of Lamech,}{listen to my words!}%%
    
    \pvbb{Because I've killed a man to my wounding\footnotemark~---}{a young man\footnotemark\ for my stripes.\footnotemark}%%
    \stepcounter{footnote}%%
    \fnthalot{xxxx}{especially one which has been caused by bruising}%%
    \stepcounter{footnote}%%
    \fnted{The verb, \Hebrew{פּצע}, means ``to wound, injure, or emasculate (by crushing the testicles).'' The syntax is a bit weird and it's plausible to think this is saying that Tubal-cain killed a man by injuring to bruising (unlikely), wounding (plausible), or emasculating him (unlikely unless the man bled out); however, the other hemistich gives a parallelism (``a young man for my stripes'') that makes this plausible rendering unlikely, as the parallel is that the wounding belongs to Lamech, not the young man.}%%
    \stepcounter{footnote}%%
    \fnted{Possibly a servant of Lamech's, although the term (``a young man'') would more probably have a pronominal suffix stating ownership (i.e., ``my young servant'').}%%
    \stepcounter{footnote}%%
    \fntalt{wounds}%%
    
    \pvab{\vn{4:24} Surely, if Cain is avenged sevenfold,}{then Lamech shall be avenged seventy and sevenfold!''}%%
    
    {\noindent\verse{4:25} Adam again had intercourse with his wife and she gave birth to a son and called him Seth,\ca{\septuagint(\vulgate) + \Greek{λέγουσα} = \Hebrew{לֵאמֹר}}{the Septuagint and Vulgate add ``saying''} ``For God has appointed further offspring to me in Abel's stead''\lit{instead of Abel, for the sake of Abel} (because Cain had murdered Abel).}%%
    \verse{4:26} A son was also born to Seth, and he was named Enosh. Then\halot{xxxx}{stylistic device for emphasized portion of sentence} the name of the \textsc{Lord} began\halot{xxxx}{be begun} to be called on.%%
\end{inparaenum}

  \heading{5}{Adam's genealogy, including lifespans, is enumerated through Noah}

\begin{inparaenum}
  \verse{5:1} The is the book of Adam's line of descendants.\halot{xxxx}{i.e., genealogical list from ancestor} On the day when God created Adam: He made him in the likeness of God;%%
  \verse{5:2} He created them as\understood\ male\halot{xxxx}{\textbf{man} (as a male, opposed to woman)} and female; He blessed them; on the day of their creation, He called them Adam.\alt{Man}%%
  
  \verse{5:3} Adam lived 130~years, had a son\understood\ born to him in his likeness and image, and named him Seth.%%
  \verse{5:4} The days of Adam after Seth was born to him were 800~years. He had sons and daughters born to him.%%
  \verse{5:5} All of Adam's days were 930~years, and he died.%%
  
  \verse{5:6} Seth lived 150~years and Enos\alt{Enosh} was born.%%
  \verse{5:7} Seth lived 807~years after Enos was born. He had sons and daughters born to him.%%
  \verse{5:8} All of Seth's days were 912~years, and he died.%%
  
  \verse{5:9} Enos lived 900~years and Cainan was born.%%
  \verse{5:10} Enos lived 815~years after Cainan was born. He had sons and daughters born to him.%%
  \verse{5:11} All of Enos's days were 905~years, and he died.%%
  
  \verse{5:12} Cainan lived 70~years and Mahalaleel was born.%%
  \verse{5:13} Cainan lived 840~years after Mahalaleel was born. He had sons and daughters born to him.%%
  \verse{5:14} All of Cainan's days were 910~years, and he died.%%
  
  \verse{5:15} Mahalaleel lived 605~years and Jared was born.%%
  \verse{5:16} Mahalaleel lived 830~years after Jared was born. He had sons and daughters born to him.%%
  \verse{5:17} All of Mahalaleel's days were 895~years, and he died.%%
  
  \verse{5:18} Jared lived 162~years and Enoch was born.%%
  \verse{5:19} Jared lived 800~years after Enoch was born. He had sons and daughters born to him.%%
  \verse{5:20} All of Jared's days were 962~years, and he died.%%
  
  \verse{5:21} Enoch lived 65~years and Methuselah was born.%%
  \verse{5:22} Enoch walked with God for 300~years after Methuselah was born. He had sons and daughters born to him.%%
  \verse{5:23} All of Enoch's days were 365~years.%%
  \verse{5:24} Enoch habitually walked\ed{\Hebrew{יִּתְהַלֵּךְ} is given in the Hithpael which gives the possible meaning of a habitual action. The Hithpael primarily serves ``as the double-status (reflexive-reciprocal) counterpart of the Piel and secondarily as a passive form'' (W\&O~26.1.1a). Other possibilities include ``estimative-declarative'' and ``benefactive reflexive'' (\textit{ibid.}~26.4a). More information and examples can be found in \textit{ibid.}~26.} with God, and he was not\lit{there was not him} because God took him.%%
  
  \verse{5:25} Methuselah lived 187~years and Lamech was born.%%
  \verse{5:26} Methuselah lived 782~years after Lamech was born. He had sons and daughters born to him.%%
  \verse{5:27} All of Methuselah's days were 969~years, and he died.%%
  
  \verse{5:28} Lamech lived 182~years and had a son.\lit{a son was born.}%%
  \verse{5:29} He named him Noah, saying, ``This one\understood\ shall comfort us regarding our work and the\lit{regarding our} hardships\alt{pain, distress; although given in singular, the plural is more idiomatic.} of our hands because of the ground that\ed{Although this could easily be argued both ways, I decided to render this as ``that'' and not ``which'' because we don't know if there is any uncursed ground (omitting the Garden of Eden), and it's better to say that all of the ground is cursed (as per \vref{Gen}{3}{17}) than to give a possibly incorrect theological implication.} the \textsc{Lord} has cursed.''%%
  \verse{5:30} Lamech lived 595~years after Noah was born. He had sons and daughters born to him.%%
  \verse{5:31} All of Lamech's days were 777~years, and he died.%%
  
  \verse{5:32} Noah was 500~years old when Shem, Ham, and Japheth were born to Noah.%%
\end{inparaenum}

  \heading{6}{Mankind multiplies on the earth~--- the Lord says that His Spirit won't always be with man~--- the earth becomes more and more wicked~--- the Lord promises to destroy every living thing~--- the Lord teaches Noah about the Ark, the coming flood, how many animals to bring on the Ark, and to bring food~--- Noah obeys with exactness}

\begin{inparaenum}
  \verse{6:1} When mankind began to multiply on the earth\lit{on the face of the earth} and daughters were born to them,%%
  \verse{6:2} the sons of God saw that the daughters of man were lovely,\alt{good, pleasing, desirable, friendly, kind, good (in character and value)} and they took wives for themselves from all that they selected.%%
  \verse{6:3} The \textsc{Lord} said, ``My Spirit shall not indefinitely\alt{always} remain\halot{xxxx}{unexplained, to stay or similar by context} with mankind. In their inadvertent sinning\alt{committing of errors} they are human.''\lit{flesh} His\ie{Man's} days were 120~years.%%
  \verse{6:4} The giants\halot{xxxx}{\textbf{giants}, produced by miscarriages or thrown out of heaven; gigantic early population of Palestine\dots\ of mythical origin}\lit{ones who have fallen, fallen ones} were in the earth in those days, and afterwards when the sons of God came in\alt{cohabitate with, sacrificed (not completely impossible because the verse goes on to say that the daughters of man bore sons, which certainly isn't out of the realm of fertility worship)} to the daughters of man and they bore children to them~--- heroes, who, from olden times, were men of renown.\alt{standing, reputation}%%
  
  \verse{6:5} The \textsc{Lord} observed\alt{perceived, became aware} the great wickedness of man on the earth, every thought\alt{impulse, tendency, imagination, idea, intention} of his heart was exclusively\lit{only} and continually evil,\lit{exclusively evil continually,}%%
  \verse{6:6} so the \textsc{Lord} was sorry\halot{xxxx}{allowed Himself to be sorry} that He had made man on the earth~--- it grieved\alt{worried, distressed} Him to His core.\halot{xxxx}{inner self, seat of feelings and impulses: \textit{hit\`{}a\d s\d s\=eb 'el-libb\^o} (God) took it to heart, felt deeply grieved \haref{Gn}{6}{8}.}%%
  \verse{6:7} The \textsc{Lord} said, ``I will wipe out\alt{destroy} mankind, that I created, from the earth:\lit{off the face of the earth} man, the large animals, the small animals,\halot{xxxx}{of animal world excluding large animals and birds: collectively \textbf{small animals}, \textbf{reptiles}}\alt{creeping things} the birds in the sky, because I am sorry I made them.''%%
  \verse{6:8} Noah found approval\alt{satisfaction, favor} in the opinion of the \textsc{Lord}.%%
  
  \verse{6:9} This is the posterity\alt{births, children born to} of Noah: Noah was a righteous man, blameless\alt{unobjectionable, sincere, honest, perfect} among his peers.\halot{xxxx}{contemporaries} Noah walked habitually with the \textsc{Lord}.\ed{cf.\ \vref{Gen}{5}{24}, footnote 1.}%%
  \verse{6:10} Noah had three sons: Shem, Ham, Japheth.%%
  \verse{6:11} The earth was corrupt\alt{spoiled} before God; the earth was full of violence.\alt{wrong}%%
  \verse{6:12} God saw the earth, and indeed, it was corrupt because every living thing had corrupted its way\alt{custom, conduct, manner, the behavior demanded by God} on the earth.%%
  
  \verse{6:13} God said to Noah, ``The end\halot{xxxx}{of someone's existence} of all living things has come before me because the earth is full of violence throughout.\lit{from their presence} I will destroy them along with the earth.%%
  \verse{6:14} Make an ark of \textit{gopher}\halot{xxxx}{unknown wood used in building ark} wood for yourself. You shall make compartments within\lit{with} the Ark and shall cover it with pitch\halot{xxxx}{\textbf{bitumen}, \textbf{asphalt} for the Ark} inside and outside.%%
  \verse{6:15} This is how\lit{And this that/which} you shall make it: the length of the Ark shall be 300~cubits, the breadth 50~cubits, the height 30~cubits.\ed{137.16~m (450~ft) $\times$ 22.86~m (75~ft) $\times$ 13.716~m (45~ft)}%%
  \verse{6:16} You shall make a covering\alt{roof, skylight}\halot{xxxx}{entry for \Hebrew{צֹהַר}: \textit{hapax legomenon} \haref{Gn}{6}{16}; meaning uncertain, two possibilities: ---a.\ from Akkadian \textit{\d s\=eru} back, top, hinterland, open country, steppe; Canaanite \textit{\d su'ru}, \textit{\d s\=uru} back; Ugaritic \textit{\d zr} back, top; Soq.\ \textit{\d thar} over; OSArb.\ \textit{\d zhr}\dots\ and Arabic \textit{\d zahr} back; JArm.\ *\Hebrew{טהרא} roof > \textit{tt}; MHeb\dots: \textbf{roof}\dots\ perhaps particularly a gabled roof\dots\ REB: make a roof for the ark; ---b. from root \Hebrew{צהר} \textbf{skylight}, hatch cf.\ Akkadian \textit{nappa\v su} small window, loophole\dots\ cf.\ Vulgate \textit{fenestra}, Targum \textit{n\=eh\=or} opening for light; Septuagint, Peshitta otherwise.} for the Ark, precisely\lit{to the cubit} you shall complete\alt{finish} it from above. You shall put the entrance\alt{opening} of the Ark in its side. You shall make the lower, second, and third story.\understood%%
  \verse{6:17} I am bringing a deluge\halot{xxxx}{\textbf{heavenly ocean}} of water onto the earth to destroy under\lit{from under} the sky\alt{heavens} every living thing in which there is a living soul\alt{breath of life}~--- everything that is in the earth shall die.\alt{expire}%%
  \verse{6:18} But I will establish a covenant with you. You shall go into the Ark, you and your children and your wife and your children's wives with you.%%
  \verse{6:19} From every living thing, from all flesh,\alt{living thing; but this doesn't really work idiomatically.} two of everything shall go into the Ark, to stay alive with you: they shall be male and female.%%
  \verse{6:20} Birds of their own species,\alt{kind} large animals of their own species, ground reptiles\alt{small animals} of their own species~--- two of every species\understood\ shall go in with you~--- to keep them alive.%%
  \verse{6:21} And for yourself,\lit{And you, for yourself,\dots} take of every type\understood\ of food that is eaten, gather it to yourself, that there may be food for you and those who are\understood\ with you.''%%
  \verse{6:22} So Noah did according to everything that God had commanded him~--- yea, so did he.%%
\end{inparaenum}

  \heading{7}{The Lord gives instructions on how many and of what kind of animals to bring onto the Ark, and how long the Flood shall last~--- everything is gathered into the Ark~--- the Flood and its effects}

\begin{inparaenum}
  \verse{7:1} The \lord\ said to Noah, ``You and all of your household, go into the Ark because I've seen you\ed{singular} righteous before Me in this generation.%%
  \verse{7:2} Of all the large, clean\ie{cultically, ritualistically} animals, you shall bring seven pairs\lit{seven seven} with\lit{to} you, male and female.\ed{What kind of split? It would make sense to bring more females than males.} But of the beasts that are not clean, you shall bring\understood\ two, male and female.%%
  \verse{7:3} Additionally, bring\understood\ the birds of the sky by sevens, male and female, in order to keep offspring alive on the face of all the earth.%%
  \verse{7:4} Because \lit{in}seven days hence I will let it rain for forty days and forty nights on the earth, until\ed{The conjuctive \textit{waw} (\Hebrew{וְ})) is generally rendered ``and,'' but can also be used to show the consequences of actions, so I feel justified in rendering it as ``until'' here.} everything living\halot{xxxx}{\textbf{what subsists}, \textbf{what is living}} that I have made is destroyed from off the face of the land.''%%
  \verse{7:5} So Noah did according to everything that the \lord\ had commanded him.%%
  
  \verse{7:6} Noah was\lit{was a son of} 600~years old when the deluge of waters was\ed{Not necessarily ``began'' because the Flood only lasted 40~days. This is simply saying that the Flood happened during the 600\super{th} year of Noah's life.} on the earth.%%
  \verse{7:7} Noah went in, and his sons and his wife and his sons' wives went in\understood\ with him into the Ark, from the presence of the waters of the Flood.\lit{deluge}%%
  \verse{7:8} The clean animals, the unclean animals, the birds, every reptile of the ground~---%%
  \verse{7:9} they came in pairs\lit{two by two} to Noah into the Ark, male and female, as God had commanded Noah.%%
  \verse{7:10} And after seven days, the waters of the Flood were on the earth.%%
  \verse{7:11} In the six hundredth year of Noah's life, in the second month, on the seventeenth day of the month, in that day there was\understood\ a breach\halot{xxxx}{a forced breach} of all the fountains\alt{springs} of the great deep\halot{xxxx}{\textbf{primeval ocean}, \textbf{deep} \haref{Gn}{1}{2}\dots\ \textbf{deeps of sea} \haref{Ex}{15}{5} (quasi-mythological)\dots\ \textbf{subterranean water} \haref{Dt}{8}{7}} and the windows\halot{xxxx}{through which rain falls} of the sky were opened.\ed{see further in Appendix~\ref{app:water-in-antiquity}}%%
  \verse{7:12} The rain was on the earth for forty days and forty nights.%%
  \verse{7:13} On the same day,\ie{the first day of the Flood} Noah went out with\lit{and} Shem, Ham, and Japheth (Noah's sons), Noah's wife, and the three wives of his sons with them, into the Ark.%%
  \verse{7:14} They, every living thing of their own species, every large animal of their own species, every reptile of their own species that flocks\halot{xxxx}{\textbf{swarm}, \textbf{teem} (of vast numbers of creatures in water, on ground in woods; in random movement)} on the earth, every bird of their own species~--- every bird,\halot{xxxx}{\textbf{birds}, (creatures) \textbf{with wings}} everything that has wings.%%
  \verse{7:15} They went in to Noah, into the Ark, in pairs\lit{two by two}~--- all living creatures in which is the breath of life.%%
  \verse{7:16} Those who came, male and female of every animal, came according as God had commanded them. And the \lord\ shut them in.%%
  \verse{7:17} The Flood was on the earth for forty days, and the waters multiplied and lifted up the Ark above the ground.\lit{from upon the ground/earth.}%%
  \verse{7:18} The waters were mighty and multiplied greatly upon the earth, and the Ark went about on the surface\lit{face} of the waters.%%
  \verse{7:19} The waters were incredibly mighty\alt{The waters swelled/rose greatly} on the earth, and covered all the high mountains that are under the heavens~---%%
  \verse{7:20} the waters swelled up fifteen cubits\ed{6.858~m (22.5~ft)} and covered the mountains.%%
  \verse{7:21} Every creature that moves\lit{swarms, teems} on the earth died: birds,\lit{of birds; this is true for most of the list.} large animals, animals, every swarming thing\halot{xxxx}{tiny animals occurring in large numbers, in water \haref{Gn}{1}{20}, in air \haref{Lv}{11}{20}, on ground \haref{Gn}{7}{21}.} that swarmed on the earth, and all mankind:%%
  \verse{7:22} everything that had a living breath\alt{soul} in its nostrils, from all that is in the dry land, have all died.%%
  \verse{7:23} Was destroyed everything that was living\halot{xxxx}{\textbf{what subsists}, \textbf{what is living}} on the face of the earth~--- from man to large animal, to reptiles, and to the\understood\ birds of the sky~--- they were destroyed from the earth. Only Noah and those who were with him in the Ark remained.\alt{were left over.}%%
  \verse{7:24} The waters were mighty on the earth for\understood\ 150~days.%%
\end{inparaenum}

  \heading{8}{The Flood subsides~--- Noah sends a raven and a dove out to scout for dry land, but both are unsuccessful~--- after seven days, the dove brings back an olive leaf~--- after seven more days, the dove does not return~--- the earth dries up~--- the Lord tells them to leave the Ark and to bring all the animals with them~--- Noah offers sacrifices to the Lord~--- the Lord accepts Noah's offering}

\begin{inparaenum}
    \verse{8:1} God remembered Noah, all of the animals, and all of the large animals that were with him in the Ark. God make a wind pass from one end of the earth to the other, and the waters went down.\alt{abated.}%%
    \verse{8:2} The fountains of the deep and the windows of heaven\halot{xxxx}{through which rain falls} were stopped up,\halot{xxxx}{\textbf{be stopped up} (of cosmic springs)} and the rain from the sky was kept back.%%
    \verse{8:3} The waters turned back from off the face of the earth, ebbing and flowing.\lit{going and returning} And the waters diminished after\lit{from [the\kern.1em] end [of\kern.1em]} 150~days.%%
    \verse{8:4} In the seventh month, on the seventeenth day of the month, the Ark rested on the Ararat mountains.%%
    \verse{8:5} The waters were ebbing and diminishing until the tenth month, on the first day of the month, when\understood\ the mountain peaks\lit{heads} were seen.%%
    \verse{8:6} After\lit{And so it was at the end of} forty days, Noah opened the window of the Ark that he had made,%%
    \verse{8:7} and he sent out the raven\halot{xxxx}{of various species, \textit{Corvus}} which went out, going about and returning until the waters had dried from the earth.%%
    \verse{8:8} He sent out the dove\halot{xxxx}{\textit{Columba}} from himself to see if the waters had gone down\alt{gotten lower} from off the surface of the land,%%
    \verse{8:9} but the dove could\alt{did} not find a resting-place for the sole of her foot; so she returned to him in the Ark because the waters were on the surface of the whole earth. He stretched out his hand and took her and brought her to him into the Ark.%%
    \verse{8:10} He again waited\alt{has the meaning to ``make someone hope.''} seven days further, and again sent the dove out from the Ark.%%
    \verse{8:11} The dove returned to him in the evening time, an olive leaf torn in pieces in her mouth. Thus\halot{xxxx}{In older Hebrew, a second clause introduced by \Hebrew{וְ} ($\textit{w}^\textit{e}$) adds accompanying circumstances, supplementary comments, etc.}\ed{I feel justified that, in this context, this \textit{waw} conjunctive is introducing a second clause and has the general meaning ``Thus,'' ``And so,'' or ``That's how''} Noah knew that the waters had gone down from off the earth.%%
    \verse{8:12} So he waited another\alt{again, yet} seven days further and send the dove out, but she didn't return again to him.%%
    \verse{8:13} In the six hundred and first year, in the first month,\understood\ on the first of the month, the waters dried up from off the earth. Noah removed\lit{turned to the side} the covering of the Ark, looked, and the surface of the land was dried.%%
    \verse{8:14} In the second month, on the twenty-seventh day of the month, the earth was dry.%%
    
    \verse{8:15} God spoke to Noah, saying,%%
    \verse{8:16} ``Go out of the Ark, you and your wife and your sons and your sons wives with you.%%
    \verse{8:17} Every animal that is with you~--- from every animal, bird, large animal, and every reptile that swarms on the earth~--- bring them out with you. They shall swarm on the earth, and be fruitful and multiply on the earth.''%%
    \verse{8:18} So Noah went out, his sons and his wife and his sons' wives with him.%%
    \verse{8:19} Every animal and every reptile and every bird. Everything that swarms\alt{is teeming} on the earth, according to their species, went out of the Ark.%%
    \verse{8:20} Noah built an altar to the \textsc{Lord}, and he took from all of the clean\alt{pure} animals and from all of the clean birds, and offered up a burnt offering on the altar.%%
    \verse{8:21} The \textsc{Lord} smelled the soothing scent,\alt{odor} and the \textsc{Lord} said to Himself,\lit{in his heart} ``I will no longer\lit{not continue anymore} curse the ground for Adam's sake, even though his thoughts\alt{impulses, tendencies}\lit{the thoughts of his heart} have been evil since his youth; and I will no longer smite every living thing as I have done.%%
    
    \pvcc{\vn{8:22} Throughout all the days of the earth~---}{during sowing and reaping,}{cold and heat,\footnotemark}%%
    \fnthalot{xxxx}{of the summer}%%
    
    \pvcc{summer and winter,}{day and night~---}{these shall not stop.}%%
\end{inparaenum}

  \heading{9}{The Lord gives stewardship of all animal life to Noah and his children~--- He sets forth expectations regarding the sanctity of life~--- the Lord makes a covenant and promises to never again flood the earth~--- rainbows are the symbol of this promise~--- Noah gets drunk and passes out naked~--- Ham sees him~--- Shem and Japheth cover Noah~--- Canaan is cursed while Shem and Japheth are blessed~--- Noah dies}

\begin{inparaenum}
  \verse{9:1} God blessed Noah and his sons, and said to them, ``Be fruitful, and multiply and fill the earth.%%
  \verse{9:2} Let the fear and horror of you be on every large animal of the earth, and on every bird in the sky, upon\ca{2 Mss \sampen\super{Mss} \Hebrew{ובכל}''}{two manuscripts of the Samaritan Pentateuch manuscripts include ``\Hebrew{ובכל}'' [meaning that ``and'' is present]} everything that is teeming on the ground, and every fish in the water: they are placed into your stewardship.\lit{hand.}%%
  \verse{9:3} Every teeming thing that is alive shall be food for you. As to the vegetables, I give them all\ed{Interestingly, ``all'' is the direct object of this sentence.} to you.%%
  \verse{9:4} However, live animals\lit{animals in their life}~--- their blood~--- you shall not eat.%%
  \verse{9:5} Indeed, your blood~--- your life blood\understood~--- I demand. I demand it from the hand of every living thing. From man's hand, from the hand of each\understood\ man's brother, I demand the life of man.%%
  
  \pvcb{\vn{9:6} Whoever sheds a man's blood,}{his blood shall be shed by a man,}%%
  
  \pvcb{because in the image of God}{was the man made.}%%
  
  \pvcb{\vn{9:7} And you: be fruitful and multiply.}{Teem on the earth. Multiply in it.''}%%
  
  \verse{9:8} God spoke to Noah and to his children with him, saying,%%
  \verse{9:9} ``I, yes I, will establish My covenant with you and your progeny after you~---%%
  \verse{9:10} with every living animal that is with you, the birds, large animals, and all the animals of the earth that\understood\ are with you~--- with everything that has gone out of the Ark: every animal of the earth.%%
  \verse{9:11} I will establish My covenant with you: never again shall anything\lit{all flesh; but meaning every man, beast, bird, etc., and therefore ``anything.''} be cut off by the waters of the Flood. There shall never again be a flood to destroy the earth.''%%
  \verse{9:12} God said, ``This is the sign of the covenant that I am offering\alt{presenting, setting, giving} between Me and you and every living animal that is with you, to generations everlasting:%%
  \verse{9:13} I set My rainbow\halot{xxxx}{metaphorical interpretation of ``bow.''} in the clouds,\halot{xxxx}{(mass of) \textbf{clouds}} and it shall be for a sign of the covenant between Me and the earth.%%
  \verse{9:14} When I cause the clouds to make an appearance over the earth, I will make the rainbow appear\lit{seen} in the clouds.%%
  \verse{9:15} I will remember the covenant that is between Me and you and every living animal of all flesh. The waters shall never again be a flood and destroy every man and animal.%%
  \verse{9:16} The rainbow shall be in the cloud and I shall see it to remember the everlasting covenant between God and every living soul of all flesh that is on the earth.''%%
  \verse{9:17} God said to Noah, ``This is the sign of the covenant that I have established between me and every man and animal that is on the earth.''%%
  
  \verse{9:18} Noah's sons who left the Ark were Shem, Ham, and Japheth. Ham is the father of Canaan.%%
  \verse{9:19} These three are the sons of Noah. The entire earth\ie{the earth's population} was spread out\alt{dispersed, scattered} from these.%%
  \verse{9:20} Noah began to be a man of the earth and planted a vineyard.%%
  \verse{9:21} He drank the\lit{from the} wine and became drunk and uncovered himself in the middle of his tent.%%
  \verse{9:22} Ham, the father of Canaan, saw his father's genital area\alt{nakedness} and reported it to his brothers outside.%%
  \verse{9:23} Shem and Japheth took a mantle\alt{wrapper}\halot{xxxx}{both garment and under-blanket} and they both put it on their shoulders\halot{xxxx}{neck and upper part of the back} and went backwards and covered their father's nakedness. Their faces were turned away\lit{backwards} and they didn't see their father's nakedness.%%
  \verse{9:24} Noah woke up from his wine and learned what his youngest son had done to him.\ed{Are we missing something here? Nothing was ``done'' to him. His son saw him naked, that's all we know.}%%
  \verse{9:25} He said,%%
  
  \pvcc{``Cursed be Canaan!}{Let him be a servant of servants}{to his brothers.''\footnotemark}%%
  \fntlit{A servant of servants\hspace*{1em} he will be to his brothers.}%%
  
  {\noindent\verse{9:26} He said,}%%
  
  \pvcc{``Blessed be the \textsc{Lord},}{the God of Shem!}{Canaan shall be his servant.}%%
  
  \pvcc{\vn{9:27} Let God provide ample space\footnotemark\ to Japheth}{and let him dwell in Shem's tents,}{and let Canaan be his servant.}%%
  \fnted{Davidson gives this verb (\Hebrew{יַפְתְּ}) as \Hebrew{פּתה} (\textit{hiphil} imperfect), meaning ``provide ample space.'' \textsc{darby} gives ``Let God enlarge Japheth'' (closer to Davidson), while \textsc{ylt} gives ``God doth give beauty to Japheth,'' thus following the \textit{piel} imperfect form of \Hebrew{יפה}.}%%
  
  {\noindent\verse{9:28} Noah lived 350~years after the Flood.}%%
  \verse{9:29} All of Noah's days were 950~years, and he died.%%
\end{inparaenum}

  \heading{10}{Noah's genealogy}

\begin{inparaenum}
  \verse{10:1} These are the generations of Noah's children, Shem, Ham, and Japheth; sons were born to them after the Flood.%%
  \verse{10:2} The sons\ed{Could also be rendered ``children,'' but that language would exclude the possibility of daughters.} of Japheth: Gomer, Magog, Madai, Javan, Tubal, Meshech, Tiras.%%
  \verse{10:3} Gomer's sons: Ashkenaz, Riphath, Togarmah.%%
  \verse{10:4} Javan's sons: Elishah, Tarshish, Kittim, Dodanim.%%
  \verse{10:5} From these branched off\halot{xxxx}{family groups genealogically} the island nations:\halot{xxxx}{the Phoenicians in \haref{Is}{23}{2$\cdot$6} (from \Hebrew{אִי יֹשְׁבֵי}). The islands and coasts of the Mediterranean are, for the Old Testament, the extremes of the western world.} by their lands, each by his language, by their family, by their nations.%%
  \verse{10:6} Ham's sons: Cush, Mizraim,\ed{Hebrew for Egypt} Phut, Canaan.%%
  \verse{10:7} Cush's sons: Seba, Havilah, Sabtah, Raamah, Sabtecha. Raamah's sons: Sheba and Dedan.%%
  \verse{10:8} Nimrod was born to Cush, and he began to be a despot\alt{mighty hunter, manly, vigorous} in the land.%%
  \verse{10:9} He was a hunting despot\halot{xxxx}{gives \Hebrew{גִבּוֺר} as ``Nimrod was a \textbf{despot}'' and \Hebrew{צַיִד} as ``\textbf{hunting}.''} before the \textsc{Lord}; hence\alt{therefore} it is said, ``Like Nimrod, the hunting despot before the \textsc{Lord}.''%%
  \verse{10:10} The beginning of his realm was Babel\alt{Babylon, Persia} and Erech and Accad and Calneh, in the land of Shinar.%%
  \verse{10:11} Assyria stretched out from that land, and he\ie{Nimrod} built Nineveh, Rehoboth-Ir,\ed{The Hebrew (\Hebrew{עִיר רְחֹבֹת}) is said to be a location in \textsc{halot}; however, \Hebrew{רְחֹבֹת} is also the plural of \Hebrew{רְחֹב} meaning ``\textbf{square}, \textbf{plaza}.'' Therefore, this could also mean ``streets (or public square or plaza) of the city'' and refer to the infrastructure of Nineveh and not a different city.} Calah,%%
  \verse{10:12} Resen (between Nineveh and Calah), which is a great city.%%
  \verse{10:13} Mizraim fathered the Ludim, the Anamim, the Lehabim, the Naphtuhim,%%
  \verse{10:14} the Pathrusim, the Casluhim (from whom came the Philistines), and the Caphtorim.%%
  
  \verse{10:15} Canaan fathered his firstborn, Sidon, and Heth,%%
  \verse{10:16} the Jebusite, the Amorite, the Girgashite,%%
  \verse{10:17} the Hivite, the Arkite, the Sinite,%%
  \verse{10:18} the Arvadite, the Zemarite, the Hamathite. Afterwards, the Canaanite tribes\halot{xxxx}{extended family, \textbf{clan} (group in which there is a felt blood-relationship)} were scattered.\alt{spread out, dispersed}%%
  \verse{10:19} The Canaanite territory was from Sidon as you go from Gerar to Gaza, as you go from Sodom, Gomorrah, and Zeboim to Lasha.%%
  \verse{10:20} These are Ham's sons, by their tribes, by their languages, by their lands, by their nations.%%
  
  \verse{10:21} To Shem, father of all of the sons of Eber (brother of Japheth, Sr.\kern-0.1em\lit{the elder}), were sons\understood\ born.%%
  \verse{10:22} Shem's sons: Elam, Asshur, Arphaxad, Lud, Aram.%%
  \verse{10:23} Aram's sons: Uz, Hul, Gether, Mash.%%
  \verse{10:24} Arphaxad fathered Salah, and Salah fathered Eber.%%
  \verse{10:25} Two sons were born to Eber: the name of the first was Peleg\lit{division}\alt{watercourse; this is an interesting meaning because the literal separation of the tectonic plates would create water channels (a possible meaning for the verb form).} because in his lifetime the nations were divided;\ed{Peleg and the countries being split up probably refers to the nations and not the tectonic plates.} his brother's name was Joktan.%%
  \verse{10:26} Joktan fathered Almodad, Sheleph, Hazarmaveth, Jerah,%%
  \verse{10:27} Haroram, Uzal, Diklah,%%
  \verse{10:28} Obal, Abimael, Sheba,%%
  \verse{10:29} Ophir, Havilah, and Jobab. All of these were Joktan's sons.%%
  \verse{10:30} Their dwelling-place was Mesha (as you go to Sephar, the eastern mountain).%%
  \verse{10:31} These are Shem's sons, by their families, by their languages, in their lands, by their nations.%%
  \verse{10:32} These are the families of Noah's sons, by their genealogical lines,\halot{xxxx}{---~1.~\textbf{(line) of descendents} (i.e., genealogical list from ancestor) \haref{Gn}{5}{1} \& often; ---~2.~(one's \textbf{generation}, \textbf{contemporaries} \haref{Gn}{6}{9}; ---~3.~story of development of generations $>$ \textbf{history} \haref{Gn}{37}{2}; $>$~origin \haref{Gn}{2}{4}; $>$~\textbf{order of birth} \haref{Ex}{28}{10}} in their nations. The nations on the earth after the Flood have come from their distribution.%%
\end{inparaenum}

  \heading{11}{The world, united, builds a city and tower to Heaven~--- the Lord confounds their work, confuses their language, and scatters them across the earth~--- Shem's genealogy through Terah to Abram~--- Terah takes Abram, Sarai, and Lot, and journeys to Canaan}

\begin{inparaenum}
  \verse{11:1} The entire world had one language\halot{xxxx}{manner of speaking} and the same\alt{one sort of} words.%%
  \verse{11:2} As they journeyed from the east, they found a valley-plain\halot{xxxx}{broad with shallow walls} in the land of Shinar and lived there.%%
  \verse{11:3} The said one to another,\lit{man to his neighbor} ``Come on! Let's mold bricks and burn them\understood\ thoroughly.'' The bricks were stone to them, the asphalt\alt{bitumen} as mortar.%%
  \verse{11:4} They said, ``Come on! Let's build\lit{to us} a city and a tower~--- the top to Heaven!\alt{the sky, the heavens} Let us make a name for ourselves so that we're not scattered across the face of the earth.''%%
  \verse{11:5} The \lord\ came down to look at\alt{consider, visit, get to know; ``to inspect'' is not too far-fetched.} the city and the tower that the children of men had built,%%
  \verse{11:6} and the \lord\ said, ``This people is one, they all have the same language~--- they've dreamed of doing this. So now, nothing they dream\lit{think, plan} of doing will be impossible.%%
  \verse{11:7} Come on! Let's go down and we'll confuse their language there so that no one will understand another's language.''\lit{so that not understand a man the language of his neighbor.}%%
  \verse{11:8} The \lord\ scattered\alt{dispersed} them from there across the entire face of the land, and they stopped building the city.\ca{\sampen\septuagint\ + \Hebrew{ואת־המגדל} sec 4.5}{The Samaritan Pentateuch and the Septuagint, according to 4.5, add ``and the tower''}%%
  \verse{11:9} For this reason\lit{Therefore} it was named Babel, because there the \lord\ confused the language of the whole earth. From there the \lord\ scattered them over the whole earth.%%
  
  \verse{11:10} \ed{Were I so bold, I would put a new chapter division here.}These are the generations of Shem. Shem was a hundred years old and fathered Arphaxad two years after the Flood.%%
  \verse{11:11} Shem lived 500~years after Arphaxad was born. He had sons and daughters born to him.%%
  
  \verse{11:12} Arphaxad lived 35~years and Salah was born.%%
  \verse{11:13} Arphaxad lived 403~years after Salah was born. He had sons and daughters born to him.%%
  
  \verse{11:14} Salah lived 30~years and Eber was born.%%
  \verse{11:15} Salah lived 403~years after Eber was born. He had sons and daughters born to him.%%
  
  \verse{11:16} Eber lived 34~years and Peleg was born.%%
  \verse{11:17} Eber lived 430~years after Peleg was born. He had sons and daughters born to him.%%
  
  \verse{11:18} Peleg lived 30~years and Reu was born.%%
  \verse{11:19} Peleg lived 209~years after Reu was born. He had sons and daughters born to him.%%
  
  \verse{11:20} Reu lived 32~years and Serug was born.%%
  \verse{11:21} Reu lived 207~years after Serug was born. He had sons and daughters born to him.%%
  
  \verse{11:22} Serug lived 30~years and Nahor was born.%%
  \verse{11:23} Serug lived 200~years after Nahor was born. He had sons and daughters born to him.%%
  
  \verse{11:24} Nahor lived 29~years and Terah was born.%%
  \verse{11:25} Nahor lived 119~years after Terah was born. He had sons and daughters born to him.%%
  
  \verse{11:26} Terah lived 70~years and Abram, Nahor, and Haran were born.%%
  
  \verse{11:27} These are the births of Terah: Terah fathered Abram, Nahor, and Haran; Haran fathered Lot;%%
  \verse{11:28} Haran died in his father Terah's presence in his birthplace,\lit{land of his birth} Ur of the Chaldees.%%
  \verse{11:29} Abram and Nahor took wives. Abram's wife was named Sarai,\lit{The name of Abram's wife was Sarai} Nahor's wife was named Milcah, Haran's daughter (Milcah and Iscah's\lit{father of Milcah, father of Iscah} father).%%
  \verse{11:30} Sarai was barren~--- she had no child.%%
  \verse{11:31} Terah took his son Abram, and Haran's son Lot (his grandchild\lit{son's son}),\ed{But not Lot's wife? Was she dead at this point? Did she make the journey later? Is the Lot's wife who later turns into a pillar of salt a different person than Milcah? xxxx} and his daughter-in-law Sarai (his son Abram's wife), and they went out together from Ur of the Chaldees to go into the land of Canaan, and came in as far as Haran, and they dwelt there.%%
  \verse{11:32} Terah's days were 205~years, and Terah died in Haran.%%
\end{inparaenum}

  \heading{12}{The Lord promises blessings upon Abram~--- Abram and his company journey to Canaan~--- Abram build altars and calls upon the Lord~--- Abram and Sarai enter Egypt~--- they lie to Pharaoh, saying that Sarai is Abram's sister~--- Pharaoh finds out the truth and sends them out of the land}

\begin{inparaenum}
  \verse{12:1} The \textsc{Lord} said to Abram, ``Go\lit{for yourself} out from your land, from your family, from your father's house, to the land that I will show you.%%
  \verse{12:2} I will make a great nation of you. I will will you. I will make your name great. You will be a blessing.%%
  \verse{12:3} I will bless those who bless you and curse those that curse you.\lit{and those that curse you I will curse.} In you\ie{Because of you, Through you} shall all of the families of the earth be blessed.''%%
  \verse{12:4} Abram went on according as the \textsc{Lord} had spoken to him, and Lot went with him. Abram was 75~years old when he left\lit{went out of} Haran.%%
  \verse{12:5} Abram took his wife (Sarai), his nephew\lit{brother's son} Lot, all the goods\halot{xxxx}{furnishings, gear, utensils} that they had acquired,\alt{gathered} and the people\halot{xxxx}{acquire people, rear persons (slaves?)} whom they had gathered\ed{Converted? Bought?} in Haran, and they went to go towards the\understood\ land of Canaan, and they came into the land of Canaan.%%
  \verse{12:6} Abram passed from one side of the land to the other, to the place of Shechem, to the great tree\alt{oak} of Moreh. And the Canaanite was then in the land.%%
  \verse{12:7} The \textsc{Lord} appeared to Abram, and said, ``I will give this land to your posterity.'' He\ed{Abram} built an altar there to the \textsc{Lord} who had appeared to him.%%
  \verse{12:8} He moved on from there to the mountain on the east of Bethel, and pitched\lit{stretched out} his tent with Bethel to the west and Ai\alt{Hai} to the east. He built an altar there to the \textsc{Lord} and called on the name of \textsc{God}.%%
  \verse{12:9} Abram traveled, going and traveling to the south.%%
  
  \verse{12:10} There was a famine in the land, and Abram went towards Egypt to stay as a foreigner and sojourn there because the famine in the land was grievous.\alt{heavy, oppressive}%%
  \verse{12:11} As he got near\alt{drew near, approached} to enter Egypt, he said to his wife Sarai, ``Please, I know that you are a beautiful woman to look at.%%
  \verse{12:12} When the Egyptians see you, they will say, `This is\understood\ his wife,' and they will kill me and keep you alive.%%
  \verse{12:13} Please say that\understood\ you're my sister in order that everything will be fine on my account\alt{for my sake} and that\understood\ my soul may live on account of you.''\lit{on your account, for your sake.}%%
  \verse{12:14} When Abram entered Egypt, the Egyptians saw the woman, that she was really beautiful.%%
  \verse{12:15} Pharaoh's princes saw her and praised her to Pharaoh and took the woman to Pharaoh's house.\ed{This is a strong argument for Abraham's seemingly unethical behavior in having his wife lie and say that she's his sister: the Egyptian princes see her and immediately take her (kidnap?) to Pharaoh. It isn't clear what they would have done to Abram if he was married to her, but him being a brother made him not a threat and might have been the thing that saved his life.}%%
  \verse{12:16} He\ed{Pharaoh?} treated Abram well for her sake\alt{because of her}~--- he had\lit{there was to him} sheep, cattle,\ca{\sampen\ | \Hebrew{מאד כבד מקנה}}{Samaritan Pentateuch: ``numerous, mighty livestock/cattle''} male donkeys, servants, female slaves,\alt{handmaids} female donkeys, and camels.%%
  \verse{12:17} The \textsc{Lord} afflicted Pharaoh and his house\ca{prb add cf 20,17}{probably added. Compare \vref{Gen}{20}{17}} with great plagues because of Abram's wife Sarai.%%
  \verse{12:18} Pharaoh called to Abram, and said, ``What is this you've done to me? Why didn't you tell me she was your wife?%%
  \verse{12:19} Why did you say, `She is my sister?' So I'd marry her?\lit{take her for myself as a wife.} Now, there's your wife. Take her\understood\ and go.''%%
  \verse{12:20} Pharaoh commanded his men concerning him,\ed{Abram} and sent him and his wife and all that they had away.%%
\end{inparaenum}

  \heading{13}{Abram's company and Lot return to the site of Abram's altar by Bethel and Ai~--- because of the overabundance of their property, they cannot live together~--- Abram and Lot's herders get in a fight and Abram and Lot go their separate ways~--- Abram lives in Canaan, Lot over by Sodom~--- the Lord promises to give all of the surrounding land to Abram and his posterity~--- Abram moves to Mamre and builds an altar}

\begin{inparaenum}
  \verse{13:1} Abram~--- he, his wife, and everything\ed{everyone?} he had, and Lot with him~--- went up from Egypt towards the south.%%
  \verse{13:2} Abram was incredibly wealthy in cattle and in silver and in gold.%%
  \verse{13:3} He went on his journeys from the south all the way to\lit{as far as} Bethel, to the place that his tent had been at first, between Bethel and Ai,%%
  \verse{13:4} to the place of the altar that he had made there in the first place\lit{at first}~--- Abram there called on the name of the \lord.%%
  \verse{13:5} Also, Lot (who had gone with Abram) had sheep and cattle and tents.%%
  \verse{13:6} But the land could not support\alt{bear} them to live\alt{dwell, habitate} together because they had too many possessions,\lit{their property was too much} so they couldn't live together.\ed{There's an interesting principle here about only hol\kern0.025em ding onto what's most important to you: if they had sacrificed possessions in order to live together, would the incident at Sodom and Gomorrah have happened?}%%
  \verse{13:7} There was a dispute\halot{xxxx}{\textbf{dispute}, \textbf{quarrel} (in public, with words, complaints, assertions, reproaches)} between Abram's cattle herders and Lot's cattle herders (the Canaanite and the Perizzite were then dwelling in the land),%%
  \verse{13:8} and Abram said to Lot, ``Please don't let there be contention between me and you, between my herders and your herders, because we're brothers.%%
  \verse{13:9} Isn't the entire land before you? Please separate yourself from me. If you go to the left,\ie{north} I'll go to the right.\ie{south} If you go to the right, I'll go to the left.''%%
  \verse{13:10} Lot looked up\lit{lifted his eyes} and saw the whole environs of Jordan\halot{xxxx}{\textbf{circuit}, \textbf{environs}: \textit{kikkar hayyard\=en} the broad southern portion of the Ghor \haref{Gn}{13}{10}}\ed{Ghor is in the region of Zoar.} that was a thoroughly watered land\understood\ (before the \lord\ wiped out Sodom and Gomorrah as the \lord's garden (like the land of Egypt)) as you go towards Zoar.%%
  \verse{13:11} Lot chose all of the Jordan environs for himself; and Lot journeyed from the east. And they parted one from another.%%
  \verse{13:12} Abram lived in the land of Canaan, and Lot lived in the cities of the circuit\alt{environs}\ed{The ``circuit'' was a group of five cities close to each other.} and pitched his tent as far as Sodom.%%
  \verse{13:13} The people\ed{men (literally), but Hebrew works much like French that a mixed group of men and women is a masculine. This, of course, assumes that it's not just the men of the city that are being wicked (more likely).} of Sodom were evil, great sinners before the \lord.%%
  \verse{13:14} The \lord\ said to Abram (after he'd separated from Lot), ``Please look up\ed{This could be easily interpreted either as (literally) looking up or (figuratively) not being downtrodden.} and see from where\lit{the place} you are now, to the north, south, east, and west,%%
  \verse{13:15} because I will give all of the land that you see to you and your posterity forever.%%
  \verse{13:16} I will make your posterity like the dust\halot{xxxx}{\textbf{dry, fine particles of dirt}, \textbf{dust}} of the earth so that if anyone is able to count\lit{number} the dust of the earth, they shall also count your posterity.%%
  \verse{13:17} Get up, walk about\ed{Hithpael} in the land to its length and breadth because I will give it to you.''%%
  \verse{13:18} Abram packed up\ed{usually means to pitch a tent, but the context here shows that Abram is moving his tent.} his tent and came and loved among the oaks\halot{xxxx}{\textbf{great tree}, `tree of God'}\ca{Relevant part: \masoretic\ semper pl, \septuagint\peshitta\vulgate\ sg}{The Masoretic Text is always plural while the Septuagint, Peshitta, and Vulgate are singular}\ed{This possibly being singular is significant because it means that the ``oak of Mamre'' is a landmark, possibly a cultic site.} of Mamre (which are in Hebron) and there built an altar to the \lord.%%
\end{inparaenum}

  \heading{14}{The prominent kings in the land battle with each other~--- Lot is taken as a spoil of war~--- Abram and his men pursue and engage the kings~--- Melchizedek blesses Abram and gives thanks to God~--- Abram negotiates with the king of Sodom}

\begin{inparaenum}
  \verse{14:1} In the days of Amraphel king of Shinar, Arioch king of Ellasar, Chedorlaomer king of Elam, and Tidal king of the nations,%%
  \verse{14:2} that\understood\ they warred\lit{made war} with Bera king of Sodom, Birsha king of Gomorrah, Shinab king of Admah, Shemeber king of Zeboim, and the king of Bela (which is Zoar).%%
  \verse{14:3} All of these joined forces in the Siddim valley\alt{low-lying plain, flat country} (which is the Dead\lit{Salt} Sea).%%
  \verse{14:4} They served Chedorlaomer for twelve years, and revolted\alt{rebelled}\halot{xxxx}{politically} the thirteenth year.%%
  \verse{14:5} In the fourteenth year, Chedorlaomer and the kings that were with him came and attacked\lit{smote} the Rephaim in Ashteroth Karnaim, the Zuzim in Ham, the Emim in Shaveh-kiriathaim,%%
  \verse{14:6} and the Horites on their\ca{1 c \sampen\ Vrs \Hebrew{בהררֵי}}{roughly: the Samaritan Pentateuch gives ``their mountain'' in construct form, making it ``Mount Seir'' and not ``their Mount Seir.''} Mount Seir to El-paran which is by the desert.%%
  \verse{14:7} The returned and came into Enmishpat (which is Kadesh) and attacked the entire field of the Amalekites, as well as the Amorites who live in Hazezon-tamar.%%
  \verse{14:8} The king of Sodom, the king of Gomorrah, the king of Admah, the king of Zeboim, the king of Bela (which is Zoar), they went out and did battle with them in the Siddim valley%%
  \verse{14:9} with Chedorlaomer king of Elam, Tidal king of the nations, Amraphel king of Shinar, and Arioch king of Ellasar~--- four kings against\lit{with} five.%%
  \verse{14:10} The Siddim valley is full of asphalt\alt{bitumen} pits, and the kings of Sodom and Gomorrah\ed{Two different people, not one person who was the king of both Sodom and Gomorrah.} fled and fell there; and those who remained fled to the mountain.\ed{The Hebrew here is a little difficult to parse: \Hebrew{נָּסוּ} \Hebrew{הֶרָה} \Hebrew{הַנִּשְׁאָרִים}: those who remained (gerund: \Hebrew{שׁאר} is the verb form of ``remain'') mountainward (\Hebrew{הר} (mountain) with a directional he) fled.}%%
  \verse{14:11} They took all of their property to Sodom and Gomorrah~--- all of their food~--- and left.%%
  \verse{14:12} They took Lot\ed{Lot was living in the area, not directly in Sodom, and was taken as a spoil of war.} (Abram's nephew\lit{brother's son})\ca{add}{added} (he was living in Sodom) and his property, and left.%%
  \verse{14:13} The fugitive\halot{xxxx}{from danger} came and told Abram the Hebrew as he lived by the oaks of Mamre the Amorite, brother of Eshcol and \lit{brother of}Aner,\ca{\sampen\ \Hebrew{ענרם}, 1QGenAp \textit{rnm}, \septuagint\ \Greek{Αυναν}}{the Samaritan Pentateuch gives Anaram; 1QGenAp (Apocryphal Aramaic Genesis from cave 1 in Qumran) gives \textit{rnm}; the Septuagint gives \textit{Aunan}} \lit{these were}Abram's allies.\lit{in a covenant/alliance with}%%
  \verse{14:14} Abram heard that his brother had been taken captive and he armed\halot{xxxx}{\Hebrew{ריק}: ``[T]he absense of the qal is noticable, and raises the question as to whether the hif.\ might not be a denominative verb from the adj.\ \Hebrew{רֵיק}. The Aramaic dialects also suggest this possibility, while the Akkadian D/\v S-themes and the Arabic IV-theme may be causatives of the basic theme, but not necessarily always.'' ``---3.\ \Hebrew{חֶרֶב} \Hebrew{הֵרִיק} [this seems to be the closest meaning, none being supplied for this verse] to draw the sword, but the interpretation of the expression is questionable: either: ---a.\ to pour out the sword, thus Gesenius-B.; or\thinspace ---b.\ to `pour out the sheath' (\Hebrew{תַּעַר}) referring to the sword, meaning to remove the sword from its sheath.''}\ed{From the above explanation, it seems that Abram is drawing \textit{his sword} (understood) for his men; in other words, arming them. The \textsc{lsg} even gives ``arma'' (armed).}\ca{1 c \sampen\ \Hebrew{וַיָּדֶק}, sic frt \septuagint\ \textit{(}\Greek{ὴρίθμησεν}\textit{)}}{one when the Samaritan Pentateuch gives this word with a dalet, not a resh; so perhaps the Septuagint gives xxxx} his experienced\halot{xxxx}{to learn, to make experienced} servants\ed{gerund, and clarified in the next verse} who were born\ed{\Hebrew{ילד} has the sense of being born and fathering, but not raising.} in his house~--- 318~men\understood~--- and they pursued them\understood\ to Dan.%%
  \verse{14:15} He and his servants divided into group during the nighttime and smote them and pursued them to Hobah (which is to the left of Damascus).%%
  \verse{14:16} He brought back all the property. He also brought back his brother Lot\ed{This is given as restrictive information because Abram had two brothers (Nahor and Haran) and because Lot is actually Abram's nephew (Haran's son).} and his property, as well as the women\ed{Why is ``women'' separate from ``people?'' Should this rather be ``wives'' instead of ``women?''} and the people.%%
  \verse{14:17} The king of Sodom went out to meet him (after he'd come back from smiting Chedorlaomer) in the Shaveh valley (which is the king's valley).%%
  \verse{14:18} Melchizedek king of Salem brought bread and wine, he being a priest of the Most High God.\alt{high god El.}%%
  \verse{14:19} He blessed him, and said,\smallskip%%
  
  \pb ``Blessed be Abram\pa of the Most High God,\pa possessor of Heaven and Earth.%%
  
  \pb \verse{14:20} And blessed be the Most High God\pa who has handed over\alt{delivered}\pa your enemies\alt{oppressors, adversaries} into your power.''\thinspace\lit{hand.}%%
  
  \noindent He gave him\ie{Abram gave Melchizedek} a tenth of everything.
  
  \verse{14:21} The king of Sodom said to Abram, ``Give me the people and take the property for yourself.''%%
  \verse{14:22} Abram said to the king of Sodom, ``I have raised\alt{lifted up} my hand to the \lord, the Most High God, possessor of Heaven and Earth:%%
  \verse{14:23} from a thread\alt{string, cord} to a sandal-thong,\halot{xxxx}{[an] object of minimum value}\halot{xxxx}{neither thread nor sandal-thong} I will not take from anything \lit{that}you have so that\understood\ you can't say, `I have made Abram rich.'%%
  \verse{14:24} Only what the young men have eaten, and the share of the booty of the men who have gone with me~--- Aner, Eshcol, and Mamre~--- let them take their share.''%%
\end{inparaenum}

  \heading{15}{The Lord promises great things to Abram~--- Abram asks the Lord what is being promised, seeing as he has no children~--- the Lord reaffirms that Abram will have posterity~--- Abram offers sacrifices to the Lord~--- the Lord promises Abram blessings}

\begin{inparaenum}
  \verse{15:1} After these things, the word of the \lord\ came to Abram in a vision, saying, ``Don't be afraid, Abram. I am your refuge.\alt{protection, shield}\ed{This is so incredibly comforting.} Your reward will be very great.''%%
  \verse{15:2} Abram said, ``Lord \god, what are You going to give me? I am\lit{go} childless.\ed{This is such a touching exchange and gives us a good look into the emotions and personality of Abram. After all this time and after all the marvelous things that the Lord has given him, when presented with being given more his first thought turns to being childless. For Abram, having a child is his ultimate wish. He's been promised countless things before~--- and has even seen some of those promises fulfilled~--- but his ultimate desire, especially in his older age, is to have a child. Additionally, he's been promised seed beyond measure. His hopes have been lifted this whole time, yet he and Sarai have never had success in having a child. How heartbroken he must be! To be promised things from God and not see them fulfilled. Did he feel empty? Forsaken? Or did he know that God would eventually fulfill that promise? And maybe he even knew that it might only be a promise to be fulfilled after this life (which doesn't really work because the promise that he received would require having children while alive in order to be fulfilled). But these things have stayed with him, festered in his mind, to the point that when God promises him a very great reward, he immediately asks how that can be because he is childless.} The [unexplained] son\halot{xxxx}{\Hebrew{מֶשֶׁק}, \sampen$.^\text{M216}$ \textit{m\=a\v saq}, \septuagint\ \Greek{Μασεκ}: \Hebrew{בֵּיתִי} \Hebrew{מֶשֶׁק} \Hebrew{בֶּן}, subsequently glossed with \Hebrew{דַמֶּשֶׁק} \Hebrew{הוּא} \haref{Gn}{15}{2}; unexplained., ? Ug.\ \textit{m\v sq}}\ed{Possibly: steward (\textsc{darby}, \textsc{kjv}), acquired son (\textsc{ylt}), inheritor (\textsc{lsg}); all of these are understood through further context given in verse~3.} is Eliezer of Damascus.''\ca{prb gl aram ad \Hebrew{בן־משׁק}}{probably an Aramaic gloss to \Hebrew{בן־משׁק} (son of [unexplained])}%%
  \verse{15:3} And Abram said, ``You haven't given me posterity, and a son of my house will be my heir.''%%
  \verse{15:4} The word of the \lord\ came\understood\ to him,\ed{It's interesting that we have no idea the timeline here. It feels like a conversation (and we know that God communicates with man as men typically communicate), but this could very easily be over the period of nights, weeks, years. Even verse~3 begins with ``And Abram said'' instead of simply continuing his previous quote, so this could very easily not be all in one typical conversation. That just adds to the emotional turmoil that Abram is having to deal with.} saying, ``He\lit{This one} won't be your heir, but the one who will come out of your body\halot{xxxx}{\textbf{(trunk of) body}, \textbf{belly} as seat of origin of man \haref{Gn}{15}{4};\dots~--- 3.~\textbf{inner parts} (as seat of feelings and excitement)} will be your heir.''%%
  \verse{15:5} He brought him outside,\ed{This is such a personal way to say this. Abram wasn't led, he was brought. He wasn't told, he was taken around the shoulder and shown.} and said, ``Look at the sky and count the stars\ed{This conversation (at least this part of it) is happening at night.} if you can.''\lit{count them.} And He said to him, ``So shall your posterity be!''%%
  \verse{15:6} And he believed in\alt{relied on, was convinced of, put trust in} \god\ and considered it to be righteousness to him.%%
  \verse{15:7} He said to him, ``I am the \lord\ who brought you out of Ur of the Chaldees, to give you this land to possess it.''%%
  \verse{15:8} He said, ``Lord \god, how will I know that I possess it?''%%
  \verse{15:9} He said to him, ``Take for me a three-year-old young cow,\alt{heifer} a three-year-old goat, a three-year-old ram, a turtle-dove,\halot{xxxx}{\textbf{turtle-dove}, \textit{Streptopelia turtur} (and other species of \textit{Columba})} and a turtledove.''\thinspace\halot{xxxx}{young bird: \textbf{turtledove} \haref{Gn}{15}{9}, young eagle \haref{Dt}{32}{11}}%%
  \verse{15:10} He took all of these and cut them in pieces\alt{cut them in half} and placed each piece\halot{xxxx}{of sacrificial meat} against its friend;\alt{fellow} but he didn't cut the birds in pieces.%%
  \verse{15:11} The birds of prey came down on the carcasses and Abram chased them away.\lit{caused them to turn away.}%%
  \verse{15:12} The sun was going down and a deep sleep fell on Abram. And a terror, a great darkness, fell on him.%%
  \verse{15:13} He said to Abram, ``\emph{Know} that your posterity will be aliens\halot{xxxx}{\textit{g\=er} is a man who, either alone or with his family, leaves his village and tribe, because of war (\haref{2S}{4}{3}), famine (\haref{Ru}{1}{1}), pestilence, blood-guilt etc., and seeks shelter and sojourn elsewhere, where his right to own land, to marry, and to participate in the administration of justice, in the cult, and in war is curtailed: \textbf{sojourner}, \textbf{alien}} in a land that is not theirs, and they shall serve them, and they\ed{ambiguous} shall afflict them for four hundred years.%%
  \verse{15:14} Additionally, I will pass judgment on\alt{pass a sentence on, execute judgment on, and derivatives} the nation they serve; afterwards, they shall go out having\lit{with}\understood\ great property.%%
  \verse{15:15} You shall go to your fathers in peace, and you shall be buried in a good old age.%%
  \verse{15:16} In the fourth generation, they will turn back to here because the Amorite's guilt\halot{xxxx}{incurred by offense, sin} is, until now, not complete.''%%
  \verse{15:17} When the sun set and it was dark,\lit{the sun went in and darkness was} there was a smoking fire-pot and a torch of fire\ed{duh} that passed between those pieces.%%
  \verse{15:18} On the same day, the \lord\ made a covenant with Abram, saying, ``I will give this land to your posterity: from the river\ca{prp \Hebrew{מִנַּחַל}}{it has been proposed: stream, often seasonal, but sometimes perennial)} of Egypt\halot{xxxx}{\textit{Wad\^i el-'{}Ar\^i\v s}} to the great river, the Euphrates river\ca{frt add}{perhaps added}~---%%
  \verse{15:19} the Kenites, Kenizzites, Kadmonites,%%
  \verse{15:20} Hittites, Perizzites, Rephaim,%%
  \verse{15:21} Amorites, Canaanites, Girgashites, Jebusites.%%
\end{inparaenum}

  \heading{16}{Sarai, barren, prompts Abram to sleep with her slave Hagar\kern0.1em\ed{We don't know if she has more than one slave so I left Hagar as restrictive information and not as an appositive.}~--- he does and the slave conceives~--- Sarai is humiliated by her decision, and Hagar leaves~--- an angel appears to Hagar and persuades her to return~--- Hagar gives birth to Ishmael}

\begin{inparaenum}
  \verse{16:1} Sarai, Abram's wife, hadn't borne any children to him; but she had an Egyptian slave\halot{xxxx}{\textbf{female slave}, \textbf{maidservant} (not clearly distinguished from \textit{'\=am\^ a})}\ed{I'm not using ``slave'' here to shake the paradigm or anything, it's just meant to jostle to reader into thinking more deeply about the situation instead of skimming through.} named\lit{whose name was} Hagar.%%
  \verse{16:2} Sarai said to Abram, ``The \textsc{Lord} has made me barren. Please go in to my slave so that I might be built up from her.'' And Abram listened to Sarai's counsel.\lit{sound, voice, call; but none of these make sense in context.}\ed{Do you think Abram immediately listened and just went for it, or was this a longer conversation? I could see it going both ways: he's spent so much time pleading with the Lord and now his wife is opening up a path so that he can receive these blessings so he just goes for it; alternatively, he's spent so much time talking about this with God and meditating that there's no way he just rushes into anything because that hasn't been his style all along.}%%
  \verse{16:3} Abram's wife, Sarai, took her Egyptian slave, Hagar, (at the end of tenth year that Abram had lived in the land of Canaan) and gave her to her husband, Abram, to be his wife.%%
  \verse{16:4} He went in to Hagar and she conceived; and when she saw that she had conceived, her mistress\ie{Sarai} was of no account in her eyes.%%
  \verse{16:5} Sarai said to Abram, ``My wrong\halot{xxxx}{\textbf{violence}, \textbf{wrong}, often a cry for help; \textit{\d{h}\superit{a}m\=as\^i} the violence which I suffer, with \textit{'al} expressing responsibility for the violence, as a curse \haref{Gn}{16}{5}} is upon you! I have given my slave into your bosom. And now she sees that she has conceived and I am of no account in her eyes. The \textsc{Lord} judge between me and you.''%%
  \verse{16:6} Abram said to Sarai, ``Your slave is in your control\lit{hand}~--- do with her however you see fit.''\lit{do to/with her what is [the] good in your eyes.} So Sarai humiliated her, and she fled from her presence.%%
  \verse{16:7} The angel of the \textsc{Lord} found her by a spring of water in the desert, by the fountain on the way to Shur,%%
  \verse{16:8} and he said, ``Hagar, Sarai's slave, where are you coming from? Where are you going?'' And she said, ``I am fleeing from before my mistress, Sarai.''%%
  \verse{16:9} The angel of the \textsc{Lord} said to her, ``Return to your mistress and humble yourself under her power.''\ed{This is a difficult passage because we simply don't have enough information. Was the Lord mad with her for leaving and that's why he sent an angel to her? (Laman and Lemuel had an angelic visitation so it's not out of the realm of possibilities.) The Lord is certainly concerned for her (because He worries about all of His children), but was He simply trying to help her make the best decision for her?} %%
  \verse{16:10} The angel of the \textsc{Lord} said to her, ``Your posterity will greatly multiply, such that it won't be numbered because of\lit{for} its\understood\ greatness.''\lit{not numbered from great.}%%
  \verse{16:11} The angel of the \textsc{Lord} said to her,%%
  
  \pc ``You have conceived,\pa and you will give birth to a son\pa and name him Ishmael%%
  
  \pc because the \textsc{Lord} has heard your humiliation.%%
  
  \pb \verse{16:12} He will be\pa a wild ass%
  \halot{xxxx}{\textbf{zebra} (other: \textbf{wild ass}, \textbf{onager})}%%
  man\pa and everyone's hand\pa will be against him.%%
  
  \pc He will live before the face of all of his brothers.''\smallskip%%
  
  \noindent\verse{16:13} She called on the name of the \textsc{Lord} who had spoken to her: ``You are the God who reveals\lit{makes Himself seen} Himself!'' because she had said, ``Here, too, I have seen Him after He's revealed Himself.''%%
  \verse{16:14} Therefore, the well was named Beer-lahai-roi\ie{The well of the Living One, my beholder} and it's between Kadesh and Bered.%%
  \verse{16:15} Hagar gave birth to a son for Abram; and Abram named his son whom Hagar had borne, Ishmael%%
  \verse{16:16} Abram was 86~years old when Hagar gave birth to Ishmael for Abram.%%
\end{inparaenum}

  \heading{17}{The Lord commands Abram to be perfect~--- continuation of Abrahamic covenant~--- name changed to Abraham~--- Canaan given to Abraham~--- covenant of circumcision set forth~--- Sarai changed to Sarah~--- Isaac promised~--- covenant to continue through him~--- Abraham and his household are circumcised}

\begin{inparaenum}
  \verse{17:1} Abram was 99~years old when the \textsc{Lord} appeared to him. He said, ``I am God Almighty\ed{\Hebrew{שַׁדַּי} is a name of deity, the meaning of which is unknown. In Ethiopic it means ``mountain,'' but \textit{Shaddai} is also the name of other gods (in the Palmyrene inscription), a group of gods (in the book of Balaam son of Beor), and also refers to the Hebrew deity.}~--- walk before Me and be perfect.%%
  \verse{17:2} I will give\alt{set up, establish} My covenant between us and I will greatly multiply you.''%%
  \verse{17:3} Abram fell on his face as God spoke with him, saying,%%
  \verse{17:4} ``My covenant is with you~--- you shall be like a father of a multitude of nations.%%
  \verse{17:5} You shall no longer be called Abram, but rather Abraham, for I have made you a father of a multitude of nations.%%
  \verse{17:6} You shall be exceedingly fruitful. You shall become many nations. Kings will be among your posterity.\lit{Kings will come from you.}%%
  \verse{17:7} I will establish a covenant between us and also between your offspring.\alt{progeny.} It shall be an everlasting covenant: to be the God of you and your children.%%
  \verse{17:8} I will give you and your descendants\lit{seed after thee} the land of your sojournings and the land of Canaan as an everlasting possession for I have become their God.''%%
  \verse{17:9} God said to Abraham, ``You and the generations that follow shall honor this covenant.%%
  \verse{17:10} This is My covenant, between Me and you and your progeny,\lit{your seed after you} that you shall observe carefully: every male shall be circumcised.''%%
  \verse{17:11} You shall circumcise the flesh of your foreskin as a sign of the covenant between Me and you.%%
  \verse{17:12} Every male in your generations shall be circumcised when they are eight days old: those born in a house and those, not your offspring (the children of a resident alien), that are bought with money.%%
  \verse{17:13} Those born in a house and those bought with your money shall certainly be circumcised. My covenant shall be an everlasting covenant in your flesh.%%
  \verse{17:14} The uncircumcised male, the flesh of whose foreskin is uncircumcised, shall be cut off\ed{excellent word choice} from My people: he has broken My covenant.%%
  
  \verse{17:15} God said to Abraham, ``Your wife will no longer be known as Sarai, but Sarah.%%
  \verse{17:16} I will bless her. Additionally, I will give you a son from her. I will bless her that she shall become a nation~--- people's kings will be among her posterity.''%%
  \verse{17:17} And Abraham fell on his face and laughed. He said in his heart, ``Shall a son be born to a hundred year old man? And shall a ninety year old woman bear him?''\lit{Shall one be born to an hundred year old? Shall a daughter who is ninety bear?}%%
  \verse{17:18} Abraham said to God, ``O that Ishmael might live in Your presence!''%%
  \verse{17:19} And God said, ``On the contrary, your wife, Sarah, shall bear a son. You shall name him Isaac. I will establish My covenant with him and it shall be an everlasting covenant with his descendants.%%
  \verse{17:20} Concerning Ishmael, I have heard you. Look, I will bless him, make him fruitful, greatly multiply him: he shall have\alt{beget, sire} twelve princes. I will make him a great nation.%%
  \verse{17:21} My covenant shall be with Isaac whom Sarah shall bear to you at this appointed time next year.''%%
  \verse{17:22} He finished speaking with him and God left\lit{went up from} Abraham.%%
  \verse{17:23} Abraham took his son, Ishmael, all who were born in his house, those bought with his money (all of the men in Abraham's house) and circumcised them\lit{the flesh of their foreskins} that day like God had told him to.%%
  \verse{17:24} Abraham was ninety-nine years old when the flesh of his foreskin was circumcised.%%
  \verse{17:25} His son, Ishmael, was thirteen when the flesh of his foreskin was circumcised.%%
  \verse{17:26} \lit{In}That same day, Abraham and his son, Ishmael, were circumcised.%%
  \verse{17:27} All of the men in the house~--- those born in the house and those bought with a resident alien's money~--- were circumcised as well.\lit{with him}%%
\end{inparaenum}

  \heading{18}{Abram entertains guests~--- a child is promised to Sarah, who had previously stopped menstruating~--- Sarah doubts the Lord's promise~--- the Lord reaffirms His promise~--- the Lord says He is going to destroy Sodom~--- Abraham argues that if there are any righteous people that the Lord not destroy it~--- they argue down to ten righteous people}

\begin{inparaenum}
    \verse{18:1} The \textsc{Lord} appeared to him by the oaks\cf{\vref{Gen}{13}{18}} of Mamre as\lit{and} he sat in the opening of his tent in the heat of day.%%
    \verse{18:2} He looked up and saw three men standing near him. When he saw them, he ran from the tent door to meet them and bowed down to the earth.%%
    \verse{18:3} He said, ``O Lord, please, if I've found approval in your eyes, then please don't overlook your servant.\ca{\sampen\ \Hebrew{עבדכם}\textbf{\ \dots\ }\Hebrew{תעברו}\textbf{\ \dots\ }\Hebrew{בעיניכם}}{the Samaritan Pentateuch pluralizes this (i.e., your servants)}%%
    \verse{18:4} Please, let some water be brought and wash your feet and stretch out and rest under the tree,%%
    \verse{18:5} and I will get a piece of bread and strengthen\halot{xxxx}{with food} you. After that you can pass through because you have therefore passed by to your servant.'' And they said, ``Do as you've said.''%%
    \verse{18:6} Abraham hurried towards the tent to Sarah, and said, ``Hurry! Three seahs\halot{xxxx}{measure of capacity, in one estimate~= approximately 7~L, 7.5~qt} of wheat\halot{xxxx}{\textbf{fine}(-ground) \textbf{wheat flour}, ground from inner kernels of wheat} flour, knead it\understood\ and make bread-cakes.''%%
    \verse{18:7} Abraham ran to the herd and took a frail\alt{tender, weak} but\ed{This could easily be ``a tender \textit{and} good calf.''} good calf\lit{son of the herd} and gave it to the young man who\lit{and he} hurried and made it.%%
    \verse{18:8} He took curdled milk\halot{xxxx}{a type of \textbf{curdled milk}, similar to yogurt} and milk and the calf that he'd made and placed them before them; and he stood in front of them under the tree and they\ed{ambiguous} ate.%%
    \verse{18:9} They said to him, ``Where is your wife, Sarah?'' He said, ``She's\understood\ in the tent.''%%
    \verse{18:10} He said, ``I will definitely return to you at this\understood\ time of life, and your wife, Sarah, will have a child.'' And Sarah was listening from behind him at the tent door.%% xxxx ylt: verse 11 starts at "and sarah was listening"
    \verse{18:11} Abraham and Sarah were old, advanced in days, and Sarah had stopped menstruating.\lit{stopped had become to Sarah the way of women.}%%
    \verse{18:12} Sarah laughed\ed{There seems to be, especially in the context of verse~14, a connotation of doubt here.} to herself,\lit{on the inside} saying, ``After I've become old shall I really have sexual pleasure? My lord\ie{Abraham} is old!%%
    \verse{18:13} The \textsc{Lord} said to Abraham, ``Why has Sarah laughed, saying, `Am I really going to give birth now that I'm old?'%% verse 14 starts here at the beginning of this verse in ylt
    \verse{18:14} Is anything\lit{any (understood) matter/thing} too difficult\alt{hard} for the \textsc{Lord}? At the appointed time I will return to you~--- at the time of life~--- and Sarah will have a son.''%%
    \verse{18:15} But Sarah denied, saying, ``I didn't laugh'' (because she was afraid). And He said, ``No, you did laugh.''%%
    \verse{18:16} The men rose up from there and looked down\halot{xxxx}{\textbf{nif.}: \textbf{look down} with \textit{`al} on (suggesting: from point of view of one standing above, i.e., `look down there'): \haref{1S}{13}{18}. \textbf{hif.}: \textbf{look down} (suggesting: from point of view of one standing below, i.e., `look down here') \haref{Gn}{18}{16}.} on Sodom, and Abraham went with them to send them off.%%
    \verse{18:17} And the \textsc{Lord} said, ``Should I keep what I'm doing secret from Abraham%%
    \verse{18:18} because Abraham will certainly become a great and vast\alt{mighty} nation, and all the nation of the earth will be blessed through him?%%
    \verse{18:19} Because I have covenanted\ed{\Hebrew{ידע} literally means ``to know,'' but in this context is has a covenental sense. See further in Appendix~\ref{app:covenants-in-antiquity}.} with him so that he will command his children and household after him that\lit{and, but can be just about any conjugation} they keep the way of the \textsc{Lord}: to do righteousness and justice so that the \textsc{Lord} can bring to pass\lit{on} that which He has spoken concerning him.''%%
    \verse{18:20} The \textsc{Lord} said, ``Sodom and Gomorrah's call for help\ca{\sampen\ \Hebrew{'צע} cf 21}{the Samaritan Pentateuch gives \Hebrew{'צע} [cry for help, call for help (mostly to God)], compare verse~21} is great and their sin is very great.%%
    \verse{18:21} I will go down and see if they've done everything according to the cry that is coming up to Me~--- and if not, I will know it.''%%
    \verse{18:22} They turned from there and went towards Sodom, but Abraham still\lit{as of yet} stood before the \textsc{Lord}.\ca{Tiq soph, lect orig \Hebrew{אברהם} \textbf{\dots} \Hebrew{ויהוה}}{Original reading of the Tiqqun sopherim [rabbinic literature meaning ``scribal correction'']: ``the \textsc{Lord}\dots\ Abraham''}%%
    \verse{18:23} Abraham came near, and said, ``Will you also take\alt{sweep, snatch away} the righteous with the wicked?%%
    \verse{18:24} Perhaps\halot{xxxx}{expression of hope, entreaty, fear} there are fifty righteous people\understood\ in\alt{within, in the midst of} the city~--- will you actually\alt{indeed} sweep away\understood\ instead of\lit{and not} pardon the place on account of the fifty righteous people who are there.\lit{in its midst}%%
    \verse{18:25} Be it far from\halot{xxxx}{originally `to the profane'} You to do this thing~--- to kill\lit{make dead} the righteous with the wicked~--- far be it from You! Won't the Judge of the whole earth do justice?''%%
    \verse{18:26} The \textsc{Lord} said, ``If I find fifty righteous people in Sodom, in the city, then I will forgive the whole place because of them.''\lit{for their sakes.}%%
    \verse{18:27} Abraham answered,\ed{It is unclear if this conversation happens all at the same time (e.g., Abraham asks for 50, rethinks and asks for 45, etc.) or if between conversations Abraham is going into the city searching for righteous people. Both options provide excellent teaching opportunities and neither is theologically incorrect.} and said, ``Please, I've begun\halot{xxxx}{\textbf{make a beginning} (usually expression of politeness or modesty)} to talk to the Lord\ca{\fragheb\super{3} mlt Mss \Hebrew{יהוה}}{multiple manuscripts from the Cairo Genizah have the Tetragrammaton [instead of \Hebrew{אֲדֹנָי}]}~--- I am dust and dust.%%
    \verse{18:28} Perhaps there are five lacking of the fifty righteous people~--- would You destroy the entire city because of the five?'' And He said, ``I won't destroy it if I can find forty-five.''%%
    \verse{18:29} He continued to speak with Him, and said, ``Perhaps there are only forty found there.'' And He said, ``I won't do\ca{\sampen\septuagint\ \Hebrew{אַשְׁחִית} cf \vulgate\ (\textit{percutiam}) et 28.31.32}{The Samaritan Pentateuch and the Septuagint both have \Hebrew{אַשְׁחִית} (bring ruin on); compare the Vulgate's \textit{percutiam}, meaning ``to strike.'' This same thing appears in verses 28, 31, and 32.} it because of the forty.''%%
    \verse{18:30} He said, ``Please don't let it anger the Lord.\lit{make to the Lord angry.} I speak: perhaps there are thirty there.'' And He said, ``I won't do it if there's thirty there.''%%
    \verse{18:31} He said, ``Please, I've begun to speak to the Lord:\ca{nonn Mss \Hebrew{יהוה}}{several manuscripts give the Tetragrammaton (\textsc{Lord})} perhaps if twenty can be found there.'' And he replied, ``I will not destroy it because of the twenty.''%%
    \verse{18:32} He said, ``Please don't let it anger the Lord. I speak just this once: perhaps there are only ten found there.'' And he replied, ``I won't destroy it because of the ten.''%%
    \verse{18:33} The \textsc{Lord} went away when He was done speaking with Abraham, and Abraham returned to his place.%%
\end{inparaenum}

  \heading{19}{Messengers come to Sodom and Lot convinces them to stay with him~--- the people of the city demand to have sex with these newcomers~--- Lot offers to give his daughters in exchange\ed{I think we should be more forgiving of Lot's daughters for the choice they made in sleeping with their father. Their father offered to let them be raped and their husbands refused to leave Sodom. However, it is interesting that Lot says that they are virgins since they're married. Did Lot have more daughters than we have record? Verse~15 tells of ``the two daughters who are present,'' but were there others or did these two (and only these two) daughters just happen to be present? And why then would the men say, ``[W]ho are present?'' Possibly he was referring to other daughters. Also, was he lying as a last-ditch ploy to save his family? But, if the men of the city did accept, what would he have done? Would he have actually given his daughters to them and let them be raped? Was he going to physically defend his family?}~--- the men tell Lot that the Lord has sent them to destroy the city~--- they instruct Lot to flee and save his family~--- the Lord destroys Sodom and Gomorrah and the surrounding area~--- Lot's wife looks back and is turned into a pillar of salt~--- Lot's daughters get him drunk and sleep with him~--- they both give birth to sons}

\begin{inparaenum}
  \verse{19:1} Two messengers\ca{prp \Hebrew{הָאֲנָשִׁים} cf 12.\thinspace16}{probably ``men,'' compare verses~12 and~16 [and 15]} came to Sodom in the evening as\lit{and} Lot was sitting at Sodom's gate. And Lot saw them and stood to greet them, and bowed down, his face to the ground.%%
  \verse{19:2} He said, ``My lords, please, please turn in to your servant's house: spend the night, wash your feet, get up early, and go on your way.'' But they said, ``No. We will spend the night in the open.''%%
  \verse{19:3} He urged them strongly, and they turned aside to him and came into his house. He prepared\lit{made} a feast for them, baked unleavened bread,\halot{xxxx}{for ordinary meals} and they ate.%%
  \verse{19:4} Before they lay down, the men\ed{Remember that ``men'' is the Hebrew mass noun for both men and women. It can technically just mean ``men,'' but I find it highly unlikely that such a sexually driven culture (cf.~v5) would solely consist of homosexuals.} of the city (the men of Sodom)\ca{frt gl}{perhaps glossed} from young to old~--- to the last man\lit{all the people from the edge}\halot{xxxx}{to the last man}~--- surrounded the house,%%
  \verse{19:5} and they called out to Lot and said to him, ``Where are the men that have come in to you tonight? Bring them out to us so we can have sex with them.''\ed{This is super messed up. Absolutely disgusting behavior.}%%
  \verse{19:6} Lot went out to them~--- to the entrance~--- closed the door behind himself%%
  \verse{19:7} and said, ``My brethren, please don't behave objectionably.\alt{do evilly}\ed{Bold talk coming from him.}%%
  \verse{19:8} I have two virgin\lit{who have not had sex with a man} daughters. Please let me bring them out to you and you can do whatever you want to them;\lit{do with them what is good in your eyes} just\alt{only} don't do anything to these men because\lit{for therefore} they've come under the shadow of my house.''\ed{Unarguably more messed up.}%%
  \verse{19:9} They said, ``Stand back!''\alt{Get away!} And they said,\ca{\missing\ \septuagint}{missing in the Septuagint} ``This one has come to stay, and he totally judges us!\understood\ Now, we will do evilly more so with you than with them!'' They strongly urged the man~--- Lot~--- and they drew near to break down the door.%%
  \verse{19:10} The men reached out\lit{stretched out their hands} and brought Lot into the house, to them, and closed the door.%%
  \verse{19:11} The men who were at the door of the house~--- from smallest to largest~--- were smitten with a dazzling light\ed{Other editions give ``blindness''} and they gave up\alt{got tired} looking for the door.%%
  \verse{19:12} The men said to Lot, ``Who else is here? A son-in-law,\alt{brother-in-law; although this doesn't make much contextual sense, especially given verse~14.} your sons, your daughters, and everyone you have in the city~--- bring them out of this\understood\ca{\sampen\septuagint\peshitta\ + \Hebrew{הזה} ut 13\thinspace.\thinspace14}{the Samaritan Pentateuch, Septuagint, and Peshitta all add ``this,'' as also in verses~13 and~14.} place%%
  \verse{19:13} because we are going to wipe this place out because their cries have been great before the \lord, so the \lord\ has sent us to wipe it\ca{\sampen\ \Hebrew{להשׁחיתה}; \vulgate\ suff 3 pl, l \Hebrew{תָם}---?}{the Samaritan Pentateuch gives a plural conjugation; the Vulgate has a third person plural suffix, read as plural}\ed{This above note is significant because it means that there could possibly be more than one city being destroyed.} out.''%%
  \verse{19:14} Lot went out and spoke to his sons-in-law (those who had married his daughters), and said, ``Get up and get out of this place because the \lord\ is going to destroy the city.'' But, in his sons-in-laws' view, he was joking around.\ed{Both Lot and his daughters made a great decision with these guys.}%%
  \verse{19:15} When the morning had broken,\lit{the dawn had ascended} the messengers\ca{prp \Hebrew{האנשׁים} cf 1\super{a--a}}{probably ``the men,'' compare verse~1} urged Lot, saying, ``Get up. Take your wife and your two daughters who are present lest you're all\understood\ swept away in the city's sin.''%%
  \verse{19:16} Lot lingered\ed{There is an incredibly rare cantillation mark used here (\Hebrew{וַֽיִּתְמַ֯הְמָ֓הּ}) called a Shalshelet (\thinspace\Hebrew{֓}\thinspace) that is said to represent a hesitation or a struggle with one's inner demons. In this case, it is said to mean that Lot is uncertain about leaving Sodom. I don't give too much credit to cantillation marks, but it's an interesting thought nonetheless.} and the men took him and his wife and his daughters by the hand, the \lord\ sparing\halot{xxxx}{\textbf{sparing}, forbearance \haref{Gn}{19}{16}, compassion \haref{Is}{63}{9}. $\dagger$} him; and they led him out and put him outside the city.%%
  \verse{19:17} When they'd brought them out, he\ca{\septuagint\peshitta\vulgate\ pl}{The Septuagint, Peshitta, and Vulgate have this in plural}\ed{\textsc{lsg} gives ``l'un d'eux,'' meaning ``one of them''} said, ``Get to safety!\lit{Escape for your life!}\halot{xxxx}{\textbf{get oneself to safety} \haref{Gn}{19}{17}} Don't look\alt{glance} behind you. Don't stand in any\lit{all} of the plain or the mountain lest you're snatched away.''%%
  \verse{19:18} Lot said to them, ``Please, not so, my Lord!\ca{prp \Hebrew{–ִי}}{it's been proposed that it's \Hebrew{–ִי} [see further in Appendix~\ref{app:adonai}]}%%
  \verse{19:19} Your servant has found favor in your eyes and you've made your loving-kindness great to me by keeping me alive,\alt{saving my life} but I can't escape to the mountain lest evil fastens itself to me and I die.%%
  \verse{19:20} Please, this city is close to fleeing\alt{escaping, slipping away} here and it's insignificant.\alt{trifling} Please let me flee to there~--- is that insignificant?~--- and I will live.''\lit{my life will live.}%%
  \verse{19:21} He\ed{One of them? The Lord?} said to him, ``I've also accepted your face for this thing\ed{What?} to not destroy the city that you've talked about.%%
  \verse{19:22} Hurry and escape from here because I can't do anything until you come here.'' That's why\lit{Therefore} the city is named Zoar.\ed{Means ``small'' or ``insignificant.''}%%
  \verse{19:23} The sun rose on the earth and Lot came into Zoar.%%
  \verse{19:24} \ca{frt tr huc 26}{Perhaps transpose verse~26 here [meaning that Lot's wife turned into a pillar of salt before the Lord destroyed the city]}The \lord\ caused sulfur and fire to rain down on Sodom and Gomorrah, from the \lord\ from Heaven,\ca{add?}{added [later]?}%%
  \verse{19:25} and He destroyed\alt{overthrew} these cities and all the plain and all the inhabitants of the cities and the growth\alt{what had sprouted} from the ground.%%
  \verse{19:26} His wife looked behind herself and she became a pillar of salt.%%
  \verse{19:27} Abraham got up early in the morning [and went]\ca{ins \Hebrew{וַיֵּלֶךְ}?}{insert ``and went?''}\ed{I used the same syntax that the \textsc{darby} uses because, even though I'm against making this an eclectic edition, I feel that the verse requires this possibly missing section.} to the place where he'd stood before the\lit{face of the} \lord.%%
  \verse{19:28} He looked toward Sodom and Gomorrah and all the\lit{face of the} plains\lit{land of the plains}\ca{pc Mss \sampen\septuagint\super{C\thinspace a\thinspace l}\peshitta\targum\super{Ms} invers, sic l et dl c Ms \Hebrew{ארץ}?}{a few manuscripts, the Samaritan Pentateuch, the Septuagint, Peshitta, and Targum (in inverse order), thus read and delete when the manuscript says ``land''?} and saw a smoke on the land going up like the smoke of a furnace.\alt{forge.}%%
  
  \verse{19:29} God destroyed the cities of the plain, and God remembered Abraham and sent Lot out of the midst of the destruction, when He destroyed the cities that Lot lived in.%%
  \verse{19:30} Lot went up to Zoar and lived in the mountain, his two daughters with him, because he was afraid to live in Zoar.\ed{Why?} So he and his two daughters\ca{\sampen\septuagint\vulgate\super{Mss}\ + \Hebrew{עִמּוֺ}}{The Samaritan Pentateuch, Septuagint, and manuscripts of the Vulgate add ``with him.''} lived in a cave.%%
  \verse{19:31} The eldest\ed{Interesting that she was unmarried.} said to the youngest,\alt{younger} ``Our father is old and there isn't a man in the world to come into us\ed{Does this work idiomatically the same as it does in modern English?} as is the custom in the whole world.%%
  \verse{19:32} Let's go and give Dad\lit{our dad} wine to drink and lie down with him so that we can preserve our father's posterity.''%%
  \verse{19:33} That night, they gave their dad wine to drink, and the eldest went in and she lay with her father; and he didn't know that she'd lied down or gotten up.\lit{and he didn't know in her lying down or her getting up.}%%
  \verse{19:34} On the next day, the eldest said to the youngest, ``I lay with my\ca{\septuagint\ suff 1 pl, l \Hebrew{אבינוּ}?}{The Septuagint pronominal suffix is first person plural [our], read ``our father?''} father last night.\alt{yesterday evening} Let us again\lit{also} give him wine to drink tonight, and you go in and lie with him so that we can preserve our father's posterity.''%%
  \verse{19:35} So again gave their father wine to drink that night, and the youngest got up and lay with him; and he didn't know that she'd lied down or gotten up.%%
  \verse{19:36} Both of Lot's daughters became pregnant by their father.%%
  \verse{19:37} The oldest gave birth to a son and named him Moab (he's the forefather of the present-day Moabites\lit{the Moabites to this day}).%%
  \verse{19:38} The youngest also gave birth to a son and named him Ben-ammi (he's the forefather of the present-day Beni-ammon\alt{children of Ammon}).%%
\end{inparaenum}

  \heading{20}{Abraham and Sarah travel southward and have an ordeal with Abimelech which tests their obedience~--- the matter is cleared up and they continue on their journey}

\begin{inparaenum}
    \verse{20:1} Abraham traveled from there towards the south\lit{the land to the south} and stayed between Kadesh and Shur and lived in Gerar.%%
    \verse{20:2} Abraham said regarding his wife, Sarah, ``She's my sister.''\ed{This word can refer to any female relative so it's not technically a lie. The important thing to remember with this story is that obedience to the Lord is primary, all other concerns secondary. Even when the Lord commands to do something contrary to the commandments, obedience is key; the Lord won't suffer his followers to disobey and will stop them before they sin. This pattern is seen time and again throughout the scriptures.} And Abimelech king of Gerar sent and took Sarah.%%
    \verse{20:3} So God came to Abimelech by night in a dream, and said to him, ``You're dead because of the woman whom you've taken~--- she's married to a husband.''%%
    \verse{20:4} But Abimelech hadn't come near to her\halot{xxxx}{to a woman (for sexual intercourse) \haref{Gn}{20}{4}} so he said, ``Lord,\ca{nonn Mss \Hebrew{יהוה}}{a few manuscripts have the Tetragrammaton} will You also kill a righteous nation?\ed{I doubt this is hyperbole. Gerar is almost certainly righteous and led by a righteous man. See verses~5--6.}%%
    \verse{20:5} Didn't he say to me, `She's my sister'? And she, even she said, `He's my brother'! With my blameless\alt{pure, complete, right, sound, orderly}\ylt{integrity} heart and with my innocent hands have I done this.''%%
    \verse{20:6} And God said to him in a dream, ``I, too, know that you've done this with your blameless heart; and I, too, have restrained you\alt{held you back} from sinning against Me. Therefore, I haven't allowed\alt{permitted} you to touch her.%%
    \verse{20:7} Now, return the man's wife because he's a prophet, and he will pray for you that you may live; but if you don't send her\understood\ back, know that you and everything you have will surely die.''%%
    \verse{20:8} Abimelech got up early in the morning and called for all his servants and spoke all these things in their ears, and the men were horribly afraid.%%
    \verse{20:9} Abimelech called Abraham, and said to him, ``What have you done to us?\ca{\peshitta\ \textit{`bdt lk}, prp \Hebrew{לְךָ} \Hebrew{עשיתִי}}{the Peshitta has xxxx, it has been proposed to be ``What have I done to you?''} How\lit{[In] what} have I sinned against you that you've brought on me and my kingdom a great sin? You've done things to me that shouldn't be done.''%%
    \verse{20:10} Abimelech said to Abraham, ``What have you seen that you've done this thing?''%%
    \verse{20:11} And Abraman said, \ca{\sampen\ + \Hebrew{יראתי} \Hebrew{כי}}{the Samaritan Pentateuch adds ``Because I've seen''}``Because I've said, `Surely, the fear of God isn't in this place and they will kill me because of my wife.%%
    \verse{20:12} Also, she's definitely\lit{truly, surely} my sister~--- my father's daughter, just\lit{only} not my mother's daughter\ed{his half-sister}~--- and she became my wife.%%
    \verse{20:13} When God caused me to wander from my father's house,\ca{\sampen\ + \Hebrew{מולדתי} \Hebrew{ומארץ} cf 12,1}{the Samaritan Pentateuch adds ``and from the land of my birth,'' compare \vref{Gen}{12}{1}} I said to her, `This is your loving-kindness that you shall do with me:\ie{This is the aid that you shall render to me:} at every place we go,\lit{enter, come} you shall say about me, ``He is my brother.''\thinspace'\thinspace''%%
    \verse{20:14} Abimelech took\ca{\sampen\septuagint\ + \Hebrew{ו} \Hebrew{כסף} \Hebrew{אלף} cf 16}{the Samaritan Pentateuch and Septuagint add ``1\thinspace000~pieces of silver and'' compare verse~16} sheep and oxen, slaves and female slaves, and gave them to Abraham, and sent back his wife, Sarah.%%
    \verse{20:15} Abimelech said, ``My land is before you: dwell where you think it's good.''\lit{in good in your eyes dwell.''}%%
    \verse{20:16} And he said to Sarah, ``I've given your brother 1\thinspace000~pieces of silver; let that be a covering for the eyes\halot{xxxx}{2.\ metaphor \textit{k\super{e}s\^ut `\^enayim} covering for the eyes: declaration of unbesmirched honor of a woman (other: veil) \haref{Gn}{20}{16}} to you and to everyone who's with you.'' And by all this she was vindicated.\alt{turned out to be in the right.}\ca{prp \Hebrew{נכחת} \Hebrew{כֻּלּוֺ} \Hebrew{ואת} (= et apud hos omnes iustificata es) vel \Hebrew{נכחת} \Hebrew{כֻּלּוֺ} \Hebrew{וְאַתְּ} (= et tu omnibus his rebus iustificata es)}{it has been proposed to be ``and in all these are justified'' or ``and you are justified of all these functions''}%%
    \verse{20:17} Abraham prayed to God, and God healed Abimelech and his wife and his maidservants, and they bore children.\understood%%
    \verse{20:18} Because the \textsc{Lord} had tightly shut all the wombs in Abimelech's household on account of Sarah, Abraham's wife.%%
\end{inparaenum}

  \heading{21}{Sarah gives birth to a son, Isaac~--- she freaks out and demands that Abraham gets rid of Hagar and her child~--- the Lord tells Abraham to listen to Sarah, but not to worry because a great nation will also come from Ishmael~--- Abraham and Abimelech deal with their problems}

\begin{inparaenum}
  \verse{21:1} The \textsc{Lord} took care of Sarah\alt{took up Sarah's cause} as He'd said, and the \textsc{Lord} did to Sarah as He'd spoken.%%
  \verse{21:2} Sarah conceived and bore a son to Abraham in his old age at the appointed time that God had told\lit{spoken to} him.%%
  \verse{21:3} Abraham named his son, who was born to him, who Sarah bore to him, Isaac.%%
  \verse{21:4} Abraham circumcised his son Isaac\ed{This can't be offset by commas because Abraham has other children. See verse~9.} when he was eight days old\lit{a son of eight days} as God had commanded him.%%
  \verse{21:5} Abraham was 100~years old when Isaac was born to him.%%
  \verse{21:6} Sarah said, ``God has made me laugh~--- everyone who hears will laugh with me.''%%
  \verse{21:7} She said, ``Who said to Abraham, `Sarah has nursed\alt{suckled} children'? Because I've given birth\ca{\sampen\targum\super{J} + \Hebrew{לוֺ}}{the Samaritan Pentateuch and Targum (Targum Pseudo-Jonathae secundum M. Ginsburger) add ``for him''} to a son in his old age.''%%
  \verse{21:8} The boy grew and was weaned, and Abraham made a great feast\halot{xxxx}{\textbf{feast}, \textbf{banquet} with wine} on the day that Isaac was weaned.%%
  \verse{21:9} Sarah saw Hagar the Egyptian's son, whom she had borne to Abraham, laughing.\ca{\septuagint\ + \Greek{μετὰ Ισαακ τοῦ υἱοῦ αὐτῆς}}{the Septuagint adds ``after Isaac's son this''}%% xxxx: fix Greek translation
  \verse{21:10} She said to Abraham, ``Drive out\alt{Divorce; but this doesn't really work because they're not married} this slave and her son because the son of a slave shall not inherit with my son~--- with Isaac.''%%
  \verse{21:11} This\understood\lit{The} thing was super displeasing to Abraham\lit{in the eyes of Abraham} on account of his son.%%
  \verse{21:12} God said to Abraham, ``Don't let it be wrong in your eyes or in the eyes of the child because of the handmaid: listen to everything \lit{that}Sarah's said to you because your posterity shall be through Isaac.\lit{in Isaac shall seed be called to you.}%%
  \verse{21:13} Furthermore, I will make a nation\ca{\sampen\septuagint\peshitta\vulgate\ + \Hebrew{גָּדוֺל}, ins}{the Samaritan Pentateuch, Septuagint, Peshitta, and Vulgate add ``great'', they insert it} of the handmaid's son because he is your posterity.''%%
  \verse{21:14} Abraham got up early in the morning and took bread and a bottle\halot{xxxx}{(goat-)\textbf{skin}, stopped at both ends with pitch, for water, wine, oil, etc.} of water and gave it to Hagar,\ca{huc tr \super{b--b} cf \septuagint\super{19.314}\peshitta}{hither transpose \Hebrew{וְאֶת־הַיֶּלֶד}, compare the Septuagint~19:314 and Peshitta} placing it on her shoulder, and to the boy; and he send her off. And she left and wandered in the desert of Beersheba.%%
  \verse{21:15} The bottle of water was consumed, and she placed the boy under one of the bushes.%%
  \verse{21:16} And she went and sat down by herself a ways away,\lit{afar off} about a bowshot, because she said, ``Don't let me see the boy die.''\lit{``Do not let me look on the death of the boy.''} So she sat a ways away, lifted up her voice, and wept.%%
  \verse{21:17} God heard the boy's voice; and the messenger of God called to Hagar from Heaven, and said to her, ``What's up, Hagar?\lit{What is it to you, Hagar?} Don't be afraid, because God has heard the boy's voice where he's at.%%
  \verse{21:18} Get up. Lift up the boy and take him by the hand because I will make a great nation of him.''%%
  \verse{21:19} God opened her eyes and she saw a well of water. She went and filled the bottle with\understood\ water and she gave the boy a\alt{to} drink.%%
  \verse{21:20} God was with the boy. He grew and lived in the desert and became an archer.%%
  \verse{21:21} He lived in the Paran desert, and his mother fetched him a wife from the land of Egypt.%%
  
  \verse{21:22} At that time, Abimelech and Phichol, the caption of his army, spoke to Abraham, saying, ``God is with you in everything you do.%%
  \verse{21:23} And now, swear to me, here, by God that you won't lie to me or my descendant\ed{\Hebrew{נִין}} or my descendant.\ed{\Hebrew{נֶכֶד}} Show to\lit{Do with} me and the land you're sojourning in according to the loving-kindness that I've shown to you.''%%
  \verse{21:24} And Abraham said, ``I so\understood\ swear.''%%
  \verse{21:25} Abraham settled a quarrel\halot{xxxx}{\textbf{argue out together} (in a legal dispute)} with Abimelech concerning the matter of a well of water that Abimelech's servants had taken with force.%%
  \verse{21:26} Abimelech said, ``I don't know who's done this~--- and even you haven't told me, neither have I heard {about it}\understood\ except today.''%%
  \verse{21:27} And Abraham took sheep and cattle and gave them to Abimelech, and they both made a covenant.%%
  \verse{21:28} Abraham put seven young ewe-lambs of the flock to one side.\lit{alone, beside.}%%
  \verse{21:29} Abimelech said to Abraham, ``What are these seven, young ewe-lambs which you've set aside?''%%
  \verse{21:30} And he said, ``That you take the seven, young ewe-lambs from me,\lit{my hand} in order that you witness that I've dug this well.''%%
  \verse{21:31} Therefore, he called that place Beersheba because they both swore there.%%
  \verse{21:32} They made a covenant in Beersheba, and Abimelech and Phichol the captain of his host got up and returned to the land of the Philistines.%%
  \verse{21:33} And he\ie{Abraham}\ca{prb ins c \sampen\septuagint\peshitta\vulgate\ \Hebrew{אַבְרָהָם}}{probably insert with the Samaritan Pentateuch, Septuagint, Peshitta, and Vulgate ``Abraham''} planted a tamarisk\halot{xxxx}{\textit{Tamarix syriaca}}\ed{Also known as a salt cedar, it's a tree commonly used as a shade tree or wind break.} in Beersheba and there called on the name of the \textsc{Lord}, the eternal God.%% xxxx add figure of tamarisk
  \verse{21:34} And Abraham sojourned in the land of the Philistines for many days.%%
\end{inparaenum}

  \heading{22}{Abraham commanded to sacrifice his only son, Isaac~--- Abraham and Isaac both submit to God's will~--- continuation of Abrahamic covenant~--- Rebekah born to Bethuel}

\begin{inparaenum}
  \verse{22:1} After these things, God tested Abraham. He said to him, ``Abraham,'' and Abraham replied, ``Yes?''%%
  \verse{22:2} He said, ``Please take your son Isaac~--- your only son~--- whom you love,\ed{This is poignantly repetitive.} and get you yonder\lit{go for yourself} to the land of Moriah. You shall raise him up as a burnt offering upon one of the mountains which I shall tell you.''%%
  \verse{22:3} So Abraham rose early in the morning and saddled his donkey. He took two of his young servants and his son Isaac with him. He chopped up some wood for the burnt offering, rose, and went to the place where God told him to go.%%
  \verse{22:4} On the third day, Abraham looked up\lit{lifted up his eyes} and saw the place afar off.\lit{from afar.; but this is written from his perspective}%%
  \verse{22:5} Abraham said to his young servants, ``Stay here with the donkey while the boy and I go off, worship, and return.''%%
  \verse{22:6} Abraham took the wood for the burnt offering and had his son, Isaac, carry it.\lit{placed it on his son Isaac} He took some fire, a knife, and the both of them went off together.%%
  \verse{22:7} Isaac said to Abraham, ``Dad?'' and he responded, ``Yes, my son?'' And Isaac said, ``I see the fire and the wood, but where is the lamb for the burnt offering?''%%
  \verse{22:8} Abraham replied, ``My son, God Himself will provide a lamb for the burnt offering,'' and they went on together.%%
  \verse{22:9} They came to the place which God had before told them and Abraham built an altar and arranged the wood. Then he bound Isaac and placed him on the wood of the altar.%%
  \verse{22:10} Abraham took the knife in his hand to slay his son.%%
  \verse{22:11} And the messenger of the \textsc{Lord} called to him from the heavens and said, ``Abraham! Abraham!'' and he said, ``Yes?''%%
  \verse{22:12} He said, ``Don't slay\lit{put forth your hand to} the boy, neither do anything to him because now I know that you fear God~--- you have not withheld your son~--- your only son~--- from me.''%%
  \verse{22:13} Abraham looked up and saw behind him a ram caught by its horns in an underbrush. So Abraham went and took the ram and offered it as a burnt offering in place of his son.%%
  \verse{22:14} Abraham named that place \textit{Jehovah-yireh}, and it is said there\understood\ today, ``The \textsc{Lord} provides on this\understood\ mountain.''%%
  \verse{22:15} The angel of the \textsc{Lord} called unto Abraham a second time from the heavens%%
  \verse{22:16} and said, ``I swear of myself, declares the \textsc{Lord}, that, because you have done this and not held back your son from me\understood~--- your only son~---%%
  \verse{22:17} I will richly bless you. I will greatly multiply your posterity as the stars of the heavens and as the sand of the seashore. Your descendants shall possess the gates of their enemies.%%
  \verse{22:18} Because you have hearkened to my words, through your posterity shall all the nations of the earth be blessed.''%%
  \verse{22:19} And Abraham returned to his young servants and they rose up and went together to Be'er Sheva; and Abraham lived in Be'er Sheva.%%
  
  \verse{22:20} After all this, word came to Abraham, saying, ``Milcah has birthed sons to your brother, Nahor.''%%
  \verse{22:21} Uz was his first-born, Buz his brother, Kemuel the father of Aram.%%
  \verse{22:22} Chesed, Hazo, Pildash, Jidlaph, and Bethuel.%%
  \verse{22:23} And Bethuel fathered Rebecca. Milcah bore these eight sons to Nahor, Abraham's brother.%%
  \verse{22:24} His concubine (whose name is Reumah), she also gave birth to Tebah, Gaham, Thahash, and Ma'acah.%%
\end{inparaenum}

  \heading{23}{Sarah dies and is buried in Kiryat Arba~--- Abraham buys the field of Ephron, including the cave of Machpelah}

\begin{inparaenum}
  \verse{23:1} Sarah's life was 127~years: the years of Sarah's life.%%
  \verse{23:2} Sarah died in Kiryat Arba, which is Hebron in the land of Canaan. And Abraham went in to mourn for Sarah, to weep for her.%%
  \verse{23:3} Abraham got up from the presence of his dead and spoke to the sons of Heth, saying,%%
  \verse{23:4} ``I am a stranger and a dweller with you. Give me property for a grave with you so that I can bury my dead before you.''%%
  \verse{23:5} The sons of Heth answered Abraham, saying to him,%%
  \verse{23:6} ``Hear us out, my lord: you are a chief\halot{xxxx}{\textbf{minor king}} of God among us: bury your dead in your choice of our burial grounds. None of us will hold back a burial site from you for you to bury your dead.''%%
  \verse{23:7} Abraham got up and bowed down before the people of the land, before the sons of Heth;%%
  \verse{23:8} and he spoke to them, saying, ``If\lit{If there is} your desire that I bury my dead from before me, hear me out and strongly urge Ephron son of Zohar for me%%
  \verse{23:9} that he gives me the cave of Machpelah which is his which is at the edge of his field. Let him give it to me, in your midst, for full money, as property for a grave.''%%
  \verse{23:10} Ephron lived in the midst of the children of Heth, and Ephron the Hittite answered Abraham in earshot\lit{the ears} of the children of Heth and all those coming into the gate of his city, saying,%%
  \verse{23:11} ``No, my lord: hear me. The field I've given you and the cave that's in it, I've given them\lit{it} to you~--- I've given it to you in the eyes of the children of my people: bury your dead.''%%
  \verse{23:12} Abraham bowed before the people of the land%%
  \verse{23:13} and spoke to Ephron in earshot of the people of the land, saying, ``If only you'd hear me!\ca{\sampen\septuagint\targum\super{J} \Hebrew{לִי}}{the Samaritan Pentateuch, Septuagint, and Targum have ``to me''}\ed{This passage literally reads: ``Only if you to them hear me.'' Idiomatically we're justified in just saying, ``If only you'd hear me,'' although literally the direction is off.} I've given money for the field: take it from me and I will bury my dead there.''%%
  \verse{23:14} Ephron answered Abraham, saying to him,%%
  \verse{23:15} ``My lord, hear me. A land worth\understood\ 400~silver shekels, what is it between us?\lit{between me and between you?} Bury your dead.''%%
  \verse{23:16} So Abraham listened to Ephron, and Abraham weighed out to Ephron the silver that he'd spoken of in earshot of the children of Heth~--- 400~silver shekels, at the current, merchant's rate.\halot{xxxx}{\textit{kesef `\=ob\=er} silver (at the) current (rate) \haref{Gn}{23}{16}}%%
  \verse{23:17} The field of Ephron in Machpelah which is before Mamre~--- the field and the cave that are in them, and all the trees of the field in its border surrounding it\ca{dl \Hebrew{׃}}{the \textit{sof pasuq} should be deleted}~--- passed over%%
  \verse{23:18} to Abraham by purchase in the eyes of the children of Heth and before everyone entering in at the gate of his city.%%
  \verse{23:19} After this, Abraham buried his wife, Sarah, in the cave of the field of Machpelah, \ca{mlt Mss \septuagint\targum\super{Ms}\targum\super{J} + \Hebrew{אֲשֶׁר}}{multiple Hebrew manuscript codices, the Septuagint, and two manuscripts of the Targum add ``which is''}in Mamre, in Hebron in the land of Canaan.%%
  \verse{23:20} The field and the cave in it were transferred to Abraham from the children of Heth as property for a grave.%%
\end{inparaenum}

  \heading{24}{Abraham commands his servant to find a wife for his son~--- the servant prays for help finding a wife for Isaac, and meets Rebecca~--- he recounts the story to Rebecca's brother Laban~--- through revelation, the servant finds Rebecca~--- he brings her back to Isaac and they're married}

\begin{inparaenum}
  \verse{24:1} Abraham grew old, come into days; and the \lord\ blessed Abraham in everything.%%
  \verse{24:2} Abraham said to his servant, the oldest in his house who governed his house and everything he had,\lit{on all which was to him} ``Please place your hand under my thigh\halot{xxxx}{area of sexual organs (hand placed there in oaths)}\ed{This is theologically incorrect: the Lord wouldn't require inappropriate touching to make oaths; in my opinion, this word was either changed throughout time or the original meaning was lost somewhere along the way.}%%
  \verse{24:3} and I will cause you to swear by the \lord, God of Heaven and God of the earth, that you won't take a wife for my son from the daughters of the Canaanites in whose midst I'm living:%%
  \verse{24:4} but you shall go to my land, to my family, and take a wife for my son, for Isaac.''%%
  \verse{24:5} The servant said to him, ``Maybe the woman won't want to follow me\lit{after me} to this land. Should I again return your son to the land that you came out of?''%%
  \verse{24:6} But Abraham said to him, ``Watch yourself lest you make my son return back here.%%
  \verse{24:7} The \lord, God of Heaven,\ca{\missing\ cod Sev; prb ins \Hebrew{הָאָרֶץ} \Hebrew{וֵאלֹהֵי} ut 3, it \septuagint}{missing in ancient, Sev(xxxx?) manuscripts; probably inserts ``the God of the earth'' like in verse~3, likewise in the Septuagint} who has taken me from the house of my fathers and the land of my birth;\ca{cod Sev \Hebrew{וּמֵאַרְצִי} \Hebrew{מִבֵּיתִי}}{ancient, Sev(xxxx?) manuscripts have ``from my house and from my land''} who's promised\alt{spoken to} me and sworn to me, saying, `I will give this land to your posterity.' He will send His messenger before you and you will take a wife from there for my son.%%
  \verse{24:8} If the woman doesn't want to follow after you, then\understood\ you will be free\halot{xxxx}{\textbf{be free} of, \textbf{exempt} from: be free of the obligation of an oath \haref{Gn}{24}{8}} from this oath of mine: just\alt{only} don't bring back my son.''%%
  \verse{24:9} The servant placed his hand under his master, Abraham's thigh, and swore to him regarding this matter.%%
  
  \verse{24:10} The servant took ten camels (from his lord's camels) and left.\ca{\missing\ \septuagint, prp dl}{[``and left'' is] missing in the Septuagint, it's been proposed to delete it} And all\ca{\peshitta(\vulgate) \textit{wmn kl}, prp \Hebrew{וּמִכָּל}}{the Peshitta (and Vulgate) have ``from all'', it's been proposed to be ``and from all''}\ed{This works best if both \textsc{ca} propositions are kept.} the goods of his lord in his hand, and he got up and went too Aram-Naharaim, to the city of Nahor.%%
  \verse{24:11} He made the camels kneel outside the city by well of water in the\lit{at} evening when the women go out to draw water.\halot{xxxx}{\textit{`\=et \d s\=e't \v s\=o'\super{e}b\^ot} = time when the women come forth to draw water \haref{Gn}{24}{11}}%%
  \verse{24:12} He said, ``O \lord, God of my master Abraham, please let something happen for me\halot{xxxx}{let something happen = \textbf{ordain}, \textbf{direct}: with \textit{l\super{e}f\=anay} for me \haref{Gn}{24}{12}} today; deal kindly with my master, Abraham.%%
  \verse{24:13} I'm standing by the well of water, and the daughters of the men of the city are going out to draw water.%%
  \verse{24:14} And the maid\ca{\sampen\ ut Q, K \Hebrew{הַנַּעַר}?}{the Samaritan Pentateuch is as read, but written as ``[masculine] servant''?} to whom I will say,\lit{the maid who I have spoken to her} `Please tilt\halot{xxxx}{\textbf{incline}, \textbf{bend down}: object pitcher, tilt \haref{Gn}{24}{12} [should be 14]} your pitcher\halot{xxxx}{\textbf{large} (pottery) \textbf{jar}, for water \haref{Gn}{24}{14}} and I will drink'; and she will say, `Drink, and I will also get water\alt{drink} for your camels.' Let it be her whom You shall determine for Your servant~--- for Isaac. I shall know because\lit{by} of this that You've dealt kindly with my master.''%%
  \verse{24:15} Before he'd finished speaking\ca{\sampen\septuagint\vulgate\ + \Hebrew{אֶל־לִבּוֺ} cf 45}{the Samaritan Pentateuch, Septuagint, and Vulgate add ``to him,'' (xxxx right?) compare verse~45} Rebecca (who was born to Bethuel, the son of Milcah, the wife of Nahor, Abraham's brother) came out with a pitcher on her shoulder.%%
  \verse{24:16} The maid had a really good appearance, and was a virgin (no man had had intercourse with her). She went down to the fountain, filled her pitcher, and went up.%%
  \verse{24:17} The servant ran to meet her, and said, ``Please let me sip a little water from your jug.''%%
  \verse{24:18} And she said, ``Drink, my lord,'' and she quickly let her pitcher down into\alt{onto} her hand, and gave him drink.%%
  \verse{24:19} When she finished giving him drink, she said, ``I will also draw water\understood\ for your camels until they've finished drinking.''%%
  \verse{24:20} She hurried and emptied\ca{\sampen\ \Hebrew{ותורד} cf 18}{the Samaritan Pentateuch has ``let down,'' compare verse~18} her pitcher into the watering-trough, and went down again to the well to draw, and drew for all his camels.%%
  \verse{24:21} The man stood gazing\ca{1 \Hebrew{משׁתעֶה}?}{1 has ``looked with favor at''?} at her, keeping quiet to know if the \lord\ had brought his way to a successful conclusion\halot{xxxx}{with accusative \textbf{make} something \textbf{succeed}, \textbf{bring} something \textbf{to successful conclusion} \haref{Gn}{24}{21}} or not.%%
  \verse{24:22} When the camels had finished drinking, the man took a gold nose-ring\halot{xxxx}{\textbf{ring} \haref{Ex}{35}{22}, nose-ring (of woman) \haref{Gn}{24}{22}, ear-ring (of woman) \haref{Gn}{35}{4}, (of man) \haref{Ex}{32}{2}}\ed{Why would he have a woman's gold nose ring on him?} (its weight was a half-shekel), and two bracelets for her hands (their weight was ten shekels\understood\ of gold),%%
  \verse{24:23} and said, ``Whose daughter are you? Please tell me. Is there place in your father's house for us to spend the night?''%%
  \verse{24:24} She said to him, ``I am a daughter of Bethuel, son of Milcah, whom she bore to Nahor.''%%
  \verse{24:25} She also said to him, ``There's tons of straw\halot{xxxx}{chopped stalks, \textbf{straw}, as fodder \haref{Gn}{24}{25}} and fodder\halot{xxxx}{\textbf{fodder}: for camels \haref{Gn}{24}{25$\cdot$32}} as well\lit{also} with us, as well as room to spend the night.''%%
  \verse{24:26} The man bowed down\halot{xxxx}{\textbf{bow down}, \textbf{kneel down} (always followed by \textit{hi\v sta\d h\super{a}w\^a}) \haref{Gn}{24}{26}} and worshiped the \lord.%%
  \verse{24:27} He said, ``Let the \lord, God of my master Abraham, be blessed, who hasn't abandoned His loving-kindness and His reliability\halot{xxxx}{\textit{\d hesed we'\super{e}met} \textbf{lasting kindness}}\ed{I chose to render this as ``reliability'' to help show that God is the same yesterday, today, and forever.} from my master. I, in the way, the \lord\ has led me to\understood\ my master's brothers' house.''%%
  
  \verse{24:28} The maid ran and told her mother's household these things.%%
  \verse{24:29} Rebecca had a brother, named Laban; and Laban ran outside to the man at\lit{to} the fountain.\ca{frt tr post 30a}{perhaps transposed to after verse~30 at mark~a}%%
  \verse{24:30} When he saw\ed{verse~30, mark~a} the nose-ring and the bracelets on his sister's hands, when he heard his sister Rebecca's words, saying, ``So has the man said to me.'' He came to the man and stood by the camels by the fountain.%%
  \verse{24:31} He said, ``Come in, you\understood\ blessed of the \lord; why are you standing outside? I've cleared up\ed{Cleaned up?} the house and place for the camels.''%%
  \verse{24:32} He brought the man into the house and unsaddled the camels, and gave the camels straw and provender, and water to wash his feet and the feet of the men who were with him;%%
  \verse{24:33} there was set before him\ed{understood: food} to eat, and he said, ``I won't eat until I've said my words.'' And he said, ``Speak.''%%
  \verse{24:34} He said, ``I am Abraham's servant.%%
  \verse{24:35} The \lord\ has greatly blessed my master, and he has become great. He has given him sheep and cattle, silver and gold, servants and maids, camels and donkeys.%%
  \verse{24:36} Sarah, my master's wife, bore a son to my master after she'd gotten old. He has given him everything that he has.%%
  \verse{24:37} My master's made me make an oath, saying, `Don't take a wife for my son from the daughters of the Canaanites in whose land I'm living;%%
  \verse{24:38} but rather you shall go among my father's house, among my family, and take a wife for my son.'%%
  \verse{24:39} So I said to my master, `Maybe the woman I take won't follow me.'%%
  \verse{24:40} And he said to me, `The \lord, whom I continually walk before, will send His messenger\alt{angel} with you, and he will make your way prosper. And you will take a wife for my son from my family, from my father's house.%%
  \verse{24:41} Then you shall be free from the obligation of my oath when you've come to my family. And if they don't give\ed{understood: a wife} to you, then you shall be free from the obligation of my oath.'%%
  \verse{24:42} So I came to the well today, and said, `O \lord, God of my master Abraham, if\ed{probably: please} You make my way successful,\halot{xxxx}{you give success \haref{Gn}{24}{42}} in the way I'm going.'%%
  \verse{24:43} I was standing by the fountain of water, and the young woman came up to draw, and I said to her, `Please give me a little water to drink from your pitcher.'%%
  \verse{24:44} She said to me, `Both you drink and I will also draw water for your camels.'\lit{Both you drink and also for your camels I will draw.} She is the woman whom the \lord\ has chosen for my master's son.%%
  \verse{24:45} Before I finish speaking in my heart, Rebecca went out, her pitcher on her shoulder, and she went down to the fountain and drew;\ed{understood: water} I said to her, `Please let me drink.'%%
  \verse{24:46} She hurried and let her pitcher down from herself, and said, `Drink. I will also give your camels drink.' So I drank, and she also gave the camels to drink.%%
  \verse{24:47} I asked her, and said, `Whose daughter are you?' And she said, `I am a\understood\ daughter of Bethuel, son of Nahor, who Milcah gave birth to.' So I put the nose-ring on her nose, and the bracelets on her hands.%%
  \verse{24:48} I bowed and showed reverence to the \lord, and I blessed the \lord, the God of my master Abraham, who's led me in the way of truth to take my master's niece\lit{brother's daughter} for his son.%%
  \verse{24:49} And now, if you will deal kindly and truthfully with my master~--- tell me. But if not, tell me, and I will turn to the right or the left.''%%
  
  \verse{24:50} Laban and Bethuel answered, and they said, ``The thing has gone out from the \lord. We can't speak good or bad to you.%%
  \verse{24:51} Rebecca is before you: take her\understood\ and go: she will be a wife to your master's son, according as the \lord\ has spoken.''%%
  \verse{24:52} When Abraham's servant heard their words, he bowed himself to the earth before the \lord.%%
  \verse{24:53} The servant brought silver vessels and gold vessels and clothes, and gave them\understood\ to Rebecca. And he gave precious things to her brother and her mother.%%
  \verse{24:54} So he and the men who were with him ate and drank and lodged. They got up in the morning, and he said, ``Send me to my master.''%%
  \verse{24:55} Her brother, and her mother as well, said, ``Let the maid dwell for days, or ten, and afterwards she'll go.''%%
  \verse{24:56} But he said to them, ``Don't delay me seeing as the \lord\ has prospered my way. Send me off and I'll go to my master.''%%
  \verse{24:57} They said, ``Let us call the maid and ask at her mouth.''%%
  \verse{24:58} They called for Rebecca, and said to her, ``Will you go with this man?'' And she replied, ``I will go.''%%
  \verse{24:59} They sent their sister Rebecca, her nurse, and Abraham's servant and his men away,%%
  \verse{24:60} They blessed Rebecca, and said to her,\smallskip%%
  
  \pd ``You are our sister;\pa to thousands may you multiply!%%
  
  \pd May your posterity possess\alt{inherit}\pa the gate of their enemies!''%%
  
  \verse{24:61} Rebecca and her maid got up and rode on camels, and they went after the man; and the servant took Rebecca and went.%%
  \verse{24:62} Isaac returned from the entrance of the\understood\ well of the living \Hebrew{רֹאִי},%
  \halot{xxxx}{\textit{la\d hay r\=o'\^i}\dots\ unexplained}
  \halotu{xxxx}{meaning uncertain\dots\ either ---a. God of seeing, of perception; or ---b. God who sees me (\Hebrew{רֳאִי} = \Hebrew{רֹאִי}\dots); ---c. \dots El of seeing, being seen\dots\ ``El of hearing''\dots\ it is most likely to mean either ---a. well of the living vision; or ---b. well of the living one (the one who is alive) who sees me; Westermann agrees that the second meaning (b) is probably preferable since it ``comes closest to v.~$_{13}$''}
  he\ed{``He''?} lives in the southern land.%%
  \verse{24:63} Isaac went out to \Hebrew{שׂוח}%
  \halot{xxxx}{unexplained; translations are only guesswork \haref{Gn}{24}{63}}
  \halotu{xxxx}{hapax legomenon, \haref{Gn}{24}{63}; Samaritan Pentateuch \textit{al'\v s\=u}: ---a. the interpretation is uncertain and the versions differ: Septuagint in order to gossip (\Greek{ἀδολεσχῆσαι}), Vulgate in order to meditate (\textit{ad meditandum}), Peshitta so as to result (\textit{lamhull\= aku}), Targum in order to pray (\textit{l\schwa \d sall\=a'\=ah}), Samaritan Targum \textit{lm\d sl'h}; modern interpretations vacillate accordingly\dots\ ---b. \Hebrew{בַּשָׂדֶה} \Hebrew{לָשׂוּחַ} [to \Hebrew{שׂוח} in the field] verse~63 is paraphrased in verse~65 with \Hebrew{בַּשָׂדֶה} \Hebrew{הַהֹלֵךְ} [to walk in the field], which accordinly suggests as the most probably meaning for \Hebrew{שׂוח} to walk, stroll, wander about, thus e.g. Westermann loco citato [in the place cited], compare also Arabic \textit{s\=a\d ha (sw\d h)} to travel, rove, roam about}
  \ed{I don't totally agree with \textsc{halotu}'s thinking that \Hebrew{שׂוח} means ``to walk about'' because although \vref{Gen}{24}{65} does contain the parallel, it's not absolutely parallel: Isaac went outside to do something, and Rebecca (the first time she's ever seen him) sees him walking about in the field. That's not enough information to definitively state what Isaac's intentions were in the field. She merely sees him walking out to meet them, but he could easily have gone into the field to meditate or to pray, there's simply not enough information to tell.}
  in the field at dusk,\lit{the beginning of evening} and he looked up and saw camels coming.%%
  \verse{24:64} And Rebecca looked up and saw Isaac, and she got down from%
  \halot{xxxx}{with \textit{m\=e`al} \textbf{get down} (quickly, attentively) from \haref{Gn}{24}{64}}\alt{fell off}
  the camel,%%
  \verse{24:65} and said to the servant, ``Who is this man who's walking in the field to meet us?'' The servant said, ``He is my master.'' So she took her veil and covered herself.%%
  \verse{24:66} The servant reported\alt{related, told} to Isaac everything that he'd done.%%
  \verse{24:67} Isaac brought her into his mother, Sarah's\ca{prb add (cf \Hebrew{האהלה})}{[``his mother, Sarah's''] probably added (compare ``her tent'')} tent; and he took Rebecca, and she became his wife. And Isaac was comforted after his mother.\ed{probably: ``after his mother's death.''}%%
\end{inparaenum}

  \heading{25}{Abraham dies and is buried~--- the generations and history of Abraham and Isaac are given~--- Jacob and Esau are born~--- Esau sells his birthright to Jacob for a bowl of lentil soup}

\begin{inparaenum}
  \verse{25:1} Abraham took another\lit{added} wife named Keturah.%%
  \verse{25:2} She gave birth to Zimran, Jokshan, Medan, Midian, Ishbak, and Shuah.%%
  \verse{25:3} Jokshan begat Sheba and Dedan. Dedan's\ca{\missing\ \caref{1Ch}{1}{33}, frt add}{[here to the end of the verse] missing in \vref{1Ch}{1}{33}, possibly added} sons were Asshurim, Letushim, and Leummim.%%
  \verse{25:4} Midian's sons were Ephah, Epher, Enoch, Abidah, and Eldaah. All of these are sons of Keturah.%%
  \verse{25:5} Abraham gave everything he had to Isaac.\ca{\sampen\septuagint\peshitta\ + \Hebrew{בְּנוֺ}, frt ins}{the Samaritan Pentateuch, Septuagint, and Peshitta add ``his son,'' [it's] possibly inserted}%%
  \verse{25:6} To the sons of the concubines whom Abraham had, Abraham gave gifts and, while he was still alive, he sent them to the east country, away from his son Isaac.%%
  \verse{25:7} These are the days of the years of Abraham's life which he lived: 175~years.%%
  \verse{25:8} Abraham expired\halot{xxxx}{1. \textbf{expire}, breathe one's last \haref{Gn}{25}{8$\cdot$17}; --- 2. \textbf{die} \haref{Gn}{6}{17}\haref{}{7}{21}} and died in a good, old age: old and satisfied.\ca{1 c pc Mss \sampen\septuagint\peshitta\targum\super{Ms} \Hebrew{וּשְׂבַע יָמִים}}{1 with a few Hebrew manuscript codices, the Samaritan Pentateuch, Septuagint, Peshitta, and Targum have ``full of days''} And he was gathered to his people.%%
  \verse{25:9} His sons Isaac and Ishmael buried him in the cave of Machpelah in the field of Ephron son of Zohar the Hittite, by Mamre~---%%
  \verse{25:10} the field Abraham bought from Heth's sons: it was\understood\ there that\understood\ Abraham buried his wife Sarah.%%
  \verse{25:11} After Abraham's death, God blessed Isaac, his son. And Isaac lived by the well of the living one \Hebrew{רֹאִי}.\ed{See the footnotes attached to \vref{Gen}{24}{62}.}%%
  
  \verse{25:12} These are the generations of Ishmael, Abraham's son who was born to Hagar the Egyptian, Sarah's maid:%%
  \verse{25:13} These are the names of Ishmael's sons, according to their names: Nebaioth (Ishmael's firstborn), Kedar, Adbeel, Mibsam,%%
  \verse{25:14} Mishma, Dumah, Massa,%%
  \verse{25:15} Hadar, Tema, Jetur, Naphish, and Kedemah.%%
  \verse{25:16} These are the sons of Ishmael, these are their names, by their settlements\halot{xxxx}{permanent \textbf{settlement} without wall, \textbf{farm} (premises) \haref{Gn}{25}{16}} and by their camps:\halot{xxxx}{\textbf{(tent-)camp} protected by stone wall \haref{Gn}{25}{16}} twelve princes of their peoples.%%
  \verse{25:17} These are the years of Ishmael's life: 137~years. And he expired and died, and was gathered to his people.%%
  \verse{25:18} They\ca{\septuagint\vulgate\ sg}{the Septuagint and Vulgate are singular [which agrees with the rest of the verse]} lived from Havilah to Shur (which is before Egypt, as you\ed{or whatever pronoun you prefer; I dislike ``one.''} go to Assyria),\ca{add?}{[``as you go to Assyria''] added?}\ca{= \Hebrew{שׁוּרָה} cf\caref{}{16}{7}\caref{}{20}{1}}{= ``Shur'' compare \vref{Gen}{16}{7},\vref{}{20}{1}}\ed{Shur makes more sense than Assyria because Shur is on the way to Egypt (it's the desert between Kadesh-Barne'a and Egypt); Assyria is in the opposite direction.} he settled opposite all of his brothers.%% xxxx this would be a great place for a figure (map of the area). Or at least a reference to one in the appendix.
  
  \verse{25:19} These are the generations of Abraham's son Isaac: Abraham fathered Isaac,%%
  \verse{25:20} and Isaac was forty\lit{had forty years} when he took Rebecca~--- the daughter of Bethuel the Aram\ae an (from Paddan Aram), Laban the Aram\ae an's sister~--- to wife.%%
  \verse{25:21} Isaac pleaded with\alt{prayed to} the \lord\ for his wife because she was barren; and the \lord\ was moved by his entreaties, and Rebecca, his wife, conceived.%%
  \verse{25:22} The children pushed each other around inside her, and she said, ``If this is so, why am I like this?'' and she went to inquire of the \lord.%%
  \verse{25:23} The \lord\ said to her,\smallskip%%
  
  \pd ``Two nations are in your womb~---\pa two people shall be cut off\ie{from each other} from your bowels.%%
  
  \pd One people will be stronger than the other people.\pa The older shall serve the younger.''%%
  
  \verse{25:24} When her days to bear were completed, there were twins in her womb.%%
  \verse{25:25} The first came out reddish all over like a fur garment; and they named him Esau.%%
  \verse{25:26} Afterwards, his brother came out, his hand seizing Esau's heel. He was named Jacob. And Isaac was sixty~years old when they were born.%%
  \verse{25:27} The boys grew, and Esau was a man who knew hunting~--- a man of the field; and Jacob was a quiet\alt{peaceful} man, living in tents.%%
  \verse{25:28} Isaac loved Esau because hunting\ca{\sampen\septuagint\ \Hebrew{צֵידוֺ}}{his hunting} was to his mouth; but Rebecca loved Jacob.%%
  \verse{25:29} Jacob boiled a boiled dish;\halot{xxxx}{\textbf{a boiled dish} (of food)} and Esau came in from the field, and he was exhausted.%%
  \verse{25:30} Esau said to Jacob, ``Please let me eat from the red, this red thing, because I'm exhausted.'' Therefore he was named Edom.%%
  \verse{25:31} Jacob said, ``Sell me your birthright today.''%%
  \verse{25:32} Esau said, ``I'm going to die, what is this birthright to me?''%%
  \verse{25:33} Jacob said, ``Swear to me today.'' And he swore to him and sold his birthright to Jacob.''%%
  \verse{25:34} Jacob gave Esau bread and a boiled, lentils dish, and he ate and drank and got up and left. And Esau despised the birthright.\ed{Did Esau even care about his birthright? It doesn't seem like it's something he cared about. I mean, I've been hungry before; but hungry enough to sell something super important like that? It just says that ``Esau despised the birthright.'' I don't think he even cared about it so in his mind he was getting rid of something of no value for food. I just feel like I've been taught incorrectly my whole life that Esau valued the birthright and that we need to be careful not to give up something of great value for instant gratification. I think the more correct lesson is that we need to recognize the value of the things that we have because that seems to be what the context is saying.}%%
\end{inparaenum}

  \heading{26}{Isaac and his family go down to live in Gerar per the Lord's instructions~--- Isaac says that his wife is his sister; Abimelech calls him out on it~--- digging of wells and related disputations}

\begin{inparaenum}
    \verse{26:1} There was a famine in the land, besides the previous famine that had happened\lit{been} in Abraham's days. So Isaac went to Gerar, to Abimelech king of the Philistines.\lit{to Abimelech king of the Philistines to Gerar.}%%
    \verse{26:2} But the \textsc{Lord} appeared to him and said, ``Don't go down to Egypt; stay in the land that I will tell you.%%
    \verse{26:3} Stay\halot{\textbf{stay as foreigner and sojourner}} in this land. I will be with you. I will bless you, because I will give all these lands to you and your posterity~--- I will carry out the oath\alt{swearing} that I swore to your father Abraham.%%
    \verse{26:4} I will multiply your posterity as the stars in the sky.\alt{heavens} I will give all these lands to your posterity. All the nations of the earth shall wish a blessing to themselves [because of] your posterity.\ed{This is a bit confusing and \textsc{halot} isn't too helpful. It gives this \textit{hithpael} form as meaning ``\textbf{wish a blessing to oneself} (to one another?), with \textit{b}\superit{e} \haref{Gn}{22}{18} (Abraham)\haref{}{26}{4} (Isaac)'' which is where I derived the present translation from.}%%
    \verse{26:5} Because that Abraham\ca{\sampen\septuagint\ + \Hebrew{אָבִיךָ}}{the Samaritan Pentateuch and Septuagint add ``your father''} heard My voice and obeyed My statutes and My commandments and My laws.''%%
    \verse{26:6} Isaac lived in Gerar.%%
    \verse{26:7} The men of the place asked him about his wife, and he said, ``She is my sister'' because he was afraid to say ``My wife,\ca{ins c \sampen\septuagint\ \Hebrew{הִיא}}{insert with the Samaritan Pentateuch and Septuagint ``she [is]''} lest the men of the place kill me for Rebecca because she's beautiful.''\lit{of good appearance.''}%%
    \verse{26:8} When he'd been there a while,\lit{When the days became long to him there} Abimelech king of the Philistines looked down through his window and saw Isaac was fondling\halot{with \textit{'}\superit{\=e}\textit{t} \textbf{fondle} (a woman) \haref{Gn}{26}{8}} his wife Rebecca.%%
    \verse{26:9} Abimelech called to Isaac, and said, ``She's definitely\lit{surely} your wife! Why\lit{How} did you say, `She's my sister'?'' And Isaac said to him, ``Because I said, `Lest I die for her.'\thinspace''%%
    \verse{26:10} Abimelech said, ``What is this thing you've done to us? One of the people, like nothing,\lit{like a little thing} could have laid with your wife, and you would have brought guilt on us.''%%
    \verse{26:11} So Abimelech commanded all the people,\ca{\sampen\septuagint\ \Hebrew{עַמּוֺ}}{the Samaritan Pentateuch and Septuagint have ``his people''} saying, ``Whoever touches this man or his wife shall certainly be killed.''\lit{shall die the death.''}%%
    \verse{26:12} Isaac sowed in that land and found a hundredfold in that field; the \textsc{Lord} blessed him.%%
    \verse{26:13} The man became great, becoming continually greater, until he had become very great.%%
    \verse{26:14} He had possession of a flock and possession of a herd and many servants. The Philistines envied him.%%
    \verse{26:15} All the wells that his father's servants had dug during his father Abraham's days; the Philistines stopped them and filled them with dust.%%
    \verse{26:16} Abimelech said to Isaac, ``Leave us\lit{Go from us} because you're much mightier than we are.''%%
    \verse{26:17} So Isaac went from that place and encamped in Garar and stayed there.%%
    \verse{26:18} Isaac returned and dug the water wells that they'd\ca{prb l c \sampen\septuagint\vulgate\ \Hebrew{עַבְדֵי} cf 15.19.25 et \peshitta\targum\super{J}}{probably read with the Samaritan Pentateuch, Septuagint, and Vulgate which have ``servants'' [in construct, so ``his father Abraham's servants'']; compare verses~15, 19, and~25 and the Peshitta and Targum} dug in the days of his father, Abraham, which the Philistines had stopped after Abraham's death. He named them\lit{called their names} after the names his father had called them.%%
    \verse{26:19} Isaac's servants dug in the valley and they there found a well of running water.%%
    \verse{26:20} The shepherds of Gerar conducted a lawsuit\halot{\textbf{dispute}, \textbf{quarrel} (in public, w. words, complaints, assertions, reproaches); \textbf{conduct a} (legal) \textbf{case}, \textbf{lawsuit}: absolute (state) \haref{Gn}{26}{21}} with Isaac's shepherds, saying, ``The water is ours!''\lit{To us the water.} And he called the name of the well ``Quarrel'' because they'd quarreled with him.%%
    \verse{26:21} They dug another well and they also quarreled for it.\ed{Possibly ``[T]hey also disputed over it.''} He called the place Sitnah.\lit{Hatred.}\halot{accusation}%%
    \verse{26:22} He moved on from there and dug another well which they didn't quarrel over.\lit{and they didn't dispute about it.} He called the place Rehoboth,\lit{Wide open} and said, ``For now the \textsc{Lord} has made room\alt{opened wide} for us and we shall be fruitful in this land.''%%
    \verse{26:23} He went from there to Beersheba.%%
    \verse{26:24} The \textsc{Lord} appeared to him in that night, and said, ``I am the God of your father, Abraham. Don't be afraid because I am with you and have blessed you. I will multiply your posterity on account of my servant Abraham.''%%
    \verse{26:25} He built an altar there and called on the name of the \textsc{Lord}. He pitched his tent there and Isaac's servant dug a well there.%%
    \verse{26:26} Abimelech went to him from Gerar along with\lit{and} his friend Ahuzzath and the captain of his army, Phichol.%%
    \verse{26:27} Isaac said to them, ``Why have you come to me, seeing as you hate me and have driven me away from you?''%%
    \verse{26:28} They said, ``We've really seen that the \textsc{Lord} is with you; so we say, `Please let there be a covenant between us\ca{\missing\ Vrs, dl?}{[``between us'' is] missing in multiple versions, deleted?} and you\lit{between you} so let us make a covenant with you:%% Keep "\lit{between you}" because it matters in the context of the critical apparatus note.
    \verse{26:29} don't do evil with us because we haven't hurt you;\lit{touched you}\alt{done you harm} and because we have only done good with you and sent you away in peace. You are now blessed of the \textsc{Lord}.''%%
    \verse{26:30} He made a feast for them, and they ate and drank.%%
    \verse{26:31} And they got up in the morning\ed{\Hebrew{שׁכם} can have the connotation to be early, although that's not given with every definition.} and swore to each other,\alt{one to another}\ca{\sampen\septuagint\ \Hebrew{לְרֵעֵהוּ}}{the Samaritan Pentateuch and Septuagint have ``to his neighbor''} and Isaac sent them and they went away from him in peace.%%
    \verse{26:32} In that day, Isaac's servants came and told him about\halot{on account of} the well that had been dug, and they said to him, ``We have found water.''%%
    \verse{26:33} He called it Sheba, thus the name of the city is Beersheba to this day.%%
    
    \verse{26:34} Esau was 40~years old when took to wife Judith, the daughter of Beeri the Hittite, and Basmath, the daughter of Elon the Hittite.\ca{frt l c \sampen\septuagint*\peshitta\ \Hebrew{הַחִוִּי}}{perhaps read with Samaritan Pentateuch, Septuagint, and Peshitta ``Hivite''}%%
    \verse{26:35} They were a bitterness\alt{affliction, grief} to Isaac and Rebecca's spirit.%%
\end{inparaenum}

  \heading{27}{Isaac is on his deathbed and requests a favor of Esau~--- xxxx}

\begin{inparaenum}
  \verse{27:1} When Isaac had aged and his eyes had become expressionless so he couldn't see, he called his elder son, Esau, and said to him, ``My son.'' And he said to him, ``I'm here.''%%
  \verse{27:2} He said, ``Look, I am old, and I don't know the day of my death.\ie{When I will die.}%%
  \verse{27:3} Now, please take your weapons\ed{Although \textsc{halot} states that \Hebrew{כְּלִי} is, in its most basic sense any ``useful object,'' but can be anything ranging from \textbf{vessel, receptable, gear} to \textbf{equipment} to \textbf{implement} or \textbf{ornaments} or \textbf{clothes}, the context most clearly means \textbf{weapons}.}~--- your quiver and your bow~--- and go out to the field and hunt game for me.%%
  \verse{27:4} Make\alt{Prepare} a delicacy\halot{מַטְעָם}{\textbf{delicacy}, \textbf{tidbit} \haref{Gn}{27}{4ff}} for me in the way that\halot{כַּאֲשֶׁר}{\textbf{as} (in the sense of `in the way that')} I like; and bring it to me and I will eat it that\halot{עֲבוּר}{\textbf{(in order) that}} my soul can bless you before I die.''%%
  \verse{27:5} And Rebecca heard when Isaac spoke to his son Esau. And Esau went out to the field to hunt game, to bring it.\ed{The ``it'' is inferred. And there's probably something missing here because this last verb is just dangling here incomplete.}%%
  \verse{27:6} Rebecca spoke to her son Jacob, saying, ``Hey, I heard your father speak to your brother, Esau, saying,%%
  \verse{27:7} `Bring me game and prepare for\understood\ me a delicacy and I will eat it: and I will bless you before the \textsc{Lord} before I die.'%%
  \verse{27:8} Now, my son, listen to my words\lit{voice} in that which I shall command you.%%
  \verse{27:9} Please, go to the flock and take\lit{for me} from there two good kids.\ed{It's literally ``goat kids'' or ``kids of the goats,'' but that's redundant in English.} I will make a delicacy from them for your father, like he likes.%%
  \verse{27:10} %%
  \verse{27:11} %%
  \verse{27:12} %%
  \verse{27:13} %%
  \verse{27:14} %%
  \verse{27:15} %%
  \verse{27:16} %%
  \verse{27:17} %%
  \verse{27:18} %%
  \verse{27:19} %%
  \verse{27:20} %%
  \verse{27:21} %%
  \verse{27:22} %%
  \verse{27:23} %%
  \verse{27:24} %%
  \verse{27:25} %%
  \verse{27:26} %%
  \verse{27:27} %%
  
  \verse{27:28} %%
  
  \verse{27:29} %%
  
  \verse{27:30} %%
  \verse{27:31} %%
  \verse{27:32} %%
  \verse{27:33} %%
  \verse{27:34} %%
  \verse{27:35} %%
  \verse{27:36} %%
  \verse{27:37} %%
  \verse{27:38} %%
  \verse{27:39} %%
  
  \verse{27:40} %%
  
  \verse{27:41} %%
  \verse{27:42} %%
  \verse{27:43} %%
  \verse{27:44} %%
  \verse{27:45} %%
  \verse{27:46} %%
\end{inparaenum}

  % \heading{31}{xxxx}

\begin{inparaenum}
  \verse{31:1} %%
  \verse{31:2} %%
  \verse{31:3} %%
  \verse{31:4} %%
  \verse{31:5} %%
  \verse{31:6} %%
  \verse{31:7} %%
  \verse{31:8} %%
  \verse{31:9} %%
  \verse{31:10} %%
  \verse{31:11} %%
  \verse{31:12} %%
  \verse{31:13} %%
  \verse{31:14} %%
  \verse{31:15} %%
  \verse{31:16} %%
  \verse{31:17} %%
  \verse{31:18} %%
  \verse{31:19} %%
  \verse{31:20} %%
  \verse{31:21} %%
  \verse{31:22} %%
  \verse{31:23} %%
  \verse{31:24} %%
  \verse{31:25} %%
  \verse{31:26} %%
  \verse{31:27} %%
  \verse{31:28} %%
  \verse{31:29} %%
  \verse{31:30} %%
  \verse{31:31} %%
  \verse{31:32} %%
  \verse{31:33} %%
  \verse{31:34} %%
  \verse{31:35} %%
  \verse{31:36} %%
  \verse{31:37} %%
  \verse{31:38} %%
  \verse{31:39} %%
  \verse{31:40} %%
  \verse{31:41} %%
  \verse{31:42} %%
  \verse{31:43} %%
  \verse{31:44} %%
  \verse{31:45} %%
  \verse{31:46} %%
  \verse{31:47} \dots Jegar-sahadutha\ed{Aramaic (\Hebrew{שָׂהֲדוּתָא יְגַר}): keep of storms of witnessing}\dots%%
  \verse{31:48} %%
  \verse{31:49} %%
  \verse{31:50} %%
  \verse{31:51} %%
  \verse{31:52} %%
  \verse{31:53} %%
  \verse{31:54} %%
  \verse{31:55} %%
\end{inparaenum}

  % \heading{37}{xxxx}

\begin{inparaenum}
  \verse{37:1} %%
  \verse{37:2} %%
  \verse{37:3} %%
  \verse{37:4} %%
  \verse{37:5} %%
  \verse{37:6} %%
  \verse{37:7} %%
  \verse{37:8} %%
  \verse{37:9} %%
  \verse{37:10} %%
  \verse{37:11} %%
  \verse{37:12} %%
  \verse{37:13} %%
  \verse{37:14} %%
  \verse{37:15} The word of the \lord\ came to me, saying,%%
  \verse{37:16} ``Son of man, take one stick and write \textit{For Judah, the children of Israel, and his companions} on it. Then take another stick and write \textit{For Joseph, the branch of Ephraim, and his companions, the whole house of Israel} on it.%%
  \verse{37:17} Bring them together as one stick because they shall be one in your hand.%%
  \verse{37:18} %% end double quote
  \verse{37:19} %%
  \verse{37:20} %%
  \verse{37:21} %%
  \verse{37:22} %%
  \verse{37:23} %%
  \verse{37:24} %%
  \verse{37:25} %%
  \verse{37:26} %%
  \verse{37:27} %%
  \verse{37:28} %%
\end{inparaenum}

  % The coat of many colors: \Hebrew{כְּתֹבֶת פּסִּים} \Greek{χιτῶνα ποικίλον} is rendered as ``coat of many colors.'' Rabbis tell us that this is more probably a long sleeve garment with markings. Additionally, this is a \textit{hapax legomenon}.
  \heading{39}{xxxx}

\begin{inparaenum}
  \verse{39:1} %%
  \verse{39:2} %%
  \verse{39:3} %%
  \verse{39:4} %%
  \verse{39:5} %%
  \verse{39:6} %%
  \verse{39:7} %%
  \verse{39:8} %%
  \verse{39:9} There is no one in this house greater than me. He hasn't held anything back\alt{withheld} from me, except for\understood\ you because you're his wife! So how can I do this great wickedness and sin against God?''%%
  \verse{39:10} %%
  \verse{39:11} %%
  \verse{39:12} %%
  \verse{39:13} %%
  \verse{39:14} %%
  \verse{39:15} %%
  \verse{39:16} %%
  \verse{39:17} %%
  \verse{39:18} %%
  \verse{39:19} %%
  \verse{39:20} %%
  \verse{39:21} %%
  \verse{39:22} %%
  \verse{39:23} %%
\end{inparaenum}

  % Genesis 48:7 - Ephrath = Bethlehem of Judaea
  % \heading{49}{XXXX}

\begin{inparaenum}
  \verse{49:1}{``}%%
  \verse{49:2}{}%%
  \verse{49:3}{}%%
  \verse{49:4}{}%%
  \verse{49:5}{}%%
  \verse{49:6}{}%%
  \verse{49:7}{}%%
  \verse{49:8}{~~To Judah~--- your brothers will praise you. Your hand will be on your enemies' necks.%%
  
  ~Your siblings}%%
  \verse{49:9}{}%%
  \verse{49:10}{}%%
  \verse{49:11}{}%%
  \verse{49:12}{}%%
  \verse{49:13}{}%%
  \verse{49:14}{}%%
  \verse{49:15}{}%%
  \verse{49:16}{}%%
  \verse{49:17}{}%%
  \verse{49:18}{}%%
  \verse{49:19}{}%%
  \verse{49:20}{}%%
  \verse{49:21}{}%%
  \verse{49:22}{}%%
  \verse{49:23}{}%%
  \verse{49:24}{}%%
  \verse{49:25}{}%%
  \verse{49:26}{}%%
  \verse{49:27}{}%%
  \verse{49:28}{}%%
  \verse{49:29}{}%%
  \verse{49:30}{}%%
  \verse{49:31}{}%%
  \verse{49:32}{}%%
  \verse{49:33}{}%%
\end{inparaenum}

  
  \book{Exodus}{\Hebrew{שמות}}
  \heading{15}{Moses and the people sing a song praises the Lord for their deliverance from Egypt~--- they come to Marah, but the waters are bitter~--- Moses makes an executive decision and brings them to Elim}

\begin{inparaenum}
  \verse{15:1} Moses and the children of Israel then sang this song to the \lord, and they spoke, saying,\smallskip%%
  
  \pb ``I will sing to the \lord\ pa because he is greatly exalted.\pa The horse and chariot\pa he's thrown into the sea.%%
  
  \pa \verse{15:2} My strength and song is \textsc{Jah}%
  \ca{\missing\ \septuagint}{[``\textsc{Jah}''] is missing in the Septuagint}
  ~---\pa He has become my%
  \ca{\peshitta\ \textit{ln}}{the Peshitta has ``our''}
  salvation.%%
  
  \pb This is my God: I will praise Him~---\pa my father's God: I will exalt\alt{praise, extol} Him.%%
  
  \pa \verse{15:3} The \lord\ is a man%
  \ca{\sampen\peshitta\ \Hebrew{גִּבּוֺר}, \septuagint\ \Greek{συντρίβων}}{the Samaritan Pentateuch and Peshitta have ``hero,'' the Septuagint has ``crusher''}%%
  of war.\pa His name is \textsc{Jehovah}.%%
  
  \pa \verse{15:4} Pharaoh's chariots and force\pa he's cast into the sea.\pa His chosen captains\pa he's sunk in the Red Sea.%%
  
  \pa \verse{15:5} The depths cover him;\pa they went into the depths like a stone.%%
  
  \pa \verse{15:6} Your right hand, O \lord,\pa has become glorious in power:\pa Your right hand, O \lord,\pa has destroyed the enemy.%%
  
  \pa \verse{15:7} You've demolished your opponents\lit{one who stands against you} in the abundance of Your excellency;\pa You've sent Your anger\pa and it consumes them like chaff.%%
  
  \pa \verse{15:8} By the breath of Your nostrils\pa You've dammed up\alt{piled up, gathered together} the water~---\pa the flowing waves stand like a dam,\alt{dike, wall}%%
  
  \pb the depths in the heart of the sea have congealed.\alt{thickened.}%%
  
  \pb \verse{15:9} The enemy said,\pa `I pursue. I overtake.\alt{catch up.}\pa I allot plunder.\pa My soul is overwhelmed by them.%%
  
  \pb I draw my sword,\pa my hand destroys them.'%%
  
  \pa \verse{15:10} You have blown\alt{blasted} with Your breath,\pa the sea covers them;\pa they sank like lead\pa in the mighty waters.%%
  
  \pa \verse{15:11} Who is like You\pa among the gods, O \lord?\pa Who is like You,\pa glorious in holiness,%%
  
  \pb awe-inspiring in praises,\pa doing miracles?%
  \alt{marvels, wonders, extraordinary things, something extraordinary?}
  \pa \verse{15:12} You've stretched out Your right hand\pa and the earth has swallowed them.%%
  
  \pa \verse{15:13} In Your kindness You've led\pa the people whom You've redeemed:\pa You've guided them%
  \ca{\sampen\super{Mss}\targum\super{J} \Hebrew{נִחַלְתָּ}}{the Samaritan Pentateuch and Targum have ``taken possession''}
  in Your strength\pa to Your holy abode.%%
  
  \pa \verse{15:14} The nations%
  \ed{people\thinspace$\rightarrow$\thinspace peoples\thinspace$\rightarrow$\thinspace nations}
  have heard, they've trembled:\pa the Philistines have been seized with anguish.\alt{pain}%%
  
  \pa \verse{15:15} Then have the chiefs of Edom\pa been amazed%
  \ed{The first two hemistichs have been rearranged; it's literally ``Then been amazed * the chiefs of Edom''}
  \pa and the mighty ones of Moab\pa have been seized with trembling.%%
  
  \pb All the inhabitants of Canaan have melted away.%%
  
  \pa \verse{15:16} Terror and dread\pa have fallen on them.%
  \ed{The first two hemistichs have been rearranged; it's literally ``Have fallen on them\pa terror and dread.''}
  \pa By the greatness of Your arm\pa they are as silent\alt{still} as stone.%%
  
  \pb Until Your people pass over, O \lord;\pa until the people You've bought have passed over.%%
  
  \pa \verse{15:17} You shall bring them in and plant them\pa in the mountain of Your inheritance,\pa in the place You've made\pa for You to dwell in, O \lord.%%
  
  \pb A sanctuary of my Lord\pa has been established by Your hand.\pa \verse{15:18} The \lord\ shall reign\pa forever and always.%%
  
  \verse{15:19} Because Pharaoh's horse came in with his chariots and his horsemen into the sea, and the \lord\ turned the water of the sea back on them; and the children of Israel went on dry ground in the midst of the sea.''%%
  
  \verse{15:20} Miriam, the prophetess, Aaron's sister, took the timbrel\alt{tambourine} in her hand, and all the women following her went out with their timbrels and dances,%%
  \verse{15:21} and Miriam answered them:\smallskip%%
  
  \pb ``Sing to the \lord\ pa because He is greatly exalted!\pa The horse and its chariot\pa He's thrown into the sea.''%%
  
  \verse{15:22} Moses brought Israel from the Red Sea, and they went out into the Shur desert. They went three days in the desert and couldn't find water.%%
  \verse{15:23} They came to Marah and hadn't been able to drink the water of Marah because they were bitter;\halot{xxxx}{\textbf{bitter} (in taste): water \haref{Ex}{15}{23}; \textit{m\^e hamm\=ar\^im} bitter-water (meaning ?)} hence they called it Marah.%%
  \verse{15:24} The people murmured\alt{grumbled} against Moses, saying, ``What shall we drink?''%%
  \verse{15:25} He\ca{\sampen\septuagint\peshitta\vulgate\ + \Hebrew{מֹשֶׁה}}{the Samaritan Pentateuch, Septuagint, Peshitta, and Vulgate add ``Moses''} cried to the \lord, and the \lord\ showed him a tree, and he threw it into the waters and the waters became sweet.\ed{There's a peculiar tabbing here in the Hebrew that's preserved with the following space.}\hspace*{4em}He there made a law and a legal decision for them. He put them to the test.%%
  \verse{15:26} He said, ``If you'll actually listen to the voice of the \lord\ your God and do what's right in His eyes and hearken to His commandments and obey all of His laws, I won't put any of the sickness I put on the Egyptians on you because I, the \lord, am your healer.''%%
  \verse{15:27} They came to Elim and there were twelve water fountains and seventy date palms,\halot{xxxx}{\textit{Phoenix dactylifera}} and they camped there by the waters.%%
\end{inparaenum}

  \heading{19}{Israel to be a peculiar treasure, a kingdom of priests, a holy nation~--- they are sanctified~--- the Lord appears in Sinai}

\begin{inparaenum}
  \verse{19:1} In the third month since the departure of the children of Israel from the land of Egypt, that day they came into the Sinai desert.%%
  \verse{19:2} They left Rephidim, came into the Sinai Desert, and camped in the desert. And Israel camped there before the mountain.%%
  \verse{19:3} And Moses went up to God, and the \textsc{Lord} called to him from the mountain, saying, ``Thus shall you say to the children of Jacob and tell to the sons of Israel:%%
  \verse{19:4} `You~--- you've seen what I've done to the Egyptians, how I bear you on eagles' wings, and how I have brought you to Myself.%%
  \verse{19:5} And now, if you will truly obey My voice and keep My commandments, you shall be a treasure to Me\ed{Refers to a temple treasure and has covenantal implications.} from among all the people:\lit{above all people} for all the earth is Mine.%%
  \verse{19:6} You shall be a kingdom of priests and a holy nation to Me.' These are the words which you shall say to the children of Israel.''%%
  \verse{19:7} So Moses entered in\ie{to the camp} and called the elders of the people. He laid before them all these things that the \textsc{Lord} had commanded him.%%%
  \verse{19:8} Then all the people answered together, saying, ``We will do everything the \textsc{Lord} has said.'' So Moses returned to tell the \textsc{Lord} what the people had said.\lit{the words of the people.}%%
  \verse{19:9} The \textsc{Lord} said to Moses, ``I will come to you in a thick cloud so that the people may hear my words when I speak with you. So shall they believe you forever.'' Then Moses told the words of the people unto the \textsc{Lord}.%%
  \verse{19:10} And the \textsc{Lord} said unto Moses, ``Go to the people and sanctify them today and tomorrow. Let them wash their clothes%%
  \verse{19:11} because on the third day the \textsc{Lord} will descend in the eyes of all the people before Mount Sinai.%%
  \verse{19:12} You shall set bounds around the people, saying, `Hearken to these words\lit{them} and don't go up into the mountain. Anyone touching the base of the mountain will certainly be put to death.%%
  \verse{19:13} No hand shall touch it\ie{the mountain} or he will surely be stoned or shot:\ie{with arrows} whether beast or man, they shall not live while the trumpet sounds\lit{continues} as they approach the mountain.'\thinspace''%% xxxx: this needs work
  \verse{19:14} So Moses went down from the mountain to the people and sanctified\alt{consecrated, made holy} the people and they washed their clothes.\ed{As a way of preparing themselves to go up into the mountain.}%%
  \verse{19:15} He said to the people, ``Prepare yourselves. Do not sleep with\lit{approach} a woman for three days.''%%
  \verse{19:16} On the third morning, when it was morning, there were noises and flashes of lightning, a thick cloud over the mountain, and the tremendously powerful sound of a shofar:\ed{used for ritualistic purposes} and all the people in the camp were afraid.%%
  \verse{19:17} Moses brought the people out from the camp to meet God and they stationed themselves at the base of the mountain.%%
  \verse{19:18} All of Mount Sinai was smoking because the \textsc{Lord} descended on it in fire. And the smoke went up like the smoke of a furnace. The whole mountain shook.%%
  \verse{19:19} The sound of the trumpet grew continually louder\lit{more and more powerful} while Moses spoke and God answered with a voice.%%
  \verse{19:20} The \textsc{Lord} descended on the summit of Mount Sinai and the \textsc{Lord} called to Moses at the summit, and Moses went up.%%
  \verse{19:21} The \textsc{Lord} said to Moses, ``Go down and solemnly charge the people, otherwise many of them will break through, see the \textsc{Lord}, and perish.''\ed{Not because of some great iniquity, merely because of the glory of the \textsc{Lord} and His inherent power.}%%
  \verse{19:22} Also, the priests who come near the \textsc{Lord} shall sanctify themselves, otherwise the \textsc{Lord} will tear them down.%%
  \verse{19:23} Moses said to the \textsc{Lord}, ``The people will not be able to come up to Mount Sinai because You have charged us, saying, `Make a border around the mount and sanctify it.'\thinspace''%%
  \verse{19:24} The \textsc{Lord} said unto him, ``Go down~--- you and Aaron with you~--- and the priests will come up. Don't let the people break through to come up to the \textsc{Lord}, otherwise His glory will break out against them.''%%
  \verse{19:25} So Moses went down to the people and spoke to them.%%
\end{inparaenum}

  \heading{20}{The Decalogue~--- Israel commanded to bear witness that the Lord has spoken~--- altars of unhewn stone are to be built~--- sacrifices performed thereon}

\begin{inparaenum}
  \verse{20:1} And God spake all these words unto them, saying,%%
  
  \verse{20:2} \textsc{Preface.} ``I am the \lord\ your God who brought you out of the land of Egypt, from the house of captivity.\alt{servitude, bondage, slavery.}%%
  \verse{20:3} \textsc{i.}\ed{There are varied approaches to numbering the commandments. The Philonic tradition is used here.} Don't have other gods besides me.\ed{There is no \textit{sof pasuq} (\Hebrew{׃}) in the \textsc{bhs}. This could be used as an argument in favor of the Philonic tradition.}%%
  \verse{20:4} \textsc{ii.} Don't make graven images\alt{idols} for yourselves, neither any image that is in the heavens above, nor in the earth, nor beneath the earth, nor in the waters beneath the earth.%%
  \verse{20:5} Don't bow down to them, neither shalt thou worship them: for I, the \lord\ your God, am a jealous God and will seek retribution unto the third and fourth generation of them that hate me,%%
  \verse{20:6} but showing kindness\alt{keeping my covenant} to those who love me\lit{my lovers} and to those who keep my commandments.%%
  
  \verse{20:7} \textsc{iii.} Don't use\alt{take, lift up} the name of the \lord\ thy God with vain intent\alt{in vain, with vanity, to/with no good purpose} for the \lord\ will not hold him innocent\alt{guiltless} who uses His name with vain intent.\ed{The real meaning here is to not take an oath in the name of God and not intend to keep it.}%%
  
  \verse{20:8} \textsc{iv.} Remember the Sabbath day to sanctify it.\alt{consecrate, make it holy}\ed{The notion of \emph{making} the Sabbath day holy is more powerful than merely \emph{keeping} it holy for the responsibility then rests upon us to be an holy nation.}%%
  \verse{20:9} You shall labor and do all your work for six days;%%
  \verse{20:10} but the seventh day, the Sabbath of the \lord\ thy God, don't do any work: neither thee, nor thy son, nor thy daughter, nor thy male or female servant,\lit{nor his/your manservant, nor his/your maidservant} nor thy beast, nor thy stranger that is within thy gates:%%
  \verse{20:11} for it took six days for the \lord\ to make the heavens and the earth and all that is upon the face thereof, and on the seventh day he rested. Therefore, the \lord\ blessed the Sabbath day and consecrated it.%%
  
  \verse{20:12} \textsc{v.} Take thy father and thy mother seriously so that thy days may be lengthened upon the land the \lord\ thy God giveth\ed{Referring to the Promised Land that they have yet to inherit. It is in the participle form showing an ongoing action.} thee.%%
  
  \verse{20:13} \textsc{vi.} Don't murder.\ed{It is not ``kill.'' The root that appears in the \textsc{bhs} (\Hebrew{רצח}) has behind it the idea of malicious forethought.}%%
  
  \verse{20:14} \textsc{vii.} Don't commit adultery.%%
  
  \verse{20:15} \textsc{viii.} Don't steal.%%
  
  \verse{20:16} \textsc{ix.} Don't answer falsely.\alt{bear false witness/testimony.}%%
  
  \verse{20:17} \textsc{x.} Don't desire\alt{covet} thy neighbor's house, neither shall you desire your neighbor's wife, nor his male or female servant, nor his ox, nor his male donkey, nor anything that is thy neighbors.''%%
  
  \verse{20:18} Then all the people were witnesses to the thunder,\lit{His voice} lightning, the sound of the trumpet, and the smoke of the mount. And they were witnesses and removed themselves.\ed{In other words, they recognized the power and glory of God and stood back so as to not be consumed by His almighty power.}%%
  \verse{20:19} They then said to Moses, ``Speak on our behalf that we hear, and let Him not speak with us lest we die.''%%
  \verse{20:20} So Moses said unto the people, ``Don't be afraid, because in order to test thee, God is coming; and in order that thy reverence for Him be before you, that you don't sin.''%%
  \verse{20:21} The people stood back as Moses approached the thick cloud where God was.%%
  
  \verse{20:22} The \lord\ said to Moses, ``Thus shalt thou say unto the sons of Israel: `You have seen that I have spoken with you from the heavens.%%
  \verse{20:23} don't make of me gods of gold or silver for yourselves.%%
  \verse{20:24} Thou shalt make for me an altar of earth and shalt offer unto me a burnt offering and a peace offering.%%
  \verse{20:25} But if you make an altar of stones to me, thou shalt not build it of hewn stones\ed{lest it's confused with an idol or graven image} nor\lit{don't} fashion those stones with tools: if thou wieldest thine tool\ie{a metal instrument or tool. Not really a sword, although that is the word used in the \textsc{bhs}.} and lay it upon it\ie{the altar} thou wilt defile it.%%
  \verse{20:26} Thou shalt not ascend on the steps to my altar in order that thy nakedness be not revealed on this altar.'\thinspace''%%
\end{inparaenum}

  \heading{21}{The Lord's law of servants, marriage, and the death penalty~--- eye for an eye, tooth for a tooth~--- damage caused by oxen}

\begin{inparaenum}
  \verse{21:1} These are the judgments\alt{laws} that you shall set before them.\lit{``their face''}%%
  \verse{21:2} If you acquire a Hebrew slave six years, he shall serve thee; in the seventh year he shall go free without having to pay.\ed{like indentured servanthood}%%
  \verse{21:3} If he comes alone, alone shall he go out. If he is married,\lit{the husband of a woman} his wife shall go with him.%%
  \verse{21:4} If his master giveth him a wife and she bears sons or daughters to him, her children shall belong to the master and the man shall go forth by himself.%%
  \verse{21:5} If the servant actually says, ``I love my master, my wife, my children~--- I will not be freed.''%%
  \verse{21:6} Then his master will bring him to the presence of God, to the door, and his master shall pierce his ear with an awl and he shall be his\ie{the master's} slave forever.%%
  
  \verse{21:7} If a man sells his daughter to be a handmaid, then she shall not go out as the bondsmen do.%%
  \verse{21:8} However, if she is unacceptable to her buyer\lit{master} (who has taken her for himself), then he shall let her be ransomed. However, because he has dealt with her unfairly, he shall not have power to sell her to foreigners.%%
  \verse{21:9} If he designates\halot{xxxx}{\textbf{designate}: --- 1.\ assign a woman} her to his son, then it shall be done to her according to the law of daughters.%%
  \verse{21:10} If he takes another for himself, then her food, clothing, and right to motherhood\alt{marital intercourse; ``right to motherhood'' is more fitting contextually}\ed{How does this cover consent? Was that an issue in the ancient world?} shall not be taken away.%%
  \verse{21:11} But if he doesn't do these three things for her, she shall go free without money.%%
  
  \verse{21:12} Anyone who strikes\halot{xxxx}{smites, (deals a) blow} a man to death shall certainly be killed.%%
  \verse{21:13} However, if he has not lain in wait, but God has delivered him\ie{the killed} into his hand, then I will appoint a place where he can flee.%%
  
  \verse{21:14} If a man schemes\lit{acts presumptuously} against his neighbor to kill him by deceit,\alt{design, scheme, subtlety} you shall take him from my altar to kill him.%%
  
  \verse{21:15} He who strikes his father or his mother, he shall surely be put to death.%%
  \verse{21:16} Anyone who kidnaps\alt{steals} someone\ie{to sell them} and he is caught with the man still in his possession:\alt{power} he shall surely be put to death.\ed{In other words, kidnapping is a capital offense.}%%
  
  \verse{21:17} He who curses\alt{reviles, insults; but harsher than we tend to think} his father or his mother shall sure be put to death.%%
  
  \verse{21:18} And if men quarrel\alt{fight} and one man hits another (whether with a stone or his fists), but not to kill him~--- merely to put him to bed~---%%
  \verse{21:19} if he rises and walks about outside\halot{xxxx}{the area outside a house} on his staff,\ed{a staff for the sick} then the smiter shall be acquitted of relevant charges.\lit{declared exempt from punishment.} Only he shall pay for his lost time until he is thoroughly healed.%%
  
  \verse{21:20} If a man hits his male or female servant with a staff and that person dies under his hand then he will surely be punished in proxy for him.\ed{or her.}%%
  \verse{21:21} But if a day or two goes by and he\ie{the slave} gets up, he\ie{the master} will not be punished because it is his property.%%
  \verse{21:22} If men fight and strike a pregnant woman so that she has a miscarriage,\lit{her children go forth} but there is no further harm, he will surely be fined according to what the husband deems right, and he\ie{the other man} will give according to what is assessed.%%
  \verse{21:23} If there is a serious injury or death, you shall take life for life:%%
  \verse{21:24} an eye for an eye, a tooth for a tooth, a hand for a hand, a foot for a foot,\ed{This is significantly more merciful than people let on. In other ancient civilizations (e.g., Hammurabi's code) there were different punishments depending on the person's social status. This is a step towards equality.}%%
  \verse{21:25} burning for burning, wound for wound, strike for strike.%%
  
  \verse{21:26} If a man strikes the eye of his male or female servant and knocks it out,\lit{destroys it} he will let him go free in penance for his eye.%%
  
  \verse{21:27} If the tooth of a male or female servant is knocked out, he must let them go free because of their tooth.%%
  
  \verse{21:28} If an ox gores a man or a woman and that person dies, the ox will surely be stoned to death, his flesh shall not be eaten, and the owner shall not be liable.%%
  \verse{21:29} If the ox has been known to gore in the past and this was made known to its master(s), and his master didn't keep him under guard, and it kills a man or a woman, the ox will be stoned and his master will be put to death as well.%%
  \verse{21:30} If a payment is laid upon him, he shall pay\lit{give} a redemption\lit{ransom} for his life according to everything that is laid upon him.%%
  \verse{21:31} Whether it runs down\alt{butts, thrusts, gores} a son or a daughter, then you shall do to it according to this judgment.%%
  \verse{21:32} If the ox runs down a male servant or a handmaid, then he shall pay the master thirty pieces of silver\alt{shekels} and the ox shall be stoned.%%
  
  \verse{21:33} If a man uncovers\alt{opens} or digs a pit, doesn't properly cover it,\ed{Probably meant ``at night.''} and a donkey or an ox falls in,%%
  \verse{21:34} then the owner of the pit shall make it right: he shall give money to the owner\ie{of the animal} and the dead animal shall be his.%%
  
  \verse{21:35} If a man's bull strikes the bull of his neighbor and he dies, then he shall sell the live ox and half the money and the dead ox.%%
  \verse{21:36} If it is known that the bull has previously run people down\lit{is known to run people down} and its owner has not kept watch of him, then the owner will have to compensate the owner of the dead ox and the dead ox will belong to him.\ie{the one who paid.}%%
  \verse{21:37} If a man should steal an ox or a sheep and slaughters\ed{to use it for food} or sells it, he shall recompense\lit{make peace} with five oxen for the ox, or four sheep for the lamb.%%
\end{inparaenum}

  \heading{23}{Laws relating to honesty and conduct given, especially as relating to peer pressure~--- sabbatical year expounded~--- three feasts set forth~--- an angel will guide Israel~--- Canaanite nations will slowly be driven out}

\begin{inparaenum}
  \verse{23:1} You shouldn't bear fraudulent hearsay. Don't extend your hand to the wicked to be a violent witness.%%
  
  \verse{23:2} Don't follow the multitude for evil; don't testify concerning a strife, to go after the multitude to turn others aside.%%
  \verse{23:3} Don't treat the helpless with distinction in their dispute.\alt{case, lawsuit.}%%
  
  \verse{23:4} If you encounter your enemy's ox or donkey wandering,\lit{going back and forth} then you shall definitely bring it back to him.%%
  
  \verse{23:5} If you see your hater's donkey lying down under its burden, then you shall stop it\alt{leave it alone} from leaving. You shall certainly set it free.\alt{abandon it}%%
  
  \verse{23:6} Don't turn aside the judgment of the poor in their dispute.%%
  \verse{23:7} Stay away from falsehood.\lit{the thing of falsehood}\alt{lie, deception} And don't kill the innocent and the righteous because I will not vindicate the wicked.%%
  \verse{23:8} Don't take a bribe because bribes blind the clear-sighted and distort\alt{twist, mislead, perverse} the words of the righteous.%%
  \verse{23:9} Don't oppress\alt{press (someone in a given direction), torment, crowd} foreigners because you know what it's like to be a foreigner\lit{know the soul of the foreigner} because you used to be foreigners in Egypt.\lit{the land of Egypt}%%
  \verse{23:10} For six years you shall sow your land and harvest the\lit{its} increase.%%
  \verse{23:11} But in the seventh year,\understood\ you shall leave it untilled; you shall leave it untilled and let the needy among your people eat it and let the beasts of the field eat the leftovers.\lit{what they leave} You shall do likewise\lit{thus shall you do} in your vineyards and olive orchards.%%
  \verse{23:12} You shall do your work for six days and rest on the seventh day in order that your ox and your donkey may rest, that the son of your handmaid and the foreigners may catch their breath.%%
  \verse{23:13} Be on your guard concerning everything I've told you. Don't mention the names of other gods~--- it shall not be heard coming from\lit{in} your mouth.%%
  
  \verse{23:14} You shall celebrate a feast in my honor\lit{to me} three times a year.%%
  \verse{23:15} You shall keep the Feast of Unleavened Bread: you shall eat unleavened bread for seven days in the way that I have commanded you: in the appointed time, the month of Abib.\ed{The first month, the month of grain (\Hebrew{אָבִיב} meaning \textit{barley}).} This because you have come out of Egypt and no one shall appear empty in my presence.%%
  \verse{23:16} The Feast of Harvests: the first-fruits of your work that you have sown in the field. The Feast of In-Gathering: at the end of the year, when you gather your work\ed{The intended meaning is probably ``the first-fruits of your work.''} from the field.%%
  \verse{23:17} Three times during the year shall all of the men appear before the face of the Lord \god.%%
  \verse{23:18} Don't offer the blood of My offering on top of leavened bread. The fat of my Festival shall not remain until morning.%%
  \verse{23:19} You shall bring the choicest of the first-fruits of your land to the house of the Lord your \god. Do not boil a kid\ie{of a goat} in its mother's milk.%%
  
  \verse{23:20} I will send a messenger\ie{a heavenly messenger} before you to guard you in the way and to bring you to the place I have prepared.%%
  \verse{23:21} Keep his face before you and hear his words. Do not anger him because he will not forgive your sins because my name is in his midst.%%
  \verse{23:22} For if ye shall surely hearken unto his voice and do all that I have said: I will be an enemy to thy enemy and a foe to thy foe.%%
  \verse{23:23} For my messenger shall go before you and bring you to the Amorites and the Hittites and the Perezites and the Canaanites and the Jebusites; and I will annihilate\alt{cut them off, efface} them.%%
  \verse{23:24} You shall not bow down to their gods, nor shall you worship\alt{serve} them. You shall not do as they do, but you shall overthrow them and shall smash their stone images to pieces.%%
  \verse{23:25} You shall serve the \lord\ thy God and He will bless your food\lit{bread} and water. I will remove sickness from among you.%%
  \verse{23:26} There shall not be a woman who miscarries, neither is sterile\alt{infertile, barren} in your land. And you shall live a full life.\lit{I will complete the number of thy days.}%%
  
  \verse{23:27} I will send forth reverence of me before thee and confuse all the people that come out against thee. I will make all thy enemies flee\lit{give to you all your enemies' backs}.%%
  \verse{23:28} I shall send hornets before thee and it shall drive out the Hivites, the Canaanites, and the Hittites from before thee.%%
  \verse{23:29} I will not drive them out before you in one year lest the land become desolate and the beasts of the field multiply and they will not be able to be controlled.%%
  \verse{23:30} Little by little will I drive them out from before you until you become\alt{are} fruitful enough to take possession of the land.%%
  \verse{23:31} I will set your boundaries from the Red Sea\lit{Reed Sea} to the sea of the Philistines, and from the desert to the river;\ed{It's difficult to tell if it's the Jordan or the Tigris River.} for I will place the inhabitants of the land in your hand and you will drive them out from before yourselves.%%
  \verse{23:32} You shall not make\lit{cut} a covenant with them, neither with their gods.%%
  \verse{23:33} They shall not dwell in your land lest they cause you to deviate\alt{miss the mark, sin, transgress} from\alt{against} me in that you serve their gods: that you become a snare to yourselves.''%%
\end{inparaenum}

  \heading{24}{Israel, by covenant, accept the Lord~--- Moses, Aaron, Nadab, Abihu, and the Seventy see God~--- the Lord calls Moses to the mountain to receive the stone tablets}

\begin{inparaenum}
  \verse{24:1} And He saith unto Moses: ``Go up unto the \lord~--- you, Aaron, and seventy of the elders of Israel~--- and you shall bow down\ie{to worship} from afar.%%
  \verse{24:2} And Moses will approach the \lord\ alone. The people\ie{Aaron and the Seventy} will not go up with him.''%%
  \verse{24:3} And Moses came and told the people all the words and judgments of the \lord; and all of the people answered with one voice, saying, ``All the words that the \lord\ hath spoken we will do.''%%
  \verse{24:4} Moses wrote all the words of the \lord. Then, he rose early in the morning, and built an altar at the foot of the mountain. He also made\alt{built} twelve memorial stones\alt{pillars} for the twelve tribes of Israel.%%
  \verse{24:5} And he sent young men of the children of Israel and they offered burnt offerings and sacrificed sacrifices, peace offerings, and oxen.%%
  \verse{24:6} Then Moses took half of the blood and placed it in bowls, and half of the blood he sprinkled\alt{cast} on\alt{at} the altar.%%
  \verse{24:7} He took the scroll of the covenant and he read it out in the ears\alt{hearing} of the people and they said, ``Everything that the \lord\ hath said we will do.''\alt{hear, obey.}%%
  \verse{24:8} And Moses took the blood and he sprinkled\alt{cast} it upon the people and said, ``This is the blood of the covenant which the \lord\ hath made with you concerning all these things.''%%
  \verse{24:9} Then Moses, Aaron, Nadab, Abihu, and seventy elders of Israel went up%%
  \verse{24:10} and saw the God of Israel. Under His feet was a slab of sapphire as bright as heaven.%%
  \verse{24:11} He did not lay His hand on the eminent children of Israel. They saw God and they ate and they drank.\ed{Could easily be covenantal or a reference to the sacrament.}%%
  
  \verse{24:12} The \lord\ said to Moses, ``Come up to me in the mountain\lit{mountain-ward} and be there. I will give you stone tablets, the Law and the Commandments that I have written for their instruction.''%%
  \verse{24:13} Moses rose up~--- and Joshua, his minister, too~--- and Moses went to the mountain of God.%%
  \verse{24:14} To the elders he said, ``Wait here\lit{in this place} for us until we return for you. And look, Aaron and Hur are here\understood\ with you, and whoever has any problems, bring it up with them.''\lit{approach them with it}%%
  \verse{24:15} Moses went up into the mountain, and the cloud covered the mountain.%%
  \verse{24:16} The glory of the \lord\ settled on Mount Sinai, and the cloud covered the mountain for six days. On the seventh day, He called to Moses from the thick\lit{midst} of the cloud.%%
  \verse{24:17} In the people's view, the appearance of the \lord\ in His glory was like fire consuming the peak of the mountain.%%
  \verse{24:18} Moses came into the thick of the cloud and climbed up the mountain. And Moses was in the mountain for forty days and forty nights.%%
\end{inparaenum}

  \heading{31}{The Lord enumerates the items to be placed in the Tabernacle~--- He reiterates through Moses the sanctity and importance of the Sabbath~--- He gives Moses the stone tablets}

\begin{inparaenum}
  \verse{31:1} The \textsc{Lord} spoke to Moses, saying,%%
  \verse{31:2} ``See, I've called Bezaleel (son of Uri, son of Hur, of the tribe of Judah) by name,%%
  \verse{31:3} and I've filled him with the Spirit of God in wisdom, in understanding, in knowledge, in every work,%%
  \verse{31:4} to devise artistic designs, to work with gold and silver and copper,%%
  \verse{31:5} working in stone, for mounting,\ca{\missing\ \fragheb\septuagint}{missing in the Cairo Genizah and the Septuagint}\halot{xxxx}{uncertain: a) traditionally: abundance of water, pool; b) setting (of eyes, like jewels); c) of teeth, (firm) mounting} and for working in lumber~--- to do all kinds of work.%%
  \verse{31:6} I have given Aholiab son of Ahisamach of the tribe of Dan with him, and I've given wisdom in the heart of everyone who's wise-hearted, and they've made everything \lit{that}I've commanded you.%%
  \verse{31:7} The tent of meeting and the Ark of the Covenant and the mercy seat which is on it and all the vessels of the tent,%%
  \verse{31:8} and the table and its vessels,\ca{\fragheb\ mlt Mss \sampen\septuagint\peshitta\ + \Hebrew{כל} cf \caref{}{30}{27\super{a}}}{the Cairo Genizah, multiple Hebrew manuscripts, the Samaritan Pentateuch, Septuagint, and Peshitta add ``all''; compare \vref{Ex}{30}{27\super{a}}} and the pure lampstand\halot{xxxx}{not candlestick} and all its vessels, and the incense altar,%%
  \verse{31:9} and the burnt-offering altar and all its vessels, and the laver and its stand,%%
  \verse{31:10} and the woven\halot{xxxx}{special kind of woven material, corduroy?} garments,\ca{pc Mss + \Hebrew{בקדשׁ} \Hebrew{לשׁרת} cf \haref{}{35}{19} \haref{}{39}{1}; \sampen\super{Mss} \Hebrew{הַשָׁרֵת} cf \septuagint\peshitta\targum\targum\super{J}}{a few Hebrew manuscripts add ``to serve in holiness,'' compare \vref{Ex}{35}{19} \vref{Ex}{39}{1}; the Samaritan Pentateuch has ``the service,'' compare with the Septuagint, Peshitta, and a couple targumim} and the holy garments for Aaron the priest, and the garments for his sons to be priests,%%
  \verse{31:11} the anointing oil, the perfumed incense for the Holy Place, according to everything I've commanded you: do it.%%
  
  \verse{31:12} The \textsc{Lord} spoke to Moses, saying,%%
  \verse{31:13} ``You, speak to the children of Israel, saying, `You shall surely keep\alt{observe} My Sabbaths because this is a sign between Me and you throughout your generations that\understood\ you may know that I am the \textsc{Lord} who sanctifies you.\alt{makes you holy, consecrates you.}%%
  \verse{31:14} Keep the Sabbath because it is holy\alt{sanctifying} to you. Everyone who profanes it will certainly\alt{shall surely} be put to death.\alt{die.}\ed{I like this more as ``be put to death'' because it shows that someone has to kill the commandment breaker other than the Lord. That said, people were almost never killed because of their disobedience to this commandment. They were rather left out of the community, ``cut off from the people.'' With this known, either rendering works well.} Because whoever does work on it, that person shall be cut off from among\ca{pc Mss \peshitta\targum\super{Ms}\targum\super{J} \Hebrew{מֵע׳}}{a few manuscripts of the Peshitta, Targum (codex manuscriptus vel editio secundum apparatum criticum Sperberi), and Targum (Targum Pseudo-Jonathae secundum M. Ginsburger, Pseudo-Jonathan 1903) simply have ``from his people''} his people.%%
  \verse{31:15} Six days is work done,\ca{\septuagint\ 2 sg et \peshitta\vulgate\ 2 pl act}{the Septuagint is in second person singular, while the Peshitta and Vulgate are in second person plural active} but the seventh day is the Sabbath, a day\understood\ of rest, holy to the \textsc{Lord}. Anyone who does work on the Sabbath day will certainly be put to death.%%
  \verse{31:16} The children of Israel shall keep the Sabbath~--- to keep the Sabbath\ca{\septuagint(\vulgate) \Greek{αὐτά}}{the Septuagint (and Vulgate) have ``these'' [instead of ``the Sabbath'']} throughout their generations is an everlasting covenant.%%
  \verse{31:17} It shall be an everlasting sign between Me and the children of Israel, because in six days the \textsc{Lord} made the heavens and the earth\ca{\septuagint\super{426}(\peshitta) + \Greek{καὶ τὴν θάλασσαν καὶ πάντα τὰ ἐν αὐτοῖς}}{the Septuagint (426) (and Peshitta) add ``and the sea and all that is in them''} and he rested\alt{ceased} on the seventh day and was refreshed.\alt{caught His breath.}''%%
  
  \verse{31:18} When He'd finished speaking with him on Mount Sinai, He gave Moses two tablets of testimony, stone tablets, written with\alt{by} God's finger.%%
\end{inparaenum}

  \heading{33}{The Lord commands Moses and the children of Israel to go into the land of Canaan~--- the people murmur, not wanting to obey~--- Moses takes the Tabernacle outside the camp and the Lord converses with him there~--- Moses talks with the Lord face to face}

\begin{inparaenum}
  \verse{33:1} The \textsc{Lord} spoke to Moses: ``Leave, get up from here, you and this people whom you've brought out from the land of Egypt to the land that I've promised\alt{sworn} to Abraham, Isaac, and Jacob, saying, `I will give it to your posterity,'%%
  \verse{33:2} (I have sent an\ca{\septuagint\ + sugg 1 sg}{the Septuagint adds a first person singular pronominal suffix [i.e., ``my angel'']} angel before you and driven the Canaanites, Amorites, Hittites,\ca{\sampen\ + \Hebrew{וְהַגִּרְגָּשִׁי} cf \septuagint}{the Samaritan Pentateuch adds ``Gergeshites,'' compare the Septuagint} Perizzites, Hivites, and Jebusites out)%%
  \verse{33:3} to a land flowing with milk and honey because I don't go up in your midst because you are a difficult and obstinate\alt{stiff-necked} people, lest I consume you in the way.''%%
  \verse{33:4} The people heard\alt{obeyed} these evil words and mourned, and no one put on their ornaments.%%
  \verse{33:5} So the \textsc{Lord} said to Moses: ``Say to the children of Israel: `You are a difficult and obstinate people: if I come into your midst for one instant,\alt{moment} I will consume you. So now, remove your ornaments\lit{from on you} and I will know what to do to you.'\thinspace''%%
  \verse{33:6} The children of Israel were stripped of their ornaments at Mount Horeb.%%
  
  \verse{33:7} Moses took the Tabernacle and pitched it outside the camp, far from the camp, and called it the Tent of Meeting. Everyone seeking the \textsc{Lord} went out to the Tent of Meeting that was outside the camp.\ed{You have to work to get to the temple.}%%
  \verse{33:8} When Moses went out to the tent, all the people stood\alt{arose}~--- each man standing at the entrance of his tent~--- watching after Moses until he'd entered the tent.%%
  \verse{33:9} When Moses had entered the tent, the pillar of the cloud came down and stood at the opening of the tent, and He\ed{understood. It's conjugated 3ms and it's already known (context) that the Lord is in the pillar: this isn't a leap.} spoke with Moses.%%
  \verse{33:10} All of the people saw the pillar of the cloud standing at the opening of the tent; additionally, all of the people got up and everyone\lit{each man} prostrated themselves\alt{bowed, worshiped}\ed{The rendering that the people worshiped the Lord doesn't seem likely since they were previously murmuring about what He'd commanded them; they don't want to obey and it's more likely that they are here prostrating themselves in fear than bowing in heartfelt worship.} at the door of their\lit{3ms} tent.%%
  \verse{33:11} And the \textsc{Lord} spoke to Moses face to face like a man speaks to his friend; and he returned to the camp, but his attendant\alt{minister} Joshua, son of Nun, a kid, didn't leave the tent.\lit{didn't depart from the tent.}%%
  
  \verse{33:12} Moses said to the \textsc{Lord}, ``Look, You say to me: `Bring up this people,' but You haven't let me know\lit{caused me to know} who You'll send with me; You've said, `I know you by name;\ca{\septuagint\ \Greek{παρὰ πάντας}}{the Septuagint has ``despite all men''} and you've also found favor in My eyes.'%%
  \verse{33:13} Now, please, if I've found favor in Your eyes, let me know Your way\ca{\septuagint\ \Greek{ἐμφάνισόν μοι σεαυτόν} cf 18\super{a--a}; \vulgate\ \textit{ostende mihi faciem tuam}}{the Septuagint has ``show me Yourself,'' compare verse~18\super{a--a} [says the same thing]; the Vulgate has ``show me Your face''}\ed{It's no surprise that the Septuagint would say ``Yourself'' while the Vulgate would say, ``Your face,'' because the Septuagint almost always avoids anthropomorphisms.} so I can know you, in order to find favor in Your eyes. Consider that this nation is Your people.''%%
  \verse{33:14} He said, ``My presence will go and I will give you rest.''%%
  \verse{33:15} But he said to Him, ``If Your presence doesn't go, then\understood\ don't take us from here.%%
  \verse{33:16} How\lit{In what} will it be known that Your people and I have found favor in Your eyes? Won't it be by Your going with us?\lit{[interrogative] not in/by Your going with us?} Your people and I have been treated differently\alt{distinguished} from all the people that are on the face of the earth.''%%
  
  \verse{33:17} The \textsc{Lord} said to Moses, ``Also, this word that I have spoken, I will do it\understood\ because you've found favor in My eyes, and I will know you by name.''\ca{ut 12\super{a}}{same as 12\super{a} [the Septuagint has ``despite all men'']}%%
  \verse{33:18} He said, ``Please let me see Your glory.''\ca{\septuagint\ \Greek{ἐμφάνισόν μοι σεαυτόν} cf 13\super{a--a}}{the Septuagint has ``show me Yourself,'' compare verse~13\super{a--a} [says the same thing]}%%
  \verse{33:19} He said, ``I will cause all of My goodness to pass before your face\ca{\septuagint\ \Greek{ἐγὼ παρελεύσομαι πρότερός σου τῇ δόξῃ μου}}{the Septuagint has ``xxxx''} and I will proclaim the name of the \textsc{Lord}\ca{\septuagint\ \Greek{ἐπὶ τῷ ὀνόματί μου}}{the Septuagint has ``xxxx''} before you. I will favor whom I will favor; I will have compassion on whom I will have compassion.''%%
  \verse{33:20} He\ie{the Lord} said, ``You aren't able to see My face because man cannot see My face and live.''%%
  \verse{33:21} The \textsc{Lord} said, ``There is\understood\ a place by Me and you shall stand on a rock.%%
  \verse{33:22} When My glory passes over, I will place you in the cleft of the rock, and I will screen you off\ie{from view} with\understood\ My hands over you until I've passed by;%%
  \verse{33:23} and I will turn My hands away and you will see My back, but you won't see My face.''%%
\end{inparaenum}

  
  % \book{Leviticus}{\Hebrew{ויקרא}}
  % \heading{26}{The people install Uzziah as king~--- Uzziah, like his father, was righteous and God made him prosper~--- he goes to war, becomes vain, and tempts God~--- he makes an illegal sacrifice and is cursed with a serious skin condition}

\begin{inparaenum}
  \verse{26:1} All the Judahites\lit{the people of Judah} took Uzziah, who was sixteen years old, and crowned\alt{made, installed} him king instead of his father, Amaziah.%%
  \verse{26:2} He built Eloth and restored it to Judah after the king lay with his fathers.\ie{after death}%%
  
  \verse{26:3} Uzziah was sixteen years old when he became king. He ruled in Jerusalem for fifty-two years. His mother's name was Jecholiah of Jerusalem.%%
  \verse{26:4} He did what was right in the eyes of \god\ like everything that his father Amaziah had done.%%
  \verse{26:5} He worshiped\alt{sought, cared about, inquired regarding} God in the days of Zechariah (who understood\alt{perceived, paid attention to, considered, gave heed to, noticed} the visions of God). In the days he worshiped the \lord, God made him prosper.%%
  
  \verse{26:6} He went and fought\alt{do battle with, come to close quarters with} against the Philistines and made a breach in\alt{broke down} the wall of Gath, the wall of Jabneh, the wall of Ashdod. He built cities in Ashdod and Philistia.%%
  \verse{26:7} God came to his aid\alt{supported, helped} against the Philistines, the Arabians living in Gur-Baal, and the Maonites.%%
  \verse{26:8} The Ammonites gave a present to Uzziah and his name went to the entrance of Egypt because he became incredibly\lit{lifted way up, escalated. From \Hebrew{עלה} meaning to lift up.} strong.%%
  \verse{26:9} Uzziah built towers\halot{xxxx}{in a defensive wall} in Jerusalem by the corner gate, the valley gate, and at the corner, and he fortified\alt{strengthened} them.%%
  \verse{26:10} He built towers in the desert and hewed out\alt{quarried; other translations render this as ``dug,'' but \textsc{halot} only mentions stonework.} many wells because he had a lot of cattle, both in the lowland and in the plains;\alt{on the plateau, \Hebrew{מִישׁוֺר} meaning both plain and plateau.} he had\understood\ serfs\halot{xxxx}{peasant, not owning land, belonging to landlord} and wine-growers\alt{vinedresser} in the mountains and in Carmel because he loved growing stuff.%%
  
  \verse{26:11} Uzziah had an army\halot{xxxx}{alt., wealthy landowner, qualified, fit for military service; (large) landowner, obligated to military service and the furnishing of a certain number of men; then valiant man without regard to property; all the armed host of a people and a satrapy (a provincial governor in ancient Persia).} who was making war, going out by troop as commissioned\lit{listed by name} by the hand of Jeiel the scribe and Maaseiah the record-keeper and by the hand of Hananiah, the second after the king.%%
  \verse{26:12} The whole number of the head honchos\lit{chief fathers} of the elite troops was 2\thinspace600.%%
  \verse{26:13} With regard to the strength\alt{power; lit., hand} of the army,\alt{host} 307\thinspace500~warriors who made war with great ability\alt{power, capacity, strength, means} to support\alt{help, come to the aid of} the king against the enemies.%%
  \verse{26:14} For the entire army, Uzziah prepared shields, spears, helmets, scale-armor,\alt{coat of mail} bows, and slinging-stones for them.%%
  \verse{26:15} In Jerusalem, he made war machines\halot{xxxx}{(skillfully contrived) \textbf{war-machines}, spec. catapults}\ed{Not a ballista. The earliest ballistas were invented circa 400~\textsc{b.c.}}~--- thought out\alt{invented} by technicians~--- to be placed\understood\ on the towers and the corners to shoot arrows and boulders.\lit{great stones} His name spread some distance\lit{out a distance} because he was extraordinarily helped  until he became strong.%%
  \verse{26:16} However, when he become strong, his heart was lifted up unto destruction. And he transgressed against the \lord\ his God by going unto the temple of the \lord\ and offering incense upon the altar of incense.%%
  \verse{26:17} And Azariah the priest went in after him with the priests of the \lord: eighty valiant men.%%
  \verse{26:18} And they withstood Uzziah the king and said unto him, ``This is not for thee, Uzziah, to burn incense unto the \lord; rather for the priests~--- the sons of Aaron~--- that are set apart\alt{consecrated, sanctified} to burn incense. Leave this holy place\lit{Go out from this sanctuary} for you have transgressed. And neither shall this be for thine honor from the \lord\ God.''%%
  \verse{26:19} Uzziah was wroth (and in his hand he had a censer of incense). And while he was angry at the priests, leprosy appeared\lit{rose up} in his forehead in the presence of the priests in the House of the \lord\ next to the altar of incense.%%
  \verse{26:20} Azariah, the chief priest, and all the priests looked upon him, and lo! he was leprous in his forehead. So they thrust him out from thence, yea, even he\ie{Uzziah} hurried out because the \lord\ has smitten him.%%
  \verse{26:21} Uzziah the king was leper unto his dying day~--- he dwelt in a separate house (being a leper) because he was cut off from the House of the \lord. And Jotham, his son, took over\lit{was over} the king's affairs,\lit{house} judging the people of the land.%%
  \verse{26:22} The rest\alt{remainder} of the history of Uzziah, the first and the last, was written by the prophet Isaiah, the son of Amoz.%%
  \verse{26:23} Uzziah lay with his fathers,\ie{in death} and they buried him with his fathers in the kings'\lit{that the kings have} cemetery,\lit{field of the graves} for the said, ``He had a skin condition.'' His son, Jotham, reigned in his stead.%%
\end{inparaenum}

  % 26:11: set footnote next to "I will establish my (presence) in your midst": See further in Welch's \textit{Experiencing the Presence of the Lord: The Temple Program of Leviticus}
  % 26:11-12: ``The term for the presence of the Lord (walking, \textit{mithalekh}) used in \vref{Leviticus}{26}{11--12} is the same used in the Eden story in \vref{Gen}{3}{8}'' (Welch, \textit{Experiencing the Presence of the Lord: The Temple Program of Leviticus})
  
  \book{Numbers}{\Hebrew{במדבר}}
  \heading{5}{How to deal with lepers~--- repentance necessary for forgiveness~--- dealing with women accused of immorality}

\begin{inparaenum}
    \verse{5:1} The \textsc{Lord} spoke to Moses, saying,%%
    \verse{5:2} ``Command the people of Israel that they send everyone outside the camp who has a skin disease, a discharge,\halot{xxxx}{\textbf{mucous discharge} of a man (\textit{gonorrhea benigna}); \textbf{discharge of blood} of a woman (within and outside the menstrual period)} or who has become ritualistically unclean from coming in contact with\understood\ a corpse.%%
    \verse{5:3} Send both men and women outside~--- send them out of the camp so that they don't make the camp wherein I live ritualistically unclean.''%%
    \verse{5:4} The people of Israel did so and sent them outside the camp. The people of Israel did just as the \textsc{Lord} had told Moses.%%
    
    \verse{5:5} The \textsc{Lord} spoke to Moses, saying,%%
    \verse{5:6} ``Speak to the people of Israel. When a man or a woman commits any human sin by being unfaithful\ed{\Hebrew{מַעַל} is always against God.} to the \textsc{Lord} and that person is guilty:%%
    \verse{5:7} they shall confess the sin that they've committed, he shall restore his wrong\halot{xxxx}{(equivalent amount of) \textbf{wrong} (in case of property damage)} by its total amount plus one fifth, and give it to the person he wronged.%%
    \verse{5:8} But if the man doesn't have a redeemer to give the compensating gift to, then the guilt shall be recompensed to the \textsc{Lord} shall be the priest's in addition to the reconciliatory ram by which the priest makes an atonement for the man.%%
    \verse{5:9} Every tribute offering of any of the holy things of the people of Israel which they present to the priest, these shall be his.%%
    \verse{5:10} Everyone's sacred things shall be his. Whatever people give to the priest shall be his.''%%
    
    \verse{5:11} And the \textsc{Lord} spake unto Moses, saying,%%
    \verse{5:12} ``Speak unto the children of Israel and say unto them: `If any man's wife go astray and trespasses against him%%
    \verse{5:13} in that a man lies sensually with her and it is hidden from the eyes\lit{kept from the knowledge} of her husband~--- kept secret~--- and she is defiled, but there is no witness against her, neither is she caught in the act:%%
    \verse{5:14} if the spirit of jealousy is passed over him and he suspects her of infidelity and she is defiled, or if the spirit of jealousy passes over him and he suspects her of infidelity and she is not defiled:\ed{e.g., if he's jealous, whether or not she's defiled, the following needs to happen.}%%
    \verse{5:15} the man shall bring his wife before the priest. He\ie{the husband} shall bring an offering for her (a tenth of an ephah of barley wheat), but he shall not pour upon it any oil neither will he place incense upon it since it is a memorial offering to bring to mind\alt{remembrance} wrong-doing.%%
    \verse{5:16} The priest shall bring her and cause her to stand before the \textsc{Lord}.%%
    \verse{5:17} The priest shall take holy water in a jar of clay and from the dust that is on the floor of the Tabernacle. And the priest shall take it and put it\ie{the dust} in the water.%%
    \verse{5:18} And the priest shall make the woman stand before the \textsc{Lord}.\ed{\dots in front of the Holy Place. It is unknown whether or not she is allowed into the grounds of the Tabernacle complex.} And he will uncover the woman and he shall place in her hands the memorial offering as an offering of jealousy. The bitter waters that cause the curse shall be in the priest's hand.%%
    \verse{5:19} And the priest shall place her under oath and say to the woman, ``If no man has laid with thee, and if thou hast not gone astray to be unclean, you will be free from these waters of bitterness that caused the curse.\ie{to take effect}%%
    \verse{5:20} But if you have gone astray to another besides your husband, or if you have defiled yourself in that a man other than your husband has lain with you,''\lit{if a man other than your husband has given you his line}%%
    \verse{5:21} the priest shall charge the woman\lit{place the woman under oath} with an oath of cursing and the priest shall say to the woman, ``May the \textsc{Lord} make you a curse and an oath amongst your people.'' The \textsc{Lord} shall cause your thigh to waste\alt{fall} away and your womb to swell.%%
    \verse{5:22} These cursed waters have come into your intestines and caused your belly\halot{xxxx}{of a man. Womb for a woman.} to swell and your crotch\halot{xxxx}{the fleshy portion of the \textbf{upper thigh}, seat of procreation, area of sexual organs.} to shrink. The woman shall say, ``Surely! Surely!''\ed{see Appendix~\ref{app:idioms}}%%
    \verse{5:23} And the priest shall write these words on a scroll and he shall wash them out into the bitter waters.%%
    \verse{5:24} He will have the woman drink the bitter waters that cause the curse and the waters will enter into her.%%
    \verse{5:25} And the priest shall take the offering of jealousy from the hand of the woman and he shall wave the offering before the \textsc{Lord} and he shall bring it\ie{the offering} to the altar.%%
    \verse{5:26} And the priest shall take a handful of the burned meal-offering\halot{xxxx}{the portion of the meal-offering which is burned; suggested meanings: reminiscence; summons; invocation; sign-offering} from the altar as her memorial offering and he will burn it on the altar, and afterwards he will have the woman drink the waters.\ie{the waters of bitterness.}%%
    \verse{5:27} And he shall cause her to drink the waters of bitterness if she has transgressed in that she is defiled. And the waters of bitterness that caused the curse shall enter into her and her stomach shall swell and her uterus shall waste away and she shall be a curse in the midst of her people.%%
    \verse{5:28} And if the woman is not defiled in that she acted unfaithfully, but she is clean, she shall be guiltless,\alt{clean} and she, conceiving, will give birth to a child.'\ie{eventually}%%
    \verse{5:29} This is the law concerning those suspected of infidelity (when the wife goes astray while married to her husband and is defiled).%%
    \verse{5:30} When a spirit of jealousy passes over a man and he's jealous of his wife, he shall cause his wife to stand before the \textsc{Lord} and the priest shall do everything in this law to her.%%
    \verse{5:31} The man is then free from wrong-doing, but the woman will pay for her iniquity.''%%
\end{inparaenum}

  \heading{6}{Law of the Nazirite~--- Aaronic blessing}

\begin{inparaenum}
    \verse{6:1} The \textsc{Lord} spoke to Moses, saying,%%
    \verse{6:2} ``Speak with the people of Israel and say to them, `If a man or woman vows the oath of the Nazirite, to consecrate themselves to the \textsc{Lord},%%
    \verse{6:3} then they\lit{he; but ``they'' is gender neutral which agrees better with verse~2} shall not drink beer\halot{xxxx}{intoxicating drink, evidently a kind of \textbf{beer}} or vinegar beer,\ed{Possibly wine. Any drink left out will begin to ferment with time.} neither shall they drink grape-juice or grape-extract or eat dried or fresh grapes.%%
    \verse{6:4} The entire time\lit{all the days} he's a Nazirite\halot{xxxx}{live as a \textit{n\=az\^\i{}r}, accept the obligations of a Nazirite} he shall not eat anything made from the wine grapevine or from the peel of unripe grapes.%%
    \verse{6:5} The whole time he's vowed to be a Nazirite, a razor shall not pass over his head until the time\lit{his days} that he's consecrated to the \textsc{Lord} is completed. The hair\halot{xxxx}{loose or unbraided} of his head shall grow.%%
    \verse{6:6} He shall not have dealings with a dead body the whole time he's a consecrated as a Nazirite to the \textsc{Lord}.%%
    \verse{6:7} He shall not be ritualistically unclean for his father, mother, or brother when they die because his head is consecrated to his God.%%
    
    \verse{6:8} For the whole time that he's consecrated he shall be holy to the \textsc{Lord}.%%
    \verse{6:9} If the dead suddenly die because of him, and his consecrated head has become ritualistically unclean, then he shall shave his head on the day of his ritualistic cleansing~--- he shall shave his head on the seventh day.%%
    \verse{6:10} In the eighth day he shall bring either two turtle doves or two pigeons to the priest at the entrance of the meeting tent.%%
    \verse{6:11} The priest shall offer one as a sin-offering and the other\lit{one} as a burnt-offering to atone for his sin with the dead body. On that same\understood\ day he shall reconsecrate\lit{consecrate; understood as ``reconsecrate''} his head.%%
    \verse{6:12} He shall vow the days of his consecration to the \textsc{Lord}. He shall bring a year-old lamb to be a compensating gift, but the first days shall be forfeited because he was ritualistically unclean.%%
    
    \verse{6:13} This is the law of the Nazirite. When his time of consecration is complete,\lit{in the day of the completion of his days of consecration,} he shall be brought to the entrance of the meeting tent.%%
    \verse{6:14} He shall bring his offerings near to the \textsc{Lord}: a year-old lamb that is free of blemish for the burnt-offering, a year-old ewe lamb that is free of blemish for the sin offering, a ram free of blemish for the peace offering,%%
    \verse{6:15} a basket of matzo, ring-shaped bread moistened with balsam oil, unleavened wafers with balsam oil spread on them, their offering, and their drink-offerings.\ed{of wine}%%
    \verse{6:16} The priest shall bring them before the \textsc{Lord} and offer his sin- and burnt-offerings.%%
    \verse{6:17} The ram shall be offered as a peace offering before the \textsc{Lord} along with the basket of matzo; the priest shall offer his sacrifice and his drink-offering.%%
    \verse{6:18} The Nazirite shall shave his consecrated head at the entrance of the meeting tent, take the hair of his consecrated head, and place it on the fire which is under the peace offering.%%
    \verse{6:19} The priest shall take the boiled ram shoulder, one piece of matzo from the basket, and one unleavened wafer and place them on the Nazirite's palms after he has shaved his consecrated head.%%
    \verse{6:20} The priest shall wave them as wave offerings before the \textsc{Lord}; in addition to the brisket of the wave offering and the tribute thigh, it is holy to the priest. Afterwards, the Nazirite may drink wine.%%
    
    \verse{6:21} This is the law of the Nazirite who vows his offering to the \textsc{Lord} in addition to \lit{his}being set apart, and besides what he is able to get his hands on. He shall do according to his vow that he has sworn~--- the law of the Nazirite.''%%
    
    \verse{6:22} God spoke to Moses, saying,%%
    \verse{6:23} ``Speak unto Aaron and his sons, saying, `Thus shall you bless the children of Israel, saying unto them,%%
    
    \pvcb{\vn{6:24} ``May the \textsc{Lord} bless}{and preserve you.}%%
    
    \pvcb{\vn{6:25} May the \textsc{Lord} cause His face to shine upon you}{and show you favor.}%%
    
    \pvcb{\vn{6:26} May the \textsc{Lord} lift His face toward you}{and may He give peace unto you.''\,'}%%
    
    \verse{6:27} They shall put my name upon the children of Israel and I will bless them.''%%
\end{inparaenum}

  \heading{10}{Trumpets to be used in calling and assembling Israel~--- cloud removes from camp~--- Israel sets forth in their orders~--- the Ark of the Covenant goes before the people}

\begin{inparaenum}
  \verse{10:1} The \textsc{Lord} spoke to Moses, saying,%%
  \verse{10:2} ``Make two silver trumpets\halot{xxxx}{long, straight instrument of metal for signaling}\ed{not to be confused with \textit{shofar}} for yourselves. You shall make them of hammered\alt{embossed} metalwork. They shall serve you when the assembly is gathered and when the order of departure for the camp is given.%%
  \verse{10:3} When they blow them, the whole company shall gather to you at the entrance of the meeting tent.%%
  \verse{10:4} If they blow with one, then the princes, the heads of thousands, shall gather to you.%%
  \verse{10:5} When you've blown an alarm, the camps which are encamped eastward shall break camp and march on.%%
  \verse{10:6} When you blow the alarm the second time, the camps which are encamped southward shall set forward. They shall blow an alarm when they set forward.%%
  \verse{10:7} In the assembling of the congregation of people, you shall blow the horn, but you shall not shout.%%
  \verse{10:8} The descendants of Aaron the High Priest shall blow the trumpets. They shall be an eternal, diving statute for you.%%
  \verse{10:9} If you go to war in your land against the enemy who is being hostile toward\alt{in a state of conflict with} you, then you shall blow the trumpets and you shall be remembered by\alt{in the sight of} the \textsc{Lord} your God and be saved from your enemies.%%
  \verse{10:10} In the day that you rejoice, in your assemblies and in your new moons, you shall blow the trumpets because of and in addition to\ed{\Hebrew{עַל}, here, carries with it both of these meanings} your burnt offerings and your peace sacrifices. They shall be a reminder for you of your God: I, the \textsc{Lord}, am your God.%%
  
  \verse{10:11} And in the second year, in the second month, on the twentieth of the month, that the cloud was taken up from the Tabernacle of the Testimony.%%
  \verse{10:12} The children of Israel set forth on their journey from the wilderness of Sinai and the cloud rested in the wilderness of Paran.%%
  \verse{10:13} And the first went not before the face of the \textsc{Lord} by the hand of Moses.%%
  \verse{10:14} The banner of the camp of Judah set forth first off, according to their ranks; and over its host was Nahshon the son of Amminadab.%%
  \verse{10:15} Over the host of the tribe of the children of Issachar: Nethaneel the son of Zuar.%%
  \verse{10:16} Over the host of the tribe of the children of Zebulun: Eliab the son of Helon.%%
  \verse{10:17} The utensils of the tabernacle were taken down, and the sons of Gershon and the sons of Merari~--- who carried the tabernacle and its accouterments~--- set forth.%%
  
  \verse{10:18} The banner of the camp of Reuben traveled according to their ranks; and over its host was Elizur the son of Shedeur.%%
  \verse{10:19} Over the host of the tribe of the children of Simeon: Shelumiel the son of Zurishaddai.%%
  \verse{10:20} Over the host of the tribe of the children of Gad: Eliasaph the son of Deuel.%%
  \verse{10:21} The carriers of the Holy Place\ed{(inferred) the Holy of Holies as well}~--- the Kohathites~--- went forth; and they would be the ones to set up the tabernacle at their\ie{the Israelites'} coming to a particular spot.%%
  
  \verse{10:22} The banner of the camp of the children of Ephraim traveled according to their ranks; and over its host was Elishama the son of Ammihud.%%
  \verse{10:23} Over the host of the tribe of the children of Manasseh: Gamaliel the son of Pedahzur.%%
  \verse{10:24} Over the host of the tribe of the children of Benjamin: Abidan the son of Gideoni.%%
  
  \verse{10:25} The banner of the camp of the children of Dan set forth, which was the rear of all the camps throughout their armies: and over its rank was Ahiezer the son of Ammishaddai.%%
  \verse{10:26} Over the host of the tribe of the children of Asher: Pagiel the son of Ocran.%%
  \verse{10:27} Over the host of the tribe of the children of Naphtali: Ahira the son of Enan.%%
  \verse{10:28} Such were the journeys of the children of Israel according to their ranks when they set forth.''%%
  
  \verse{10:29} And Moses said to Hobab,\ie{Jethro} the son of Reuel the Midianite, the father-in-law of Moses, ``We are setting off to the same place that the \textsc{Lord} said, `I will give to you.' Come with us and we will treat you kindly because the \textsc{Lord} hath dealt kindly with Israel.''%%
  \verse{10:30} But he said, ``I will not go; rather, I will go unto mine own land and to mine own home country.''%%
  \verse{10:31} And he\ie{Moses} said, ``Please don't leave me because you know how we should encamp in the desert~--- you shall be our eyes.%%
  \verse{10:32} ``If you will go with us, it will be good to you in the same measure as the \textsc{Lord} will be\alt{was}\ed{Some slight eisegesis here, but I feel the rendering is more accurate with an eternal perspective. However, it works perfectly well with ``was'' because the Israelites have already been delivered from Egypt.} with us.''%%
  \verse{10:33} So they set out from the mountain of the \textsc{Lord} a journey of three days, and the Ark of the Covenant of the \textsc{Lord} went before them three days to seek out a place of rest for them.%%
  \verse{10:34} And the \textsc{Lord}, as a cloud, went over them by day when they journeyed from the camp.%%
  
  \verse{10:35} And when the Ark went forth, Moses said,\smallskip%%
  
  \pa ``Rise up, O \textsc{Lord},\pa and let thine enemies be scattered;\pa yea, let thine haters flee from before thy face.''%%
  
  \noindent\verse{10:36} And when it rested he said,\smallskip%%
  
  \pa ``Return, O \textsc{Lord},\pa to the tens of thousands of Israelites.''%
  \ed{10\thinspace000 is the largest root describing numbers in Biblical Hebrew.}%%
\end{inparaenum}

  \heading{11}{The Lord destroys the rebels with fire~--- Israel complains about manna~--- Moses cannot take the burden alone~--- the Lord commands Moses to call the Seventy~--- meat will be given until it is loathsome~--- Seventy are called and chosen~--- Eldad and Medad prophesy~--- the people lust~--- many are destroyed by a plague}

\begin{inparaenum}
  \verse{11:1} And the people complained that which is evil\alt{complained bitterly, complained and it was evil} in the ears of the \lord, and the \lord\ heard it and was angry;\lit{His anger was kindled} and the \lord's fire burned against them and consumed the outer part of the camp.%%
  \verse{11:2} Then the people complained to Moses, so Moses prayed unto the \lord\ and the fire was quenched.%%
  \verse{11:3} He called the name of the place Taberah for the fire of the \lord\ burned there among them.%%
  \verse{11:4} And the crowd that was in its midst\alt{among them} as well as the children of Israel lusted and cried again, saying, ``Who shall give us flesh to eat?%%
  \verse{11:5} We remember the fish that we would eat in Egypt for no cost;\halot{xxxx}{xxxx what? free fish, but not freely eat. That connotes something different.} and the cucumbers, melons, leeks, onions, and garlic~---%%
  \verse{11:6} now our souls are dried up for there is nothing that we can see besides manna!''%%
  \verse{11:7} The manna was like coriander\ed{Cilantro. However, this probably refers to the fruit (seed), not the leaves; additionally, it probably refers to the taste, not the color.} and was the color of bdellium.\halot{xxxx}{xxxx A tree resin varying from yellow to green, but usually a brown color.}%%
  \verse{11:8} The people roamed and they gathered the manna and they ground with millstones, beat in a mortar, and cooked,\halot{xxxx}{xxxx boiled in a pot} or made round loafs. Its taste was like the taste of a cake with olive oil.%%
  \verse{11:9} When the dew descended on the camp at night the manna would likewise descend upon it.%%
  \verse{11:10} Moses heard the people crying according to their tribes\ed{xxxx each with about five generations}~--- every man at the opening of his tent~--- and the anger of the \lord\ was great. And Moses was displeased.%%
  \verse{11:11} Moses said to the \lord, ``Why have you afflicted your servant? How have I not found favor in your sight in that you lay the burden of all this people on me?%%
  \verse{11:12} Have I conceived all this people? Have I given birth to them? You say to me, `Carry them in your bosom as a nursing father the infant,' to the land which you swore by oath to their fathers.%%
  \verse{11:13} Where am I supposed to get\lit{From where do I have} meat to give to this entire people? Because they bemoan to me, saying, `Give us meat to eat!'%%
  \verse{11:14} I am not able to bear all this people alone because they are too heavy for me.%%
  \verse{11:15} If you deal thusly with me, just kill me~--- please~--- if I have found favor in your eyes, that I may not dwell on my displeasure.''%%
  
  \verse{11:16} The \lord\ said to Moses, ``Gather seventy men for me among the elders of Israel whom you know to be elders and overseers of the people. Take them unto the meeting tent\ed{part of the tabernacle complex} and cause them to stand there with you.%%
  \verse{11:17} I will come down and converse with you there. I will take from the spirit that is upon you\ie{the burden of the people; delegation of priesthood responsibility} and shall put it upon them. They shall bear the load with you~--- you shall not bear it alone.%%
  \verse{11:18} You shall say to the people, `Sancitify yourselves for tomorrow. You shall eat meat because you have whined in the ears of the \lord, saying ``Who will give us meat to eat? Because we had it good in Egypt.'' The \lord\ will give you meat to eat.'%%
  \verse{11:19} You shall not eat meat\understood\ for one day, neither two days, nor five, ten, or twenty days,\ed{A lot of repetition of ``days'' omitted.}%%
  \verse{11:20} but for an entire month until it comes out your nose and becomes nauseating to you. Because you have despised the \lord\ who is among you and cried in front of Him, saying, `Why did we come out of Egypt?'\thinspace''%%
  \verse{11:21} Moses said, ``There are six hundred thousand footman among this people and I am in their midst.\lit{in whose midst I am} And now You say, `I will give them meat which they shall eat for a month'?%%
  \verse{11:22} Shall we slay all the flock and cattle for them? That would suffice. Or should we gather all the fish in the sea for them? That would be enough to satisfy them.''%%
  
  \verse{11:23} But the \lord\ said to Moses, ``Is the hand of the \lord\ shortened? You shall see whether or not My word shall come to pass.''%%
  \verse{11:24} Moses went forth and he spoke the \lord's words unto the people. He gathered up seventy of the elders of the people and caused them to stand surrounding the tent.\ed{probably the tabernacle}%%
  \verse{11:25} The \lord\ went down in the cloud and spoke unto them. He took of the spirit that was on Moses and he conferred it on the seventy elders and they prophesied without ceasing.%%
  \verse{11:26} Two men remained in the camp. One was named Eldad and the other Medad. They were in the register, but did not go to the tent. When the spirit rested upon them they went into the camp and prophesied.%%
  \verse{11:27} And a youth ran and told Moses, saying, ``Eldad and Medad are prophesying in the camp.''%%
  \verse{11:28} Joshua, the son of Nun and the servant of Moses, answered saying, ``My lord, Moses, restrain them!''%%
  \verse{11:29} But Moses said unto him, ``Are you jealous on my behalf? I wish that all the \lord's people were prophets and that the \lord\ would so place His spirit upon them all.''%%
  \verse{11:30} So Moses gathered himself and the elders of Israel%%
  \verse{11:31} and the spirit of the \lord\ went forth and drove quails from the sea who passed over and fell on the camp: a day's journey this way and about a day's journey that way round about the camp: ten cubits above the face of the earth.%%
  \verse{11:32} The people arose all that day, that night, and the morrow to gather quail. The least gathered ten homers\halot{xxxx}{xxxx An ancient Hebrew measure of capacity, equal to ten ephahs or ten baths, and approximately equal to ten or eleven bushels. About eight gallons} and they spread out for themselves a place surrounding the camp.%%
  \verse{11:33} While the flesh was still in between their teeth, before is was consumed, the anger of the \lord\ was kindled against the people and the \lord\ smote the people with a great plague.%%
  \verse{11:34} He called the name of that place Kibroth-hattaavah\lit{burial place of the lusters} because there they buried the people who had lusted.%%
  \verse{11:35} The people traveled from this place to Hazeroth and they stayed there.%%
\end{inparaenum}

  \heading{12}{Miriam and Aaron complain against Moses~--- Miriam is cursed with leprosy~--- Moses prays and Miriam is healed}

\begin{inparaenum}
  \verse{12:1} Miriam and Aaron spoke against Moses because of the Kushite\ed{possibly ``Midianite''} woman whom he married; for he had taken a Kushite woman in marriage.%%
  \verse{12:2} They said, ``Has the \textsc{Lord} only spoken through Moses? Hasn't He also spoken through us?'' And the \textsc{Lord} heard it.%%
  \verse{12:3} Now, the man (Moses) was incredibly unassuming\alt{meek, humble}~--- more than any other man on the face of the earth.\ed{In other words, he is too humble to fight his own battles. Instead, he lets the Lord fight for him.}%%
  
  \verse{12:4} The \textsc{Lord} spoke suddenly unto Moses, Aaron, and Miriam, ``The three of you shall go forth unto the tent of meeting.'' So the three of them went forth.%%
  \verse{12:5} Then the \textsc{Lord} went down in the pillar of a cloud and stood in the door of the tent. He called for Aaron and Miriam and they both came out.%%
  \verse{12:6} He said,%%
  
  \pvbb{``Heed my words:}{if you have a prophet, the \textsc{Lord}}%%
  
  \pvbb{will reveal Himself to him in a vision.}{I will speak unto him in a dream.}%%
  
  \pvab{\vn{12:7} Is it not true that Moses is My servant?}{That of all My house he is faithful?}%%
  
  \pvab{\vn{12:8} I speak to him face to face: visibly. Not in riddles.}{He has seen the form of the \textsc{Lord}.}%%
  
  \pvbb{Why weren't you hesitant}{to speak against my servant Moses?''}%%
  
  {\noindent\verse{12:9} The \textsc{Lord} grew angry with them and He left.}%%
  \verse{12:10} The cloud turned away from the tent\ie{the Tabernacle; see Appendix~\ref{app:tabernacle}} and Miriam was leprous\ed{What is mentioned here as ``leprosy'' is a curable skin condition. See Appendix~\ref{app:psalm-110}.} as snow. Aaron turned to face Miriam and she was leprous.%%
  \verse{12:11} Moses said, ``My \textsc{Lord}, please do not lay this sin that we foolishly sinned upon us.%%
  \verse{12:12} Please, don't let her look like\lit{be like} a stillborn, half of whose skin is eaten when it comes out of its mother's womb.''\ie{a malformed stillborn.}%%
  \verse{12:13} Moses cried unto the \textsc{Lord}, saying, ``Oh God, please heal her!''\ed{\Hebrew{נָא} is used twice to emphasize Moses' plea.}%%
  
  \verse{12:14} So the \textsc{Lord} said to Moses, ``If her father had spat in her face, would she not be in disgrace for seven days? Let her be removed from the rest of the people and after that\understood\ she shall be received.''\ed{We don't know why nothing happened to Aaron. Perhaps it's because he merely gave in to peer pressure (as he was wont to do).}%%
  \verse{12:15} Miriam was confined outside the camp for seven days, and the people did not travel until she was received back into the camp.%%
  \verse{12:16} Then the people up and left Hazeroth and camped in the desert of Paran.%%
\end{inparaenum}

  \heading{13}{Scouts are sent to do reconnaissance in Canaan~--- they perform their duties, return, and give a false report of the land}

\begin{inparaenum}
    \verse{13:1} The \textsc{Lord} spoke to Moses, saying,%%
    \verse{13:2} ``You shall send men to do reconnaissance in the land of Canaan~--- the land that I shall give to the people\lit{descendants} of Israel. You shall send one man~--- just one man~--- for each father's tribe, each of whom is a prince among these men.''\lit{among them}%%
    \verse{13:3} And Moses sent them from the desert of Paran according to the commandment of the \textsc{Lord}. All of the men were chiefs of the tribes of Israel.%%
    \verse{13:4} These are their names: for the tribe of Reuben: Shammua, the son of Zaccur.%%
    \verse{13:5} For the tribe of Simeon: Shaphat, the son of Hori.%%
    \verse{13:6} For the tribe of Judah: Caleb, the son of Jephunneh.%%
    \verse{13:7} For the tribe of Issachar: Igal, the son of Joseph.%%
    \verse{13:8} For the tribe of Ephraim: Hoshea, the son of Nun.%%
    \verse{13:9} For the tribe of Benjamin: Palti, the son of Raphu.%%
    \verse{13:10} For the tribe of Zebulun: Gaddiel, the son of Sodi.%%
    \verse{13:11} For the tribe of Joseph: for, the tribe of Manasseh, Gaddi the son of Susi.%%
    \verse{13:12} For the tribe of Dan: Ammiel, the son of Gemalli.%%
    \verse{13:13} For the tribe of Asher: Sethur, the son of Michael.%%
    \verse{13:14} For the tribe of Naphtali: Nahbi, the son of Vophsi.%%
    \verse{13:15} For the tribe of Gad: Geuel, the son of Machi.%%
    \verse{13:16} Here are the names of the men whom Moses sent to do reconnaissance in the land. Moses renamed Hoshea the son of Nun, Jehoshua.%%
    \verse{13:17} Moses sent them to explore the land of Canaan, and said to them, ``Go up by the south and go up into the mountains.%%
    \verse{13:18} You shall examine the land, what it is; the people who live there, if they're strong or feeble, few or numerous;%%
    \verse{13:19} what the land that they live in is like, if it's good or bad; what cities they live in, if they're camps or fortified cities;%%
    \verse{13:20} what the land is like, if it's fertile or lean; whether or not there are trees in it.\lit{are there trees in it or are there not?} Show some courage and bring back the fruit of the land.'' It was the time of\lit{The days were the days of} the first grapes.%%
    \verse{13:21} And they went up and did reconnaissance from the Zin Desert to Rehob where you enter Hamath.%%
    \verse{13:22} They went up in the south and came into Hebron where were the children of Anak: Ahiman, Sheshai, and Talmai. Hebron was built seven years before Zoan, Egypt.%%
    \verse{13:23} They came to the Wadi Eschol, and there they cut down a branch with a bunch of grapes on it and carried it between two people\understood\ on a pole. They also brought some pomegranates and figs.%%
    \verse{13:24} That place was called the Wadi Eschol because of the bunch of grapes that the people of Israel cut down from there.%%
    \verse{13:25} They returned after forty days of reconnaissance in the land.%%
    \verse{13:26} They came and went in to Moses and Aaron and the entire congregation of the people of Israel (to the Paran Desert and to Kadesh). They brought back word to them\ie{Moses and Aaron} and to the entire congregation and showed them the fruit of the land.%%
    \verse{13:27} They proclaimed to the people and said, ``We came into the land that you sent us to. Additionally, it flows with milk and honey and here is its fruit.%%
    \verse{13:28} However,\halot{xxxx}{limiting, \textbf{only that}} the people who live in the land are strong,\alt{defiant, shameless} the cities are inaccessible and very great. Furthermore, we have seen Anak's descendants there.%%
    \verse{13:29} Amalek lives in the land to the south. The Hittites, Jebusites, and Amorites live in the mountain range. The Canaanites live by the sea and by the Jordan River.''\understood%%
    \verse{13:30} Caleb hushed the people about Moses, and said, ``We will definitely go up and possess it because we are totally capable of doing so.''%%
    \verse{13:31} But the men who went up with him said, ``We aren't able to go up against the people because they're stronger than us.''%%
    \verse{13:32} They discredited\halot{xxxx}{bring bad news of, bring into discredit}\halot{xxxx}{rumor, calumny} the land that they had done reconnaissance in to the people of Israel, saying, ``The land that we passed through to do reconnaissance is a land that devours its inhabitants. Everyone we saw who lives there is huge.%%
    \verse{13:33} We saw giants there, the sons of Anak. We were grasshoppers in our own estimation~---their's, too!''%%
\end{inparaenum}

  \heading{14}{The people of Israel complain to Moses and Aaron, wishing to have died in Egypt~--- Joshua speaks up in favor of the land of Canaan~--- the Lord and Moses discuss the future of Israel~--- the men who gave a false report are reproved by Moses}

\begin{inparaenum}
  \verse{14:1} All the congregation raised their voices and the people cried on that night.%%
  \verse{14:2} All the children of Israel murmured against Moses and Aaron. The entire congregation said to them, ``Would that we had died in the land of Egypt. Or would that we had died in this desert.%%
  \verse{14:3} Why did the \lord\ bring us to this land? To fall by the sword? That our wives and little ones become prey? Wouldn't it be better for us to return to Egypt?''%%
  \verse{14:4} Everyone said to their brother, ``We will choose a leader and return to Egypt.''%%
  \verse{14:5} So Moses and Aaron fell on their faces before the entire congregation of the people of Israel.%%
  \verse{14:6} Joshua, the son of Nun, and Caleb, the son of Jephunneh, from among the explorers of the land, rent their clothes.%%
  \verse{14:7} They said to the whole congregation of the people of Israel, saying, ``The land that we have explored is really good.%%
  \verse{14:8} If the \lord\ is pleased with us and brings us to this land, then He will give it to us: the land that is flowing with milk and honey.%%
  \verse{14:9} Only, don't rebel against the \lord. Do not fear the people of the land for they shall be our food. Their shadow is turned away and the \lord\ is with us. Don't be afraid of them.''%%
  \verse{14:10} All the congregation said to stone them\ie{Joshua and Caleb; possibly Moses and Aaron, too} with stones. But the glory of the \lord\ appeared over the tabernacle\ed{probably over the Holy of Holies} in front of all of the people of Israel.%%
  
  \verse{14:11} The \lord\ said to Moses, ``For how long will this people despise Me? How long will they not believe Me despite all the signs that I have performed in their midst?%%
  \verse{14:12} I will smite them with a pestilence\alt{plague} and I shall disinherit them. I shall make you\ed{singular} a great nation, mightier and more powerful than them.''%%
  \verse{14:13} But Moses said to the \lord, ``The Egyptians shall hear how You have, by Your power, brought up this people out of their midst%%
  \verse{14:14} and they will tell the other inhabitants of the land how they have heard that you, \lord, are in this people's midst, that you have appeared eye to eye, that your cloud stands above them. That you walk before them, a cloud by day, a pillar of fire by night.%%
  \verse{14:15} If you kill this people together,\lit{as one man} then the nations will say that which they've heard of You, saying,%%
  \verse{14:16} `The \lord\ was not able to bring this people to the land that He swore to them because He has destroyed them in the desert.'%%
  \verse{14:17} And now, I plead with You, let the power of the \lord\ be great according as You have previously spoken, saying,%%
  \verse{14:18} `The \lord\ is full of patience, incredibly merciful, forgiving wrongdoing and sin, not clearing those who are guilty, visiting the wrongdoings unto the third and fourth generation.'%%
  \verse{14:19} Please, pardon the transgression of this people according to Your great kindness, even as You have borne this people from Egypt until now.''%%
  \verse{14:20} The \lord\ said, ``I have forgiven according to your word.%%
  \verse{14:21} And yet, even as I live, the whole earth will be filled with the glory of the \lord.%%
  \verse{14:22} All mankind\ed{referring just to the Israelites} who see My glory and signs that I have done in Egypt and the desert, those who have tested Me these ten times and have not hearkened to My voice:%%
  \verse{14:23} surely they shall not see the land that I have sworn to their fathers. Everyone who has provoked me shall not see it.%%
  \verse{14:24} My servant Caleb, because there was another spirit in him and because he fully follows\lit{is fully after} Me: I shall bring him into the land that he has entered. His posterity shall possess it,%%
  \verse{14:25} and the Amalekites and the Canaanites who are dwelling in the valley. On the morrow, turn yourselves unto the desert by the way of the Red Sea.''%%
  
  \verse{14:26} The \lord\ spoke unto Moses and Aaron, saying,%%
  \verse{14:27} ``How long do I have to bear this wicked congregation that murmur against me? I have heard the murmurings of the people of Israel which they whine against me.%%
  \verse{14:28} Say to them: `As I, the \lord\ live, as you have spoken in My ears, I will not do to you.%%
  \verse{14:29} Your corpses shall fall in this desert. All of your numbered ones, even all those numbered from a son twenty and up who has murmured against Me.%%
  \verse{14:30} You will not come into the land that I have raised My hand\ed{raised His arm to the square and sworn to them}, which I have caused you to settle in with the exception of Caleb, the son of Jephunneh, and Joshua, the son of Nun.%%
  \verse{14:31} Your small children~--- who have become the objects of plunder, you said~--- I have brought them in. They know the land that you hate.%%
  \verse{14:32} Your corpses will fall in this desert.%%
  \verse{14:33} Your children shall be shepherds in this desert. I will bear your fornications until your corpses are consumed in the desert.%%
  \verse{14:34} By the number of the days in which you spied out the land (forty days), a day for a year\lit{a day for a year (again)} shall you bear your iniquities~--- forty years~--- and you shall know My displeasure.%% xxxx: This needs some serious work.
  \verse{14:35} I, the \lord, have spoken. Surely I shall do this thing to this whole, evil company xxxx'\thinspace''%%
  \verse{14:36} The men that Moses sent to explore the land, they returned and caused the entire congregation to murmur against Moses by bringing an evil report of the land.%% xxxx: better word than evil? Change in next verse as well.
  \verse{14:37} Even those men who brought an evil report of the land died by the plague before the \lord.%%
  \verse{14:38} Joshua, the son of Nun, and Caleb, the son of Jephunneh, survived all the men who went out to explore the land.%%
  \verse{14:39} Moses spake these words to all the people of Israel and the people mourned exceedingly.%%
  \verse{14:40} They rose early in the morning and went up to the top of the mountain, saying ``Here we are. Because we have sinned, we have come up to the place that the \lord\ told us.''%%
  \verse{14:41} Moses said, ``Why is this? All of you are transgressing the \lord's commands. It shall not succeed.\ed{For two reasons: they had sinned and they would be consumed by the glory of the Lord.}%%
  \verse{14:42} Don't go up! The \lord\ is not in your midst and all of you are not smitten before your enemies.%% xxxx: this verse needs complete retranslation
  \verse{14:43} The Amalekites and the Canaanites are before you there and you will fall on the sword because you have turned from following the \lord\ and therefore\understood\ the \lord\ is not with you.%%
  \verse{14:44} They have the audacity to go up to the base of the mountain. However, the Ark of the Covenant of the \lord\ and Moses did not leave the midst of the camp.%%
  \verse{14:45} The Amalekites and the Canaanites who are dwelling in the mountain shall come down and smite you and beat you even as far as Hormah.''%%
\end{inparaenum}

  \heading{16}{Korah, Dathan, and Abiram rebel against Moses accusing him of being holier-than-thou~--- the Lord, through Moses, challenges them to figure out who the Lord has actually called to lead~--- they fail the test, are consumed in the earth, and those who had rebelled against them are consumed by fire from Heaven}

\begin{inparaenum}
  \verse{16:1} Korah, the son of Izhar, the son of Kohath, the son of Levi, xxxx Dathan and Abiram, the sons of Eliab, xxxx%%
  \verse{16:2} They rose up before Moses with two hundred and fifty men of the sons of Israel, leaders of the congregation, and called men\ed{Other men?} from the congregation.%%
  \verse{16:3} They were assembled against Moses and against Aaron. They said to them,\ie{to Moses and Aaron} ``xxxx-> That's enough!\lit{It is too much for you [to do this].} All of the congregation are holy and the \lord\ is in their midst. Why do you lift yourselves up above the congregation of the \lord?''%%
  \verse{16:4} And Moses and Aaron heard this\understood\ and fell on their faces.\ed{a sign of distress}%%
  \verse{16:5} He said to Korah and to all of his assembly, saying, ``xxxx Let the \lord\ make His known~--- who is holy\ed{xxxx does this refer to the Lord or his holy called people (as in verse~7)?}~--- and bring them near to Him xxxx.%%
  \verse{16:6} Do this: get some censers for yourselves: for Korah and all his assembly.%%
  \verse{16:7} Place among them and set incense before the \lord\ tomorrow and the men the \lord\ chooses, he is the holy one: it is too much for you, sons of Levi.''%%
  \verse{16:8} Moses said to Korah, ``Please here, sons of Levi.%%
  \verse{16:9} It is unimportant to you that the God of Israel has separated you from the congregation of Israel to present you\lit{bring you near} to Him to do the service of the tabernacle of the \lord\ and to stand before the congregation to administer to them.%%
  \verse{16:10} He has presented you and your male relatives, the sons of Levi, with you and you have demanded the priesthood!\ed{\Hebrew{כְּמֹרָה} is non-Levitical priesthood; it's Melchizedek priesthood.}%%
  \verse{16:11} Therefore you and all of your associates who have gathered together against the \lord\ and against Aaron\ed{who has the priesthood they're seeking}, what is He that you murmur against Him?''%%
  \verse{16:12} Moses sent out to call for Dathan and Abiram, sons of Eliab; and they said, ``We're not coming up.%%
  \verse{16:13} It is unimportant that you brought us up out of a land flowing with milk and honey\ed{Referring blasphemously to Egypt instead of Canaan, the real promised land.} to kill us in the desert that you can make yourself a ruler among us.%%
  \verse{16:14} You have not brought us to a land flowing with milk and honey, neither have you given us an inheritance of fields and vineyards.\ed{The plural is idiomatic, though non-literal.} Are you going to put out the eyes of three men? We aren't coming up.''%%
  \verse{16:15} It was very displeasing to Moses, and he said to the \lord, ``xxxx''%%
  \verse{16:16} Moses said to Korah, ``You and all your associates: stand before the \lord\ tomorrow: you and them and Aaron.%%
  \verse{16:17} Each man will take his censor and put incense on them and come before the \lord, each with his censor: 250~censors: you and Aaron with your censors.''%%
  \verse{16:18} Each man took his censor, put fire on it, and put incense on it. And they stood at the opening of the tabernacle\ed{Most likely before it and not actually inside the tabernacle complex.} with Moses and Aaron.%%
  \verse{16:19} And Korah assembled all of his associates who were against them at the entrance to the tabernacle complex\lit{opening of the tent of meeting; cf.\ verse~18} and the glory of the \lord\ was seen by all the congregation.\ed{Think about that! There are about two~million people there seeing this!}%%
  
  \verse{16:20} The \lord\ spoke to Moses and Aaron, saying,%%
  \verse{16:21} ``Separate yourselves from the multitude because I will consume them in a moment.''%%
  \verse{16:22} And they fell on their faces and said, ``O God, the God of the spirits of all flesh: will you be wroth with the whole congregation because\understood\ one man sins?''%%
  
  \verse{16:23} The \lord\ spoke to Moses,\ed{not Aaron here} saying,%%
  \verse{16:24} ``Speak to the company, saying, `Go up from around the tabernacle of Korah, Dathan, and Abiram.'\thinspace''%%
  \verse{16:25} So Moses got up and went to Dathan and Abiram, and the elders of Israel went after him.%%
  \verse{16:26} He spoke to the assembly, saying, ``Please turn from the tents of these wicked men and don't come against anything that they have lest you be consumed because of their sins.''%%
  \verse{16:27} So they went up from round about from the tabernacle\ed{xxxx ?} of Korah, Dathan, and Abiram. And Dathan and Abiram took a stand at the opening of their tents xxxx with their wives, sons, and infants.%%
  \verse{16:28} Moses said, ``You will know by this that the \lord\ has sent me to do all these things\ed{As if the miracles bringing them out of Egypt weren't sufficient.}~--- that they're from my heart.%%
  \verse{16:29} If these people die according to the death of all men\ie{If they die in some normal way}~--- of the visitation of all men is visited on them~--- then the \lord\ has sent me.%%
  \verse{16:30} But if they \lord\ creates something new in that the ground opens her mouth and swallows them and all that they have and they go down alive to Sheol~---\ed{the realm of departed spirits (with no regard to reward or punishment)} you will then know that these men have xxxx the \lord.''%%
  \verse{16:31} And now, when he was done speaking these words to them, the ground beneath them cleaved open\understood%%
  \verse{16:32} and the earth opened its mouth and swallowed them, their posterity, as well as all the men who were for Korah, with all their movable property.%%
  \verse{16:33} They went down alive to Sheol and the earth closed on them and they perished from the midst of the congregation.%%
  \verse{16:34} All of Israel who was round about them fled at the sound of their cry, saying, ``Lest the earth swallow us up.''%%
  \verse{16:35} And fire came down from the \lord\ and consumed the 250~men xxxx who offered the incense to the \lord.%%
\end{inparaenum}

  \heading{17}{The children of Israel murmur, saying that Moses has killed the people of the Lord~--- the Lord sends a plague through the camp~--- Aaron offers atonement and the plague stops~--- in a further test of who the Lord has chosen, the Lord has Aaron place a staff from each tribe in the Holy of Holies, the one that buds shows who was chosen~--- Aaron's staff buds and produces almonds}

\begin{inparaenum}
  \verse{17:1} \ed{In English versions, the first part of this chapter (\vref{Num}{17}{1--15}) is given as \vref{Num}{16}{36--50}.}The \textsc{Lord} spoke to Moses, saying,%%
  \verse{17:2} ``Say to Eleazar the son of Aaron the priest: `xxxx%% the ashes are holy
  \verse{17:3} xxxx? even the censors of the sinners xxxx%%
  \verse{17:4} Eleazar the priest took the brazen censors which those who were burned had brought near and they beat them out as an overlay for the altar.%% xxxx?
  \verse{17:5} This is a memorial for the children of Israel xxxx%% end double quote
  \verse{17:6} On the morrow, the entire congregation of the children of Israel complained against Moses and against Aaron, saying, ``You've killed the people of the \textsc{Lord}.''%%
  \verse{17:7} When the congregation was assembled against Moses and against Aaron before the Tabernacle; and the cloud covered it and the glory of the \textsc{Lord} was seen.%%
  \verse{17:8} And Moses and Aaron came to the xxxx of the Tabernacle.%%
  
  \verse{17:9} The \textsc{Lord} spoke to Moses, saying,%%
  \verse{17:10} ``xxxx''%%
  \verse{17:11} Moses said to Aaron, ``Take the censor and put fire\halot{xxxx}{xxxx} from the altar on it and place incense on it and go quickly to the congregation and make\halot{xxxx}{xxxx} an atonement for them because the wrath of the \textsc{Lord} has come out upon them. The plague has begun.''%%
  \verse{17:12} So Aaron took, just as Moses had said, and ran into the midst of the assembly and now\ed{xxxx yeah?} the plague had begun. xxxx%%
  \verse{17:13} He stood between the dead and the living and the plague was stopped.%%
  \verse{17:14} And 14\thinspace700 people\understood\ died in the plague, apart from those who died from the Korah incident.%%
  \verse{17:15} And Aaron returned to Moses, to the opening of the Tabernacle, and the plague was stopped.%%
  
  \verse{17:16} \ed{In English versions, this is \vref{Num}{17}{1}; at the end of this chapter, verse numbering will be back in sync.}The \textsc{Lord} spoke to Moses, saying,%%
  \verse{17:17} ``Speak to the children of Israel and take for each of them a staff for each tribe: twelve~rods. Each will write their name on their staff.%%
  \verse{17:18} You shall write Aaron's name for the tribe of Levi because one staff is for the head of the father's house.%%
  \verse{17:19} xxxx%% end double quote
  \verse{17:20} The man whoom I choose, his staff will sprout xxxx%%
  \verse{17:21} Moses said to the children of Israel, ``xxxx''%%
  \verse{17:22} Moses placed the staffs before the \textsc{Lord} in the Tabernacle.%%
  \verse{17:23} xxxx a sprout xxxx and it blossomed and produced almonds.%%
  \verse{17:24} Moses brought all the staffs out from the presence of the \textsc{Lord}\ie{from the Holy of Holies} to the children of Israel, and they saw and each took his staff.%%
  
  \verse{17:25} The \textsc{Lord} said to Moses, ``Put Aaron's staff back before the Testimony as a witness to the sons of the rebellion\lit{bitter ones [xxxx yeah?]} and remove their their murmurings from off me xxxx that they do not die.''%% xxxx is this complete? Feels like we're missing something.
  \verse{17:26} Moses did as the \textsc{Lord} commanded him: so did he.%%
  
  \verse{17:27} The children of Israel spoke to Moses, saying, ``We have expired, we've perished,\alt{become lost} we have all become lost.%%
  \verse{17:28} Anyone who comes near to the Tabernacle of the \textsc{Lord} shall die. Have we not been xxxx?''%%
\end{inparaenum}

  \heading{20}{The people enter the Zin Desert~--- they complain about leaving Egypt~--- God commands Moses and Aaron to speak to the rock and water will come out~--- they strike the rock with their staff and water comes out~--- God tells them that because of their disobedience they will not enter the Promised Land~--- Aaron is brought up to Mount Hor and Eleazar takes over his responsibilities~--- Aaron dies and is mourned throughout Israel}

\begin{inparaenum}
  \verse{20:1} The children of Israel~--- the whole congregation~--- entered the Zin Desert in the first month and the people dwelt in Kadesh. Miriam died and was buried there.%%
  \verse{20:2} There was no water for the congregation so they assembled against Moses and Aaron.%%
  \verse{20:3} The people quarreled\alt{disputed}\ed{In other contexts (attested both earlier and later), this verb (\Hebrew{ריב}) means to conduct a legal case or lawsuit against someone.} with Moses\ca{Ms \peshitta\ + \Hebrew{אַהֲרֹן} \Hebrew{וְעִם־} (Ms bis \Hebrew{עַל} pro \Hebrew{עם})}{a manuscript of the Peshitta adds ``and with Aaron'' (another manuscript twice has ``with'' instead of [another form of]``with'')} and spoke, saying, ``If only we had xxxx when our brothers expired\halot{xxxx}{xxxx} before the \lord%%
  \verse{20:4} Why have you brought the congregation of the \lord\ into this desert? To die there, we and our animals?%%
  \verse{20:5} Why have you brought us up from Egypt to bring us into this evil place? No place of seed, fig, or pomegranate. There is no water to drink!''%%
  \verse{20:6} Moses and Aaron entered into the Tabernacle from the presence of the congregation and they fell on their faces and the glory of the \lord\ appeared to them.%%
  
  \verse{20:7} The \lord\ spoke to Moses, saying,%%
  \verse{20:8} ``Take the staff and assemble the congregation, you and your brother Aaron. Speak to the rock before their eyes and it will give them water. You shall bring forth water from the rock and water the congregation and their animals.''%%
  \verse{20:9} So Moses took the staff from the presence of the \lord\ as He had commanded him.%%
  \verse{20:10} Moses and Aaron assembled the people before the face of the rock and Moses said to them, ``Please hear, you rebels, we will bring forth water from this rock for you.''%%
  \verse{20:11} Moses lifted up hand hand and twice smote the rock\alt{boulder}\halot{xxxx}{xxxx} with his staff, and water went forth and the company and their animals drank.%%
  
  \verse{20:12} The \lord\ said to Mose and to Aaron, ``Because you have not believed in Me and have not sanctified Me in this eyes of the children of Israel, you shall therefore not bring in this congregation to the land I shall give them.''%%
  \verse{20:13} These are the waters of Meribah\ed{Hebrew for ``rebellion''} because the children of Israel strove against the \lord, but He xxxx%%
  
  \verse{20:14} Moses sent messengers from Kadesh to the king of Edom: ``Thus says your brother Israel: `You know all the travail xxxx that has befallen us,%%
  \verse{20:15} how our fathers went down from Egypt and we have dwelt many days in Egypt, and the Egyptians did evil to us and to our fathers.%%
  \verse{20:16} We cry to the \lord\ and He hears our voice and sends a messenger and brings us out from Egypt. And now we are in Kadesh, a city at your border.%%
  \verse{20:17} Please, let us pass over your land. We will not pass through a field or a vineyard, neither will we drink well water. We will go on\understood\ the king's highway.\alt{way of the king} We won't deviate to the left or the right until we've passed through your border.'\thinspace''%%
  \verse{20:18} Edom\ed{probably an official representative of the country, an ambassador} said to him, ``You will not pass over me lest I come out to meet you with the sword.''%%
  \verse{20:19} The children of Israel said to him, ``xxxx''%%
  \verse{20:20} Edom said, ``You shall not pass through.'' And Edom came out to confront him with many people and a strong hand.%%
  \verse{20:21} Edom refused to give Israel passage through his border, so Israel turned away from him.%%
  
  \verse{20:22} The entire congregation of Israel journeyed from Kadesh and came to Mount Hor.%%
  \verse{20:23} The \lord\ spoke to Moses and Aaron at Mount Hor, near the border of the land of Edom, saying,%%
  \verse{20:24} ``Aaron is going to be gathered\halot{xxxx}{xxxx} to his people.\ed{euphemism for death} He shall not go into the land which I have given to the children of Israel because you provoked me\ed{Understood: ``\dots and Aaron didn't stand up for Me.'' He tends to lack a backbone.} at the waters of Meribah.%%
  \verse{20:25} Take Aaron and his son Eleazar and have them go up to Mount Hor,\ed{John Taylor (ref xxxx) taught that Aaron had the Melchizedek Priesthood which helps make sense of why they would need to go up into a mountain.}%%
  \verse{20:26} and strip Aaron of his garments and place them upon his son Eleazar xxxx''%%
  \verse{20:27} Moses did as the \lord\ commanded. And they went up to Mount Hor before the eyes of the congregation,%%
  \verse{20:28} and Moses stripped Aaron of his clothes and put them on his son Eleazar. Aaron died there\ca{\missing\ \septuagint*}{[``there''] missing in the Septuagint (textus Graecus originalis)} on the top of the mountain. Moses and Eleazar went down from the mountain.%%
  \verse{20:29} \ca{\bomberg\ hic init cp~21}{this is the beginning of chapter~21 in Bomberg's edition of the Bible}The entire congregation saw that Aaron had died,\alt{expired, breathed his last breath} and the entire house of Israel mourned Aaron for thirty~days.%%
\end{inparaenum}

  \heading{21}{God sends fiery serpents into Israel's camp~--- Moses makes a brass serpent~--- anyone who looks at it lives~--- the Israelites continue to travel through the desert~--- they sing praises to the Lord~--- the Israelites war against Sihon and Bashan}

\begin{inparaenum}
  \verse{21:1} The Canaanite king Arad who was dwelling in the south, heard that Israel was coming by way of Atharim. He fought against Israel and took some of them\understood\ captive.%%
  \verse{21:2} Israel vowed a vow to the \lord, and said, ``If you truly deliver this people into my power, then I will destroy their cities.''%%
  \verse{21:3} The \lord\ heard the voice of Israel and delivered the Canaanites, and they\ie{Israel} destroyed them and their cities. And the name of the place was called Hormah.\ed{Hebrew for ``anathema''}%%
  
  \verse{21:4} They journeyed from Mount Hor via the Red Sea to circumnavigate the land of Edom, and the soul of the people was discouraged in the way.%%
  \verse{21:5} The people spoke to God and to Moses: ``Why have you brought us up from Egypt to die in the desert? Because there is no bread or water, our souls abhor this worthless food.''%%
  \verse{21:6} So the \lord\ sent fiery serpents among the people and they bit the people. And many of the people of Israel died.%%
  \verse{21:7} The people came in to Moses and they said, ``We have sinned because we have spoken against the \lord\ and against you. Pray to the \lord\ that he take the serpents away from us.'' So Moses prayed on behalf of the people.%%
  \verse{21:8} The \lord\ said to Moses, ``Make a fiery serpent for yourself and place it on a pole.\halot{xxxx}{xxxx} And it shall be that everyone that is bitten, when he looks at it he shall live.''%%
  \verse{21:9} So Moses made a bronze serpent and put it on a pole and if a serpent had bitten a man, when he looked at the brazen serpent he lived.%%
  \verse{21:10} The children of Israel set off and camped in Oboth.%%
  \verse{21:11} They journeyed from Oboth and camped in Iye Abarim in the desert that is before Moab in the east.\lit{at the sunrise.}\ca{\sampen\ + \caref{Dt}{2}{9}}{the Samaritan Pentateuch here adds \vref{Deut}{2}{9}}%%
  \verse{21:12} From there they journeyed and encamped in Wadi Zered.\ca{\sampen\ + \caref{Dt}{2}{17--19} (sec Syh in 13)}{the Samaritan Pentateuch here adds \vref{Deut}{2}{17--19} (according to the Syro-hexaplar version in 13)}%%
  \verse{21:13} From there they journeyed and encamped on the other side of Arnon which\ca{dl cf \septuagint\peshitta}{delete [``which''], compare the Septuagint (textus Graecus originalis) and Peshitta} is in the desert that comes out of the Amorite border, because Arnon is on the border of Moab, between Moab and the Amorites.%%
  \verse{21:14} Therefore it's said in the scroll \textit{The Wars of the \lord}:\smallskip%%
  
  \pa ``Waheb%
  \ca{nonn Mss \Hebrew{אתוהב}; \septuagint\ \Greek{τὴν Ζωοβ} = \Hebrew{את־זָהָב} (cf~\caref{Dt}{1}{1}?), \peshitta\ \textit{\v slhbjt'} = \Hebrew{להב}}{several Hebrew manuscript codices have \Hebrew{אתוהב} (xxxx?); the Septuagint has ``gold'' (compare~\vref{Deut}{1}{1}?), and the Peshitta has ``flame/blade''}
  in Suphah, and the Wadi\pa Arnon; \verse{21:15} the spring of the wadi%%
  
  \pa that bends to the dwelling of Ar%
  \ca{\sampen\ \Hebrew{עיר}}{the Samaritan Pentateuch has ``city''}
  \pa and leans on the border of Moab.''%
  \ed{This verse is an absolute mess. I've omitted references to the critical apparatus, but the long and short of it is that most of these words are possibly wrong or missing, especially in the Septuagint.}\smallskip%%
  
  \noindent\verse{21:16} From there\ed{understood: they journeyed} to Beer, it is the well that the \lord\ spoke to Moses: ``Gather the people and I will give them water.''%%
  
  \verse{21:17} Israel then sang this song:\smallskip%%
  
  \pd ``Spring up,%
  \ed{Literally, ``get up'' or ``arise,'' but in the context of a well it would be ``spring up.''}
  O well!\pa Sing to it!%%
  
  \pc \verse{21:18} A well, dug by princes,\pa dug by the nobles of the people,%%
  
  \pd with the scepter\alt{leader, director} and with the support.''\pa From the wilderness\ed{understood: they journeyed} to Mattanah,\smallskip%%
  
  \noindent\verse{21:19} from Mattanah to Nahaliel,\ie{the wadi of God} from Nahaliel to Bamoth,%%
  \verse{21:20} from Bamoth to the valley that's in the field of Moab, to the peak of Pisgah which\understood\ looks down on the desert.%%
  
  \verse{21:21} Israel sent forth messengers unto Sihon king of the Amorites, saying,%%
  \verse{21:22} ``Let me pass through your land. We will not turn aside into a field or a vineyard. We will not drink water from a well. We will go in the king's highway until we pass through your border.''%%
  \verse{21:23} Sihon wouldn't allow Israel to pass through his border; Sihon gathered all his people and went out against\ed{It's easy to mistake this verse as saying ``Sihon gathered all his people and went out to meet Israel,'' but \Hebrew{קרא} has two meanings: \begin{inparaenum}\item to \textbf{call}, \textbf{name}, or \textbf{summon}, and \item (to go to battle) \textbf{against}.\end{inparaenum}} Israel in the desert. And they came to Jahaz and fought against Israel.%%
  \verse{21:24} Israel smote him with the edge of the sword xxxx and took possession of his land from Arnon to Jabbok to the sons of Ammon because the border of the children of Ammon was strong.%%
  \verse{21:25} Israel took all of these cities and Israel dwelt in all the cities of the Amorites in Heshbon\ed{not a city of the Amorites} and their accompanying villages.\lit{and in all its villages.}%%
  \verse{21:26} Because Heshbon is the city of Sihon king of the Amorites and he who fought against the xxxx king of Moab. He took all of his\ie{Moab's} land out of his hand, even unto Arnon.%%
  \verse{21:27} Therefore, those who speak in proverbs\ie{the poets} say:\smallskip%%
  
  \pb Come to Heshbon.\pa Let the city of Sihon be built up and restored%%
  
  \pa \verse{21:28} Because fire has gone out from Heshbon,\pa a flame from the city of Sihon.%%
  
  \pb It has consumed Ar%
  \ca{pc Mss \septuagint(\peshitta) \Hebrew{עַד}; l \Hebrew{עָרֵי}}{a few Hebrew manuscript codices and the Septuagint (and Peshitta) had ``until'' [i.e., ``It has consumed up to Moab,'']; read ``the city of'' [i.e., ``It has consumed the city of Moab,'']}
  of Moab,\pa lords of the high place of Arnon.%%
  
  \pa \verse{21:29} Wo to you, Moab!\pa You have perished, people of Chemosh!%%
  
  \pb He's given his sons who have escaped\pa and his daughters into captivity\pa to Sihon, the king of the Amorites.%%
  
  \pa \verse{21:30} We shoot xxxx\pa Heshbon to Dibon;%
  \ed{This verse is an absolute clustercuss. Just about every word has a variant reading and some phrases can possibly be almost entirely omitted. See further in the following notes from the critical apparatus.}%%
  
  \pb we xxxx\pa xxxx%%
  
  \verse{21:31} Israel dwelt in the land of the Amorites.%%
  \verse{21:32} Moses sent spies to scout out Jaazer and they captured the outlying villages\ed{How many spies did he send?}\alt{suburbs} and they dispossessed the Amorites who were there.%%
  \verse{21:33} They turned and went up the way of Bashan, and Og king of Bashan came out to confront them, he and all of his people, to battle at Edrei.%%
  \verse{21:34} The \lord\ said to Moses, ``Don't be afraid of him because I have put him and all of his people and his land into your power. You shall do to him as you did to Sihon king of the Amorites who dwelt in Heshbon.''%%
  \verse{21:35} They smote him and his sons and all of his people until no remnant of him remained, and they possessed the land.\ed{Earlier it was polite expulsion, now it is total death.}%%
\end{inparaenum}

  \heading{24}{Balaam praises the Lord~--- xxxx}

\begin{inparaenum}
  \verse{24:1} Balaam saw that it was good in the eyes of the \lord\ to bless Israel, so he didn't go as at other times to summon\alt{proclaim, call for, meet, look for} magic curses\alt{bewitchments} and set his face toward the desert.%%
  \verse{24:2} Balaam looked up\lit{lifted his eyes} and saw Israel dwelling by their tribes, and the Spirit of God was upon them.%%
  \verse{24:3} He declaimed his saying,\ed{from \textsc{halot}} and said,\smallskip%%
  
  \pb ``The declaration of Balaam, his son%
  \ca{Seb \Hebrew{בֵּן}}{Sebir has ``son''}
  of Beor,\pa and the declaration of the man whose eyes are closed:%
  \ca{\peshitta\ \textit{(d)glj'} = apertus, \vulgate\ \textit{obturatus} = \Hebrew{שְׂתֻם};\dots.}{the Peshitta has ``open,'' the Vulgate has ``closed'' (which supports the \textsc{halot} definition for \Hebrew{שְׂתֻם});\dots [the rest says that the Septuagint and Targum have ``closed'' in one form or another]}
  \halot{xxxx}{[entry for \Hebrew{שְׁתֻם} has ``\haref{Nu}{24}{3$\cdot$15}: read \Hebrew{שְׂתֻם}. \hadagger'' which entry reads:] \textbf{obstruct} (the way for something)\dots\ with closed eye \haref{Nu}{24}{3$\cdot$15}}
  \ed{Oddly, \textsc{darby} has ``open,'' \textsc{nrsv} has ``clear'' (with footnotes for both ``closed'' and ``open''), \textsc{niv} has ``whose eye sees clearly,'' and \textsc{kjv} has ``open.''}%%
  
  \pa \verse{24:4} the declaration of the one who hears the words of God,%
  \ca{\missing\ \sampen\septuagint*}{[this stich] missing in the Samaritan Pentateuch and Septuagint}%%
  
  \pb who sees the vision of the Almighty,\pa who falls and his eyes%
  \ca{\sampen\ sg}{singular in the Samaritan Pentateuch}
  are opened:%%
  
  \pb \verse{24:5} How great are your tents, Jacob,\pa and\understood%
  \ca{nonn Mss \sampen\peshitta\targum\super{J}\vulgate\ \Hebrew{וּמ׳}}{several Hebrew manuscript codices, the Samaritan Pentateuch, Peshitta, Targum, and Vulgate have ``and''}
  your tabernacles, Israel!%%
  
  \pa \verse{24:6} The wadis have been turned aside\pa like gardens by a river,%%
  
  \pb like aloes%
  \halot{xxxx}{\textbf{aloes} (aromatic wood), \textit{Aloexyllon Agallochum} \& \textit{Aquilaria Agallocha}}
  the \lord\ planted,%
  \ca{\sampen\septuagint\vulgate\ \Hebrew{נָטָה}}{the Samaritan Pentateuch, Septuagint, and Vulgate have ``turned aside''}
  \pa like cedars by the water.%%
  
  \pb \verse{24:7} xxxx\pa xxxx%%
  
  \pb xxxx\pa xxxx%%
  
  \pa \verse{24:8} xxxx\pa xxxx%%
  
  \pc xxxx\pa xxxx\pa xxxx%%
  
  \pa \verse{24:9} xxxx\pa xxxx%%
  
  \pb xxxx\pa xxxx%%
  
  \verse{24:10} Balak was super ticked at Balaam and forcefully clapped his hands together, and Balak said to Balaam, ``I called you to curse my enemies and you've freaking blessed them three times!%%
  \verse{24:11} And now you run away to your place. I said I'd greatly honor you, but the \lord\ has withheld that honor from you.''\alt{held that honor back from you.''}%%
  \verse{24:12} So Balaam said to Balak, ``xxxx%%
  \verse{24:13} `If Balak gave me his house full of gold and silver, I couldn't go against the mouth\halot{xxxx}{xxxx} of the \lord\ to do good or evil from my heart. That which the \lord\ says, I will say'?%%
  \verse{24:14} Now, I'm going back to my people xxxx I will xxxx to you that this people do to your people in the last days.''%% xxxx what?
  \verse{24:15} He took a parable\alt{proverb} and said,%%
  
  \pb ``The declaration of Balaam son of Beor~---\pa a declaration of the man whose eyes are shut:\halot{xxxx}{xxxx}%%
  
  \pa \verse{24:16} He declares:%
  \ca{\missing\ \septuagint*}{[the preceding] missing in the Septuagint (textus Graecus originalis)}
  `Those who obey the sayings of God\pa and know the knowledge of the Most High,%%
  
  \pb who perceive the Almighty's visions\pa and fall down and have their understandings\lit{eyes} opened.%%
  
  \pb \verse{24:17} xxxx\pa xxxx%%
  
  \pb xxxx\pa xxxx%%
  
  \pb xxxx\pa xxxx%%
  
  \pb \verse{24:18} xxxx\pa xxxx%%
  
  \pb xxxx\pa xxxx(?) \verse{24:19} xxxx%%
  
  \pb xxxx%%
  
  \verse{24:20} xxxx%%
  
  \pb xxxx\pa xxxx%%
  
  \verse{24:21} xxxx%%
  
  \pb xxxx\pa xxxx%%
  
  \pa \verse{24:22} xxxx\pa xxxx%%
  
  \verse{24:23} xxxx%%
  
  \pb xxxx\pa xxxx(?) \verse{24:24} xxxx%%
  
  \pb xxxx\pa xxxx%%
  
  \verse{24:25} xxxx%% Is this line poetry? If not, add \smallskip before it.
\end{inparaenum}

  \heading{27}{Zelophehad's daughters risk losing their inheritance because their father had no sons~--- they ask Moses about the situation, he inquires of the Lord, and an answer is given~--- Joshua is ordained}

\begin{inparaenum}
    \verse{27:1} The daughters of Zelophehad (the son of Hepher, the son of Gilead, the son of Machir, the son of Manasseh, from the families of Manasseh the son of Joseph) came\alt{drew} near, and these are the names of his daughters: Mahlah, Noah, Hoglah, Milcah, Tirzah.%%
    \verse{27:2} They stood before Moses and Eleazar the priest and the military\halot{xxxx}{xxxx} leaders and the entire congregation at the door of the Tabernacle,\lit{opening of the tent of meeting} saying,%%
    \verse{27:3} ``Out father died in the wilderness~--- he was not in the midst of the company of Korah who gathered together against the \textsc{Lord}, but from his own sins. And he had no sons.%%
    \verse{27:4} Why should the name of our father be removed from his family? Because he had no sons? Please, give property to us in the midst of our uncles.''\lit{father's brothers.}%%
    \verse{27:5} Moses brought their matter before the \textsc{Lord}.%%
    
    \verse{27:6} The \textsc{Lord} spoke to Moses, saying,%%
    \verse{27:7} ``The daughters of Zelophehad have rightly spoken. You shall surely give to them an inheritance among their uncles by giving their father's inheritance to them.%%
    \verse{27:8} %%
    \verse{27:9} If he doesn't have a daughter, his inheritance shall go to his brethren.\alt{[nearest?] male relative.}%%
    \verse{27:10} If he has no brothers, you shall give it to his uncles.%%
    \verse{27:11} If his father has no brothers, you shall give his inheritance to his next of kin and he shall possess it.'' Thus it was a statue of judgment to the children of Israel, as the \textsc{Lord} had commanded Moses.%%
    
    \verse{27:12} The \textsc{Lord} spoke\ca{\sampen\ ut 6\super{a} $\rightarrow$ 2 Mss \sampen\septuagint\ \Hebrew{וַיְדַבֵּר}}{the Samaritan Pentateuch as verse~6a, which says ``two Hebrew manuscripts, the Samaritan Pentateuch, and Septuagint have `The word of'\thinspace''} to Moses,\ca{Ms \sampen\ + \Hebrew{לאמר} cf 6}{a Hebrew manuscript and the Samaritan Pentateuch add ``saying,'' compare verse~6} ``Go up to this Mount Abarim and see the land that I have given to the children of Israel.%%
    \verse{27:13} You shall see it, and you also shall be gathered to your people like your brother, Aaron,%%
    \verse{27:14} because you've been obstinate against\understood\ My word in the Sin Desert in the strife of the congregation~--- to sanctify me\ca{\peshitta(\vulgate) \textit{wl' qd\v stwnnj} = \Hebrew{אוֺתִי} \Hebrew{וְלֹא־קִדַּשְׁתֶּם} cf \caref{Dt}{32}{51}; \septuagint\ + \Greek{οὐχ ἡγιάσατ με}}{the Peshitta (and Vulgate) have ``and have not sanctified me,'' compare \vref{Deut}{32}{51}; the Septuagint has ``xxxx''} at the waters before their eyes'' (them being\lit{they} the waters of Meribah in Kadesh in the Sin Desert).%%
    
    \verse{27:15} Moses spoke to the \textsc{Lord}, saying,%%
    \verse{27:16} ``O \textsc{Lord}, God of the spirits of all flesh, appoint a man over the congregation%%
    \verse{27:17} who will go out before them, who will come in before them, who will take them out, who will bring them in, so that the congregation of the \textsc{Lord} won't be like sheep without a shepherd.''%%
    \verse{27:18} The \textsc{Lord} said to Moses, ``Take Joshua, son of Nun, a man\ie{Joshua} in whom is the Spirit, and lay your hand on him\halot{xxxx}{(in consecration)}%%
    \verse{27:19} and cause him to stand before Eleazar the priest and \lit{before}the entire congregation, and appoint\alt{command, order, direct} him before their eyes.%%
    \verse{27:20} Put your honor on him so that the entire congregation of the children of Israel will obey him.%%
    \verse{27:21} He shall stand before Eleazar the priest, and you shall consult the Urim for judgment about him before the \textsc{Lord}: at his word they shall go out, at his word they shall come in~--- he and all of the children of Israel with him, even the entire congregation.''%%
    \verse{27:22} Moses did as the \textsc{Lord} had commanded him, and he took Joshua and caused him to stand before Eleazar the priest and the entire congregation,%%
    \verse{27:23} and laid his hands on him and commanded him, as the \textsc{Lord} had spoken\ca{nonn Mss \septuagint\targum\super{J}\vulgate\ \Hebrew{צִוָּה} cf 22a}{several Hebrew manuscripts, the Septuagint, Targum, and Vulgate have ``commanded,'' compare verse~22} through\lit{by the hand of} Moses.\ca{\missing\ \vulgate; nonn Mss \septuagint\targum\super{J} \Hebrew{אֶת־מ׳} cf \super{b}; \sampen\ + \Hebrew{אֵלָיו} \Hebrew{וַיֹּאמֶר} et \caref{Dt}{3}{21 b.\ 22}}{missing in the Vulgate; several Hebrew manuscripts, the Septuagint, and Targum have ``to Moses,'' compare the previous [\textsc{ca}] footnote; the Samaritan Pentateuch adds ``and He said to him'' and \vref{Deut}{3}{21--22}}%% xxxx: 21 b. 22? Is that "21 and the beginning of 22"?
\end{inparaenum}

  \heading{36}{Zelophehad's daughters risk losing their inheritances~--- Moses asks the Lord what they should do and receives an answer~--- they remarry and retain their inheritances}

\begin{inparaenum}
  \verse{36:1} The heads\halot{xxxx}{xxxx} of the clans of the sons of Gilead son of Machir son of Manasseh of the clan of the sons of Joseph spoke before Moses and the princes, the heads\halot{xxxx}{xxxx here, too} of the fathers of the children of Israel,%%
  \verse{36:2} and said, ``The \textsc{Lord} commanded my lord to give an inheritance xxxx%%
  \verse{36:3} xxxx they are married to one of the sons xxxx%%
  \verse{36:4} If it is the jubilee\halot{xxxx}{xxxx} to the sons of Israel xxxx''%%
  \verse{36:5} Moses commanded the children of Israel according to the command\alt{word} of the \textsc{Lord}, saying, ``The tribe of Joseph's plea is just.\lit{The tribe of the children of Joseph has said correctly.}%%
  \verse{36:6} This is the matter that the \textsc{Lord} commanded concerning the daughters of Zelophehad, saying, `Let them marry who they deem good,\lit{those who are good in their eyes, let them be wives to you} but only within the family of their father's tribe,%%
  \verse{36:7} lest the children of Israel's inheritance pass from tribe to tribe for every man\ed{xxxx person?} of the children of Israel shall keep for himself the inheritance of the tribe of his fathers.'%%
  \verse{36:8} Every daughter who possesses an inheritance among the tries of the children of Israel shall be married to one of the family of the tribes of her father that every one of the children of Israel may possess the inheritance of his fathers.%%
  \verse{36:9} The inheritance shall not pass from one tribe to another tribe, because each of the tribes of the children of Israel shall keep his inheritance.''%%
  \verse{36:10} The daughters of Zelophehad did as the \textsc{Lord} commanded Moses.%%
  \verse{36:11} Mahlah, Tirzah, Hoglah, Milcah, and Noah (Zelophehad's daughters) married\lit{became wives to} their cousins,\lit{the sons of their father's brothers}%%
  \verse{36:12} men from the clans of the sons of Manasseh, son of Joseph, and they became wives and their inheritance remained in the tribe of the clan of their father.%%
  \verse{36:13} These are the commandments and the judgments that the \textsc{Lord} commanded by the hand of Moses concerning the sons of Israel in the plains of Moab (by Jordan of Jericho).%%
\end{inparaenum}

  
  \book{Deuteronomy}{\Hebrew{דברים}}
  \heading{5}{Moses restates the Decalogue and lends historical context to it~--- commands Israel to follow the Lord in all they do}

\begin{inparaenum}
  \verse{5:1} Moses proclaimed to all of Israel, and said to them, ``Listen, Israel, to the statutes and legal decisions that I am speaking into your ears today. Learn them. Observe\ie{to do} them.%%
  \verse{5:2} The \lord\ our God made a covenant with us in Horeb.%%
  \verse{5:3} The \lord\ did not make this covenant with our fathers, but with us: all of us who are here and alive today.%%
  \verse{5:4} On the mountain, the \lord\ spoke face to face with you from the midst of the fire.%%
  \verse{5:5} I stood between you and the \lord\ at that time in order to declare the word of the \lord\ to you because you were afraid of\lit{by reason of} the fire and didn't go up into the mountain, where he said,\lit{saying. This rendering is meant to clarify that it was the Lord speaking, not Moses or Israel.}%%
  
  \verse{5:6} \textsc{Preface.} `I am the \lord\ your God. I brought you out from the land of Egypt~--- from the house of captivity.%%
  \verse{5:7} \textsc{i.} There will be no other gods in preference to me.%%
  \verse{5:8} \textsc{ii.} Never make for yourselves graven images, or any image,\alt{idol, picture, image, or any likeness} in the heavens above, neither that is in the earth nor the waters underneath the earth.%%
  \verse{5:9} Never bow down to them or worship\alt{serve} them because I, the \lord\ your God, am a jealous God: I visit the punishment\alt{iniquity} unto the third and fourth generation of my haters%%
  \verse{5:10} Showing mercy unto thousands who love me and obey my commandments.%%
  
  \verse{5:11} \textsc{iii.} Never take the name of the \lord\ thy God with vain purpose for the \lord\ shall not hold him guiltless who uses His name to vain purpose.%%
  
  \verse{5:12} \textsc{iv.} Keep the Sabbath day holy, just as the \lord\ your God has commanded you.%%
  \verse{5:13} You shall work and perform all your responsibilities in six days,%%
  \verse{5:14} but the seventh day is the Sabbath of the \lord\ your God: you shouldn't work, neither your son, daughter, handmaid, ox, donkey, any of your cattle, nor an outsider who's living in Israel. This so that your manservant and maidservant may rest just like you.%%
  \verse{5:15} Remember when you were slaves in the land of Egypt and the \lord\ your God brought you from there with a strong hand and an outstretched arm.\ed{So that He can show the way; fight our battles} Therefore the \lord\ your God has commanded you to keep the Sabbath day.%%
  
  \verse{5:16} \textsc{v.} Take your mom and dad seriously just as the \lord\ your God has commanded you. This that your days may be lengthened and in order that it may be well for you upon the land which the \lord\ your God is going to give you.%%
  
  \verse{5:17} \textsc{vi.} Don't murder.%%
  
  \verse{5:18} \textsc{vii.} Don't commit adultery.%%
  
  \verse{5:19} \textsc{viii.} And don't steal.%%
  
  \verse{5:20} \textsc{ix.} Never give a vain witness against your neighbor.%%
  
  \verse{5:21} \textsc{x.} Don't desire your neighbor's wife
  
  or his house, field, bondman, handmaid, ox, donkey, or anything that's his.'%%
  
  \verse{5:22} The \lord\ spoke these words on the mountain to the entire congregation. With a great voice from out of the midst of the fire, the cloud, the gloom, and he added no more. He wrote it on two stone slabs and gave them to me.%%
  \verse{5:23} When you heard the voice from out of\alt{the midst of} the darkness and the mountain burned with fire, all of the heads of your tribes and your elders came to me\ie{Moses, not the Lord}%%
  \verse{5:24} and said, `The \lord\ our God showed us His magnificence and greatness. We have heard his voice from the midst of the fire today.\ed{It's ambiguous if ``today'' should go with this sentence or the next. It works idiomatically with either. However, there does remain a theological implication in placing it with the next sentence, viz.: did they not know this before?} We have seen that God speaks with man~--- that He lives.%%
  \verse{5:25} Now, why should we die? Because this great fire consumes us? Additionally, if we again hear the voice of the \lord\ our God, we shall die.%%
  \verse{5:26} Because who among all living\lit{flesh} has heard the \lord's voice speaking from out of the fire as we have and lived?%%
  \verse{5:27} You,\ie{Moses} come here and hear everything that the \lord\ God is going to say. You shall tell\lit{speak to} us everything that the \lord\ our God has told\lit{spoken to} you. We will hear it and obey.'%%
  \verse{5:28} The \lord\ has heard your voice\lit{the sound/voice of your speaking/words} when you spoke to me, and the \lord\ said, `I have heard the sound of this people's words that they've spoken to you. Everything they've spoken has been well put.\lit{spoken}%%
  \verse{5:29} Who shall give?\halot{xxxx}{xxxx} There was a heart in them to fear Me and to keep all of My commandments forever\lit{all of the days} so that it's good for them and their children forever.'%% xxxx this beginning (->in them) needs complete reworking.
  \verse{5:30} Go. Say to them, `Return to your tents.'%%
  \verse{5:31} However, you shall remain\ed{Understood through context. Without, it is literally ``you here,'' but it's the Lord telling Moses to remain where he is.} here and shall certainly stand\ed{Repeated} by Me as I tell you all of the commandments and statutes and legal decisions that you shall teach them; they shall observe these in the land that I shall give them to possess:%%
\end{inparaenum}
  % xxxx work on this formatting. Bulleted lists should be allowed. Issue reported. Verses~32 and~33 should be bulleted.
\begin{enumerate}
  \setcounter{enumi}{31}
  \verse{5:32} Observe to do as the \lord\ your God commands you; don't fall away to the right or to the left.%%
  \verse{5:33} In all His ways that the \lord\ your God commands you to walk in order that you live, these are good for you and shall prolong your\lit{the; understood as \textit{your}.} days in the land which you shall possess.%%
\end{enumerate}

  \heading{6}{Moses promises Israel that if they keep the commandments they will be blessed}

\begin{inparaenum}
  \verse{6:1} ``These are the commandments, statutes, and legal decisions that the \textsc{Lord} your God has commanded me\understood\ to teach you, that you apply them in the land thither that you will pass through and\lit{to} possess.%%
  \verse{6:2} Fear the \textsc{Lord} your God so that you keep all of His statutes and commandments that I have command you and your children and your children's children. Keep them\ed{Repeated} all the days of your life so that your days may be prolonged.%%
  \verse{6:3} You, Israel, shall hear and observe to do that which is good for you in order that you may greatly multiply in the land flowing with milk and honey, even as your fathers' God, the \textsc{Lord}, has told you.%%
  
  \verse{6:4} Hear, O Israel! I, the \textsc{Lord} am God: there is only one \textsc{Lord}\lit{the \textsc{Lord} is one}.%%
  \verse{6:5} Thou shalt love the \textsc{Lord} thy God with all thy heart, with all thy soul, and with all thy being.\alt{strength.}%%
  \verse{6:6} These are the words\alt{commands} that I am going to command you in your heart.%%
  \verse{6:7} You shall repeat them to your children. You shall talk about them when you sit in your home, when you go in the road, or when you get up.%%
  \verse{6:8} You shall bind them as a sign upon your hand. They shall be as frontlets.%%
  \verse{6:9} Ye shall write them upon the lintel posts of your house and gate\ie{any official entrance to your house, even doors to separate rooms: these words shall be upon them.}.%%
  
  \verse{6:10} It shall be, when the \textsc{Lord} your God brings you into the land that He swore to your fathers~--- to Abraham, Isaac, and Jacob~--- that He would give to you, the great and good cities that you didn't have to\understood\ build.%%
  \verse{6:11} Houses that you didn't fill, but which are full of everything good; hewed cisterns that you did not quarry; vineyards and olive orchards that you did not plant; of these\understood\ you have eaten and been satiated.%%
  \verse{6:12} Be careful lest you forget the \textsc{Lord} who brought you out from the land of Egypt, from the house of servitude.%%
  \verse{6:13} Reverence the \textsc{Lord} your God, serve Him, swear by His name.%%
  \verse{6:14} You shall not follow\alt{adhere to} other gods, the gods of the people who are round about you,%%
  \verse{6:15} because the \textsc{Lord} your God, who is\understood\ in your midst, is an envious God. Don't do this\understood\ lest the \textsc{Lord} your God becomes indignant with you and exterminates you from off the face of the land.%%
  
  \verse{6:16} Don't put the \textsc{Lord} your God to the test as you did\lit{put Him to the test} in Massah.%%
  \verse{6:17} You shall constantly keep the commandments, testimonies,\ed{From \textsc{halot}, seems to be more in a legal sense.} and statutes of the \textsc{Lord} your God that He has commanded you.%%
  \verse{6:18} You shall do what the \textsc{Lord} sees as\lit{in His opinion is} fitting and good so that it's good to you. That you may go and possess the good land that the \textsc{Lord} has sworn to your fathers,%%
  \verse{6:19} He shall drive out all of your enemies from before you, even as the \textsc{Lord} has spoken.%%
  
  \verse{6:20} In the future, when your son asks you,\lit{saying} `What are the testimonies, statutes, and judgments that the \textsc{Lord} our God has commanded you?'%%
  \verse{6:21} You shall say to your son, `We were servants to Pharaoh in Egypt and the \textsc{Lord} brought us out of Egypt with a powerful\alt{strong, firm, violent, severe. All of which fit.} hand.'%%
  \verse{6:22} The \textsc{Lord} gave great and sad signs\alt{distinguishing marks} and omens,\alt{wonders} while we watched,\lit{in our eyes, in our view} upon Egypt and upon Pharaoh and upon all of his household.%%
  \verse{6:23} He brought us out from there in order to bring us in so that\understood\ to give to us the land that He swore to our fathers.%%
  \verse{6:24} The \textsc{Lord} commanded us to keep all his statutes, to reverence the \textsc{Lord} our God. This\understood\ that it would be good for us all of our days, that He would keep us alive, even as He has\understood\ at this time.%%
  \verse{6:25} It shall be righteousness to us if we observe to keep all of these commandments before the \textsc{Lord} our God, even as He has commanded us.%%
\end{inparaenum}

  \heading{25}{Rules about beating transgressors~--- Levirite law given~--- fair measurements to be used~--- commandments regarding how and why to deal with Amalek enumerated}

\begin{inparaenum}
  \verse{25:1} ``If there's a lawsuit between men and they come to\alt{decide upon, resort to} judgment, then they shall judge and pronounce the righteous not guilty and pronounce the transgressor guilty.%%
  \verse{25:2} If the transgressor is supposed to be beaten, the judge shall cause him to fall down and someone shall hit him in the judge's presence, sufficient by number for his transgression.%%
  \verse{25:3} He shall be smitten no more than forty times. They shall not add to this, otherwise, if they hit him more than these many strikes, then your brother shall be of low esteem in your eyes.%%
  
  \verse{25:4} You shouldn't muzzle an ox while it's threshing.\alt{Seems like something is missing here. This is very \textit{non sequitur}, although it is in a separate section than the surrounding verses.}%%
  
  \verse{25:5} When brothers dwell together\ed{This does not necessarily denote two men with the same parents, but rather male relatives. Same goes for whenever sister-in-law is said: it is simply a female relative.} and one of them dies and he has no son\lit{there is not a son to him}, his wife shall not go unto foreigners\lit{outside} to find a husband, but her brother-in-law will come to her and he shall take her unto himself as a wife. And thus\ed{``Thus'' is not in the verse, but helps the flow.} shall he perform the duty of a brother-in-law.%%
  \verse{25:6} He shall raise up for the deceased\lit{over the name of the deceased} the oldest child that she shall bear that his\ie{the deceased} name be not erased from Israel.\lit{The oldest child that she shall bear, he shall raise up for the deceased that his name be not erased from Israel.}%%
  \verse{25:7} And if the man is not inclined\alt{pleased} to take his sister-in-law then let her\lit{his sister-in-law} go up to the gate to the elders and say, ``My brother-in-law hath refused to raise up into his brother a name in Israel.''%%
  \verse{25:8} The elders of the city will call for him and he will stand and he shall say, ``I do not desire to take her.''%%
  \verse{25:9} Then his sister-in-law will approach him in the eyes\alt{presence} of the elders and she will remove his sandal from off his foot. And she shall spit in his face and say,\lit{and she will answer and say} ``So shall it be done\lit{it is done} to the man who will not build up the house of his brother.''%%
  \verse{25:10} And his name will be called \textit{The house of the man whose sandal was removed}.%%
  
  \verse{25:11} When men quarrel together\lit{one with another} and one of their wives comes near to deliver her husband from the guy who's hitting him,\lit{from the hand of his smiter} and she's stretched out her hand and grabbed\alt{taken hold of, seized} his shame,\ed{\textsc{halot} doesn't seem to have a rendering which means \textit{privates} or \textit{private parts}. The verb form, \Hebrew{בושׁ}, means \textit{to be ashamed} from which we surmise that the noun form refers to something of which a man would be ashamed, therefore his genitals.}%%
  \verse{25:12} then you shall cut off her hand. You shall not pity her.\lit{You shall not be troubled about (or look compassionately) in your opinion (understood: on her).}%%
  
  \verse{25:13} There shall not be any stones in your bag, even a great or a small stone.%%
  
  \verse{25:14} There shall not be an ephah, great or small, in your house.%%
  \verse{25:15} You shall have a complete and just stone. You shall have a complete and just ephah. This\understood\ in order that your days may be lengthened in the land that the \lord\ your God shall give you.%%
  \verse{25:16} Because anyone doing these things~--- anyone doing wickedness~--- is abhorred by the \lord\ your God.%%
  
  \verse{25:17} Remember what Amalek did to you in the road on your way out of Egypt,%%
  \verse{25:18} how he happened upon you in the way and seized and destroyed the rear-guard (all those who are unfit to travel\alt{stragglers} who were behind you); and you were weary and tired.\alt{extremely weary. Both \Hebrew{עָיֵף} and \Hebrew{יָגֵעַ} mean weary, but the latter also means tired.} But he did not fear God.%%
  \verse{25:19} Since the \lord\ your God has given you peace and quiet\ed{\Hebrew{נוח} means to give rest or quiet.} from all the enemies which surround you in the land that the \lord\ your God has given to you~--- a hereditary possession for you\understood\ to possess~--- that\understood\ the remembrance of Amalek shall be wiped out from under the heavens. You shall not forget it.%%
\end{inparaenum}

  
  \book{Joshua}{\Hebrew{יהושע}}
  \heading{3}{Joshua teaches the Israelites how to follow the Ark~--- the people commanded to be pure~--- priests and Levites commanded to enter the Jordan River~--- the river is stopped in its course}

\begin{inparaenum}
  \verse{3:1} Joshua got up early in the morning. He, and all the children of Israel, journeyed from Shittim until they came to Jordan where\understood\ they passed the night\lit{they passed the night there} before passing through.%%
  \verse{3:2} At the end of three days, the officers\alt{record-keepers} passed through into the midst of the camp%%
  \verse{3:3} and commanded the people, saying, ``When you see the Ark of the Covenant of the \lord\ your God with the priests and Levites carrying it, and they journey from where you are:\lit{your place} you shall follow them.%%
  \verse{3:4} However, there shall be a distance between you and \lit{between}them~--- by measurement, about 2\thinspace000~cu\-bits.\ed{A little over half a mile, 0.568~mi (0.914~km)). At a good walking pace, that's about ten minutes following distance.} In order for you to know the way that you're supposed to go~--- because you've never passed over this way before now~--- don't come near it.''%%
  
  \verse{3:5} Joshua said to the people, ``Sanctify yourselves because the \lord\ is going to do something extraordinary in your midst tomorrow.''\alt{in the future, in time to come. However, we know through context that it is literally the next day.}%%
  \verse{3:6} Joshua spoke to the priests, saying ``Carry the Ark of the Covenant, pass over in front of the people.'' So they carried the Ark of the Covenant and walked before the people.%%
  
  \verse{3:7} The \lord\ said to Joshua, ``Today\lit{this day} I will begin to make you great in the opinion of all of Israel, so that they will know that I am with you even as I was with Moses.\ed{Moses actually comes first, but this is more idiomatic in English.}%%
  \verse{3:8} You shall command the priests to carry the Ark of the Covenant, saying, `When you come to the edge of the waters of Jordan, stand in the Jordan.'\thinspace''%%
  
  \verse{3:9} Joshua said to the children of Israel, ``Come here and hear the words of the \lord\ your God.''%%
  \verse{3:10} And Joshua said, ``By this you know that the living God is in your midst. He will certainly take possession of the Canaanites', Hittites', Perizzites', Girgashites', Amorites', and Jebusites' property from before you.%%
  \verse{3:11} Look, the Ark of the Covenant of the Lord of the whole earth is passing before you into the Jordan.%%
  \verse{3:12} Now, take\lit{for you} twelve men from the tribes of Israel, but only one man\lit{one man one man} from each tribe.%%
  \verse{3:13} When the soles of the priests' feet who bear the Ark of the \lord~--- the Lord of the whole earth~--- when they\understood\ rest in the waters of Jordan, the waters of Jordan shall be cut off. The waters that come down from above\ed{\Hebrew{הַיַּרְדֵּן} comes from a root meaning ``waters going from a higher to a lower place.''} shall be dammed up.''\lit{stand in one heap.}%%
  \verse{3:14} During the people's journey from their tents to pass over the Jordan, the priests bore the Ark of the Covenant before the people.%%
  \verse{3:15} And when those bearing the Ark came to the Jordan, the feet of the priests bearing the Ark dipped into the edge of the waters (the Jordan being full over all its banks all the days of harvest)%%
  \verse{3:16} and the flowing waters stood and rose up in a dam, far above Adam city (which is beside Zaretan); and those going down by the sea of the plain~--- the Salt Sea~--- were completely cut off. And the people passed through to Jericho.%%
  \verse{3:17} And the priests bearing the Ark of the Covenant of the \lord\ stood established on the dry ground in the midst of the Jordan. And all Israel passed over on dry ground, even until the whole nation had completely passed over the Jordan.%%
\end{inparaenum}

  \heading{24}{The Lord, through Joshua, recounts Israelite history to the people~--- He tells the people to make a choice: serve Him or the other gods~--- the people choose to serve the Lord~--- Joshua challenged them and they stand true~--- Joshua and Eleazar die, they and Joseph's bones are buried}

\begin{inparaenum}
  \verse{24:1} Joshua gathered all the tribes of Israel in Shechem and called for the elders of Israel, their chiefs, judges, and officers,\halot{xxxx}{\textbf{record-keepers}} and presented them before God.%%
  \verse{24:2} Joshua said to all the people, ``Thus says the \lord, God of Israel: `Your fathers~--- Terah, Abraham's and Nahor's father~--- used to\lit{of old} live on the other side of the river\ed{the trans-Euphrates} and they served other gods.%%
  \verse{24:3} I took your father Abraham from the trans-Euphrates and caused him to go through the land of Canaan, to multiply his posterity, and I have him Isaac.%%
  \verse{24:4} I gave to Isaac, Jacob, and Esau. And I give Mount Seir to Esau to possess.\lit{to possess it.} And Jacob and his posterity have gone down to Egypt.\lit{have gone down Egypt.}\ca{nonn Mss \Hebrew{יְמָה}\hspace*{0em}---}{several Hebrew manuscript codices have ``Egypt-ward'' [``to Egypt'']}%%
  \verse{24:5} I sent Moses and Aaron\ca{\missing\ \septuagint}{[``and Aaron''] missing in the Septuagint} and I struck\ed{understood: with a plague} Egypt as I did in their midst, and afterwards brought you out.%%
  \verse{24:6} I brought out\
  \ca{\missing\ \septuagint}{[``you'' in verse~15 and ``I brought out'' in verse~16 are] missing in the Septuagint}
  your fathers\
  \ca{\missing\ \septuagint\super{Mss}}{[``I brought out your fathers''] missing in the Septuagint}
  from Egypt and you came to the sea and the Egyptians pursued after your fathers with chariots and riders to the Red Sea.%%
  \verse{24:7} They cried to the \lord\ and He put darkness between you and the Egyptians, and the sea came upon them and covered them; and your eyes saw what I did in Egypt. And you lived in the desert for many days.%%
  \verse{24:8} I will bring you into the land of the Amorites who live on the other side of the Jordan River,\understood\ and they'll fight with you. I will give them into your power\lit{hand} and you will take possession of their land, and I will exterminate them from before you.%%
  \verse{24:9} Balak son of Zippor, king of Moab, got up and fought against Israel. He sent and called for Balaam son of Beor to curse you.%%
  \verse{24:10} I would not listen to Balaam and he greatly blessed you, and I delivered you out of his power.%%
  \verse{24:11} You crossed the Jordan River\understood\ and came to Jericho, and the citizens of Jericho~--- the Amorites, Perizzites, Canaanites, Hittites, Girgashites, and Jebusites\ca{prb add}{[``Amorites\dots\ Jebusites'' was] probably added}~--- fought against you, and I gave them into your hand.%%
  \verse{24:12} I sent depression\halot{xxxx}{\textbf{depression}, \textbf{discouragement} \haref{Ex}{23}{28} \haref{Dt}{7}{20} \haref{Jos}{24}{12}}\ed{other translations have ``hornet''} before you and it cast them (two\ca{\septuagint\ 12 = \Hebrew{עָשָׂר} \Hebrew{שׁנים}}{the Septuagint has 12} Amorite kings) out from before you, not by your sword neither by your\ca{\peshitta\ suff pl}{[both ``yours'' here in] the Peshitta have plural endings} bow.%%
  \verse{24:13} I've given you the land that you didn't exert yourself for and cities you didn't build, and you live in them: vineyards and olive orchards that you didn't plant, and you eat them.\understood%%
  \verse{24:14} And now, fear the \lord\ and serve Him in honesty\alt{sincerity} and truth, and turn away from the gods whom your fathers served in the trans-Euphrates and Egypt, and serve the \lord.%%
  \verse{24:15} If it seems wrong to you to serve the \lord, then choose right now whom you will serve: whether the gods of the trans-Euphrates whom your fathers served, or the gods of the Amorites in whose land you dwell; but my family and I will serve the \lord.''%%
  
  \verse{24:16} The people\ca{\peshitta\ + \textit{k(w)lh} = \Hebrew{כָּל־}}{the Peshitta adds ``all''} answered and said, ``Be it far from us\lit{To the profane} to abandon the \lord\ to serve other gods%%
  \verse{24:17} because \textsc{Jehovah} is our God: He has brought us and our fathers\ca{\missing\ \peshitta}{[``and our fathers''] missing in the Peshitta} out of the land of Egypt~--- out of the house of bondage~--- who's done these great works in our eyes,\ca{\missing\ \septuagint*}{[``out of the house\dots\ our eyes''] missing in the Septuagint} who protects us in the path\alt{road, way} we've walked and among the people in whose midst we've passed.%%
  \verse{24:18} The \lord\ drove out all the people and the Amorites who lived in the land from before us. Also, we serve the \lord\ because He is our God.''%%
  
  \verse{24:19} Joshua said to all the people: ``You won't be able to serve the \lord\ because He is a holy God, a jealous\halot{xxxx}{1.\ \textbf{passion}, \textbf{ardor}: (of man for God)\dots\ \textbf{passion} (of sexual attraction)\dots\ with \textit{min}, \textbf{envy} [not the case here]\dots\ Y.'s jealous love\dots\ 2.\ \textbf{jealousy}} God:\ca{\septuagint\ \Greek{καί}, \septuagint\super{\textit{O}}(\peshitta) \Greek{καὶ θεός}}{the Septuagint has ``and,'' the Septuagint (and Peshitta) have ``and God''} He won't pardon your transgressions and sins.\ed{see further in Appendix~\ref{app:sins-transgressions}}%%
  \verse{24:20} If you forsake\alt{abandon} the \lord\ and serve foreign gods, He will return and do you evil; He will consume you, even\understood\ after He's done you good.''%%
  \verse{24:21} But the people said to Joshua, ``No! We will serve the \lord!''%%
  \verse{24:22} So Joshua said to the people, ``You are witnesses against yourselves that you've chosen\lit{for yourselves} the \lord, to serve Him.'' And the people said, ``We are witnesses!''%%
  \verse{24:23} ``So now, put away the foreign gods that are among you and turn your hearts to the \lord, God of Israel.''%%
  \verse{24:24} They,\ca{pc Mss \Hebrew{—ֶר} cf \septuagint\peshitta\super{W}\vulgate}{a few Hebrew manuscript codices have ``It said'' [Hebrew handles mass nouns like British English], compare the Septuagint, Peshitta, and Vulgate} the people, said to Joshua, ``We will serve the \lord\ our God\ca{\missing\ 2 Mss \septuagint*}{[``our God''] missing in two Hebrew manuscript codices and the Septuagint} and hearken to His voice.''%%
  \verse{24:25} Joshua made a covenant with the people that day, and set a statute and judgment in Shechem.%%
  \verse{24:26} Joshua wrote these words in the book of the law\ed{``[B]ook'' and ``law'' are lowercase because I'm not sure if this is the proper name of the book.} of God, and he took a great stone and put it there under the might tree\halot{xxxx}{unspecified, like [definition] II \Hebrew{אָיִל} [another word meaning ``not a specific tree''], \textbf{mighty tree}, often with cultic significance} of the \lord.%%
  
  \verse{24:27} Joshua spoke to all\ca{\missing\ \septuagint* cf 16.19.21}{missing in the Septuagint, compare verses~16, 19, and 21} the people: ``This stone is a witness against us because it's heard all the words of the \lord\ that He's spoken to us\ca{\septuagint\ + \Greek{σήμερον} = \Hebrew{הַיּום}}{the Septuagint adds ``today''}~--- it will be a witness against you lest you deny God.''\ca{\septuagint(\peshitta\vulgate) \Greek{κυρίῳ τῷ θεῷ}\dots\ = \Hebrew{אֱ׳} \Hebrew{בַּיהוה}}{the Septuagint (and Peshitta and Vulgate) have ``the \lord\ God.''}%%
  \verse{24:28} Joshua sent the people away, everyone to their inheritances.%%
  
  \verse{24:29} After these things, Joshua the son of Nun, the servant of the \lord, died being\lit{a son of} 110~years old.%%
  \verse{24:30} They buried him in the border of his inheritance, in Timnath-serah which is\ca{\missing\ nonn Mss Edd \septuagint*\targum\super{Mss} et\caref{}{19}{50} \caref{Jdc}{2}{9}}{missing in several Hebrew manuscript codices, Kennicott's manuscripts, the Septuagint, and the Targum, and in \vref{Josh}{19}{50} and \vref{Judges}{2}{9}} in Mount Ephraim, north of Mount Gaash.%%
  \verse{24:31} \ca{\septuagint\ tr post 28}{the Septuagint transposes [this verse] after verse~28}Israel served the \lord\ all of Joshua's days and all the elders' days who outlived Joshua, who knew all the works of the \lord\ which He'd done for Israel.%%
  \verse{24:32} They buried Joseph's bones (that the children of Israel had brought up from Egypt) in Shechem in the plot of ground\alt{field}\halot{xxxx}{portion of common field of a locality distributed to an individual by lot, \textbf{piece of land}, \textbf{plot of ground}} that Jacob had bought\alt{acquired} for 100~kesitah\halot{xxxx}{old measure of weight, amount unknown} from the children of Hamor, Shechem's father, and it became an inheritance for Joseph's posterity.%%
  \verse{24:33} Eleazar\ca{\peshitta\ + \textit{khn'} = \Hebrew{הַכֹּהֵן}}{the Peshitta adds ``the priest''} son of Aaron\ca{2 Mss \peshitta\arabica\ + \Hebrew{הַכֹּהֵן}}{two Hebrew manuscript codices, the Peshitta, and versio Arabica add ```the priest''} died, and they buried him in the hill\ed{easily not ``Gibeah'' in this context} of his son Phinehas which had been given to him in Mount Ephraim.%%
\end{inparaenum}

  
  \book{Judges}{\Hebrew{שופטים}}
  \heading{7}{Gideon brings the Israelites to the River Jordan~--- through revelation, Gideon dismisses men until only three hundred remain~--- they surround the Midianite camp, wreak havoc, and are miraculously protected through combat}

\begin{inparaenum}
  \verse{7:1} Jerubbaal (formerly\lit{who is} Gideon) and all the people who were with him rose early. They set up\alt{pitched} a military camp by Harod's well. The Midianite camp was on the north in the valley by the hill Moreh.%%
  \verse{7:2} And the \textsc{Lord} said unto Gideon, ``The people with thee are too great\alt{many} for me to just give the Midianites into their hand, lest Israel vaunt themselves against me, saying, `My own hand hath saved me!'%%
  \verse{7:3} Now, please proclaim within earshot of the people, saying, `Whosoever is fearful and trembling, let him turn back and leave early from Mount Gilead.'\thinspace'' And 22\thinspace000 of the people turned back, and 10\thinspace000 remained.%%
  
  \verse{7:4} And the \textsc{Lord} saith unto Gideon, ``Still, the people are too many. Bring them down to the water and I shall refine\alt{test} them for you there. It shall be that he to whom I say to thee, `This man shall go with you,' shall go with you; and any that I say unto you, `This man shall not go,' shall not go.''%%
  \verse{7:5} So he brought the people down to the water.%%
  
  And the \textsc{Lord} said unto Gideon, ``Any who laps water with his tongue (like a dog), you shall set by himself. Likewise with those that kneel down to drink.''%%
  \verse{7:6} The number of those who lapped~--- putting their hands to their mouth~--- was 300 men; but all the rest of the people knelt down to drink water.%%
  
  \verse{7:7} And the \textsc{Lord} saith unto Gideon, ``I will deliver you with the 300 men who lapped; and I will give the Midianites into your hand. Let all the rest go unto their homes.''%%
  \verse{7:8} They took the people's provisions\ed{\Hebrew{צֵדָה}, from the root \Hebrew{צוד}, is not attested in \textsc{halot}. However, Davidson includes a Hithpael rendering which can mean ``to furnish oneself with provision'' (641) from which we can ascertain the root to have something to do with provisions. \textsc{ylt} renders it as ``provision'' and \textsc{darby} renders it as ``victuals'' so this seems to be verified.} and trumpets into their hands. And he sent every male Israelite away, every man to his tent; but he held onto the three hundred men. The Midianites' camp was in the valley below him.%%
  
  \verse{7:9} That night, the \textsc{Lord} said to him, ``Get up. Go down into the camp because I have delivered the camp into your hand.%%
  \verse{7:10} But if you're afraid to go down, then go down~--- you and your servant Phurah~--- to the camp%%
  \verse{7:11} and hear what they have to say.\lit{what they say} Afterwards, your hands shall be strengthened and you shall go down to the camp.'' So he went down with his servant, Phurah,\ed{xxxx: If he has more than one servant, remove commas.} to the border of the fifty men who were in the camp.%%
  \verse{7:12} And Midian and Amalek and all the Easterners\lit{children of the east} were lying in the valley like a swarm\alt{multitude} of locusts. Their camels are beyond number,\lit{to their camels there is no number} as multitudinous as the sand on the seashore.%%
  \verse{7:13} Gideon came and, look, a man was telling his friend about a dream; he said, ``I've had\lit{dreamed} a dream: a round loaf of barley bread was turning this way and that in the Midianites' camp, and it entered the tent, smote it, and the tent\understood\ fell and was turned upside down. The tent had fallen.''%%
  \verse{7:14} His friend replied and said, ``This is nothing other than the sword of Gideon the son of Joash, an Israelite. God put Midian and the entire camp in his hand.''%%
  
  \verse{7:15} When Gideon had heard the dream being told\lit{the telling of the dream} as well as its interpretation, he bowed down and returned to the Israelites' camp and said, ``Get up because the \textsc{Lord} has given you\lit{placed in your hands} the Midianites' camp.''%%
  \verse{7:16} He divided the three hundred men into three companies, and gave everyone a trumpet,\lit{gave a trumpet into the hand of everyone} an empty jar,\alt{pitcher}\halot{xxxx}{large (pottery) jar for flour or water} and a torch to put in it.%%
  \verse{7:17} He said to them, ``Watch me and do what I do.\lit{thus do ye} I will come to the edge of the camp, and when I've done that, do the same.%%
  \verse{7:18} When we\ed{conjugated 1cs, but understood 1cp} blow the trumpet, me and everyone who's with me, blow your trumpets as well all around the camp and say, `For the \textsc{Lord} and for Gideon.'\thinspace''%%
  
  \verse{7:19} Gideon and his hundred\lit{the hundred men who were with him} came to the edge of the camp and got there at the beginning of the middle watch;\ed{\textsc{halot} notes this as being ``evidently the last watch of the night.'' However, since it is modified by \Hebrew{הַתִיכוֺנָה} (middle) it refers to the middle watch. This watch runs from 10~\textsc{pm} till 2~\textsc{am}.} however, the guards just took their station.\lit{however, it confirmed confirmed [verily/just confirmed] the guards.} They blew their trumpets and smashed the jars in their hands.%%
  \verse{7:20} The three companies blew their trumpets and smashed the jars. With the lamp in their left hand and the trumpets blowing in their right, they cried out, ``The sword of the \textsc{Lord} and of Gideon.''%%
  \verse{7:21} The men stood in their places surrounding the camp, the entire camp running, shouted in alarm, fleeing.%%
  \verse{7:22} The three hundred blew their\lit{the} trumpets and the \textsc{Lord} set every man's sword against his friend throughout the entire camp. And the camp fled to Beth-shittah towards Zererah, to the border of Abel-meholah towards Tabbath.%%
  \verse{7:23} Israelites were called\alt{summoned} from Naphtali, Asher, and from all of Manasseh to pursue the Midianites.%%
  \verse{7:24} Gideon sent messengers to the mountains\lit{all the mountains (alt., mountain range)} of Ephraim, saying, ``Come here to meet Midian. Capture them at the waters by Beth-barah and the Jordan River.'' The Ephraimites were summoned and they captured the waters from Beth-barah to the River Jordan.%%
  \verse{7:25} They captured two Midianite officials, Oreb and Zeeb. They slew Oreb at the Oreb Boulder, they slew Zeeb at the Zeeb Wine-vat, and they pursued the Midianites. They brought Oreb's and Zeeb's heads to Gideon to the opposite bank of the Jordan River.%%
\end{inparaenum}

  
  \book{Ruth}{\Hebrew{רות}}
  \heading{1}{Naomi's husband and sons die~--- one of her daughters-in-law, Orpah, leaves~--- the other, Ruth, promises to stay}

\begin{inparaenum}
  \verse{1:1} In the days when the judges judged there was a famine in the land. A man of Bethlehem-Judah went to sojourn in the field of Moab\ed{Moabite territory or country}~--- he, and his wife, and his two sons.%%
  \verse{1:2} And the name of the man~--- Elimelech; and the name of his wife~--- Naomi; and the names of his two sons~--- Mahlon and Chilion: Ephrathites from Bethlehem-Judah. And they entered the field of Moab and they stayed\alt{continued, sojourned} there.%%
  \verse{1:3} And Naomi's husband, Elimelech, died, and she was left with\lit{and} her two sons.%%
  \verse{1:4} And they took to\lit{raised up unto} themselves Moabite wives, the name of the one, Orpah; and the name of the other, Ruth. And they stayed there about ten years.%%
  \verse{1:5} Both Mahlon and Chilion died. And the woman was bereft of her two boys and her husband.%%
  \verse{1:6} And\alt{Then} she rose up with her daughters-in-law to return to the land of Moab, for she had heard that in the land of Moab the \textsc{Lord} had visited His people by giving them bread.%%
  \verse{1:7} She went out from the place where she was with her two daughters-in-law and they went on the road to return to the land of Judah.%%
  \verse{1:8} And Naomi said to her two daughters-in-law to go return to their mothers' houses. ``The \textsc{Lord} will do kindly with you according as ye have done with the dead and myself.%%
  \verse{1:9} The \textsc{Lord} will give you rest in your husbands' houses.'' Then she kissed them, and they lifted up their voices and wept.%%
  \verse{1:10} But they said to her, ``Surely we will return to thy people.''%%
  \verse{1:11} But Naomi said, ``Return, my daughters. Why would you go with me? Do I yet have sons in my womb that they may be husbands to you?%%
  \verse{1:12} Return, my daughters, for I have grown too old to have a husband.\lit{I have grown old from having a man.} If I were to say I have hope, and if I were to have a man tonight, and have sons,%%
  \verse{1:13} would you wait for them until they were grown and keep yourselves from a husband? Not so, my daughters, for it grieves me greatly for your sakes that the hand of the \textsc{Lord} is against me.''%%
  \verse{1:14} They lifted up their voices and wept again. And Orpah kissed her mother-in-law, but Ruth clung to her.%%
  \verse{1:15} And she\ie{Naomi} said, ``Your sister-in-law\ed{The Hebrew, \Hebrew{בּמת}, denotes a relationship by marriage, not necessarily that they are sisters. In general it can refer to any female relative.} is returning to her people and to her gods. Return after your sister-in-law.''%%
  \verse{1:16} And Ruth said, ``Do not ask me to abandon you nor to turn from following thee: for wherever you go, will I go; and where you lodge, I will lodge; your people shall be my people; and your God shall be my God;%%
  \verse{1:17} and where thou diest, will I die and be buried there: the \textsc{Lord} do to me, and more also, if anything but death parts me and thee.''%%
  \verse{1:18} When she saw how bold\alt{steadfastly minded} she\ie{Ruth} was to go with her, she ceased speaking with her.%%
  \verse{1:19} And the two of them went to Bethlehem. And when they got to Bethlehem the city was abuzz about them, saying, ``Is this Naomi?''%%
  \verse{1:20} And she said unto them, ``Don't call me Naomi. Call me Mara, for the Almighty hath treated me harshly.%%
  \verse{1:21} I left full, and empty the \textsc{Lord} returned me. Why would you call me Naomi since the \textsc{Lord} hath afflicted me and the Almighty hath broken me in pieces?''%%
  \verse{1:22} And Naomi returned with Ruth the Moabitess, her daughter-in-law, with her to the land of Moab and they entered into Bethlehem in the beginning of barley harvest.%%
\end{inparaenum}

  \heading{2}{Ruth gathers gleanings of grain~--- Boaz asks about Ruth, gives her permission to glean, and has his servants make sure that there will be gleanings to gather}

\begin{inparaenum}
  \verse{2:1} Naomi got to know one of her husband's relatives, a man of great substance\alt{heroic landowner} from Elimelech's extended family whose name was Boaz.%%
  \verse{2:2} Ruth the Moabitess said to Naomi, ``Please, let me go to the field and I'll glean grain\lit{ears of grain} behind the man\understood\ in whose opinion I shall find grace.'' And she said to her, ``Go, my daughter.''%%
  \verse{2:3} So she went and she came and gleaned behind the harvesters. And it so happened that she came upon one of the fields allotted to Boaz (who is in Elimelech's extended family).%%
  \verse{2:4} Boaz came from Bethlehem and said to the harvesters, ``The \lord\ be with you!'' And they said to him, ``The \lord\ bless you!''%%
  \verse{2:5} Boaz said to his servant who was in charge of the harvesters, ``Who is this girl?''\halot{xxxx}{xxxx \Hebrew{נערה}, in this context, refers to an unmarried girl (who is a virgin)}%%
  \verse{2:6} The servant who was in charge of the harvesters answered and said, ``The girl is a Moabitess who came back from the fields of Moab with Naomi.%%
  \verse{2:7} She has said, `Please let me gather gleanings. I have harvested ears of grain\halot{xxxx}{not sheaves; the stalks were cut off right under the ears.} behind the harvesters.' She's come and been here\lit{remained} from morning until now. She did\understood\ sit in the house a little.''%%
  \verse{2:8} So Boaz said to Ruth, ``Hello girl,\lit{my daughter} haven't you heard? Don't go gather gleanings in another field. Also, don't pass through here, but\lit{thus} stick with my girls.\alt{female servants}%%
  \verse{2:9} Let your eyes be on the field that is being harvested and go after them. I've commanded the servants to not hurt\alt{touch} you, haven't I? When you're thirsty, go to the vessels and drink from the water\understood\lit{what, that, which} the servants draw.''%%
  \verse{2:10} She fell on her face and bowed to the ground, and said to him, ``Why have I found grace in your eyes that you recognize me, \lit{I am; I being}a stranger?''%%
  \verse{2:11} And Boaz, answering, said unto her, ``I've been told about \emph{everything} you've done to your mother-in-law: how you left\alt{abandoned} your father and your mother and the land of your birth, and you've come to a people you've never before known.%%
  \verse{2:12} May the \lord\ recompense your efforts and may your reward be full\alt{complete} from the \lord\ God of Israel under whose wings you have come to seek refuge.''%%
  \verse{2:13} And she said, ``May I find favor in your eyes for you have been kind to us, and because you have been kind to your maidservant even though I am not like\lit{among one of} your maidservants.''%%
  \verse{2:14} And Boaz said to her at mealtime, ``Come hither and eat bread and dip the morsel\ie{broken pieces} in vinegar.'' And she stayed by the harvesters and he\ed{This can refer to either Boaz or the harvesters~--- it's ambiguous.} offered\alt{reached out, extended} her parched\alt{roasted} grain and she ate, was satiated, and had some left over.%%
  \verse{2:15} And she stood up to go glean, and Boaz commanded his servants saying, ``Let her glean between the sheaves and do not bother\alt{rebuke} her.%%
  \verse{2:16} And also draw out for her some bundles\alt{sheaves} and leave and she shall gather. And rebuke her not.''%%
  \verse{2:17} And she gathered in the land\alt{field} until eventide and she threshed that which she had gathered:\alt{gleaned} about an ephah of barley.%%
  \verse{2:18} And she took it up and went into the city. And her mother-in-law saw her and she\ie{Naomi} took that which she\ie{Ruth} had gathered and she brought it out to her and she was satiated.%%
  \verse{2:19} And her mother-in-law said unto her, ``Where did you glean today? And where did you work? May her who helped thee be blessed.'' And she told her mother-in-law, ``The person with whom I worked today was Boaz.''%%
  \verse{2:20} And Naomi said to her daughter-in-law, ``May he be blessed of the \lord\ who did not abandon his loving-kindness toward the living nor toward the dead.'' And Naomi said to her, ``The man that is near to us, he is our levir.''\lit{is among our levirs}\ed{A levir was a tribal leader, avenger of blood, or a redeemer.}%%
  \verse{2:21} And Ruth the Moabitess said, ``He also said to me, `You shall stay with my workers until they have ended my harvests.'''%%
  \verse{2:22} And Naomi said to Ruth her daughter-in-law, ``My daughter, it is well that you go out with his maidservants and that they don't bother you in any other field.''%%
  \verse{2:23} And she stayed fast by the maidservants of Boaz to glean until the end of the barley and wheat harvests. And she stayed with her mother-in-law.%%
\end{inparaenum}

  \heading{3}{Naomi has Ruth entreat herself to Boaz, their levir~--- Boaz tells Ruth that there is another levir who has more immediate responsibility over her than he~--- he comes up with a plan to redeem her himself~--- she spends the night and he sends her off}

\begin{inparaenum}
    \verse{3:1} So Naomi, her mother-in-law, said to her, ``My daughter, rest that it may be well for you.%%
    \verse{3:2} Is not Boaz our kinsman, among whose maidservants you were with? He is threshing barley at the threshing floor tonight.%%
    \verse{3:3} Wash and anoint yourself, put your clothes on, and get down to the threshing floor, but don't make yourself known to the man until he's done eating and drinking.%%
    \verse{3:4} And when he lies down you shall know the place where he lie. And you shall enter and uncover his feet and lie down. And he shall tell you that which you shall do.''%%
    \verse{3:5} And Ruth\understood\ said unto her, ``All that thou hast said unto me I will do.''%%
    \verse{3:6} And she went down to the threshing floor and she did just about everything that her mother-in-law told her.%%
    \verse{3:7} And Boaz ate and drank and was happy. So he entered to lie down at the edge of the heap of grain. And she entered in quietly\alt{secretly} and uncovered his feet and lied down.%%
    \verse{3:8} And night came, and the man was shocked and turned and there was a woman lying at his feet.%%
    \verse{3:9} And he said, ``Who is there?'' And she said, ``I am Ruth, thy handmaiden. Spread the hem of thy garment over thy maidservant for you have the right of a levirite.''%%
    \verse{3:10} And he said, ``Blessed be thou of the \textsc{Lord}, my daughter. You've done a finer act of kindness at the end than in the beginning in not going after the young men,\halot{xxxx}{fully grown, vigorous, still unmarried.} be they rich or poor.%%
    \verse{3:11} Now, my daughter, don't be afraid because all I have said unto you, that will I do. All of my people think that thou art a virtuous woman.%%
    \verse{3:12} And now, truly I am a levir; however, there's a levir nearer than I.%%
    \verse{3:13} Stay the night. If he redeems you in the morning, great. If he's not inclined to redeem you, then as the \textsc{Lord} lives, I will redeem you. Lie down till the morning.''%%
    \verse{3:14} She laid at his feet till morning and got up before you can tell who's who.\lit{before you can tell a man from another.} He said, ``Don't let it be known that a woman came to the threshing floor.''%%
    \verse{3:15} He said, ``Give me the scarf from off your head.'' And he measured out for her six measures of barley. Then he\ed{xxxx ca? The Septuagint version can be translated as either ``he'' or ``she.'' The BHS leaves no such ambiguity and can only be rendered as ``he.''} went into the city.%%
    \verse{3:16} And she came to her mother-in-law and Naomi\understood\ said, ``How did it go, my daughter?'' And then she told her everything that the man had done for her,%%
    \verse{3:17} saying\lit{``and she said''}, ``He gave me six measures because he said to me, `Don't go to your mother-in-law empty-handed.'\thinspace''%%
    \verse{3:18} Then she said, ``Sit, my daughter, until you know how the thing will play out, for the man will not be quiet until he has finished the matter today.''%%
\end{inparaenum}

  \heading{4}{The other levir refuses Ruth because it would hurt him economically~--- Boaz redeems Ruth and they are married~--- Ruth gives birth to Obed whom Naomi takes care of~--- Boaz's genealogy given to show that David comes from this line}

\begin{inparaenum}
  \verse{4:1} And Boaz went up the the gate and he sat down there. And the levirite\ed{The text renders this in such a fashion as to not give away the man's identity out of respect to his posterity. It can be determined, though, that this man was the levirite.} was passing by, and Boaz\understood\ said, ``Turn away, sit here.'' And he turned aside and sat down.%%
  \verse{4:2} And he took ten men from among the elders of the city and told them, ``Sit down here.'' And they sat down.%%
  \verse{4:3} And the levir said to his kinsman, ``Naomi~--- who is returned from the plains of Moab~--- is selling the land of our kinsman Elimelech.%%
  \verse{4:4} Now, let me speak to you\lit{I will uncover thy ear} to acquire it before of those who are sitting here and before the elders of my people. If you are going to redeem it, then redeem it. If he will not\ed{It seems that at this point Boaz has turned and is addressing the people standing by.} redeem it, tell me so that I can know since there is no one besides you to redeem it.''\ie{I am next after you.} And he said, ``I will redeem it.''%%
  \verse{4:5} And Boaz said, ``When you\lit{In the day that you} acquire the property\alt{land, field} from the hand of Naomi, you will acquire Ruth the Moabitess~--- the wife of the deceased kinsman~--- to raise up the name of the deceased over his inheritance.''%%
  \verse{4:6} Then the levir said to him, ``I will not be able to redeem it for myself lest I mar\alt{dilute} my inheritance. Redeem my right\ie{the property that I would redeem} to thyself for I cannot redeem it.''%%
  \verse{4:7} (Now this was the manner of redemption in Israel in order to effect redemption or exchange\ie{of Levirite marriage or property} in order to establish anything. The man took off his sandal and gave it to his neighbor as a testimony to his neighbor.)%%
  \verse{4:8} The levir said to Boaz, ``Acquire it for thyself.'' And he took off his sandal\ed{In \vref{Deut}{25}{5--10} we learn that the spurned woman~--- Ruth in this case~--- would have removed the man's sandal and then spit in his face.}.%%
  \verse{4:9} So Boaz said to the elders and to all of the people, ``Today you are all witnesses that I have acquired all that which belonged to Elimelech, Chilion, and Mahlon, from the hand of Naomi.%%
  \verse{4:10} And also Ruth the Moabitess~--- the wife of Mahlon~--- have I obtained\lit{for myself} to be my wife, to raise up the name of the deceased over his inheritance that the name of the deceased be not cut off from among his brethren and from the gate of his place: of this ye are witnesses this day.''%%
  \verse{4:11} Then all the people that were within the city limits and the elders said, ``We are witnesses. May the \lord\ make the wife come into thine house like Rachel and Leah who built\lit{the two of whom built} the house of Israel.\ed{\S~--- quite the thing to say, especially considering there is no mention of the Patriarchs.} Now, do thou worthily\alt{act virtuously} in Ephratah and make a name for thyself in Bethlehem.%%
  \verse{4:12} And now let thy house be as the house of Pharez, whom Tamar bare for Judah, of the seed which the \lord\ will give thee of this young woman.''%%
  \verse{4:13} So Boaz took Ruth unto himself to be his wife. And he went in unto her and the \lord\ gave unto her conception\alt{pregnancy} and she bare a son.%%
  \verse{4:14} And the women\ie{her female neighbors} said unto Naomi, ``Blessed be the \lord\ who hath not abandoned thee this day without a redeemer: that His name may be glorified in Israel.%%
  \verse{4:15} And He shall restore your soul\lit{shall be a restorer of life unto thee} and sustain\lit{support you in} your old age because your daughter-in-law, who loves you, hath given birth to him, and she is better to you than seven sons.''%%
  \verse{4:16} And Naomi took the child\alt{boy} and laid him\lit{it} on her bosom and became his nurse.%%
  \verse{4:17} And the women gave him a name, saying, ``A son is born to Naomi.'' And they named him Obed. He is the father of Jesse the father of David.%%
  
  \verse{4:18} These are the generations of Perez: Perez fathered Hezron;%%
  \verse{4:19} Hezron,\ed{For all following, \textit{begat}, \textit{fathered}, or \textit{sired} is understood.} Ram; Ram, Amminadab;%%
  \verse{4:20} Amminadab, Nahshon; Nahshon, Salmon;%%
  \verse{4:21} Salmon, Boaz; Boaz, Obed;%%
  \verse{4:22} Obed, Jesse; and Jesse, David.%%
\end{inparaenum}

  
  \book{1 Samuel}{\Hebrew{א שמואל}}
  \heading{15}{Samuel commands Saul to annihilate the Amalekites~--- Saul and the people disobey God, spare Agag, and bring back some of the animals to sacrifice~--- Saul is sharply reprimanded by Samuel, denies any wrongdoing, and is stripped of his royalty}

\begin{inparaenum}
  \verse{15:1} And Samuel spake unto Saul, ``The \textsc{Lord} didst send me to anoint thee king over His people~--- over Israel~--- therefore, now, hearken unto the voice\alt{sound} of the words of the \textsc{Lord}:%%
  
  \verse{15:2} Thus saith the \textsc{Lord} of Hosts, `I have reviewed\alt{looked after} that which Amalek did to Israel, that which he placed for himself in the way going up out of Egypt.%%
  \verse{15:3} Go now, and smite Amalek and all that he hath; show no mercy.\lit{have no pity on them} Put to death every man, woman, infant, suckling, ox, sheep, camel, and ass.'''%%
  
  \verse{15:4} So Saul summoned the people and inspected them in Telaim: 200,000 footmen and 10,000 men of Judah.%%
  \verse{15:5} Then Saul came unto the city of Amalek and waited in a valley.%%
  \verse{15:6} And Saul saith to the Kenite, ``Go, turn aside. Leave from the midst of Amalek lest I consume thee as well, for thou didst show kindness unto the children of Israel when they came up out of Egypt.'' So the Kenite turned aside from the midst of Amalek.%%
  \verse{15:7} Then Saul smote Amalek from Havilah to Shur (on the borders of Egypt).%%
  \verse{15:8} He caught Agag, king of Amalek, alive; but all the people were destroyed by the sword.%%
  \verse{15:9} But Saul~--- and the people as well~--- had pity on Agag, the best of the flocks, herds, garments, rams, and all that was beautiful and they would not destroy them.%%
  
  \verse{15:10} Then the word of the \textsc{Lord} came unto Samuel, saying,%%
  \verse{15:11} ``I'm sorry that I made Saul king: he hath forsaken Me in that he did not keep My commandments.'' And Samuel was grieved and cried unto the \textsc{Lord} all night.%%
  \verse{15:12} So Samuel rose early in the morning to meet Saul. And it was declared to Samuel, saying, ``Saul hath come in to Carmel and is setting up to himself a monument. Then he shall go round, pass over, and go down into Gilgal.''%%
  \verse{15:13} Samuel came to Saul, and Saul saith to him, ``Blessed be thou of the \textsc{Lord}; I have done what the \textsc{Lord} hath said.''%%
  \verse{15:14} So Samuel said, ``Why then do I hear bleating of sheep and the noise of a herd?''%%
  \verse{15:15} And Saul said, ``We brought them from Amalek because the people had pity on the best of the flocks and herds so that we could sacrifice them to the \textsc{Lord} thy God; the rest we destroyed.''%%
  
  \verse{15:16} Samuel said to Saul, ``Hold on,\alt{Relax, Wait} let me tell you what the \textsc{Lord} told me tonight.'' And Saul said, ``Go on.''%%
  
  \verse{15:17} Samuel said, ``When you were little\lit{in your own eyes}, were you not made head of the tribes of Israel and anointed by the \textsc{Lord} to be king over Israel?%%
  \verse{15:18} And the \textsc{Lord} sent you on a journey and said, `Go and utterly destroy the Amalekites and all that they hath.'%%
  \verse{15:19} Why then did you not obey the voice of the \textsc{Lord}, but instead did fly on their spoil and do evil in the eyes of the \textsc{Lord}?''%%
  
  \verse{15:20} And Saul said to Samuel, ``I \emph{have} hearkened to the voice of the \textsc{Lord}. I went the way the \textsc{Lord} sent me, and I have brought Agag, king of Amalek, and destroyed the Amalekites.%%
  \verse{15:21} And the people took the spoil of the flocks and herds\ed{Idiomatically rendered in plural}~--- the choicest\alt{chief, first, best} part of the devoted things\ie{that which should have been destroyed}~--- to sacrifice unto the \textsc{Lord} their God in Gilgal.''%%
  \verse{15:22} And Samuel saith:%%
  
  \pb ``Does the \textsc{Lord} delight in sacrifices and burnt offerings\pa as much as He delights in hearkening to the voice of the \textsc{Lord}?%%
  
  \pb Listen! Obedience is better than sacrifice.\pa Paying attention is more important than ram's fat.%%
  
  \pa \verse{15:23} Since rebellion is like the sin of witchcraft\pa and stubbornness is like idolatry.%%
  
  \pb Because you've rejected the word of the \textsc{Lord},\pa He has rejected you from being king.%%
  
  \verse{15:24} Then Saul said to Samuel, ``I have sinned, for I transgressed the commandment of the \textsc{Lord}\lit{I have passed over (or by) the command (or mouth) of the \textsc{Lord}} and thy words\ed{Repetitious because they're the same in this instance. Compare \vref{D\&C}{1}{38}. Shows a serious lack of understanding on Saul's part.} in that I showed reverence to\alt{feared} the people by hearkening to their voice.\ie{instead of the \textsc{Lord}}%%
  \verse{15:25} Please forgive me\lit{Now, please bear my sin} and come with me as I bow down before the \textsc{Lord}.''%%
  \verse{15:26} Samuel said to Saul, ``No, I'm not going with you\lit{I will not turn back with thee} because you've rejected the word of the \textsc{Lord} and because the \textsc{Lord} hath rejected you from being king over Israel.''%%
  
  \verse{15:27} Then Samuel turned around to go, but he\ie{Saul} caught hold of his\ie{Samuel's} robe's mantle\alt{upper skirt} and it rent.%%
  \verse{15:28} And Samuel said to him, ``The \textsc{Lord} hath rent the Kingdom of Israel from thee today and given it to thy neighbor (who is better than thee).%%
  \verse{15:29} Additionally, the eminence\alt{preeminence, perpetuity, everlastingness} of Israel\ed{Is this a name-title of the \textsc{Lord}? If so, ``eminence'' should be capitalized.} neither lies nor repents for He is not a man, that He is penitent.''%%
  \verse{15:30} And he said, ``I have sinned. Come, now, and honor me before the elders of the people\ed{This is horribly pretentious on Saul's part: if Samuel wouldn't bow down to the \textsc{Lord} with Saul, why in the world would he honor him before the elders of the people and all of Israel? It's like Saul doesn't think before he speaks.} and before Israel. Come with me as I bow down before the \textsc{Lord}.''%%
  \verse{15:31} So Samuel turned back after Saul as Saul bowed before the \textsc{Lord}.\ed{I know, I'm as shocked as you.}%%
  
  \verse{15:32} And Samuel said, ``Bring Agag, king of Amalek, unto me.'' And Agag came in unto him delicately and said, ``Certainly the bitterness of death has past.''%%
  
  \verse{15:33} Samuel said, ``As thy sword hath bereaved women, even so shall thy mother be bereaved among women,'' and Samuel hewed Agag into pieces before the \textsc{Lord} in Gilgal.%%
  
  \verse{15:34} Samuel went to Ramath and Saul went to his house in Gibeah of Saul.%%
  \verse{15:35} Samuel never again came to see Saul, even until his death; nevertheless, Samuel mourned for Saul. And the \textsc{Lord} was sorry that he had made Saul king over Israel.%%
\end{inparaenum}

  \heading{16}{The Lord tells Samuel to visit Jesse for one of his sons will be king~--- Samuel is told to not judge on outward appearance, but to trust in revelation~--- Samuel anoints David, Saul is forsaken~--- David becomes Saul's armor bearer~--- the evil spirit leaves Saul}

\begin{inparaenum}
  \verse{16:1} The \lord\ said to Samuel, ``How long will you mourn for Saul? I have rejected him from being king of Israel. Fill your horn with oil and go: I will send you Jesse the Bethlehemite because I've seen for Myself a king among his sons.''%%
  \verse{16:2} Samuel said, ``Where should I go? If Saul hears, he'll kill me.''%%
  
  \ca{mlt Mss om interv}{multiple Hebrew manuscript codices omit the interval [the section spacing]}The \lord\ said, ``Take\ca{\qumran\ \Hebrew{ק[ח} cf \septuagint\peshitta}{the Dead Sea Scrolls have [a letter missing], compare the Septuagint and Peshitta}\ed{This is an excellent example of a place where we've been able to use other manuscripts to fill in the blanks in the Dead Sea Scrolls.} a young cow\alt{heifer} from the herd in your hand and say, `I have come to sacrifice to the \lord.'%%
  \verse{16:3} Call for Jesse for the sacrifice, and I will let you know what you're do to, and you shall anoint him whom I've told\lit{said to} you.''%%
  \verse{16:4} Samuel did what the \lord\ told him: he came to Bethlehem and the city elders trembled to meet him, and one\ie{of the elders}\ca{mlt Mss Seb pl cf \septuagint\peshitta\targum\super{-Ms\thinspace ed\thinspace princ}\vulgate}{multiple Hebrew manuscript codices and Sebir are in plural [``they said''], compare the Septuagint, Peshitta, Talmud, and Vulgate} said, ``Do you come in peace?''\ca{\qumran\ + \Hebrew{הראה} cf \septuagint}{the Dead Sea Scrolls add ``\dots, Seer?''}%%
  \verse{16:5} He said, ``Peaceably. I've come to sacrifice to the \lord. Put yourself in a state of dedication\halot{xxxx}{1.\ \textbf{behave as} (people who have been) \textbf{consecrated, made holy} \haref{Ex}{19}{22}; --- 2.\ \textbf{prove oneself} (to be) \textbf{holy} (subject Y.) \haref{Ez}{38}{23}; --- 3.\ \textbf{put oneself} (or each other) \textbf{into the state of dedication} or \textbf{cultic purity} \haref{1S}{16}{5}} and come with me to sacrifice.'' He sanctified Jesse and his sons and called them to the sacrifice.\ed{I initially rendered this ``called them to sacrifice,'' but that doesn't work with Samuel having Priesthood authority and them not.}%%
  \verse{16:6} When they came, he saw Eliab and said, ``Surely the \lord's anointed is before Him.''%%
  \verse{16:7} But the \lord\ said to Samuel, ``Don't set a value on\halot{xxxx}{look at something = set a value on \haref{1S}{16}{7}} his appearance or his height\halot{xxxx}{his tall figure} because I have rejected him. Because it's not like\lit{that}\ca{cit \Hebrew{כא׳} cf \septuagint\peshitta\targum}{cited as ``like,'' compare the Septuagint, Peshitta, and Targum} man sees: men look on the outward appearance,\lit{looks} but the \lord\ looks on the heart.''%%
  \verse{16:8} Jesse called Abinadab and had him walk past Samuel. And he\ie{Samuel} said, ``\lord\ hasn't chosen this one either.''\lit{``Also on this not chosen the \lord.''}%%
  \verse{16:9} So Jesse had\lit{made} Shammah pass by, but he\ie{Samuel} said, ``\lord\ hasn't chosen this one either.''%%
  \verse{16:10} Jesse had seven of his sons\ed{but not ``his seven sons''} pass by in front of Samuel, but Samuel said to Jesse, ``The \lord\ has not chosen these.''%%
  \verse{16:11} So Samuel said to Jesse, ``Is this all the young men?''\lit{``Are the young men finished?''} He replied,\lit{said} ``There is still\alt{There still/yet remains} the youngest. He is tending the sheep.'' Samuel said to Jesse, ``Send and get him because we will not sit around the table until he's come here.''%%
  \verse{16:12} So he sent and brought him in. He was reddish with a beautiful appearance and handsome.\lit{good appearance}%%
  
  \ca{mlt Mss om interv}{multiple Hebrew manuscript codices omit the interval [the section spacing]}And the \lord\ said, ``Arise. Anoint him. Because this is he.''%%
  \verse{16:13} Samuel took the horn of oil and anointed him in the midst of his brothers. The Spirit of the \lord\ ca{pc Mss \Hebrew{אלהים}}{a few Hebrew manuscript codices have ``God''} was powerful\alt{effective, strong} on David from that day on. And Samuel got up and went to Ramah.%%
  
  \verse{16:14} The Spirit of the \lord\ ca{cit \Hebrew{אלהים} cf \septuagint\super{Ms}\peshitta}{cited as ``God,'' compare the Septuagint and Peshitta} stopped being with Saul, and an evil spirit from the \lord\ terrified him.%%
  \verse{16:15} Saul's servants said to him: ``Please, an evil spirit from God\ca{pc Mss \Hebrew{יהוה} cf \septuagint\latina\super{93.94}}{a few Hebrew manuscript codices have ``the \lord,'' compare the Septuagint and the old Latin version} terrifies you.%%
  \verse{16:16} Let our lord please speak, and your servants before you will seek a skilled,\halot{xxxx}{understand how to do something \haref{1S}{16}{18}} string-player\lit{a man skilled/understanding from a string-player} on the lyre;\halot{xxxx}{\textbf{lyre} (stringed instrument with sounding-board or -chest)} and it shall be, when the evil Spirit of God\ca{\missing\ \septuagint\superit{-L}\latina\super{93.94}\peshitta}{missing in the Septuagint, old Latin version, and Peshitta}\ed{``Spirit'' and ``God'' are punctuated with a \textit{maqaf}. This also makes it difficult to know if the \textsc{ca} is referring to ``Spirit of God'' or just ``God.''} is on you, and he plays with his hand, and it's good to you.''%%
  
  \verse{16:17} Saul said to his servants, ``Please look for a man\lit{for me} who plays well, and bring him to me.''%%
  \verse{16:18} One of the servants answered, and said, ``I've seen a son of Jesse the Bethlehemite, skilled at playing~--- a warrior, strong~--- a man of war, smart\halot{xxxx}{xxxx} with words, a man of dignity.\halot{xxxx}{of man, \textbf{dignity}, fine, impressive \textbf{appearance} \haref{1S}{16}{18}} The \lord\ is with him.''%%
  \verse{16:19} Saul sent messengers to Jesse, saying, ``Bring\lit{Send} me your son David who's with the flock.''%%
  \verse{16:20} Jesse took a donkey, bread, a bottle of wine, and one goat kid, and sent them\understood\ by the hand of his son David to Saul.%%
  \verse{16:21} David came in to Saul and stood before him. And he\ie{David} loved him greatly; and he was his armor bearer.%%
  \verse{16:22} Saul sent to Jesse, saying, ``Please let David stand before me because he has found favor in my eyes.''%%
  \verse{16:23} When the Spirit of God\ca{nonn Mss cit + \Hebrew{רעה}\dots}{a few Hebrew manuscript codices cited add ``evil''\dots [xxxx: lots of Greek and Syriac I don't know]} was on Saul, David took the lyre and played with his hand and Saul felt relieved\halot{xxxx}{it becomes wide for him = \textbf{he feels relieved}} and it was good to him and the evil spirit\ed{It's obvious that either ``God'' needed to be missing earlier in this verse, or that ``evil'' needed to be added because otherwise this verse equates ``the Spirit of God'' with ``the evil spirit.''} departed from him.%%
\end{inparaenum}

  \heading{17}{xxxx}

\begin{inparaenum}
  \verse{17:1} The Philistines gathered their armies to battle, and they gathered to Sokho (which is Judah's) and camped between Sokho and Azekah, in Ephes Dammim.%%
  \verse{17:2} Saul and the men of Israel were gathered and encamped in the valley of Elah, and entered into battle against the Philistines.%%
  \verse{17:3} The Philistines stood by the mountain on this side, and the Israelites stood by the mountain on that side, and the valley was between them.%%
  \verse{17:4} A champion\alt{single fighter} went out from the Philistines' camps named Goliath of Gath, whose height was six\
  \ca{\septuagint\superit{L}\super{Mss} 4, \septuagint\super{Mss} 5, \latina\super{94} 16}{one manuscript of the Septuagint has 4, another manuscript of the Septuagint has 5, and the versio Latina has 16}
  cubits and a span.\
  \ed{4.5~cubits = 6'9''; 5.5 cubits = 8'3''; 6.5~cubits = 9'9''; 16.5~cubits = 24'9''}%%
  \verse{17:5} %%
  \verse{17:6} %%
  \verse{17:7} %%
  \verse{17:8} %%
  \verse{17:9} %%
  \verse{17:10} %%
  \verse{17:11} %%
  
  \verse{17:12} %%
  \verse{17:13} %%
  \verse{17:14} %%
  
  \verse{17:15} %%
  \verse{17:16} %%
  
  \verse{17:17} %%
  \verse{17:18} %%
  \verse{17:19} %%
  \verse{17:20} %%
  \verse{17:21} %%
  \verse{17:22} %%
  \verse{17:23} %%
  \verse{17:24} %%
  \verse{17:25} %%
  \verse{17:26} %%
  \verse{17:27} %%
  \verse{17:28} %%
  \verse{17:29} %%
  \verse{17:30} %%
  \verse{17:31} %%
  \verse{17:32} %%
  \verse{17:33} %%
  
  \verse{17:34} %%
  \verse{17:35} %%
  \verse{17:36} %%
  
  \verse{17:37} %%
  
  \verse{17:38} %%
  \verse{17:39} %%
  \verse{17:40} %%
  \verse{17:41} %%
  \verse{17:42} %%
  \verse{17:43} %%
  \verse{17:44} %%
  
  \verse{17:45} %%
  \verse{17:46} %%
  \verse{17:47} %%
  \verse{17:48} %%
  \verse{17:49} %%
  \verse{17:50} %%
  \verse{17:51} %%
  \verse{17:52} %%
  \verse{17:53} %%
  \verse{17:54} %%
  
  \verse{17:55} %%
  \verse{17:56} %%
  
  \verse{17:57} %%
  \verse{17:58} %%
\end{inparaenum}

  
  \book{2 Samuel}{\Hebrew{ב שמואל}}
  \heading{6}{King David gathers the people together~--- they take the Ark from Abinadab's house to Obed-Edom's~--- the cart carrying the Ark rocks, Uzzah tries to steady it and is struck dead~--- David rejoices and dances about half-naked~--- Saul's daughter Michal reproves him~--- David responds like a jerk}

\begin{inparaenum}
    \verse{6:1} David once again gathered every firstborn in Israel, 30\thinspace000.%%
    \verse{6:2} David and all the people that were with him got up and went from Baale-Judah\ed{Known as Kiriath-jearim in \vref{1~Chr}{13}{6}.} to take up the Ark of God. God,\understood\ whose name has been invoked upon it~--- that name being ``The \textsc{Lord} of Hosts who sits between the cherubs.''%%
    \verse{6:3} The Ark of God rode on a new wagon.\halot{xxxx}{not a chariot.} They lifted it up from Abinadab's house in Gibeah. Uzzah and Ahio, Abinadab's sons, led the new wagon.%%
    \verse{6:4} They carried the Ark of God from Abinadab's house on the hill, Ahio going before the Ark.%%
    \verse{6:5} David and the Israelites played \lit{with}all kinds of instruments\understood\ before the \textsc{Lord}, instruments\understood\ of juniper wood,\halot{xxxx}{Phoenician juniper, \textit{Juniperus phoenicea} (tree \& wood).} lyres,\alt{harp} harps, drums, sistrums,\halot{xxxx}{small percussion instrument which is rattled.} and cymbals.%%
    \verse{6:6} They came to Nachon's threshing floor and Uzzah put forth his hand\understood\ to the ark of God and took hold of it because the oxen stumbled.%% xxxx: Reword so the oxen stumbling is earlier in the sentence.
    \verse{6:7} And the \textsc{Lord} was exceedingly wroth with Uzzah, so God smote him there for his error that he died there before the ark of God.\ed{John Taylor said, referring to verses~6--7, ``The ark of God does not need steadying, especially by incompetent men without revelation and without knowledge of the kingdom of God and its laws'' (\textit{The Gospel Kingdom}, 166).}%%
    \verse{6:8} It was angering to David that the \textsc{Lord} had torn a breach through Uzzah. This place, to this day, is called Perez-Uzzah.\ed{Hebrew for ``the breach of Uzzah.''}%%
    \verse{6:9} That day, David reverenced the \textsc{Lord} and said, ``Why should the Ark of the \textsc{Lord} come to me?''%%
    \verse{6:10} But David himself was not willing to remove the Ark of the \textsc{Lord} to the City of David; rather, he\lit{David} turned aside to the Gittite\ed{someone from Gath (where Goliath was from)} Obed-Edom's house%%
    \verse{6:11} and the Ark of the \textsc{Lord} stayed there,\understood\ in the house of the Gittite Obed-Edom, for three months. And the \textsc{Lord} blessed Obed-Edom and his family.%%
    \verse{6:12} This was explained to King David, saying, ``The \textsc{Lord} blessed Obed-Edom's family and everything that he has because of the Ark of God.'' So David went and joyfully\lit{with joy} brought up the Ark of God from Obed-Edom's house to the City of David.%%
    \verse{6:13} When those who were bearing the Ark of the \textsc{Lord} had walked six steps, he sacrificed a ox and a fattened cattle.%%
    \verse{6:14} David, dressed in a linen ephod, danced before the \textsc{Lord} with all his might.%%
    \verse{6:15} David and all the people of Israel brought up the Ark of the \textsc{Lord} while shouting and playing the shofar.\lit{with shouting and the voice/sound of the shofar.}%%
    \verse{6:16} The Ark of the \textsc{Lord} came into the City of David. Michal, Saul's daughter, looked down from her window and saw King David being nimble and dancing before the \textsc{Lord}. And she thought contemptuously of him in her heart.%%
    \verse{6:17} They brought the Ark of the \textsc{Lord} and placed it in its spot\alt{set it in its place} in the tent that David had set up. And David offered burnt-offerings and peace-offerings before the \textsc{Lord}.%%
    \verse{6:18} When David completed offering the burnt-offering and the peace-offering, he blessed the people in the name of the \textsc{Lord} of Hosts%%
    \verse{6:19} and allotted to all of the people~--- the whole crowd of Israel, men and women~--- to each he allotted\understood\ a ring-shaped loaf\understood\ of bread, a date-cake, a raisin-cake. And everyone went to their homes.%%
    \verse{6:20} David returned to bless his house and Saul's daughter, Michal, went out to call on David, and said, ``How magnificent\alt{honorable} was the king of Israel today that he exposed himself today in front of\lit{in the eyes of} his servants' handmaids, just how\alt{like, as} an uncovered, vain person is exposed.''%%
    \verse{6:21} David said to Michal, ``It was before the \textsc{Lord} (who chose me instead of your dad,\ed{Jerk comment. Granted she wasn't being very polite, but this is no way to respond.} instead of his whole household, and appointed me leader over the people of \textsc{God}, the Israelites),\lit{over the people of Israel} so I danced before the \textsc{Lord}.%%
    \verse{6:22} I have been more humble\ed{This is a difficult verb to render. \Hebrew{קלל} appears here in the Niphal, \Hebrew{וּנְקַלֹּתִי}, and means: %%
    \begin{inparaenum}
        \item \textbf{prove swift}
        \item \textbf{humble oneself, demean oneself}
        \item be a small matter to someone
        \item \textbf{be easy to \dots}
        \item \textbf{be too light a thing to \dots}
        \item \textbf{be easy; superficially}
    \end{inparaenum}%%
    . It seems that David is saying that he is humbling himself, yet \textsc{darby} renders it as ``I will make myself yet more vile than thus'' and \textsc{ylt} renders it as ``I have been more vile than this.'' In essence, however, \Hebrew{קלל} means ``to humble oneself.''} than this. In my opinion, I've been lower.\halot{xxxx}{in social respect} I will be honored along with the handmaids you've referred to.''\lit{spoken of}%%
    \verse{6:23} \lit{to\dots}Michal, Saul's daughter, didn't have a child until the day of her death.\ed{How is this relevant? Does this mean that she never had a child, or that she had a child and died that selfsame day?}%%
\end{inparaenum}

  
  \book{1 Kings}{\Hebrew{א מלכים}}
  \heading{17}{Elijah prophesies no rain~--- he stays with the widow of Zarephath~--- the widow's son dies~--- the Lord, through Elijah, heals the boy}

\begin{inparaenum}
  \verse{17:1} Elijah the Tishbite, one\understood\ of the inhabitants of Gilead, said to Ahab, ``As\ed{Understood from covenantal language formula. See further in Appendix~\ref{app:covenants-in-antiquity}.} the \lord~--- the God of Israel before whom I've stood~--- lives, there shall not be any\understood\ dew\alt{light rain} or rain these years except as I order.''%%
  
  \verse{17:2} The word of the \lord\ came to him, saying,%%
  \verse{17:3} ``Leave.\lit{Go from this [place]} Turn east and hide in the wadi-Cherith\ed{Modernly, Beit She'an in northern Israel.} which is over by\lit{on the face of} the Jordan River.\understood%%
  \verse{17:4} You shall drink from the brook. I have commanded the ravens to feed you there.''%%
  \verse{17:5} So he went and did according to the \lord's word. He went and lived by the Cherith wadi which is by the Jordan River.%%
  \verse{17:6} In the morning the ravens brought him bread and meat; in the evening they brought him\understood\ bread and meat. And he drank from the brook.%%
  \verse{17:7} And it came to be after a while\ed{A Hebrew idiom is here used. Literally, ``the days are cut off,'' but meaning that this is at the end of a period of time and the seasonal river (brook) has been cut off from lack of rain.} that the brook dried up because there hadn't been any rain in the land.%%
  
  \verse{17:8} And the word of the \lord\ came unto him, saying,%%
  \verse{17:9} ``Arise. Go unto Zarephath~--- that is, unto Zidon~--- and dwell there. Behold! I have there commanded a widow woman to sustain thee.''%%
  \verse{17:10} So he arose and went unto Zarephath, came unto the city's entrance, and lo! there was a widow woman gathering sticks. So he called unto her and said, ``Please bring me a little water in a vessel so I can drink.''%%
  \verse{17:11} So she went to bring it and he calleth after her, and said, ``Please bring me some bread in thy hand.''%%
  \verse{17:12} Then she said, ``As the \lord\ thy God liveth, I don't even have a cake, but only a handful of meal in a pitcher and a little oil in a dish. Behold, I am gathering two sticks so that I can go in, prepare it for myself and my son, so that we can eat it and then die.''%%
  \verse{17:13} Elijah said unto her, ``Fear not: go and do as I've said, only make for me from thence a little cake first and bring it out to me. Then make for thee and thy son last.%%
  
  \verse{17:14} For thus saith the \lord\ God of Israel, `The pitcher of meal shall not be consumed and the oil dish shall not be lacking until the day the \lord\ makes it rain on the face of the land.' ''%%
  \verse{17:15} So she went and did as Elijah had said. And she ate~--- she, he, and her household~--- for days.%%
  \verse{17:16} And the pitcher of meal was not consumed, neither did the oil dish lack, even according to the word of the \lord\ that He had spoken by the hand of Elijah.%%
  
  \verse{17:17} After these things, the son of the woman (the mistress of the house) became sick. And his sickness was so severe that there was no breath left in him.%%
  \verse{17:18} And she said unto Elijah, ``What's this to me and thee, O man of God? You came to me to make me remember my iniquity and to kill my son!''%%
  \verse{17:19} So he said to her, ``Give me your son.'' And he took him from her bosom, and took him to the upper chamber where he was abiding, and laid him on the bed.%%
  \verse{17:20} Then he cried unto the \lord\ and said, ``O \lord\ my God, have you really brought evil upon the widow with whom I sojourn by killing her son?''%%
  \verse{17:21} He stretched himself upon the child thrice and cried to the \lord\ and said, ``O \lord\ my God, please let this child's soul return to him!''%%
  \verse{17:22} And the \lord\ heard the voice of Elijah and the child's soul returned and he lived.%%
  \verse{17:23} And Elijah took the child and brought him down from the upper room into the house and gave him to his mother. And Elijah said, ``Look, your son lives.''%%
  \verse{17:24} And the woman said to Elijah, ``By this I now know that thou art a man of God and that the word of the \lord\ in thy mouth is true.''%%
\end{inparaenum}

  \heading{18}{God commands Elijah to go meet Ahab~--- Obadiah and Elijah meet~--- Obadiah is afraid to tell Ahab about the meeting lest he is killed~--- Elijah and Ahab meet~--- Elijah challenges the prophets of Baal to call down fire from Heaven, but they are unsuccessful~--- Elijah calls down fire from Heaven, commands Israel to follow God, and kills the false prophets~--- the drought finally ends}

\begin{inparaenum}
  \verse{18:1} After quite some time\lit{(temporal setter) many days; however, we learn in just a bit that it's been three years so ``many days'' seems insufficient to show that.}~--- three years~--- the word of the \textsc{Lord} came to\lit{was} Elijah, saying, ``Go. Show yourself to Ahab. I will send rain on the face of the land.''%%
  \verse{18:2} So Elijah went and showed himself to Ahab. The famine was violent\alt{severe} in Samaria.%%
  \verse{18:3} Ahab called for Obadiah, who is in charge of the house. (Obadiah reverenced the \textsc{Lord} a lot.%%
  \verse{18:4} When Jezebel cut off the prophets of \textsc{God}, Obadiah took a hundred prophets and hid each fifty in a cave and provided them with bread and water.)%%
  \verse{18:5} Ahab said to Obadiah, ``Go throughout the land to all the fountains of water and wadis. If you find grass,\alt{leeks, reeds} use it\understood\ to keep the horses and mules alive, this so we don't kill off any of our cattle.''%%
  \verse{18:6} They split up the land between themselves in order\ed{idiomatically understood from \Hebrew{ל}} to pass through it. Ahab went down one road alone, Obadiah went down another\lit{one} road alone.%%
  \verse{18:7} Obadiah was in the road and, what do you know, Elijah was there to meet him. He recognized him and fell on his face and said, ``Are you he? My master Elijah?''%%
  \verse{18:8} He said to him, ``I am. Go, tell your master, `Elijah's here!'\thinspace''%%
  \verse{18:9} He replied, ``What have I done to be at fault\alt{How have I sinned} that you have given your servant into Ahab's power in order to kill him?%%
  \verse{18:10} The \textsc{Lord} your God lives. Is\lit{If} there no nation or kingdom to which my master hasn't sent you out to seek you there? They say, `He is not here.' The kingdom and the nation swore that they didn't find you.%%
  \verse{18:11} Yet now you're saying, `Go. Say to your master, ``Look, Elijah!''\thinspace'%%
  \verse{18:12} I'm going to leave\lit{go from} you and the Spirit of the \textsc{Lord} will carry you I don't know where. I've come to tell\alt{announce to} Ahab: he shall not find you, but he is going to kill me. I, your servant, have reverenced the \textsc{Lord} since my youth.%%
  \verse{18:13} Hasn't my master been told\lit{Hasn't it been told to my master} what I did when Jezebel killed the prophets of \textsc{God}? I hid one hundred men of the \textsc{Lord}'s prophets, fifty by fifty in a cave and fed them bread and water.%%
  \verse{18:14} And now you say, `Go, tell your master, ``Elijah's here!''\thinspace'? He'll kill me!''%%
  
  \verse{18:15} Elijah replied, ``As the \textsc{Lord} of Hosts lives, before whom I stand, I will definitely appear before him today.''%%
  \verse{18:16} Obadiah went to meet Ahab and announced it all\understood\ to him. So Ahab went to meet Elijah.%%
  \verse{18:17} When Ahab saw Elijah, Ahab said unto him, ``Are you the troubler\lit{the one who causes trouble} of\alt{in} Israel?''%%
  \verse{18:18} So he\ie{Elijah} said, ``I have not troubled Israel, rather you and your father's house have in forsaking the commandments of the \textsc{Lord} and going after Baalim.\ed{``false gods.'' The definite article precedes the noun, so ``Baalim'' is probably more accurate than ``false prophets.''}%%
  \verse{18:19} Send now and gather all of Israel unto me at Mount Carmel as well as the 450 prophets of Baal and the 400 prophets of Asherah who eat at Jezebel's table.''%%
  \verse{18:20} So Ahab sent unto all the children of Israel and gathered all the prophets unto Mount Carmel.%%
  \verse{18:21} Elijah drew near to all of the people and said, ``How long will you stall between two opinions?\lit{How long will you hop between two branches?} If the \textsc{Lord} is God, follow him; but if Baal, then follow him.'' And the people didn't say a thing.%%
  \verse{18:22} Then Elijah said unto the people, ``I alone remain a prophet of the \textsc{Lord}, yet the prophets of Baal are 450 men.%%
  \verse{18:23} Alright, have them\ie{the false prophets} bring\lit{let them give us} two bullocks: they shall choose one bullock for themselves, cut it in pieces, and place it on wood, but they shall not put any fire on it;\lit{place no fire} and I shall prepare the other bullock, place it on wood, and not light it.%%
  \verse{18:24} You shall call on the name of your god and I will call on the name of the \textsc{Lord}\ed{That's pretty powerful language (i.e., the Tetragrammaton) to be using around heathens.}. And thus it shall be that the god who answers by fire, He is God.'' And all the people answered and said, ``Sounds good.''\lit{Good is the word.}\ed{This brings up an interesting thought: either the prophets of Baal had before been able to call down fire from heaven or they knew that since they couldn't do it that Elijah couldn't either, and they would use that in the soon to come (or so they thought) argument.}%%
  \verse{18:25} And Elijah said unto the prophets of Baal, ``Choose a bullock and prepare it first because there are more of you.\lit{you are the multitude, you are many.} But don't put any fire on it, merely call on the name of your god.''\lit{Call on the name of your god and (but) place no fire.}%%
  \verse{18:26} So they took the bullock given to them, prepared it, and called on the name of Baal from the morning until noon, saying, ``Baal, answer us!'' Yet there was no voice\alt{sound} and no one answered. And they leapt on the altar they had made.%%
  \verse{18:27} And when it was midday, Elijah toyed with them and said, ``Call with a loud voice for he is a god and maybe he is meditating or busy or gone on an adventure.\alt{journey.} Maybe he's asleep and just waking up!''%%
  \verse{18:28} So they cried with a loud voice and cut themselves (according to their ordinances) with swords and spears until their blood flowed.%%
  \verse{18:29} And when midday had passed, they prophesied until the time of the evening sacrifice, and yet there was neither a voice nor an answer nor any that regarded them.%%
  \verse{18:30} Then Elijah said unto all the people, ``Come here.'' So all the people drew near as he repaired the altar of the \textsc{Lord} that was broken down.%%
  \verse{18:31} And he took twelve stones (according to the number of the tribes of the sons of Jacob~--- unto whom the word of the \textsc{Lord} was, saying, ``Thy name shall be Israel.'')%%
  \verse{18:32} and built an altar from the stones in the name of the \textsc{Lord}. And he made a trench about the space of two measures of seed encircling the altar.%%
  \verse{18:33} And he arranged the wood, cut the bullock in pieces, and placed it on the wood.%%
  \verse{18:34} \ed{Still verse~33 in English translations.} And he said, ``Fill four pitchers of water and pour them on the burnt offering and on the wood.'' And he told them to do it a second time, and they did it a second time; and he told them to do it a third time, so they did it a third time.%%
  \verse{18:35} \ed{back to normal versing} And the water encircled the altar insomuch that the trench was filled with water.%%
  \verse{18:36} And at eventide, Elijah the prophet drew near and said, ``\textsc{Lord} God of Abraham, Isaac, and Israel: let it be known today that Thou art God in Israel, that I am Thy servant, and that by Thy word I have done all of these things.%%
  \verse{18:37} Answer me, \textsc{Lord}. Answer me and this people shall know that Thou art the \textsc{Lord} God: that Thou hast turned back their heart.''\ed{i.e., that they shall know that Thou, Lord, hast turned back their hearts. Not me. Not this miracle that Thou shalt perform. But that Thou has done this thing.}%%
  \verse{18:38} Then the fire of the \textsc{Lord} fell and consumed the burnt offering, wood, stones, and dust; it licked up the water that was in the trench.%%
  \verse{18:39} And all the people saw it, fell on their faces, and said, ``The \textsc{Lord} is God! The \textsc{Lord} is God!''%%
  \verse{18:40} Elijah said unto them, ``Seize the prophets of Baal~--- don't let one of them escape!'' So they seized them. Then Elijah brought them down to the brook~Kishon and slew them there.\ed{That's 450~men! This doesn't even talk about the other 400~false prophets of Asherah who were probably slain as well.}%%
  \verse{18:41} Elijah said unto Ahab, ``Get up. Eat. Drink. You can hear a bunch of rain coming.''\lit{There is a sound of an abundance of rain.}%%
  \verse{18:42} So Ahab went up to eat and drink. Elijah went up to the summit of Carmel, bowed down on the earth, and put his face between his knees.%%
  \verse{18:43} Then he said unto his servant, ``Go up\ed{Go up where? They're at the summit.} and look at the sea.'' So he went up, looked, and said, ``There's nothing.''\lit{I don't see anything.} So Elijah said, ``Go again seven times.''%%
  \verse{18:44} And on the seventh time he said, ``Behold! There is a cloud~--- small as a man's hand~--- rising up out of the sea!'' And he\ie{Elijah} said, ``Go up and say to Ahab, `Prepare\ie{thy chariot} and go down that the rains do not restrain thee.'''%%
  \verse{18:45} In the meantime the heavens blackened with clouds and wind, and there was a great pouring of rain. And Ahab got on his chariot and went to Jezreel.%%
  \verse{18:46} Thus the hand of the \textsc{Lord} was upon Elijah, and he girded up his loins and ran before Ahab to the border of Jezreel.%%
\end{inparaenum}

  \heading{22}{Jehoshaphat and Ahab discuss going to battle against the Syrians at Ramoth-Gilead~--- they inquire of the prophets who each say that this is okay~--- Micaiah prophesies the same, but is adjured by Ahab to not lie to him~--- Micaiah then tells the king that the Lord is sending false revelation through His prophets so that Ahab will die at the battle at Ramoth-Gilead~--- Ahab is angry and throws Micaiah in jail~--- Micaiah tells him that if Ahab returns then the Lord doesn't speak through him~--- Ahab and Jehoshaphat go to battle at Ramoth-Gilead~--- the king of Syria commands his chariot captains to disregard the Israelite army and go after the king of Israel directly~--- Ahab is killed~--- Jehoshaphat rules righteously, but does not tear down the high places~--- Ahaziah rules unrighteously}

\begin{inparaenum}
  \verse{22:1} They\ca{\septuagint\ sg}{singular in the Septuagint} stayed three years without war between Aram\ie{Syria} and Israel.%%
  
  \verse{22:2} In the third year, Jehoshaphat king of Judah came down to the king of Israel.%%
  \verse{22:3} The king of Israel said to his servants, ``Don't you know that Ramoth-Gilead is ours? We are hesitating\alt{postponing, delaying, being silent} from taking it out of the hand of the king of Aram.''%%
  \verse{22:4} He said to Jehoshaphat, ``Will you go down with me\ca{\septuagint\superit{-L}\ \Greek{μεθ᾽ ἡμῶν}}{the Septuagint (except the textus Graecus ex recensione Luciani) says ``after us'' [instead of ``with me'']} to battle in\alt{at} Ramoth-Gilead?'' Jehoshaphat said to the king of Israel,\ca{\missing\ \septuagint*\peshitta}{missing in the Septuagint (textus Graecus originalis) and Peshitta} ``I am like you, my people like your people, my horses like your horses.''%%
  \verse{22:5} Jehoshaphat said to the king of Israel, ``Please inquire the word of\ca{\missing\ \septuagint}{[``the word''] missing in the Septuagint} the \textsc{Lord} today.''\alt{now, (understood) as soon as possible.''}%%
  \verse{22:6} The king of Israel assembled the prophets\ed{probably pagan prophets} (about 400~men\lit{400~in number}) and said to them, ``Should I go against Ramoth-Gilead to battle, or should I leave it alone?''\alt{forbear, discontinue, stop?} They said to him, ``Go up and the Lord\ca{\fragheb\ mlt Mss \tetragrammaton}{multiple manuscripts in the Cairo Genizah have the Tetragrammaton} will give it into the hand of the king.''%%
  \verse{22:7} Jehoshaphat said, ``Is there not still\ca{\missing\ \septuagint*}{missing in the Septuagint (textus Graecus originalis) [makes more sense this way]} a prophet of the \textsc{Lord} here that we might inquire of\alt{seek} him?''\ed{Jehoshaphat here specifies that he specifically wants to ask a prophet of the Lord.}%%
  \verse{22:8} The king of Israel said to Jehoshaphat, ``There is yet\ca{\missing\ \septuagint*}{missing in the Septuagint (textus Graecus originalis)} one man by whom we may\understood\ inquire of the \textsc{Lord}, Micaiah\ed{Hebrew for ``Who is like the \textsc{Lord}?''} son of Imlah, but I hate\alt{am unwilling to put up with} him because he does not prophesy good concerning me, rather evil.'' And Jehoshaphat said, ``Don't let the king speak thusly.''\alt{``Please don't say that.''}%%
  \verse{22:9} The king of Israel called to one court official,\halot{xxxx}{(eunuch who is a) \textbf{court official}} and said, ``Quickly! Micaiah son of Imlah.''%%
  \verse{22:10} The king of Israel and Jehoshaphat king of Judah were sitting, each on his throne, clothed in garments, in the threshing-floor,\ca{dub; \septuagint\ \Greek{ἔνοπλοι}}{doubtful [``clothed in garments, in the threshing-floor'']; the Septuagint has ``garments''} at the entrance of the gate of Samaria, and all the prophets prophesied before them.%%
  \verse{22:11} Zedekiah the son of Chenaanah made iron imitation,\lit{horns of iron}\halot{xxxx}{\textbf{horns} (part of body, not the material)\dots\ iron imitation \haref{1~K}{22}{11}} and said, ``Thus says the \textsc{Lord}: `You will knock down\alt{butt, thrust, gore} the Aram\ae{}ans with these until you have exterminated them.'\thinspace''%%
  \verse{22:12} All the prophets prophesied thus, saying, ``Go up to Ramoth-Gilead and be successful\alt{succeed, enjoy success} because the \textsc{Lord} will give it\understood\ into the control of the king.''%%
  \verse{22:13} The messenger who went to call Micaiah spoke to him, saying, ``Please, the words of the prophets are good towards the king with one accord.\lit{with one mouth.} Let your word please be like the word of one of them, and you will have spoken favorably.''\alt{well, good.''}%%
  \verse{22:14} Micaiah said, ``As the \textsc{Lord} lives, whatever the \textsc{Lord} says to me, that will I say.''\ed{Micaiah here shows the proper relationship that man is to have with God. We are not to ask the Lord to conform to what we want, but rather to be humble and be willing to do whatever he asks, even if it's against our will or to our own harm.}%%
  \verse{22:15} He came in to the king, and the king said to him, ``Micaiah, should we\ca{\septuagint\targum\super{f} sg}{singular in the Septuagint and Targum (codex Reuchlinianus (qui Olim \targum\super{L} dicebatur) secundum apparatum criticum Sperberi)} go down to Ramoth-Gilead to battle, or should we\ca{pc Mss \septuagint\targum\super{f} sg}{singular in a few manuscripts of the same} leave it alone?'' He said to him, ``Go up and succeed because the \textsc{Lord} will give it into the control of the king.''\ed{We have no real way of knowing, but it's possible that Micaiah said this dripping with sarcasm, thus prompting Ahab's response.}%%
  \verse{22:16} The king said to him, ``How many times do I have to urge\alt{adjure}\halot{xxxx}{\textbf{adjure}, \textbf{urge} (with an oath)} you that you only tell me the truth in the name of the \textsc{Lord}?''%%
  \verse{22:17} He said:\ed{There is a weird indentation here in the \textsc{bhs}; the colon is to symbolize that.} ``I've seen all of Israel scattered on\alt{to} the mountains like sheep without a shepherd.\lit{like sheep that to them there is not a shepherd} And the \textsc{Lord} has said, `These shall have no master. Let each of them return to his house in peace.'\thinspace''%%
  \verse{22:18} The king of Israel said to Jehoshaphat, ``Didn't I tell you that he doesn't prophesy good about me, but evil?''%%
  \verse{22:19} He\ie{Micaiah} said, ``Therefore, hear the word of the \textsc{Lord}: I have seen the \textsc{Lord}\ca{\septuagint\ + \Greek{θεὸν Ισραηλ}}{the Septuagint adds ``God of Israel''} sitting on His throne, and all the host of Heaven standing by Him, on His right and on His left.%%
  \verse{22:20} And the \textsc{Lord} said, `Who will lure\alt{persuade, entice} Ahab\ca{\septuagint\vulgate\ ut \caref{2~Ch}{18}{19} + \Hebrew{יִשְׂרָאֵל} \Hebrew{מֶלֶךְ}}{the Septuagint and Vulgate (as in \vref{2~Chr}{18}{19}) adds ``king of Israel''} that he go up and die\lit{fall} in Ramoth-Gilead?' And one said this and another said that.\lit{One spoke like thus, and another spoke like thus.}%%
  \verse{22:21} And the spirit went out and stood before the \textsc{Lord}, and said, `I will entice him.' And the \textsc{Lord} said, `How?'\lit{By what?}\ed{Some translations have ``And the \textsc{Lord} said, `How?'\thinspace'' as part of verse~22.}%%
  \verse{22:22} He said, `I will go and be a lying\alt{deceptive} spirit in the mouth of all of the prophets.' So He said, `You shall persuade him\understood\ and you shall also be successful. Go out and do so.'%%
  \verse{22:23} And now, the \textsc{Lord} has put a lying spirit in the mouth of all of your prophets and the \textsc{Lord} has spoken evil concerning you.''%%
  \verse{22:24} So Zedekiah son of Chenaanah slapped\lit{struck, smote} Micaiah on the cheek, and said, ``How did the spirit of the \textsc{Lord} pass over me to speak to you?''\lit{``Where is this, he's passed over, the spirit of the \textsc{Lord}, from me to speak to you?''}%%
  \verse{22:25} Micaiah said, ``You shall see on that day when you flee\lit{enter, come} into an inner room\ie{a room within a room}\alt{from one room to another} to hide yourself.''\ca{\fragheb\ mlt Mss et \caref{2~Ch}{18}{24} \Hebrew{בא}\hspace*{0em}---}{multiple manuscripts of the Cairo Genizah have [a different root, the more common verb ``to hide (oneself)'']}%%
  \verse{22:26} The king of Israel said, ``Take Micaiah and carry him back to Amon\ca{\septuagint\super{B19.82} \Greek{Σεμ(μ)ηρ} ex \Greek{Εμ(μ)ηρ} (cf \septuagint\super{135.93}) = \Hebrew{אִמֵּר}?}{the Septuagint (codex Vaticanus~19:82) has ``Sem(m)ir'' from ``Em(m)ir'' (compare the Septuagint~135:93) which might be ``Amer'' [instead of ``Amon'']} the city administrator,\ie{head of the city} and to Joash son of the king,%%
  \verse{22:27} and you shall say,\ca{\peshitta\vulgate\ ut \caref{2~Ch}{18}{26} \Hebrew{וַאֲמַרְתֶּם}}{the Peshitta and Vulgate (as in \vref{2~Ch}{18}{26}) have ``you (plural) shall say''} `Thus says the king:\ca{\missing\ \septuagint*}{[``thus says the king''] missing in the Septuagint (textus Graecus originalis)} ``Place this one in the prison\alt{house of confinement, house of imprisonment} and feed him short rations of bread and water\halot{xxxx}{2.\ \textit{mayim la\d{h}a\d{s}} (appositive) water such as is appropriate to hardship (siege), short rations of water \haref{1K}{22}{27}, so \textit{le\d{h}em la\d{h}a\d{s}} \haref{1K}{22}{27}} until I come\ca{Ms \septuagint\vulgate\ et 2~Ch \Hebrew{שׁוּבִי}}{a manuscript of the Septuagint and Vulgate and 2~Chr have ``return''} in peace.''\thinspace'\thinspace''\ed{This is pretty messed up. Ahab commands Micaiah to not lie to him, Micaiah tells the truth, and Ahab throws Micaiah in prison.}%%
  \verse{22:28} Micaiah said, ``If you actually return in peace,\ed{Meaning ``If you don't die like I said you would.''} the \textsc{Lord} hasn't spoken through me.'' And he said, ``Listen, O people! All of you!''\ca{\missing\ \septuagint*, gl cf \caref{Mi}{1}{2}}{[``And he said, `Listen, O people! All of you!'\thinspace''] missing in the Septuagint (textus Graecus originalis), glossed (compare \vref{Mic}{1}{2})}%%
  \verse{22:29} The king of Israel and Jehoshaphat king of Judah went up to Ramoth-Gilead.%%
  \verse{22:30} The king of Israel told\lit{said to} Jehoshaphat, ``Disguise yourself! and go into battle; put on your\ca{\septuagint\ suff 1 sg}{first person singular pronominal suffix in the Septuagint [i.e., ``my clothes'' (which would be a better disguise than his own clothes)]} clothes.'' And the king of Israel disguised himself and went into battle.%%
  \verse{22:31} The king of Aram commanded his 32~chariot captains who were with him,\lit{the captains of the chariots who were to him 32} saying, ``Don't fight with small,\ca{pc Mss et 2~Ch \Hebrew{הַקּ׳}}{a few manuscripts and 2~Chr have ``the small''} or great,\ca{pc Mss et 2~Ch \Hebrew{הַגּ׳}}{a few manuscripts and 2~Chr have ``the great''} but with the king of Israel directly.''\lit{only, by himself.''}%%
  \verse{22:32} When the chariot captains saw Jehoshaphat, they said, ``He's definitely the king of Israel!'' So they turned aside\ca{\septuagint\ ut \caref{2Ch}{18}{31} \Hebrew{וַיָּסֹבּוּ}}{the Septuagint (as \vref{2~Chr}{18}{31}) has ``encircled him''} to fight him, and Jehoshaphat cried out.%%
  \verse{22:33} When the chariot captains saw that he wasn't the king of Israel, they turned back from following him.%%
  
  \verse{22:34} A man innocently\ed{Has the sense of ``by chance,'' but here it is more probably ``innocently.'' That is, he shot (into the air?) and wherever it landed, it landed.} drew a bow and smote the king of Israel between the scales (of the coat of mail)\halot{xxxx}{2.\ `appendage,' i.e., \textbf{scales} of coat of mail} and the coat of mail.\halot{xxxx}{\textbf{coat of mail}, \textbf{scale-armor}}\ie{between his breastplate and armor} He\ie{the king of Israel} said to his chariot driver,\understood\ ``Turn around\lit{Turn (around) your hand} and take me out of the camp\ca{1 \Hebrew{הַמִּלְחָמָה}? cf \septuagint}{1 has ``the battle''? Compare the Septuagint} because I am severely wounded.''%%
  \verse{22:35} And the battle became more violent\lit{increased, went up}\halot{xxxx}{became more violent} that day, and the king was stood up in his chariot against\lit{opposite} the Aram\ae{}ans and died in the evening.\ca{\missing\ 2 Mss; \septuagint\ tr ad fin v}{missing in two manuscripts; the Septuagints transposes [``in the evening''] to the end of the verse} The blood of the wound spread\alt{poured out} into the midst of the chariot.%%
  \verse{22:36} A cry of lamentation\alt{moaning, joy} passed from one side of the camp to the other at sunset,\lit{the going of the sun} saying, ``Every man to his city and every man\ca{pc Mss \septuagint\vulgate\ \Hebrew{ואל}}{a few manuscripts of the Septuagint and Vulgate have [``Every man to his city] and to [his land'']} to his land.''\ed{If the subject of this sentence is the Syrians, this would be a cry of joy and everyone saying that since the king is dead (checkmate) to return home. Go home, battle's over. However, if this is the Israelites, it is probably a cry of lamentation and everyone returning home, tail between their legs. Although there does exist the possibility that since they might not have liked Ahab much, that it was still a cry of joy.}%%
  \verse{22:37} The king died\ca{1 frt \Hebrew{ה׳} \Hebrew{מֵת} \Hebrew{כִּי} (et cj c~36) cf \septuagint}{one perhaps has ``Because the king died,'' compare the Septuagint} and came\ca{1 \Hebrew{ויבֹאוּ}? cf \septuagint}{one has ``they came''? Compare the Septuagint} into Samaria, and they buried the king in Samaria.%%
  \verse{22:38} They washed off\ca{\septuagint\vulgate\ pl}{the Septuagint and Vulgate are in plural [i.e., ``They washed off'']}\alt{washed out} the chariot in the pool of Samaria, and the dogs lapped up his blood and the prostitutes bathed~--- according to the word of the \textsc{Lord} that he'd spoken.%%
  \verse{22:39} The remainder of the works of Ahab~--- everything he'd done~--- the ivory house he'd built, all the cities he'd built, aren't they written on the scroll of the xxxx xxxx of the kings of Israel?%%
  \verse{22:40} Ahab slept with his fathers, and his son Ahaziah reigned in his place.%%
  
  \verse{22:41} \ca{\septuagint\superit{L} om 41--51 (\septuagint* om 47--50) cf 16,28\super{a}}{the Septuagint (textus Graecus ex recensione Luciani omits verses~47--50 (the Septuagint (textus Graecus originalis) omits verses~47--50), compare chapter~16, verse~28 footnote~a [where it's added])}Jehoshaphat son of Asa ruled over Judah in the fourth\ca{16,28 \septuagint* \Greek{τῷ ἑνδεκάτῳ τοῦ Αμβρι}}{in the Septuagint (textus Graecus originalis) [in 3~Kings] 16\thinspace:\thinspace28 it says ``In the eleventh~year of Ambri [Ahab]'' [not the fourth~year]} year of Ahab king of Israel.\ie{Ahab's reign}%%
  \verse{22:42} Jehoshaphat was 35~years old\lit{a son of thirty and five~years} when he reigned,\ie{when he began to reign} 25~years old when he reigned in Jerusalem. His mother's name was Azubah daughter of Shilhi.%%
  \verse{22:43} He walked in all the ways\ed{understood idiomatically as plural} of his father, Asa; he didn't turn away from it, he did\lit{to do} what was right in the eyes of the \textsc{Lord}. %%
  \verse{22:44} \ed{In other translations, this is still part of verse~43. Every verse from here on is one less in other translations.}Only the high places\halot{xxxx}{4.\ (cultic) \textbf{high place}\dots\ associated with pagan worship and cultic prostitution} weren't removed. The people still sacrificed and had sacrifices go up in smoke on the high places.%%
  \verse{22:45} Jehoshaphat made amends\alt{peace, was at peace} with the king\ca{\septuagint\super{A}\peshitta\ pl}{the Septuagint (codex Alexandrinus) and Peshitta are in plural [i.e., ``with the kings of Israel.'']} of Israel.%%
  \verse{22:46} The rest of the deeds of Jehoshaphat, his might that he did, that he fought,\ca{\missing\ \septuagint* (sed hab in 16,28)}{missing in the Septuagint (textus Graecus originalis) (but it is in 16:28)} aren't they written in the scroll of the works xxxx of the kings of Judah?%%
  \verse{22:47} He removed the rest of the cultic prostitutes\alt{consecrated persons} who were left from the days of his father, Asa.%%
  \verse{22:48} There was no king in Edom,\ca{16,28 \septuagint* \Greek{ἐν Συρίᾳ} = \Hebrew{בַּאֲרָם}}{the Septuagint (textus Graecus originalis) in 16:28 has Aram (Syria)} the governor reigned.\ca{\missing\ \peshitta; \septuagint\superit{O}(\vulgate) et 16,28 \septuagint* \Greek{καὶ ὁ βασιλεύς} (cj c 49)}{missing in the Peshitta; the Septuagint (textus Graecus ex recensione Origenis) (and Vulgate) as well as the Septuagint (textus Graecus originalis) in 16:28 have ``and the king'' (connected with verse~49)}%%
  \verse{22:49} Jehoshaphat tithed\ca{nonn Mss Vrs ut Q \Hebrew{עָשָׂה}, sic 1}{several manuscripts in all or most translations as [have] qere [read] ``made,'' thus 1} ships\alt{fleet}\ca{16,28 \septuagint* sg}{the Septuagint (textus Graecus originalis) in 16:28 is in singular [``a ship''] [this goes for all later plural usages in this chapter]} at Tarshish to go to Ophir for gold; but they didn't go because the fleet was wrecked at Ezion-Geber.%%
  \verse{22:50} Then Ahaziah son of Ahab\ca{16,28 \septuagint* \Greek{ὁ βασιλεὺς Ισραηλ}}{the Septuagint (textus Graecus originalis) in 16:28 has ``the king of Israel''} said to Jehoshaphat, ``Let my servants go in the ships with your servants,'' but Jehoshaphat would not consent.%%
  \verse{22:51} Jehoshaphat lay with his fathers and was buried with his fathers\ca{\missing\ 2 Mss et 16,28 \septuagint\super{BNmin}}{[``and was buried with his fathers'' is] missing in two manuscripts and the Septuagint (codex Vaticanus, codex Basiliano-Vaticanus jungendus cum codice Veneto, codices minusculis scripti) 16:28} in the City of David, his father, and his son Jeroram ruled in his place.%%
  
  \verse{22:52} Ahaziah son of Ahab ruled over Israel in Samaria in the seventeenth\ca{\septuagint\superit{L} 24}{the Septuagint (textus Graecus ex recensione Luciani) has ``twenty-fourth''} year of Jehoshaphat king of Judah, and he ruled over Israel for two years.%%
  \verse{22:53} He did evil in the eyes of the \textsc{Lord} and walked in the way of his father and in the way of his mother and in the way\ca{1 \Hebrew{וּבְחַטֹּאת}? cf \septuagint}{1 has ``in the sins of''; compare the Septuagint} of Jeroboam son of Nebat who caused Israel to sin.%%
  \verse{22:54} He served Baal and worshiped him and provoked the \textsc{Lord} God of Israel according to everything that his father had done.%%
\end{inparaenum}

  
  \book{2 Kings}{\Hebrew{ב מלכים}}
  \heading{2}{Elijah is taken by a fiery carriage into heaven~--- the people doubt him and send a search party out to find his body~--- they're unsuccessful~--- the people petition Elisha to make their water potable, which he does~--- some children mock Elisha and are eaten by bears}

\begin{inparaenum}
  \verse{2:1} When the \lord\ took Elijah into Heaven in the gale,\alt{heavy windstorm} Elijah and Elisha departed from Gilgal.%%
  \verse{2:2} Elijah said to Elisha, ``Please stay here, because the \lord\ has sent me to Bethel.'' Elisha responded, ``As the \lord\ lives and as you\lit{your personality, individuality, life, desire, mood, state of mind, will, man, person} live, I will not leave\alt{abandon} you.'' So they went down to Bethel.%%
  \verse{2:3} The sons of the prophets who were in Bethel came out to Elisha, and said, ``Are you aware\lit{Do you know} that the \lord\ is taking your master from being in charge\lit{from your head} today?'' and he said, ``I know. Be quiet!''\alt{silent}%%
  \verse{2:4} Elijah said to him, ``Elisha, please stay here because the \lord\ has sent me to Jericho.'' And he responded, ``As the \lord\ lives and as you live, I will not leave you.'' So they came to Jericho.%%
  \verse{2:5} The prophets' children (in Jericho) approached Elisha, and said to him, ``Are you aware that the \lord\ is taking your master from being in charge today?'' and he said, ``I know. Be quiet!''%%
  \verse{2:6} Elijah said to him, ``Please stay here because the \lord\ has sent me to Jordan.'' And he responded, ``As the \lord\ lives and as you live, I will not leave you.'' So they both went on.%%
  \verse{2:7} Fifty men, sons of the prophets, went and stood aside, far off; and both of them stood by the Jordan.%%
  \verse{2:8} Elijah took his mantle,\ed{Traditionally rendered as \textit{mantle} and kept that way here.}\lit{robe (of state), fur garment} rolled it up, beat\alt{strike, hit} the water with it, and it divided on this side and the other.\alt{here and there} And they crossed over on dry ground.%%
  \verse{2:9} When they'd crossed over, Elijah said to Elisha, ``Ask what you'd have me do for you before I'm taken from you.'' And Elijah said, ``Please let there be a double portion of your spirit upon me.''%%
  \verse{2:10} He said, ``You've asked for difficult\alt{hard} thing. If you see me taken from you, then it shall be so to you; otherwise,\lit{and if not} it shall not be so.''%%
  \verse{2:11} As they were going about talking, a fiery chariot and fiery horses came\lit{separated} between them both, and Elijah went up in a gale to Heaven.%%
  \verse{2:12} Elisha saw it and cried out, ``My father, my father! The chariot of Israel and its horsemen!''\alt{riders} And he didn't see him again. He took hold\lit{strength} of his clothing and ripped it in two.\lit{two pieces}%%
  \verse{2:13} He picked up Elijah's mantle that had fallen off him and went back and stood on the bank of the Jordan.%%
  \verse{2:14} He took Elijah's mantle that had fallen off him, struck the water, and said, ``Where is the \lord, the God of Elijah? Where is He?''\lit{even He} He struck the water and it divided here and there, and Elisha crossed over.%%
  \verse{2:15} The children of the prophets in Jericho, on the opposite bank, saw him and said, ``Elijah's spirit has rested on Elisha!'' They came out to meet him and bowed to the ground in front of him.%%
  \verse{2:16} They said to him, ``Please pay attention. There are fifty men with your servants, qualified people\lit{sons, children}~--- please let them go. Seek your master lest the Spirit of the \lord\ takes him and throws him down onto a mountain or into a valley.'' And he replied, ``You shall not send.''\ie{You shall not send any men to go try and find him. It would be futile.}%%
  \verse{2:17} They strongly urged him until he was ashamed and said, ``Send.'' So they sent fifty men who\lit{and they} sought for three days, but didn't find him.%%
  \verse{2:18} They returned to him, and he was living in Jericho. He said to them, ``Didn't I tell you, `Don't go?'\thinspace''%%
  
  \verse{2:19} The men of the city said to Elisha, ``Look, please, the city's position is good, as my master sees it, but the water is bad, the land is barren.''%%
  \verse{2:20} He replied, ``Bring me a new dish and put salt in it.'' So they brought it to him.%%
  \verse{2:21} He went out to the water's source and threw salt in it,\lit{there} and said, ``Thus saith the \lord, `I've made this water drinkable. Death and sterility shall no longer come from it.'\thinspace''%%
  \verse{2:22} The waters are drinkable even\understood\ until now, according to the words that Elisha had spoken.%%
  
  \verse{2:23} And he went from there up to Bethel, and as he was going up on the highway, little boys came out of the city and mocked him, saying unto him, ``Go up, bald head! Go up, bald head!''%%
  \verse{2:24} And he turned back, faced them, and cursed them in the name of the \lord. And two mama bears came forth and out of the woods and tore 42 children among them.%%
  \verse{2:25} He then went from there up to Mount Carmel, and from there he returned to Samaria.%%
\end{inparaenum}

  \heading{5}{Naaman, an Aram\ae{}an, beseeches Elisha to cure him of his skin disease~--- Elisha, through his servant, commands Naaman to wash seven times in the Jordan River~--- at first, he disbelieves, but after being prompted by his own servant, he obeys and is healed~--- Naaman tries to give a gift to Elisha, is refused~--- Elisha's servant accepts and is struck with a similar skin condition}

\begin{inparaenum}
  \verse{5:1} And Naaman the head of the king of Aram's\ie{Syria} army was a great and honorable man before the his superior officer\alt{lord, master} because through him the \textsc{Lord} delivered Aram. Yea, he was a mighty man of valor, but he was a leper.%%
  \verse{5:2} And the Aram\ae{}ans had gone forth in bands and captured a little maid from out of the land of Israel; and she waited upon Naaman's wife.%%
  \verse{5:3} And she said unto her maidservant, ``I wish that my lord was in the presence of the prophet in Samaria! Then he might be healed of his leprosy.''%%
  \verse{5:4} So he went and told his lord, saying, ``Thus and thus said the maid that is from the land of Israel.''%%
  \verse{5:5} The king of Aram said, ``Go! I will send a letter to the king of Israel.'' So he went and took 10~talents of silver,\ed{\textsc{halot} states that a talent is ``normally about 35~kg\thinspace=\thinspace75~lbs.'' Ten talents of silver would be equal to roughly 750~lbs (340~kg) of silver, or \$215,687.50 as of 2014-04-26.} 6\thinspace000~gold pieces, and 10~changes of clothes with him.\lit{in his hand; however, this is impossible with this amount of silver.}%%
  \verse{5:6} He brought the letter to the king of Israel, saying, ``Now, as this letter comes to you, see that I have sent my servant, Naaman, to you that he may cure your skin disease.''%%
  \verse{5:7} When the king of Israel had read the letter, he rent his garments and said, ``Am I God? Can I\understood\ kill and give life? Because this man\understood\ has sent me to heal a man of his skin condition. Please be concerned about this and see because he's presenting himself to me.''%%
  \verse{5:8} When Elisha, the man of God, had heard that the king of Israel had rent his garments, he sent to the king, saying, ``Why have you rent your clothes? Please let him come to me and he shall know that there's a prophet in Israel.''%%
  \verse{5:9} So Naaman came with his horse and chariot and stood in the doorway of Elisha's house.%%
  \verse{5:10} Elisha sent a messenger to him, saying, ``Go. Bathe in the Jordan River seven times and your skin will be clean\alt{pure, genuine, (cultically) clean} again.''\alt{turn back, return, revert}%%
  \verse{5:11} And Naaman became angry and left. He said, ``Look, I said that\understood\ he would definitely come out to see me and stand and call on the name of the \textsc{Lord} his God and wave his hand over the place\ie{the affected place} and heal the skin condition.%%
  \verse{5:12} Aren't the Abana\ed{Modernly, the Barada River, the main river of Damascus, Syria.} and Pharpar\ed{Modernly, either the A`waj or Taura River in Damascus, Syria.} rivers in Damascus better than all the waters of Israel? May I not\alt{Can't, Shouldn't} just\understood\ wash in them and be clean?'' And he turned and left in a fury.\alt{anger}%%
  \verse{5:13} His servant came near, spoke to him, said, ``My father,\halot{xxxx}{fatherly \textbf{protector}, honorable title: of one's elder, prophet, husband.} if\understood\ the prophet had commanded you to do some great thing, wouldn't you have done it? How much more then when he tells you, `Wash, and be clean?'\thinspace''%%
  \verse{5:14} He went down and dipped in the Jordan River seven times according to the words of the man of God. His skin became\alt{turned back, returned} like a young man's\lit{a young (little) boy (youth, young man). However, \Hebrew{קָטֹן} is redundant in an English rendering and \textit{young man} suffices.} skin. He was clean.%%
  \verse{5:15} He and his entire company returned to the man of God. He came and stood before him and said, ``Please look. I know that there is no god in the entire world except in Israel. Now, please receive\lit{take. This doesn't work well idiomatically.} a blessing from your servant.''%%
  \verse{5:16} He said, ``As the \textsc{Lord}, before whom I stand, lives, I cannot receive it.'' He strongly urged him to take it, but he refused.%%
  \verse{5:17} Naaman said, ``If not, then please let a mule team's load of earth be given to your servant because he\lit{your servant} shall make no more burnt offerings and sacrifices to other gods, only to the \textsc{Lord}.%%
  \verse{5:18} In this matter, may\understood\ the \textsc{Lord} forgive your servant. When my master comes to the house in Rimmon,\ed{A Syrian cult image, only mentioned in this verse, identified as Baal.} bows himself there, and leans on my hand as I bow in the house of Rimmon~--- when I bow in the house of Rimmon, may the \textsc{Lord} please forgive your servant in this matter.''%%
  \verse{5:19} He said to him, ``Go in peace,'' and he went away a little.%%
  
  \verse{5:20} Gehazi, the servant of Elisha the man of God, said, ``Look, my master has taken good care of Naaman, this Aram\ae{}an, not to receive from his hand what he'd brought. As the \textsc{Lord} lives, I will run after him and take something from him,''%%
  \verse{5:21} so Gehazi pursued\lit{pursued after} Naaman. And Naaman saw someone running after him and he got down from his chariot to meet him, and said, ``Is everything alright?''\lit{Is there peace?}%%
  \verse{5:22} And he said, ``Everything's alright.\lit{Peace, All is well.} My master has sent me, saying, `Pay attention to this right now.\lit{Look now this.} Two young men from Mount Ephraim, sons of the prophets, are coming to me. Please give them a talent of silver\ed{75~lbs, \$21,284.38 as of 2014-05-03} and two changes of clothes.'\thinspace''%%
  \verse{5:23} Naaman said, ``Agree to take two talents,''\ed{dual form} and he urged him and tied up two talents of silver in two bags,\halot{xxxx}{xxxx can't find this noun in either Lexicon} two changes of clothes, and gave it to two of his servants who bore it before him.%%
  \verse{5:24} When he came to the Ophel,\ed{A high, fortified part of a city, either in the City of David or in the Old City of Jerusalem. Sometimes refers to a place in Samaria, the ancient capital of the kingdom of Israel.} he took them out of their possession\lit{hand} and put them away in the house. He sent the men away and they left.%%
  \verse{5:25} He entered and stood before his master. And Elisha said to him, ``Where did you come from,\lit{Whence} Gehazi?'' And he replied, ``Your servant didn't go here nor there.''%%
  \verse{5:26} He said to him, ``My heart didn't go when the man turned from his chariot to meet you.  Is it time to take silver? To take clothes? Olive yards, vineyards, sheep, oxen, servants, maids?%%
  \verse{5:27} Naaman's skin condition shall cling\alt{cleave} to you and your posterity forever.'' So he went out from his presence, suffering from a skin eruption like snow.%%
\end{inparaenum}

  \heading{6}{Elisha miraculously causes an ax head to float~--- the Syrian king surrounds Elisha on a hill~--- Elisha's servant is afraid so Elisha prays for his eyes to be opened and the servant sees angelic hosts surrounding and protecting them~--- Samaria is besieged and the people suffer~--- the Syrian king commands Elisha's head to be struck from his body, but Elisha is forewarned}

\begin{inparaenum}
  \verse{6:1} The sons of the prophets said to Elisha, ``Please, the place that we've been living in before you is too narrow for us.%%
  \verse{6:2} Please let us go to the Jordan River\understood\ and every one will bring a beam\ie{a framing beam} and there we'll make a place for ourselves where we can live.'' So he said, ``Go.''%%
  \verse{6:3} One of them\understood\ said, ``Please agree and go with your servants,'' and he said, ``I will go.''%%
  \verse{6:4} He went with them and they came to the Jordan River\understood\ and cut down the trees.%%
  \verse{6:5} As one of them\understood\ was felling the beam, the iron fell into the water, and he cried out and said, ``Ah!\halot{xxxx}{a cry for help} My master! It was borrowed!''%%
  \verse{6:6} The man of God said, ``Where did it fall?'' And he showed him the place. He cut a stick and threw it out there and made the iron float.%%
  \verse{6:7} He said, ``Pick it up.''\lit{Pick it up to you.} And he stretched out his hand and took it.%%
  
  \verse{6:8} The king of Aram\ie{Syria} hath been fighting against Israel and took counsel with his servants, saying, ``My encamping is wherever.''\ed{Seriously ``wherever.'' They use \Hebrew{פְּלֹנִי אַלְמֹנִי} which means ``whoever'' or ``whatever.'' It's the same usage as found in \vref{Ruth}{4}{1} to obfuscate who or what is being referred to.}%%
  \verse{6:9} The man of God sent unto the king of Israel, saying, ``Beware of passing through this place for the Aram\ae{}ans are coming down thence.''%%
  \verse{6:10} So the king of Israel sent to the place of which the man of God had told and warned him. And he\ie{the king of Israel} stayed on guard. This happened not once, but twice.%%
  \verse{6:11} And the heart of the king of Aram was troubled because of this thing\ie{these words}, so he called his servants and saith unto them, ``Will you not tell me which of us is for the king of Israel?''\ie{Is there a double agent among us?}%%
  \verse{6:12} One of the servants said, ``None, my lord the king. However, Elisha, the prophet that is in Israel, tells the king of Israel the things you have spoken in private.''\lit{in thy bedchamber.}%%
  \verse{6:13} Then he said, ``Go and see where he is. Then I will send for and fetch him.'' It was then told him, saying, ``He is in Dothan.''%%
  \verse{6:14} So he sent forth horses, chariots, and a great host, and they came by night and surrounded the city.%%
  \verse{6:15} And the man of God's servant arose early and went out, and lo! an army, horses and chariots, surrounded the city. Then his servant said unto him, ``My lord, what shall we do?''\lit{how do we do? Or, how will we do?}%%
  \verse{6:16} And he said, ``Don't be afraid~--- there are more with us than with them.''\alt{greater are they who are with us than they who are with them.}%%
  \verse{6:17} And Elisha prayed and said, ``\textsc{Lord}, I pray that thou wilt open his eyes and let him see.'' So the \textsc{Lord} opened the servant's eyes and he saw, and lo! the mountain was full of fiery horses and chariots surrounding Elisha.%%
  \verse{6:18} They went down to him and Elisha prayed to the \textsc{Lord} and said, ``Please smite this nation with blindness.'' And he smote them with blindness according to the word of Elisha.%%
  \verse{6:19} Elisha said to them, ``This is not the way. This is not the city. Follow me and I will bring you to the man whom you seek.'' And he led them to Samaria.%%
  \verse{6:20} When they entered Samaria, Elisha said, ``O \textsc{Lord}, open their eyes so that they can see!'' And the \textsc{Lord} opened their eyes and they saw. And they were in the midst of Samaria.%%
  \verse{6:21} When the king of Israel saw them, he said to Elisha, ``My father, shall I certainly smite them?''%%
  \verse{6:22} He said, ``Don't smite them. Would you smite those whom you've taken captive with your sword and your bow? Place bread and water before them so that they can eat and drink and return to their master.''%%
  \verse{6:23} He prepared many provisions for them, and they ate and drank. He sent them and they returned to their master and the Aram\ae{}an robbers\alt{military troops, raiding parties} never again came\lit{did no more continue to come} into the land of Israel.%%
  
  \verse{6:24} After this, Ben-hadad, king of Aram, assembled his entire army, went up, and besieged Samaria.%%
  \verse{6:25} There was a great famine in Samaria and they besieged it until a donkey head was worth\understood\ eighty pieces of silver and a fourth of a cav\alt{rendered \textit{cab}}\halot{xxxx}{a measure of capacity, about 1.5~L (1.33~qt).} of doves' dung\ed{Some translations render this as ``seed pods'' (\textsc{niv}) or ``wild onions'' (\textsc{njb}), but \textsc{halot} states that it is ``doves' dung'' (note that it is given as plural possessive where other translations render it as singular possessive). ``The Geneva Bible posits that the dung was used as a fuel for fire. Jewish historian Josephus suggested that dove's dung could have been used as a salt substitute. An alternative view is that `dove's dung' was a popular name for some other food, such as Star-of-Bethlehem or falafel. A third position, based on amending the Hebrew text, is that the passage actually refers to locust-beans, the fruit of the carob tree'' (Wikipedia). The Hebrew here is \Hebrew{חִרְייֹונִים} (the footnote gives \Hebrew{יוֺנִים חֲרֵי}), meaning ``doves' dung.''} was worth\understood\ five silver pieces.\ed{Some translations render this as being five shekels, but there is no way to know for certain that the unit of measurement is a shekel; the Hebrew simply says ``five silver.''}%%
  \verse{6:26} As the king of Israel passed from one side of the wall to the other, a woman cried to him, saying, ``O King! Help\alt{save, aid} me,\understood\ my master!''%%
  \verse{6:27} He said, ``If the \textsc{Lord} doesn't come to your aid, from where\alt{with what} do I save you?\ed{In other words, ``How am I supposed to save you?''} From the threshing floor? From the the wine press?''%%
  \verse{6:28} And the king said to her, ``What is troubling\understood\ you?''\lit{What to you? \textit{or} What is it to you?} And the woman replied, ``This woman said to me, `Give me your son. We'll eat him today and we'll eat my son tomorrow.'%%
  \verse{6:29} So we cooked\alt{boiled, roasted} my son and ate him. Then I said to her the next day, `Give me your son and we'll eat him,' but she hid her son.''%%
  \verse{6:30} When the king had heard the woman's words, he rent his garments. He passed from one side of the wall to the other and he saw the people and they were covered in sackcloth.%%
  \verse{6:31} He said, ``Thus does God do to me and more if Shaphat's son Elisha's head remain on him today!''%%
  \verse{6:32} Meanwhile, Elisha sat in his house and the elders sat with him. A man was sent before him, but before the messenger got to him, he said to the elders, ``Do you see that this murderer's son has sent someone\understood\ to remove my head? Watch for\understood\ when the messenger comes. Close the door and press him\halot{xxxx}{\textbf{crowd}, \textbf{press} s.one in a given direction} with the door. Isn't the sound of his master's feet following him?''%%
  \verse{6:33} While he was still\alt{yet} talking with them, the messenger came down to him and said, ``Hey! This evil is from the \textsc{Lord}. Why should I hope in\alt{wait on} \textsc{God} any more?''\alt{longer}%%
\end{inparaenum}

  \heading{7}{Elisha prophesies relief from the siege~--- four men with a skin condition go into the Syrian camp and find it empty~--- they tell the king~--- Elisha prophesies that the gatekeeper will be killed~--- both of Elisha's prophesies are fulfilled}

\begin{inparaenum}
    \verse{7:1} Elisha said, ``Hear the word of the \textsc{Lord}! Thus saith the \textsc{Lord}: `About this\understood\ time tomorrow, a seah\halot{xxxx}{measure of capacity, in one estimate~= approximately 7~L (7.5~qt)} of fine-ground wheat flour\halot{xxxx}{ground from inner kernels of wheat} shall be a shekel and two seah of barley shall be a shekel in the gate of Samaria.'\thinspace''%%
    \verse{7:2} The captain on whose hand the king had depended,\alt{leaned} answered the man of God and said, ``Look, if the \textsc{Lord} made windows\halot{xxxx}{in the wall through which smoke escapes, \textbf{chimney}; windows through which rain falls} in Heaven, would this thing happen?'' He said, ``You shall see it\understood\ with your eyes, but you shall not eat from it.''%%
    
    \verse{7:3} There were four men with a skin condition at the gateway\lit{the entrance/opening of the gate} and they said one to another, ``What? Are we going to sit here until we die?''%%
    \verse{7:4} If we say, ``Let's go into the city,''\alt{We shall come to the city} then the famine is\alt{there's famine in} in the city and we die there. If we stay here, we die. So now,\alt{And so} come and let's invade\alt{attack}\lit{fall on} the Aram\ae{}an camp~--- if we live, we live; if we die, we die.%%
    \verse{7:5} So they got up before sunrise\lit{twilight; either before sunrise or after sunset, but \textsc{halot} states that in this instance it's before sunrise.} to go to the Aram\ae{}an camp. When\understood\ they came to the outposts of the Aram\ae{}an camp, there wasn't a man there.%%
    \verse{7:6} The Lord made the Aram\ae{}an army hear the sound of chariots and horses and a great army, and they said one to another, ``The king of Israel has hired the king of the Hittites and the king of the Egyptians against us, to come against us!''%%
    \verse{7:7} So they got up and fled\alt{escaped, slipped away} before sunrise and left their tents, horses, donkeys~--- they left\understood\ the camp as it was and ran\lit{fled} for their lives.%%
    \verse{7:8} The men with the skin condition came to the edge of the camp, went into a\lit{one} tent, ate, drank, and took silver, gold, and garments from there, and they went and hid it. Then they came back, entered another tent, took stuff\understood\ from it, and went and hid it.%%
    \verse{7:9} They said one to another, ``We aren't doing right. Today is a day of good news\halot{xxxx}{either good or bad news, sometimes neutral.} and yet\understood\ we stay silent. If we wait for the morning light, the iniquity will find us. So now, let us go and come in and declare it\understood\ to the king's household.''%%
    \verse{7:10} They came and called the city gatekeepers and told them,\lit{saying} ``We came into the Aram\ae{}an camp and no one was there, not even\understood\ the sound of a man; but the horses were tied up, the donkeys were tied up, the tents were as they should be.''\thinspace\understood%%
    \verse{7:11} He called the gatekeepers and they told it to the king's house inside.%%
    \verse{7:12} The king got up in the night and said to his servants, ``Let\ed{\Hebrew{נָּא} is used here which can either mean ``please'' (which is not likely because it's the king addressing his servants) or it can be a logical connector, connecting the previous thought with the current thought.} me tell you what the Aram\ae{}ans have done to us: they know we're hungry and they've gone out of the camp and hidden in the field, saying, `When they come out of the city, we'll capture them alive and enter the city.'\thinspace''%%
    \verse{7:13} One of his servants answered and said, ``Please, let someone take five of the remaining horses which are left in the city\understood\lit{it}~--- they're like the entire Israelite crowd that remains in it, they're like the entire Israelite crowd that's died\alt{perished, been annihilated, been cut off} in it. Let's send them and see.''%%
    \verse{7:14} So they took two chariots and their horses, and the king sent them after the Aram\ae{}an army, saying, ``Go and see.''%%
    \verse{7:15} They went after them to the Jordan River.\understood\ The entire road was full of garments and equipment\alt{gear, clothes. Basically, accoutrements.} that the Aram\ae{}ans had thrown out in their hurriedness.\alt{haste} The messengers returned and told the king.%%
    \verse{7:16} The people set out and plundered the Aram\ae{}an camp. A measure of fine flour was a shekel and two measures of barley were a shekel, just as the \textsc{Lord} had said.%%
    \verse{7:17} The king appointed the adjutant\halot{xxxx}{the 3rd man in a war-chariot, who is the shield- and armer-bearer of the warrior}~--- upon whose hand he leaned~--- in charge of the gate. The people trampled him down in the gate and he died, just as the man of God had spoken, which he had said when the king came down to him.%%
    \verse{7:18} It happened just as the man of God had said to the king:\,\lit{saying} ``Two measures of barley shall be a shekel; a measure of fine flour shall be a shekel. So shall it be this time tomorrow in the gate of Samaria.''%%
    \verse{7:19} The adjutant answered the man of God, and said, ``If the \textsc{Lord} made windows in Heaven, would this thing happen?'' And he replied, ``Look, you'll see this with your\understood\ own eyes, but you won't eat of it.''%%
    \verse{7:20} And so it happened to him: the people trampled him down in the gate and he died.%%
\end{inparaenum}

  \heading{20}{Hezekiah becomes ill and Isaiah tells him to get his house in order~--- Hezekiah humbly prays and beseeches the Lord to remember his righteousness~--- the Lord hears Hezekiah's prayer and prolongs his life~--- the sundial moves backwards as a sign that the Lord has heard Hezekiah's prayer\ed{As if Isaiah miraculously knowing of Hezekiah's prayer was not enough.}~--- the Babylonian king sends messengers to Hezekiah who, in his sick and delirious state, shows them the secrets of his kingdom~--- Isaiah prophesies that everything shall be carried away into Babylon~--- Hezekiah dies and his son, Manasseh, reigns in his stead}

\begin{inparaenum}
    \verse{20:1} In those days, Hezekiah became mortally ill. The prophet\ed{Ambiguous whether this refers to Isaiah or Amoz. Talmudic tradition states that if a prophet's father's name is given that the father is a prophet.} Isaiah, son of Amoz, came in to him and said,\lit{to him} ``Thus saith the \textsc{Lord}: `Set your house in order\halot{xxxx}{prepare for death} because you're dying and you won't live through it.'\thinspace''%%
    \verse{20:2} He turned his face to the wall and prayed to the \textsc{Lord}, saying,%%
    \verse{20:3} ``O \textsc{God}, please remember me. Remember\understood\ that I've walked before you in truth, with a peaceful heart. I've done that which is good in your sight.'' And Hezekiah wept bitterly.\lit{a lot}%%
    
    \verse{20:4} Isaiah had not gone out of the middle quarter when the word of the \textsc{Lord} came to him, saying,%%
    \verse{20:5} ``Return and tell Hezekiah, the leader of My people: `Thus saith the \textsc{Lord}, God of your forefather David: I have heard your prayers, I have seen your tears.\lit{prayer\dots\ tear} I will heal you. On the third day, go up to the House of the \textsc{Lord}%%
    \verse{20:6} and I will add fifteen years to your life.\lit{days.} I will deliver you and this city from out of the hand of the king of Assyria~--- I will protect this city for My sake and for the sake of My servant, David.''%%
    \verse{20:7} Isaiah said, ``Take fig cakes.'' So they took and laid them on the inflamed spots\halot{xxxx}{\textbf{boils}, possibly \textbf{smallpox}} and he lived.%%
    
    \verse{20:8} Hezekiah said to Isaiah, ``What is the sign that the \textsc{Lord} has healed me? That I shall go up to the House of the \textsc{Lord} on the third day?''%%
    \verse{20:9} Isaiah said, ``This shall be your sign\lit{a sign to you} from the \textsc{Lord}: the \textsc{Lord} shall do the thing that He's said he'll do:\understood\ will the shadow of the sundial\alt{steps, degrees. Also possible that they measured time by how many steps up a staircase the sun/shadow went.} go forward ten or backwards ten?''%%
    \verse{20:10} Hezekiah said, ``It is easy for the shadow to go down ten degrees; rather, let the shadow go\lit{return} backwards ten degrees!''%%
    \verse{20:11} The prophet Isaiah cried to the \textsc{Lord} and He brought the shadow back by the amount it had gone down on Ahaz's sundial: ten degrees backwards.%%
    
    \verse{20:12} At that time, Berodach-baladan, son of Baladan, king of Babylon, sent letters and a gift to Hezekiah because he'd heard that Hezekiah was sick.%%
    \verse{20:13} Hezekiah obeyed them\ed{It's important to note that Hezekiah was sick and not thinking straight.} and showed them everything in the treasury:\lit{treasure-house} the silver and the gold and the balsam\alt{balsam shrub, balsam oil (which easily congeals), perfume (in general), sweet-smelling cinnamon, (sweet) cane)} and the good oil\halot{xxxx}{either olive (\textit{Olea europ\ae a}), oleaster (\textit{Eleagnus hortensis}), or Aleppo pine (\textit{Pinus halepensis}).} and the house of vessels\alt{armor (though not attested in \textsc{halot}), items, equipment, clothing; \textsc{halot} notes that, in the wildest sense, it is any sort of useful object.} and everything that had been found in his storehouses.\alt{supplies, treasure} There wasn't a single\understood\ thing\lit{There was nothing} that Hezekiah didn't show them throughout his house and his entire dominion.\alt{authority}%%
    \verse{20:14} The prophet Isaiah came in to King Hezekiah and said to him, ``What did these men say? Where do they come from?'' Hezekiah replied, ``They come from a faraway land: Babylon.''\lit{from Babylon.}%%
    \verse{20:15} He said, ``What did they see in the house?'' and Hezekiah said, ``They saw everything in my house. There isn't a single thing among my treasures that I didn't show them.''%%
    \verse{20:16} Isaiah said to Hezekiah, ``Listen to the word of the \textsc{Lord}:%%
    \verse{20:17} `Look, the days are coming when everything in your house and everything\understood\ that your father has stored up until this day shall be carried into Babylon. Nothing will be left,'\alt{remain} says the \textsc{Lord}.%%
    \verse{20:18} Of your sons that go out from you, whom you've fathered: they shall be taken away and become eunuchs\ed{It is unclear if they are simply eunuchs or eunuchs that serve in the capacity of court officials.} in the Babylonian king's palace.''%%
    \verse{20:19} Hezekiah said to Isaiah, ``The word of the \textsc{Lord} that you've spoken is good,'' and he said, ``Is it not so if there's peace and truth in my days?''%%
    
    \verse{20:20} The remainder of Hezekiah's acts, all his might, that he made the pool and the aqueduct\lit{circuit for water} and brought water into the city~--- aren't these written written in the book of the kings of Judah's lifetimes?%%
    \verse{20:21} Hezekiah lay with his fathers. And his son, Manasseh, reigned in his stead.\alt{place}%%
\end{inparaenum}

  
  \book{2 Chronicles}{\Hebrew{ב הימים דברי}}
  \heading{26}{The people install Uzziah as king~--- Uzziah, like his father, was righteous and God made him prosper~--- he goes to war, becomes vain, and tempts God~--- he makes an illegal sacrifice and is cursed with a serious skin condition}

\begin{inparaenum}
  \verse{26:1} All the Judahites\lit{the people of Judah} took Uzziah, who was sixteen years old, and crowned\alt{made, installed} him king instead of his father, Amaziah.%%
  \verse{26:2} He built Eloth and restored it to Judah after the king lay with his fathers.\ie{after death}%%
  
  \verse{26:3} Uzziah was sixteen years old when he became king. He ruled in Jerusalem for fifty-two years. His mother's name was Jecholiah of Jerusalem.%%
  \verse{26:4} He did what was right in the eyes of \god\ like everything that his father Amaziah had done.%%
  \verse{26:5} He worshiped\alt{sought, cared about, inquired regarding} God in the days of Zechariah (who understood\alt{perceived, paid attention to, considered, gave heed to, noticed} the visions of God). In the days he worshiped the \lord, God made him prosper.%%
  
  \verse{26:6} He went and fought\alt{do battle with, come to close quarters with} against the Philistines and made a breach in\alt{broke down} the wall of Gath, the wall of Jabneh, the wall of Ashdod. He built cities in Ashdod and Philistia.%%
  \verse{26:7} God came to his aid\alt{supported, helped} against the Philistines, the Arabians living in Gur-Baal, and the Maonites.%%
  \verse{26:8} The Ammonites gave a present to Uzziah and his name went to the entrance of Egypt because he became incredibly\lit{lifted way up, escalated. From \Hebrew{עלה} meaning to lift up.} strong.%%
  \verse{26:9} Uzziah built towers\halot{xxxx}{in a defensive wall} in Jerusalem by the corner gate, the valley gate, and at the corner, and he fortified\alt{strengthened} them.%%
  \verse{26:10} He built towers in the desert and hewed out\alt{quarried; other translations render this as ``dug,'' but \textsc{halot} only mentions stonework.} many wells because he had a lot of cattle, both in the lowland and in the plains;\alt{on the plateau, \Hebrew{מִישׁוֺר} meaning both plain and plateau.} he had\understood\ serfs\halot{xxxx}{peasant, not owning land, belonging to landlord} and wine-growers\alt{vinedresser} in the mountains and in Carmel because he loved growing stuff.%%
  
  \verse{26:11} Uzziah had an army\halot{xxxx}{alt., wealthy landowner, qualified, fit for military service; (large) landowner, obligated to military service and the furnishing of a certain number of men; then valiant man without regard to property; all the armed host of a people and a satrapy (a provincial governor in ancient Persia).} who was making war, going out by troop as commissioned\lit{listed by name} by the hand of Jeiel the scribe and Maaseiah the record-keeper and by the hand of Hananiah, the second after the king.%%
  \verse{26:12} The whole number of the head honchos\lit{chief fathers} of the elite troops was 2\thinspace600.%%
  \verse{26:13} With regard to the strength\alt{power; lit., hand} of the army,\alt{host} 307\thinspace500~warriors who made war with great ability\alt{power, capacity, strength, means} to support\alt{help, come to the aid of} the king against the enemies.%%
  \verse{26:14} For the entire army, Uzziah prepared shields, spears, helmets, scale-armor,\alt{coat of mail} bows, and slinging-stones for them.%%
  \verse{26:15} In Jerusalem, he made war machines\halot{xxxx}{(skillfully contrived) \textbf{war-machines}, spec. catapults}\ed{Not a ballista. The earliest ballistas were invented circa 400~\textsc{b.c.}}~--- thought out\alt{invented} by technicians~--- to be placed\understood\ on the towers and the corners to shoot arrows and boulders.\lit{great stones} His name spread some distance\lit{out a distance} because he was extraordinarily helped  until he became strong.%%
  \verse{26:16} However, when he become strong, his heart was lifted up unto destruction. And he transgressed against the \lord\ his God by going unto the temple of the \lord\ and offering incense upon the altar of incense.%%
  \verse{26:17} And Azariah the priest went in after him with the priests of the \lord: eighty valiant men.%%
  \verse{26:18} And they withstood Uzziah the king and said unto him, ``This is not for thee, Uzziah, to burn incense unto the \lord; rather for the priests~--- the sons of Aaron~--- that are set apart\alt{consecrated, sanctified} to burn incense. Leave this holy place\lit{Go out from this sanctuary} for you have transgressed. And neither shall this be for thine honor from the \lord\ God.''%%
  \verse{26:19} Uzziah was wroth (and in his hand he had a censer of incense). And while he was angry at the priests, leprosy appeared\lit{rose up} in his forehead in the presence of the priests in the House of the \lord\ next to the altar of incense.%%
  \verse{26:20} Azariah, the chief priest, and all the priests looked upon him, and lo! he was leprous in his forehead. So they thrust him out from thence, yea, even he\ie{Uzziah} hurried out because the \lord\ has smitten him.%%
  \verse{26:21} Uzziah the king was leper unto his dying day~--- he dwelt in a separate house (being a leper) because he was cut off from the House of the \lord. And Jotham, his son, took over\lit{was over} the king's affairs,\lit{house} judging the people of the land.%%
  \verse{26:22} The rest\alt{remainder} of the history of Uzziah, the first and the last, was written by the prophet Isaiah, the son of Amoz.%%
  \verse{26:23} Uzziah lay with his fathers,\ie{in death} and they buried him with his fathers in the kings'\lit{that the kings have} cemetery,\lit{field of the graves} for the said, ``He had a skin condition.'' His son, Jotham, reigned in his stead.%%
\end{inparaenum}

  \heading{31}{The Israelites destroy all the false worship sites in their land~--- Hezekiah appoints priests to serve~--- the people are given goods from King Hezekiah and pay a tithe on their increase~--- the people, as a direct result of their tithing, now have enough to eat~--- the Levites' services enumerated~--- Hezekiah served his God faithfully}

\begin{inparaenum}
    \verse{31:1} When all this was finished, all of the Israelites who were found went out to the cities of Judah and smashed\alt{shattered; this doesn't work as well in context.} the standing-stone,\halot{xxxx}{usually an unhewn, upright stone for cult, burial-marking, or memorial purposes.} cut down the Asherah statues,\alt{shrines}\ed{The Hebrew simply says Asherah (lit., as \Hebrew{הָאֲשֵׁרִים}, the plural of \Hebrew{אֲשֵׁרָה}). In Ugarit, Asherah is a goddess, the wife of El and the mother of the gods (\textsc{halot} xxxx: get full quote and quote it). Since it is in plural, it refers to something representing her, either statues, shrines, or something used to represent her, probably for cultic worship.} and tore down\alt{broke up, demolished} the high places\halot{xxxx}{associated with pagan worship and cultic prostitution.} and altars in all of Judah and Benjamin, in Ephraim and Manasseh, until they'd all been destroyed.\lit{until completion} All the children of Israel returned, everyone to their property\alt{landed property, city where they own land} in their cities.%%
    
    \verse{31:2} Hezekiah stationed\alt{appointed} the division of priests and Levites according to their divisions~--- everyone appointed, priests and Levites~--- for burnt offerings and peace offerings, to minister\alt{serve} and to praise\alt{give voice to (praise and thanksgiving)} in the gates of the camps of the \textsc{Lord}.%%
    
    \verse{31:3} The king apportioned\alt{counted out, divided in parts} from his property\halot{xxxx}{goods (gained by work, not by purchase); goods (furnishings, gear, utensils); baggage-train; personal property, domain (of king)} for the morning and evening burnt offerings, for Sabbath-day\lit{Sabbath; more accurately ``\textit{Shabbat},'' the Jewish weekly sabbath observed from sundown Friday through sundown Saturday.} burnt offerings, for new moons, for appointed feasts, as it is written in the law of the \textsc{Lord}.%%
    \verse{31:4} He said to the people who lived in Jerusalem to give the priests' and Levites' portion in order that they may be strengthened in the law of the \textsc{Lord}.%%
    \verse{31:5} As the matter spread forth, the children of Israel multiplied the first-fruits of corn, new wine, oil, honey, and all the increase of the field: they brought in the tithe of the whole in abundance.%%
    \verse{31:6} The children of Israel and Judah (those dwelling in the cities of Judah) also brought forth a tithe of their herds and flocks,\ed{``[T]heir'' is taken from context (unless they're presenting a tithe of things they do not own in which case they're following the poor example of Saul (cf. \vref{1~Sam}{15}{15})) and the plurality is given for idiomatic purposes.} as well as a tithe of the holy things that are consecrated unto the \textsc{Lord} their God: heaps and heaps were brought in.%%
    
    \verse{31:7} In the third month they began to lay the foundation of the heaps and finished in the seventh month.%%
    
    \verse{31:8} Hezekiah and the heads of the people\lit{heads. ``[O]f the people'' is understood, unless we're to assume that Hezekiah came in with a bunch of heads.} came in to see the heaps and to bless the \textsc{Lord} and His people Israel.%%
    
    \verse{31:9} Hezekiah asked\alt{inquired at} the priests and the Levites about the heaps%%
    \verse{31:10} and Azariah, the chief priest (of the house of Zadok), spoke unto him and said, ``Since we began taking in the heave offerings to the House of the \textsc{Lord}, there's plenty to eat, be satisfied, and leave abundantly for the \textsc{Lord} hath blessed his people, and this is the surplus.''%%
    
    \verse{31:11} Hezekiah ordered\alt{commanded} them\understood\ to prepare\alt{ordered that [understood] the chambers be prepared} the hall\halot{xxxx}{a room on three walls of which were benches where worshipers ate sacrificial meal, fourth open to the courtyard.} of the House of the \textsc{Lord}, and they prepared it.\understood%%
    \verse{31:12} They faithfully brought in the heave offering and the tithe and the holy things; the leader over them was Conaniah the Levite and his brother, Shimei, was second in command.\understood%%
    \verse{31:13} Jehiel, Azaziah, Nahath, Asahel, Jerimoth, Jozabad, Eliel, Ismachiah, Mahath, and Benaiah were officers\alt{commissioners} under the authority\lit{hand} of Conaniah and his brother, Shimei, by King Hezekiah and Azariah the leader of the House of God's appointment.%%
    \verse{31:14} Kore, the son of Imnah the Levite, the eastward gatekeeper, was in charge of\lit{over} the free-will offerings of God, the distribution\alt{giving} of\lit{to distribute; however, this enumeration better lends itself to nouns than infinitive verbs.} the heave offerings of the \textsc{Lord}, and the holiest items.\alt{things}%%
    \verse{31:15} Under his command\lit{hand} were Eden, Miniamin, Jeshua, Shemaiah, Amariah, and Shecaniah, in the cities of the priests, who\understood\ faithfully distributed to their brothers in their divisions, to great and small,%%
    \verse{31:16} except the men, three years old\lit{the son of three years} and up, who were registered in a genealogical table\halot{xxxx}{\textbf{have oneself registered in a genealogical table} by the establishment of one's descent.}~--- everyone who came into the House of the \textsc{Lord} for their day to day work, for their service in their observance and according to their divisions~---%%
    \verse{31:17} along with the priests who were registered in their fathers' genealogical table; the Levites, twenty years old and up, in their observance and divisions;%%
    \verse{31:18} and their small children, wives, sons, and daughters~--- the whole congregation of those entered in the genealogical table~--- because they, in their faithfulness, sanctified themselves to be holy.%%
    \verse{31:19} For the descendants of Aaron, the priests, in the fields and in the pasture land\halot{xxxx}{(belonging to a city), the belt of land outside the walls under the jurisdiction of a city.} of their cities, in all their many cities, were men, specified by name, who gave portions to every male priest and to all the Levites who were registered in the genealogical table.%%
    \verse{31:20} Thus Hezekiah did in all of Judah~--- he did what was good and right and true before the \textsc{Lord} his God.\ed{What a good man.}%%
    \verse{31:21} In everything he began for the service of the House of God and for the Law and for the commandments, to seek his God with all his heart, he did and was successful\alt{succeeded} at.\understood%%
\end{inparaenum}

  \heading{32}{The Assyrians besiege Jerusalem~--- Hezekiah encourages the people~--- Sennacherib blasphemes and tries to make the people doubt Hezekiah~--- Hezekiah and Isaiah pray and the Lord kills the Assyrian leaders~--- Hezekiah's last doings and his obituary}

\begin{inparaenum}
    \verse{32:1} After these\lit{the} things and this truth, Sennacherib king of Assyria entered and came into Judah; he encamped against the inaccessible cities and said to breach them for himself.%%
    \verse{32:2} But Hezekiah saw that Sennacherib had come and that his face was against Jerusalem.%%
    \verse{32:3} He counseled with his leaders and his young\halot{xxxx}{sometimes, \textit{strong}} men to stop the fountains of waters outside the city, and they helped him.%%
    \verse{32:4} Many people gathered and they dammed up\alt{stopped} all the fountains and the seasonal river that floods throughout the midst of the land, saying, ``Why do the Assyrian kings come and find a lot of water?''%%
    \verse{32:5} He strengthened himself, rebuilt\lit{built up}\thinspace\understood\ the entire wall that was breached, raised up defensive walls,\alt{towers} built\understood\ another wall outside, strengthened Millo (in the City of David), and made javelins\alt{darts} and shields in abundance.\ed{It's ambiguous whether ``in abundance'' refers to just the javelins or to both the javelins and shields.}%%
    \verse{32:6} He put the battle chiefs over the people and assembled them to himself to a plaza\halot{xxxx}{(open) \textbf{square}, \textbf{plaza} (of a town, village)} gate of the city, and spoke to their heart, saying,%%
    \verse{32:7} ``Be strong and brave. Don't be afraid. Don't be discouraged\alt{filled with terror, terrified, struck down} before\lit{the face of} the king of Assyria or \lit{before}the multitude that is with him because there are more with us than with him.%%
    \verse{32:8} With him is an arm of flesh, but with us is the \textsc{Lord} our God: to help us, to fight our battles.''\ed{\S~--- this is so powerful} And the people were supported by the words of Hezekiah king of Judah.%%
    
    \verse{32:9} After this, Sennacherib king of Assyria sent his servant\ed{Other translations render this in plural (i.e., ``servants'') but it seems to be singular to me.} to Jerusalem~--- but he was by Lachish\ed{around 45~mi (72~km) from Jerusalem} and all of his military forces with him~--- against Hezekiah king of Judah, against all Judah who were in Jerusalem, saying,%%
    \verse{32:10} ``Thus says Sennacherib king of Assyria: `Why\halot{xxxx}{\textit{`al-m\^a} on what basis, \textbf{why}?}\ed{This carries an interesting theological component: Sennacherib is calling them out on their faith. The other rendering, ``On what basis\dots'' is less blasphemous and more inquisitive; however, the former seems to be more true to Sennacherib's character.} do you trust\alt{trust, feel safe, full of confidence} and \ed{understood: still}live in the siege of Jerusalem?%%
    \verse{32:11} Isn't Hezekiah persuading you, to give you to die by famine and thirst, saying, ``The \textsc{Lord} our God will deliver us from the hand\lit{palm} of the king of Assyria''?%%
    \verse{32:12} Hasn't Hezekiah removed his high places\halot{xxxx}{associated with pagan worship and cultic prostitution} and altars, and said to Judah and Jerusalem, saying, ``Bow before one\ed{\textsc{darby} emphasizes ``one.''} altar and make sacrifices\understood\ go up in smoke''?%%
    \verse{32:13} Don't you know what I've done, me and fathers, to all of the people of the earth? Were the gods of the land in any way able to deliver their land from my power?%%
    \verse{32:14} Who among all the gods of those nations whom my fathers have destroyed\halot{xxxx}{\textbf{devoted to the ban}\dots\ dedicate to destruction} that have been able to deliver his people from out of my power, that your God is able to deliver you from my power?%%
    \verse{32:15} So now, don't let Hezekiah lift you up, don't let him lead you astray like this,\alt{stir you up, get you provoked} and don't believe him, because no god of any nation or kingdom is able to deliver his people from my power or from the power of my forefathers; also because your God won't\alt{doesn't} deliver you from my power.'\thinspace''%%
    
    \verse{32:16} His servants have once again\lit{again} spoken against the \textsc{Lord} God and against His servant Hezekiah.%%
    \verse{32:17} He wrote scrolls to revile\alt{reproach, taunt} the \textsc{Lord} God of Israel; to speak against Him, saying, ``Like the gods of the nations of the earth who won't deliver their people from my power, so shall the god\ed{lowercase because remember who's speaking} of Hezekiah not deliver his people from my power.''%%
    \verse{32:18} They\ca{pc Mss \septuagint*\vulgate\ sg}{a few Hebrew manuscripts, the Septuagint, and Vulgate are in singular} called in Hebrew\lit{Jewish} with a great voice to the people of Jerusalem who were on the wall to make them afraid, to terrify them, to trouble them, so they could capture the city.%%
    \verse{32:19} They\ca{\septuagint\vulgate\ sg}{the Septuagint and Vulgate are in singular} spoke against the God of Jerusalem like they speak against\understood\ the gods of the land, works of man's hands.%%
    
    \verse{32:20} King Hezekiah and Isaiah the prophet, son of Amoz, prayed about this and cried to Heaven.%%
    
    \verse{32:21} The \textsc{Lord} sent a messenger and destroyed\alt{hid} every mighty man of valor, every chief and leader in the camp of the king of Assyria, and who\understood\ returned to his land with a face full of shame. When he came to the house of God, those who came from\ca{K lapsus calami}{written [the way it is because of a] slip of the pen} his bowels made him fall on his sword there.%%
    \verse{32:22} The \textsc{Lord} saved Hezekiah and the inhabitants of Jerusalem from the power of Sennacherib king of Assyria, and from the power of everyone~--- He guided\halot{xxxx}{\textbf{guide}, \textbf{help along}, \textbf{lead carefully} (especially the handicapped)} the about.%%
    \verse{32:23} Many brought gifts to the \textsc{Lord}, to Jerusalem, and choice gifts\ed{Does this say anything about the people, that they brings gifts to God, but choice gifts to Hezekiah? Or is that simply eisegesis in translation?} to Hezekiah king of Judah. And he was exalted in the eyes of all the nations after this.%%
    
    \verse{32:24} In those days, Hezekiah was deathly sick,\lit{sick to death} and he prayed to the \textsc{Lord} and spoke to Him and gave Him a sign.%%
    \verse{32:25} Hezekiah didn't do\alt{render} to him according to the thing\alt{benefit, deed}\ed{This sentence uses a Hebrew grammatical structure that doesn't exist in English so it's really hard to render, let alone idiomatically. Literally, it's ``Not according to done upon him the thing Hezekiah.'' It's a mess.} because of the haughtiness of his heart, and there was anger\alt{rage} upon him, Judah, and Jerusalem.%%
    \verse{32:26} Hezekiah humbled himself because of the haughtiness of his heart, he and the inhabitants of Jerusalem,\ed{who are understood to have been haughty} and the \textsc{Lord}'s anger did not come upon them in the days of Hezekiah.%%
    \verse{32:27} Hezekiah had tons of\lit{very much} riches and lots of\lit{very much} honor, treasures he'd made for himself of silver and gold and precious stone, for balsam oil\alt{perfume} and shields and for every valuable\alt{precious} vessel;%%
    \verse{32:28} warehouses\halot{xxxx}{\textbf{stores}, \textbf{supplies}, \textbf{storerooms}\dots\ military bases with armories and supply depots} to increase corn, wine, oil, stables for every cattle, and flocks\alt{herds} for stables;%%
    \verse{32:29} cities\ca{1 \Hebrew{וַעֲדָרִים}}{one has ``flocks''} he's made for himself, possessions of many flocks and cattle, because God has given him so much\lit{very much} property.%%
    \verse{32:30} But Hezekiah, he plugged up\ed{Possibly ``dammed up.'' This verb has the sense of stopping or plugging up, but it says he did it at the outlet, which could either be interpreted as the source (interesting that the word for source was not used) or upper outlet (debatable what that refers to), in which case it would be a dam.} the upper outlet of the waters of Gihon and directed them beneath\alt{below} the westward side\understood\ of the city of David. And Hezekiah prospered in all of his works.%%
    \verse{32:31} Thus, with the prince of Babylon's envoys\halot{xxxx}{1. official, \textbf{middleman}: a) \textbf{interpreter} (of foreign language) \haref{Gn}{42}{23}; b) \textbf{envoy} \haref{2C}{32}{31}; c) \textbf{intermediary}, \textbf{mediator} (i.e. prophet) \haref{Is}{43}{27}; --- 2. \textbf{subordinate heavenly being}, angel-intercessor \haref{Jb}{33}{23}}~--- those sent to inquire of him\lit{to him to inquire} about\understood\ the sign\alt{omen} that was in the land~--- God abandoned him to\ie{in order to} test him to know everything in his heart.%%
    \verse{32:32} The rest of the works of Hezekiah, his kind deeds, are written in the vision of the prophet Isaiah, son of Amoz, in the scroll of the kings of Judah and Israel.%%
    \verse{32:33} Hezekiah slept with his fathers, and he was buried in the upper graves of the sons\ca{\peshitta\ \textit{bqrjth} = in urbe}{the Peshitta has ``in the city'' [instead of ``in the upper graves of the sons'']} of David. All of Judah and the inhabitants of Jerusalem did honor to him at his death. And his son Manasseh reigned in his stead.%%
\end{inparaenum}

  
  \book{Ezra}{\Hebrew{עזרא}}
  \heading{4}{Rehum, Shimshai, and others conspire against Israel to halt the building of the temple~--- they write a letter to Artaxerxes who commands that the construction cease}

\begin{inparaenum}
  \verse{4:1} Judah and Benjamin's oppressors\alt{adversary, enemy, foe} heard that the exiles\lit{children of the exiles} were building a temple for the \textsc{Lord}, the God of Israel,%%
  \verse{4:2} and they approached Zerubbabel and the chief fathers, and said to them, ``Allow us to build with you because, like you, we worship\alt{seek, care about} your God; and we haven't sacrificed since the time\lit{days} of Esarhaddon, king of Asshur, who brought us up here.''%%
  \verse{4:3} Zerubbabel, Jeshua, and the chief fathers of Israel said to them, ``It's not for you, but for us, to build a house to our God, because only we\alt{we alone} shall build to the \textsc{Lord} God of Israel, as was commanded by King Cyrus, king of Persia.''%%
  \verse{4:4} And so the people of the land discouraged\alt{demoralized} the \lit{the hands of the}people of Judah and terrified\alt{made them be out of their senses} them in building.%%
  \verse{4:5} They bribed counselors against them to frustrate\alt{put an end to, invalidate} their plans\alt{schemes} throughout the days of Cyrus, king of Persia, until the reign of Darius, king of Persia.%%
  \verse{4:6} During the reign of Ahasuerus, in the beginning of his reign, they wrote an accusation\ed{Interestingly, from \Hebrew{שׂטן} (satan), to accuse. However, this verse contains no such theological implications.} against the inhabitants of Judah and Jerusalem.%%
  
  \verse{4:7} In the days of Artaxerxes, Bishlam,\halot{xxxx}{xxxx (trust me)} Mithredath, Tabeel, and the rest of his companions, wrote to Artaxerxes, king of Persia. The letter\lit{The writing of the letter} was written in Aramaic and interpreted in Aramaic.%%
  
  \verse{4:8} Rehum the commander and Shimshai the scribe wrote a letter to the king, Artaxerxes.%%
  \verse{4:9} Then Rehum the commander and Shimshai the scribe and all their companions~--- the judges, envoys, officials, secretaries, Urukites, Babylonians, Susaites (who are Elamites),%%
  \verse{4:10} and the rest of the people whom the great and noble Osnappar brought over and settled in the cities of Samaria; the rest on this side of the river, and so forth.%%
  \verse{4:11} Here is a copy of the letter which they sent to Artaxerxes the king, the servants, the men on this side of the river, and so on:%%
  
  \verse{4:12} ``Let it be known to the king that the Jews that have come up from you to us have gone into Jerusalem. They're building the rebellious and evil city\ed{The Aramaic word here, \Hebrew{קִרְיָה}, refers specifically to Jerusalem.} and are finishing the walls and laying\halot{xxxx}{The form, etymology, and meaning of \Hebrew{חוט} are uncertain; suggested ``repair,'' or ``lay,'' or ``inspect.''} the foundations.%%
  \verse{4:13} Let it be known to the king that if this city is built and its walls completed, they will not give tax, toll, or custom. Eventually,\halot{xxxx}{either ``treasury'' or as an adverb meaning ``eventually'' or ``positively.''} it will injure\alt{wrong} the king.%%
  \verse{4:14} Now, because we're bound in loyalty to the king,\lit{the salt of the palace is our salt}\halot{xxxx}{\textbf{eat} (the) \textbf{salt} (of the palace)}\ed{idiomatically: be bound in loyalty to the king} and it isn't right for us to see the king's dishonor,\lit{nakedness}\halot{xxxx}{metaphorically, \textbf{dishonor}} for this reason we have sent and made this\understood\ known to the king%%
  \verse{4:15} so that someone can investigate in the book of the minutes\alt{memorandum} of your fathers. You shall find\ed{Understood: what you're looking for} in the book of the minutes and shall know that this city is a rebellious city which\understood\lit{and} shall injure the king and the provinces;\halot{xxxx}{administrative district, \textbf{province}, specifically the satrapies of the Persian empire.} that they strive to make revolt\lit{pride, arrogance} in its midst just like in olden times. Hence why this city was destroyed.\alt{devastated}%%
  \verse{4:16} We make it known to the king that if this city is built and its walls finished, because of this you will have no portion on this side of the river.''%%
  
  \verse{4:17} The king sent a decree to Rehum the chief commander\alt{of report(ing)} and to Shimshai the scribe and the rest of their companions who live in Samaria and the rest of the people\understood\ on the other side of the river: ``Peace and so on.%%
  
  \verse{4:18} The document which you sent to us has been interpreted and read to me.%%
  \verse{4:19} I have established a decree and they have investigated and found that this city, from the days of old, raises itself against the kings~--- rebellion and sedition are made therein.%%
  \verse{4:20} And there have been strong kings over Jerusalem, mighty officers on all the other side of the river, to them is given toll, tribute, and customs.%%
  \verse{4:21} Now, make a decree to stop these men. This city will not be built until I make a decree.%%
  \verse{4:22} And be warned of doing this negligence: why should hurt come to the detriment of the kings?''%%
  
  \verse{4:23} Then, from the time that a copy of king Artaxerxes' letter was read before Rehum and Shimshai the scribe and their companions, they went in haste to Jerusalem against the Jews and stopped them by force.\lit{with a strong arm.}%%
  
  \verse{4:24} The work of the house of God in Jerusalem ceased and remained stopped until the second year of king Darius of Persia's reign.%%
\end{inparaenum}

  \heading{5}{Haggai and Zechariah prophesy~--- Zerubbabel and others begin construction of the temple~--- Tatnai, Shethar-boznai, and others write a letter to King Darius asking for an order to stop the Jews}

\begin{inparaenum}
  \verse{5:1} Haggai the prophet and Zechariah the son of Iddo prophesied in the name of the God of Israel concerning the Jews in Judah and in Jerusalem.%%
  
  \verse{5:2} Then Zerubbabel, son of Shealtiel, and Jeshua, son of Jozadak, rose and began to rebuild the house of God in Jerusalem. And the prophets supported them.%%
  \verse{5:3} At that time, Tatnai, the governor on the other side of the river, came to them with Shethar-boznai and their companions; and they said to them, ``Who gave you orders to build this house\alt{temple} and to finish this wall?''%%
  
  \verse{5:4} They then said thus unto them, ``What are the names of the men who are building this house?''%%
  \verse{5:5} The eye of their God had been upon the elders\lit{hoar-headed} of the Jews and they had not caused them to stop until the matter went to Darius. They then sent back a letter concerning this.%%
  
  \verse{5:6} A copy of the letter that Tatnai the governor on the other side of the river, Shethar-boznai, and his companions the Apharsachites\ed{title for an official in Assyria} who are on the other side of the river sent unto Darius the king.%%
  \verse{5:7} In the letter they sent to him was written: ``Peace be unto Darius the king.%%
  
  \verse{5:8} Be it known to the king that we have gone to the province of Judah, to the great temple of God, and it is build with square stones, wood is placed in the walls, and this work is done speedily. This work prospers in their hand.%%
  
  \verse{5:9} Then we asked these elders~--- thus did we ask them: `Who hath made a decree for you for this house to be built and this wall to be completed?'%%
  \verse{5:10} Additionally, we asked them for their names, to make it known unto you, so we can write the names of their leaders.\lit{the men who are at the head.}%%
  
  \verse{5:11} Thus have they returned the word, saying, `We are the servants of the God of heaven and earth. We have been building the house for many years'\lit{from before this year} and that a great king of Israel had built and finished it.%%
  \verse{5:12} But after that, they made the God of heaven angry and He delivered them into the hands of the Chaldean Nebuchadnezzar, king of Babylon, who demolished this house and removed the other people to Babylon.%%
  
  \verse{5:13} But in the first year of Cyrus, king of Babylon, Cyrus, the king, gave an order to build this house of God.%%
  \verse{5:14} Furthermore, the gold and silver vessels from the house of God that Nebuchadnezzar removed from the temple in Jerusalem and brought to the temple in Babylon, Cyrus, the king, removed them from the temple in Babylon and brought them to Sheshbazzar whom he had appointed as governor.%%
  \verse{5:15} He\ie{Cyrus} said to him, ``Lift up these vessels and go and put them in the temple in Jerusalem. Let the house of God be build on its place.''%%
  
  \verse{5:16} Then did Sheshbazzar come and lay the foundations of the house of God in Jerusalem. From then until now it has been being built and is not finished.%%
  \verse{5:17} And now, if it be good to the king, let an investigation\alt{search, inquiry} be made into the treasury of the house of the king in Babylon whether it be that Cyrus the king made a decree to build this house of God in Jerusalem. Let the king's will concerning this be sent to us.''%%
\end{inparaenum}

  \heading{6}{Darius starts an investigation to find if sanctions were ever given to the Jews to build a temple~--- evidence is found that Cyrus gave sanctions through a decree~--- Darius states that this decree is still in effect~--- he commands that the Jews be provided with whatever they need to build the temple~--- the temple is completed and dedicated}

\begin{inparaenum}
  \verse{6:1} King Darius then made a decree and they investigated in the records house where the treasures of Babylon were laid up.%%
  \verse{6:2} A scroll was found in Achmetha, a fortress in the province of Media, in the midst of which was written a record:%%
\end{inparaenum}

\begin{quotation}
  \begin{inparaenum}\setcounter{enumi}{2}
    \verse{6:3} ``In the first year of Cyrus, the king, King Cyrus made an order regarding the house of God in Jerusalem: let the house be built in the place where they are sacrificing sacrifices. Let its foundations be strongly laid: its height sixty cubits and its breadth sixty cubits.%%
    \verse{6:4} Three layers of square stones and a layer of new wood. Let the expense be charged\lit{given} to the king's house.%%
    \verse{6:5} Additionally, the gold and silver of the house of God which Nebuchadnezzar removed from the Jerusalem temple and brought to Babylon, let it be returned to the Jerusalem temple: you shall put it in the house of God.%%
    
    \verse{6:6} And now, Tatnai, governor of the trans-Euphrates,\lit{on the opposite side of the river} Shethar-boznai, and your companions the Apharsachites\ed{officials in Assyria} of the trans-Euphrates, stay away from this.\lit{be ye far from thence}%%
    \verse{6:7} Leave behind the work of the house of God; let the governor of the Jews and the elders of Judah build this house of God in its place.%%
    \verse{6:8} From me I give an order regarding what you should do to the work of those elders of Judah and the building of this house of God: the riches of the king which are on the other side of the river, the expenses be speedily given to these men that their work cease not.%%
    \verse{6:9} What they need, both young bullocks, rams, and lambs, for burnt offerings to the God of heaven; wheat, salt, wine, and oil, according to the saying of the Levite priest who is in Jerusalem, let it be given to them daily without negligence%%
    \verse{6:10} so that they can offer sweet odors to the God of heaven and pray for the life of the king and his sons.%%
    \verse{6:11} From me I give a command to anyone\lit{all mankind} who changes this decree: let wood be pulled out from their house and let them be impaled\lit{in impaling let them be impaled} and their house shall be made a dunghill for this.%%
    \verse{6:12} God caused His name to dwell there. He overthrew all the kings and people who try and cause His hand to change and hurt this house of God in Jerusalem. I, Darius, have made a decree~--- let it be done with all diligence.''\lit{let it be speedily done}%%
  \end{inparaenum}
\end{quotation}

\begin{inparaenum}\setcounter{enumi}{12}
  \verse{6:13} Then Tatnai, the governor of the trans-Euphrates, Shethar-boznai, and their companions speedily did as King Darius had sent.%%
  \verse{6:14} The elders of the Jews were continuing to build and prosper because of the prophecies of Haggai the prophet and Zechariah the son of Iddo. Since the decree of the God of Israel, Cyrus, Darius, and Xerxes\lit{Artaxerxes (\Hebrew{ְאַרְתַּחְשַׁ֖שְׂתְּא}), but historically this is more probably Xerxes.} king of Persia, they built and finished.\ed{Understood: the house of God in Jerusalem}%%
  \verse{6:15} This house was completed on the third day of the month Adar which was in the sixth year of the reign of King Darius.%%
  
  \verse{6:16} The sons of Israel, the priests, Levites, and the rest of the sons of the captivity made a joyous dedication to this house of God.%%
  \verse{6:17} They brought one hundred bullocks, two hundred rams, four hundred lambs, and twelve young goats (for a sin offering for all Israel according to the number of tribes of Israel) near to this temple of God for its dedication.%%
  \verse{6:18} They put their priests in their divisions, the Levites in their courses over the work of God that is in Jerusalem according to the writing in the scroll of Moses.%%
  
  \verse{6:19} The exiles\lit{children of the exiles} held\lit{made} Passover on the fourteenth of the first month%%
  \verse{6:20} because the priests and the Levites together\lit{as one man} were ritualistically clean. They were all clean. They slaughtered the Paschal lamb for all of the exiles, for all of their brethren the priests, for themselves.%%
  \verse{6:21} The children of Israel~--- the ones returning from exile, everyone who had separated themselves from the\ed{The definite article is used to clearly show that the state of uncleanliness was of the foreign nations, not of the returning Israelites.} state of ritualistic uncleanliness of the nations of the land~--- to worship the \textsc{Lord}, the God of Israel, ate.\alt{offered sacrifice; this is justified contextually because in verse~20 they slew the Paschal lamb, but since the lamb was already slain it makes more sense for them to eat than to again offer sacrifice.}%%
  \verse{6:22} They held with joy a festival of unleavened bread for seven days, for the \textsc{Lord} had made them happy.\alt{rejoice, joyful, glad} The \textsc{Lord}\understood\ made the Assyrian king change to their side,\lit{turn} to strengthen their hands in the work of the House of God, the God of Israel.%%
\end{inparaenum}

  \heading{7}{Genealogy of Ezra~--- Ezra receives permission from the king to return to Jerusalem~--- he prepares himself to learn and teach the Lord's law~--- Artaxerxes gives Ezra permission to use the national treasury to purchase animals for sacrifice~--- further permission is given to take all of the temple vessels~--- Ezra praises God}

\begin{inparaenum}
  \verse{7:1} After these things, during the reign of Artaxerxes, king of Persia, Ezra the son of Seraiah, son of Azariah, son of Hilkiah,%%
  \verse{7:2} son of Shallum, son of Zadok, son of Ahitub,%%
  \verse{7:3} son of Amariah, son of Azariah, son of Meraioth,%%
  \verse{7:4} son of Zerahiah, son of Uzzi, son of Bukki,%%
  \verse{7:5} son of Abishua, son of Phinehas, son of Eleazar, son of Aaron the chief priest;%%
  \verse{7:6} this Ezra came up from Babylon, a ready scribe in the Mosaic Law that the \lord\ God of Israel gave. The king gave him his every request when the hand of the \lord\ his God was upon him.%%
  
  \verse{7:7} In the seventh year of King Artaxerxes' reign, some of the children of Israel, some of the priests, Levites, singers, gatekeepers, and temple slaves,\alt{bondsmen; lit., those who are donated} went up to Jerusalem.%%
  \verse{7:8} He came to Jerusalem in the fifth month in the seventh year of the king.%%
  \verse{7:9} For on the first of the month he determined\alt{founded} the going up from Babylon and on the first of the fifth month he came up to Jerusalem according as the good hand of his God was upon him.%%
  \verse{7:10} For Ezra readied his heart to seek the law of the \lord\ and to do it~--- to teach the definitions, rules, prescriptions,\ed{These three (definitions, rules, and prescriptions) are all possible meanings of \Hebrew{חֹק}, with the connotation that they are given by God.} and judgments in Israel.%%
  
  \verse{7:11} This is a copy of the decree\halot{xxxx}{official document} that King Artaxerxes gave to Ezra~--- the priest, the scribe, scribe of the words of the commandments of the \lord\ and of His rules concerning Israel:%%
\end{inparaenum}
\begin{quote}
  \begin{inparaenum}\setcounter{enumi}{11}
    \verse{7:12} Artaxerxes, king of kings, to Ezra the priest, a scribe of the law of the perfect God of heaven,\ed{Horribly ambiguous: \textit{perfect} could modify \textit{law}, \textit{God}, or \textit{heaven}. Most probably God.} and so forth:%%
    \verse{7:13} ``From me has a declaration been made to all of the people of Israel in my kingdom: any priest or Levite who is willing to go to Jerusalem with them shall go.%%
    \verse{7:14} Because that from before the king and his seven counselors you are sent to investigate concerning the Jews and Jerusalem with the law of God which is in your hand.%%
    \verse{7:15} To bring the silver and gold which the king and his counselors willingly offered to the God of Israel whose tabernacle\ed{The root of this word, \Hebrew{מִשְׁכְּנֵהּ}, means ``a dwelling-place.''} which is in Jerusalem.%%
    \verse{7:16} All the silver and gold you find in all the provinces of Babylon with the donations\alt{free-will offerings} of the people and of the priests shall be freely offered\alt{donated} to the house of their God which is in Jerusalem.%%
    \verse{7:17} Therefore, you shall speedily buy with this money bullocks, rams, lambs, and their presents and libations, and bring them to the altar which is in the house of their God in Jerusalem.%%
    \verse{7:18} That which is good to you and your brethren with the rest of the silver and gold to do according to the will of your God, that shall you do.%%
    \verse{7:19} The vessels which are given to you for the service of your God's house~--- completely finish it before the God of Jerusalem.%%
    \verse{7:20} The remainder of the needful things of your God's house which have fallen to you to give, give them\understood\ from the treasure house of the king.%%
    \verse{7:21} From me, I, Artaxerxes the king, an order is made to all treasures of the trans-Euphrates that all the requirements of Ezra the priest, scribe of the law of the God of heaven, let them be diligently done%%
    \verse{7:22} unto one hundred talents of silver, one hundred cords of wheat, one hundred baths of wine, and one hundred baths of oil. And salt without writing.\ie{permission}%%
    \verse{7:23} All that is from the decree of the God of heaven, let it be done with zeal\alt{eagerness} to the house of the God of heaven: for why should there be wrath upon the kingdom of the king and his sons?%%
    \verse{7:24} We are informing you that there is no officer to lift a tribute or tax upon all the priests, Levites, singers, gate keepers, sanctuary servants, and servants of this house of God.%%
    \verse{7:25} You, Ezra, according to the wisdom of your God which is in your hand, place judges and magistrates who will judge all the people of the trans-Euphrates and all who know the laws of your God; and unto those who don't know, teach them.%%
    \verse{7:26} All who will not do the law of your God and the law of the king with exactness, let judgment come upon the, whether to death, banishment, or a fine of riches and a bond.''%%
  \end{inparaenum}
\end{quote}
\begin{inparaenum}\setcounter{enumi}{26}
  \verse{7:27} Blessed be the \lord\ God of our fathers who has put it like this into the king's heart to glorify the House of the \lord\ in Jerusalem!%%
  \verse{7:28} Who has\understood\ extended His loving-kindness before the king, his counselors, and all the king's manly officials.\halot{xxxx}{outside of Israel, a representative of the king} I have become powerful when the hand of the \lord\ my God was upon me. I have assembled a company of soldiers\understood\ from Israel to go up with me.%%
\end{inparaenum}

  
  \book{Nehemiah}{\Hebrew{נחמיה}}
  \heading{6}{xxxx}

\begin{inparaenum}
  \verse{6:1} And when Sanballat, Tobiah, Geshem the Arabian, and the rest of our enemies had heard\lit{when it was heard by} that I had built the wall and that there was no breach in it (also, up until that time\lit{until then} I had not set up the gates)%%
  \verse{6:2} that Sanballat and Geshem sent unto me, saying, ``Come, and we shall meet together in the villages in the valley of Ono,'' but they were thinking of doing me wrong.\lit{thinking to do evil unto me.}%%
  \verse{6:3} So I sent messengers unto them, saying, ``I am doing a great work so I cannot come down. Why should the work cease while I leave it and come down to you?''%%
  \verse{6:4} So they sent unto me~--- as I've said\lit{according to this word}~--- four times, and I told them likewise.%%
  
  \verse{6:5} So Sanballat sent his servant unto me~--- as I've said~--- the fifth time with an open letter in his hand,%%
  \verse{6:6} and it was written therein: ``It hath been heard among the nations, and Gashmu hath said, `You and the Jews are thinking of rebelling. Therefore, you are building the wall and hast been a king unto them. Even according to these words!%%
  \verse{6:7} Moreover, thou hast appointed prophets who\lit{to} call for thee in Jerusalem, saying, ``A king is in Judah!''' Now, it hath been heard by the king, even according to these words. Come now and let us reason together.''%%
  
  \verse{6:8} So I sent unto him, saying, ``No, it hath not been as thou hast said\lit{according to these words that thou hast said} because you've made this up\ie{devised} from thine own heart.%%
  \verse{6:9} For all of them are making us afraid, saying, `Their hands are too weak\alt{feeble} for the work~--- that's why it's not done.' O God, strengthen my hands!''\ed{Some slight eisegesis to get ``O God'' into this clause, but other translations seem to agree with me. Logic: why would he be asking his enemies to strengthen his hands?}%%
  \verse{6:10} Then I entered the house of Shemaiah, son of Delaiah (son of Mehetabeel), who was shut up,\alt{locked in, under house arrest, not going outside} and he said,%%

  \pvcb{``Let us meet at the House of God,}{inside the temple.}%%

  \pvcb{So we shall shut the doors of the temple}{because they are coming to slay thee.}%%

  \pvca{Even by night are they coming to slay thee.''}%%
  
  \verse{6:11} And I said, ``Would a man like me flee?\lit{A man such as I (or, like me)~--- would he flee?} And who, like me, would go in unto the temple and live? I will not go.''\lit{go in.}%%
  \verse{6:12} And I perceived that lo! God had not sent him because he had prophesied this word against me: Tobiah and Sanballat had hired him.\ed{That's some serious transgression of taking the name of the Lord in vain.}%%
  \verse{6:13} He was hired to frighten me.\lit{Therefore was he hired, that I might be afraid} That I might do so: sin~--- that I might have something to go off of to spread an evil report so that they could reproach me.%%
  
  \verse{6:14} My God, remember Tobiah and Sanballat according to these, their works. And also the prophetess Noadiah and the rest of the prophets who would have made me afraid.%%
  \verse{6:15} So the wall was finished in fifty-two days: on the twenty-fifth of Elul.%%
  
  \verse{6:16} And when all of our enemies had heard it, and when all the nations round about us were afraid (and much cast down in their own perception) for they knew that this work was brought to pass by our God.%%
  \verse{6:17} Additionally, in those days, the noblemen of Judah sent a lot of letters to Tobiah, so those who are with Tobiah came to them.%%
  \verse{6:18} For there are many in Judah sworn to him because he hath a son-in-law of Shechaniah, son of Arah; his son Johanan hath married the daughter of Meshullam, son of Berechiah.%%
  \verse{6:19} They also spoke before me of his goodness and then reported my words to him. So Tobiah sent letters to make me afraid.%%
\end{inparaenum}

  
  \book{Job}{\Hebrew{איוב}}
  {\noindent{\small \textsc{ca} $\nicefrac{1}{5}$ fere huius libri deest in \septuagint\ (nearly one fifth of this book is missing in the Septuagint)}}
  \heading{1}{xxxx~--- the Lord and Satan discuss Job, a perfect man, and the Lord allows Satan to afflict Job~--- xxxx}

\begin{inparaenum}
    \verse{1:1} There was a man (his name was Job) in the land of Uz. This man was pure,\alt{blameless} had integrity:\ed{more properly ``integrous,'' but that's not really a common word} he revered God and turned from evil.%%
    \verse{1:2} Seven sons and three daughters were born to him.%%
    \verse{1:3} His property was: 7\thinspace000~sheep, 3\thinspace000~camels, 500~head\lit{pair, yoke} of oxen, 500~donkeys, and so many servants. Such it was that\ie{And it came to pass that} this man was greater than all the children of the east.%%
    \verse{1:4} His sons went and made a feast in each one's house on their day. They sent and invited\lit{called} their three sisters to eat and drink with them.%%
    \verse{1:5} When the daily, feast rotation\halot{xxxx}{go in a series around a circle (of the year), make a (yearly) round \haref{Is}{29}{1}} was done,\understood\ Job sent and sanctified them. He got up early in the morning and offered a burnt offering for all of them,\lit{for the summation of them all} because Job said, ``Perhaps\halot{xxxx}{adverb: \textbf{perhaps} (expression of hope, entreaty, fear)} my children\ed{Perhaps just ``sons'' here, but not enough context to rule one way or the other; ``children'' is more consistent with the rest of this translation.} have sinned and blessed\ca{correctio vel euphemismus pro \Hebrew{וְקִלְלוּ} vel sim}{correction or euphemism for ``curse'' or similar} God in their hearts.'' Thus did Job do all the time.\alt{continually.}%%
    
    \verse{1:6} There was a day when the sons of God came to appear before the \textsc{Lord}, and Satan also came among them.%%
    \verse{1:7} The \textsc{Lord} said to Satan, ``Where are you coming from?'' Satan answered the \textsc{Lord}, and said, ``From roaming around in the earth, going about in it.''%%
    \verse{1:8} The \textsc{Lord} said to Satan, ``Have you considered\lit{set your heart on/against}\alt{paid attention to, noticed} my servant Job? Because there is no one like him in the world: a blameless\alt{pure, perfect, complete, morally pure} and upright man who\understood\ fears God and desists\alt{leaves off} from evil.''%%
    \verse{1:9} Satan answered the \textsc{Lord}, and said, ``Does Job fear God without cause?\alt{without compensation, for nothing/naught, in vain?}%%
    \verse{1:10} Have You made a hedge round about him and his home and everything he has? You've blessed the works of his hands, his substance increases\alt{is spread out} in the land.%%
    \verse{1:11} xxxx please, stretch out Your hand and afflict\alt{touch, hurt, buffet} everything he has to see\understood\ if he'll bless\ca{correctio vel euphemismus pro \Hebrew{וְקִלְלוּ} vel sim}{correction or euphemism for ``cursed'' or a similar word} You to Your face.''%%
    \verse{1:12} The \textsc{Lord} said to Satan, ``Everything he has is in your power, just don't lay a hand on him.''\lit{only on him don't put your hand.''} So Satan went out from the presence of the \textsc{Lord}.%%
    
    \verse{1:13} This day, his sons and daughters ate and drank wine in their brother, the firstborn's house.%%
    \verse{1:14} %%
    \verse{1:15} %%
    \verse{1:16} \smallskip%%
    
    \pvba{}%%
    
    \pvba{}%%
    
    {\noindent\verse{1:17} \smallskip}%%
    
    \pvbb{}{}%%
    
    \pvba{}%%
    
    \pvba{}%%
    
    {\noindent\verse{1:18} \smallskip}%%
    
    \pvbb{}{}%%
    
    \pvab{\vn{1:19} }{}%%
    
    \pvbb{}{}%%
    
    \pvba{}%%
    
    {\noindent\verse{1:20} \smallskip}%%
    \verse{1:21} %%
    
    \pveb{}{}%%
    \pveb{}{}%%
    
    {\noindent\verse{1:22} }%%
\end{inparaenum}

  \heading{19}{xxxx}

\begin{inparaenum}
  \verse{19:1} %%
  \verse{19:2} %%
  \verse{19:3} %%
  \verse{19:4} %%
  \verse{19:5} %%
  \verse{19:6} %%
  \verse{19:7} %%
  \verse{19:8} %%
  \verse{19:9} %%
  \verse{19:10} %%
  \verse{19:11} %%
  \verse{19:12} %%
  \verse{19:13} %%
  \verse{19:14} %%
  \verse{19:15} %%
  \verse{19:16} %%
  \verse{19:17} %%
  \verse{19:18} %%
  \verse{19:19} %%
  \verse{19:20} %%
  \verse{19:21} %%
  \verse{19:22} %%
  \verse{19:23} %%
  \verse{19:24} %%
  \verse{19:25} I know that my Redeemer lives and at last He\alt{I have known my living Redeemer, the Last; I have known my Redeemer, the Living and the Last, Who} shall rise\alt{stand} on the earth.%%
  \verse{19:26} After my skin has been destroyed, then shall I see God in the\ie{my} flesh.%%
  \verse{19:27} %%
  \verse{19:28} %%
  \verse{19:29} %%
\end{inparaenum}

  \heading{38}{xxxx}

\begin{inparaenum}
  \pa \verse{38:1} The \textsc{Lord} answered Job from the gale\halot{xxxx}{heavy whirlwind} and said,%%
  
  \pb \verse{38:2} ``Who is this who darkens counsel\pa with words%
  \halot{xxxx}{\textbf{word} (utterance rather than lexical unit)}
  without knowledge?%%
  
  \pb \verse{38:3} Gird up your loins%
  \halot{xxxx}{\textbf{loins} (body between ribs and hipbones)}
  like a youth%
  \ca{Ms \Hebrew{כְגִיבּוֺר}}{a Hebrew manuscript has ``like a warrior''}
  \pa and I will inquire of\alt{ask, consult, demand, request} you and cause you to know.%%
  
  \pb \verse{38:4} Where were you when I laid the foundation of the earth?\pa Tell it, if you know.\lit{have a knowledge of understanding.}%%
  
  \pb \verse{38:5} Who placed its measurements, if you know?\pa Or, who's stretched it out on its measuring line?%%
  
  \pb \verse{38:6} On what have its pedestals\alt{sockets} been sunk?\pa Or, who has laid its cornerstone?%%
  
  \pb \verse{38:7} When the morning stars sang\alt{shouted} together for joy together\pa and all the children of God shouted for joy,%%
  
  \pb \verse{38:8} and He shut up of the sea with doors%
  \ed{not in construct, so it's ``sea with doors''}
  \pa when bursting forth, it went out of the womb.\ie{of the sea.}%%
  
  \pb \verse{38:9} When I made clothing for a cloud\pa and thick darkness its swaddling;%%
  
  \pb \verse{38:10} when I established\alt{set} my statute upon it\pa and place bars and doors,\ed{understood: on/over it}%%
  
  \pb \verse{38:11} and said, `You shall come to here, but no farther.\pa And here is placed on the haughtiness of your proud waves.'\lit{in the pride/haughtiness of your waves.'}%%
  
  \pb \verse{38:12} Have you, from your days, commanded the morning?\pa Have you made the light before dawn to know its place?%%
  
  \pb \verse{38:13} xxxx\pa xxxx%%
  
  \pb \verse{38:14} xxxx\pa xxxx%%
  
  \pb \verse{38:15} xxxx\pa xxxx%%
  
  \pb \verse{38:16} xxxx\pa xxxx%%
  
  \pb \verse{38:17} xxxx\pa xxxx%%
  
  \pb \verse{38:18} xxxx\pa xxxx%%
  
  \pb \verse{38:19} xxxx\pa xxxx%%
  
  \pb \verse{38:20} xxxx\pa xxxx%%
  
  \pb \verse{38:21} xxxx\pa xxxx%%
  
  \pb \verse{38:22} xxxx\pa xxxx%%
  
  \pb \verse{38:23} xxxx\pa xxxx%%
  
  \pb \verse{38:24} xxxx\pa xxxx%%
  
  \pb \verse{38:25} xxxx\pa xxxx%%
  
  \pb \verse{38:26} xxxx\pa xxxx%%
  
  \pb \verse{38:27} xxxx\pa xxxx%%
  
  \pb \verse{38:28} xxxx\pa xxxx%%
  
  \pb \verse{38:29} xxxx\pa xxxx%%
  
  \pb \verse{38:30} xxxx\pa xxxx%%
  
  \pb \verse{38:31} xxxx\pa xxxx%%
  
  \pb \verse{38:32} xxxx\pa xxxx%%
  
  \pb \verse{38:33} xxxx\pa xxxx%%
  
  \pb \verse{38:34} xxxx\pa xxxx%%
  
  \pb \verse{38:35} xxxx\pa xxxx%%
  
  \pb \verse{38:36} xxxx\pa xxxx%%
  
  \pb \verse{38:37} xxxx\pa xxxx%%
  
  \pb \verse{38:38} xxxx\pa xxxx%%
  
  \pb \verse{38:39} xxxx\pa xxxx%%
  
  \pb \verse{38:40} xxxx\pa xxxx%%
  
  \pb \verse{38:41} xxxx\pa xxxx%%
\end{inparaenum}

  \heading{42}{Job answers the Lord's questions from chapter~38~--- xxxx}

\begin{inparaenum}
  \verse{42:1} Job answered the \lord, and said,\smallskip%%
  
  \pb \verse{42:2} ``I know that You can do anything,\alt{all things}\pa no plan\alt{intention, concept, device} is impossible for you.%%
  
  \pb \verse{42:3} Who's this who hides counsel without knowledge?\pa Therefore, I have declared and not understood; things too wonderful to me that I don't know.%%
  
  \pb \verse{42:4} Please hear me and I will speak~---\pa I will ask You and You will let me know.%%
  
  \pb \verse{42:5} By the hearing of the ear I'd heard of You,\pa but now my eyes have seen You.%%
  
  \pb \verse{42:6} Therefore, I blame myself and have repented\pa in dust and ashes.''\smallskip%%
  
  \verse{42:7} After the \lord\ had spoken these words to Job,\ed{Isn't this the wrong direction?} the \lord\ said to Eliphaz the Temanite: ``My anger burns against you and your two friends because you haven't spoken well of Me like My servant Job.%%
  \verse{42:8} So now, take seven bulls and seven rams and go to My servant Job and make a burnt offering\ed{for purification or guilt} for yourselves; My servant Job will pray for you because I will definitely accept him\lit{his face:} without treating you disdainfully because you haven't spoken well\lit{that which is right} of Me like My servant Job.''%%
  \verse{42:9} So Eliphaz the Temanite, Bildad the Shuhite, Zophar the Naamathite, went and did as the \lord\ had told them, and the \lord\ accepted Job.%%
  \verse{42:10} The \lord\ turned Job's fortune when he prayed for his friends, and the \lord\ added to Job twice as much as he'd before had.%%
  \verse{42:11} All his brothers, sisters, %%
  
  \verse{42:12} %%
  \verse{42:13} %%
  \verse{42:14} %%
  \verse{42:15} There was not found in all the land a woman as hot as Job's daughters; and their father gave them an inheritance\ed{probably an equal inheritance} among their brothers.%%
  
  \verse{42:16} %%
  \verse{42:17} Job died, old and full of days.%%
\end{inparaenum}

  
  \book{Psalms}{\Hebrew{תהילים}}
  \heading{1}{Those who follow the Lord are blessed and happy~--- the Lord knows the way}

\begin{inparaenum}
  \pa \verse{1:1} Happy\alt{Blessed} is the man\pa who does not walk\pa in the counsel of the wicked.%%
  
  \pb Who doesn't stand in the way of sinners\pa and doesn't sit in the assembly\alt{seat} of scorners.%%
  
  \pa \verse{1:2} Truly his joy\alt{pleasure, delight} is in the law of the \textsc{Lord}~---\pa he ponders%
  \halot{xxxx}{c) \textbf{read in an undertone} \haref{Ps}{1}{2}; d) \textbf{ponder (by talking to oneself)} with accusative \haref{Is}{33}{18}, with \textit{b}\superit{e} \haref{Ps}{63}{7}; e) \textbf{plan} with accusative \haref{Ps}{2}{1}; f) \textbf{speak}, proclaim \haref{Is}{59}{3}}
  day and night on His law.%%
  
  \pa \verse{1:3} He is like a tree,\pa planted by streams of water,\ie{so even in the heat it will grow.}%%
  
  \pb which gives fruit in its season;\pa its leaves will not wither.\alt{fade}%%
  
  \pb He will succeed in everything he does.%%
  
  \pb \verse{1:4} The wicked aren't so!\lit{Not thus the wicked!}%%
  
  \pb Because they are like chaff,\pa driven by the wind.\alt{that the wind blows away.}%%
  
  \pa \verse{1:5} Because of this, the wicked won't rise up in judgment,%
  \ed{They won't be able to object during judgment. They will be invalid witnesses.}
  \pa neither sinners in the assembly%
  \ca{\septuagint\ \Greek{ἐν βουλῇ} cf 1}{The Septuagint gives ``in the council of,'' compare verse~1}
  of the righteous%%
  
  \pa \verse{1:6} because the \textsc{Lord} knows the way of the righteous,\pa and the way of the wicked is lost.\alt{shall perish.}%%
\end{inparaenum}

  \heading{2}{A kingship and temple psalm~--- the nations revolt against the Lord~--- He retaliates and anoints His own king~--- blessed are they who trust in the Lord}

\begin{inparaenum}
  \pb \verse{2:1} Why are the nations in unrest?\alt{agitated [to the point of shaking]?}\pa Why do the people meditate on vanity?%%
  
  \pb \verse{2:2} The kings of the earth establish themselves\pa and the princes are in cahoots%%
  
  \pc against the \lord\ and His anointed.%
  \ed{Not necessarily the Anointed, but any of the Lord's servants.}%%
  
  \pb \verse{2:3} Let's tear away their fetters\pa and cast off their cords from us.%%
  
  \pc \verse{2:4} He who dwells in heaven\ie{deity} shall laugh.\pa The Lord%
  \ca{\fragheb\ mlt Mss \tetragrammaton; \septuagint\peshitta\ pr cop}{manuscripts in the Cairo Genizah as well as multiple, Hebrew manuscript codices have the Tetragrammaton; the Septuagint and the Peshitta put it before the copula [so, right where it is in translation]}
  shall have them in scorn.%%
  
  \pb \verse{2:5} He will then speak to them in His anger~---\pa in His fierce fury%
  \halot{xxxx}{\Hebrew{חרוֺן} is ``anger (only of God)''}
  He will strike them.%%
  
  \pb \verse{2:6} I have anointed my king%
  \ca{\septuagint\ suff 3 sg}{the Septuagint has a third singular pronominal suffix [i.e., ``his king'']}
  \pa on Zion, my holy mountain.%
  \ca{\septuagint\ suff 3 sg}{the Septuagint has a third singular pronominal suffix [i.e., ``his holy mountain'']}%%
  
  \pc \verse{2:7} Let me relate the decree%
  \ca{\peshitta\ + suff 1 sg}{the Peshitta adds a first singular pronominal suffix [i.e., ``my decree:'']}%%
  the \lord%%
  
  \pc has said to me: ``You are my son.\pa Today I have begotten you.%%
  
  \pb \verse{2:8} Ask of me\pa that I may give nations as your inheritance\pa and the ends of the earth as your possession.%
  \ed{wonderful example of reverse parallelism}%%
  
  \pb \verse{2:9} You shall break them with an iron scepter;\pa you shall break them in pieces like a potter's utensil.%%
  
  \pc \verse{2:10} Now, O kings, be prudent.\alt{wise}\pa You judges of the earth, listen to reason.\lit{(niphal imperative) be disciplined}%%
  
  \pc \verse{2:11} Serve the \lord\ with fear%
  \ca{Ms \Hebrew{בְּשִׂמְחָה}}{a manuscript has ``with joy/gladness''}
  \pa and rejoice with trembling. \verse{2:12} Kiss the Son%
  \ed{The \textit{Jewish Study Bible} renders this as ``Pay homage in good faith.''}%%
  
  \pc lest He be angered and you lose your way,\pa even though His anger only burns a little.%%
  
  \pc Blessed are all they\lit{those} who trust in Him.%%
\end{inparaenum}

  \heading{15}{Only those who are pure and clean my enter the temple~--- qualifications for cleanliness enumerated}

\begin{inparaenum}
  \noindent\verse{15:1} A psalm of David:\smallskip%%
  
  \pc O Lord, who shall dwell in Your tent?\pa Who will reside in Your holy mountain?%%
  
  \pb \verse{15:2} Those who walk uprightly,%
  \ed{Although \Hebrew{תָּמִים} can mean ``perfectly,'' it is more modest to here render it as ``uprightly.''}
  \pa work righteousness~---%%
  
  \pc who speaks truth from his heart.\pa \verse{15:3} Who doesn't slander with his tongue,%%
  
  \pc do evil to his friends,\pa or lift up reproaches on his neighbor.%%
  
  \pb \verse{15:4} The rejected%
  \ca{crrp}{corruption}
  \ed{This makes enough sense as is that the corruption doesn't seem to make a noticeable difference.}
  are despised%
  \ca{2 Mss `\Hebrew{וְנ}, \septuagint\ \Greek{πονηρευόμενος}, \peshitta\ \textit{mrgzn'} irritator}{two manuscripts have ``and he honors,'' the Septuagint has ``doing evil,'' the Peshitta has ``he who irritates''}
  in his eyes;\pa he honors those who fear the \lord.%%
  
  \pc If he's sworn to his own inconvenience,\alt{hurt}%
  \ca{\septuagint(\peshitta) \Greek{τῷ πλησίον αὐτοῦ}, \symmachus\ \Greek{ἐταῖρος εἶναι}}{Septuagint (and Peshitta) ``to his neighbor,'' Symmachus' Greek translation of the Old Testament ``fellow companion''}
  \pa he doesn't retract.%%
  
  \pb \verse{15:5} He doesn't put his money to usury,\pa doesn't take bribes\alt{rewards} against the innocent.%%
  
  \pc He who does these things\pa shall never be shaken.%%
\end{inparaenum}

  \heading{23}{The Lord is my shepherd~--- He will comfort and provide}

\begin{inparaenum}
  \noindent\verse{23:1} A psalm of David:\smallskip%%
  
  \pc The \lord\ is my shepherd: I will not go lacking.\pa \verse{23:2} He makes me to lie down in grassy pastures.%
  \ca{huc tr \Hebrew{׃}}{hither transpose a \textit{sof pasuq} (symbol for end of verse)}%%
  
  \pc He leads me beside calm waters.\pa \verse{23:3} He rescues my soul.%%
  
  \pc He leads me in\alt{on} paths of righteousness\pa for His name's sake.%%
  
  \pb \verse{23:4} Although I walk\pa through the valley of darkness%
  \ed{Compare the Ugaritic \textit{\d{s}lmt} (sometimes with a \textit{waw} as \textit{\d{s}lmwt}) (to the Hebrew \Hebrew{צַלְמָוֶת}) meaning ``deepest darkness.''}
  \ed{Classically understood as ``shadow (\Hebrew{צל}) of death (\Hebrew{מות})'' but a bit better as ``darkness.''}
  \pa I won't fear evil%%
  
  \pc because You are with me.\pa Your rod and your staff\pa comfort me.%%
  
  \pb \verse{23:5} You prepare a table before me\pa in the presence of my enemies.%%
  
  \pc You have anointed my head with oil;\pa my cup overflows.\ed{probably with wine}%%
  
  \pb \verse{23:6} Surely, goodness and loving-kindness%
  \ca{\septuagint\peshitta\ + suff 2 sg}{the Septuagint and Peshitta have a second singular pronominal suffix [i.e., ``your loving-kindness'']}
  will follow me\pa all the days of my life.%%
  
  \pc I will live in the house of the \lord\ pa for the rest of my days.%%
\end{inparaenum}

  \heading{24}{Ancient equivalent of a temple recommend interview~--- only those who are ritualistically and ethically clean may enter the temple~--- the King of Glory is the Lord}

\begin{inparaenum}
  \noindent\verse{24:1} A psalm of David:\smallskip%%
  
  \pc The earth and its fullness are the \lord's:\pa the world and its inhabitants.%%
  
  \pb \verse{24:2} For%
  \ca{\missing\ \septuagint\symmachus\theodotion}{missing in the Septuagint, Symmachus' and Theodotion's Greek translations of the Old Testament}
  it was He that founded it upon the sea~---\pa He established it upon the rivers.%%
  
  \pc \verse{24:3} Who shall ascend to the \lord's mountain?\pa Who shall stand in His place of holiness?%%
  
  \pb \verse{24:4} Those with clean palms and soles%
  \ie{ritualistically and ethically clean palms and soles. This is a higher order of cleanliness than the Mosaic Law which only requires ritualistic cleanliness.}
  and a pure heart;\pa who don't lift their souls in vanity%%
  
  \pc and don't swear\ie{in a covenantal sense} deceitfully.%%
  
  \pb \verse{24:5} He shall bear a blessing from the \lord\ pa and righteousness from the God of his salvation.%%
  
  \pb \verse{24:6} This is the generation of those who seek Him~---\pa that seek your face,%
  \ca{\septuagint\ om suff, 2 Mss \septuagint\peshitta\ + \Hebrew{אֱלֹהֵי}, \targum\ suff 3 sg}{The Septuagint omits the [pronominal] suffix, two manuscripts and the Septuagint and Peshitta add ``of God,'' the Targum has a third person singular suffix [``his face'']}
  \ed{It is shocking that any manuscript of the Septuagint would include ``the face of God'' because the Septuagint tends to exclude phrases referring to the anthropomorphic nature of God.}
  O Jacob. Selah!%%
  
  \pa \verse{24:7} O gates, lift up your heads!\pa Be lifted up, O everlasting doors!%
  \ed{The story of the battle of Baal and Yam has an interesting parallel to this story. On line~27 of column~iv it reads ``Lift up, gods, your heads,'' the same structure found here. This is of interest because it shows the same structure in a scriptural text of a pagan religion. See further in \textit{Canaanite Myths and Legends}, J.C.L. Gibson.}
  \pa The King of Glory shall come in!%%
  
  \pa \verse{24:8} Who is this, the King of Glory?\pa The \lord: powerful and manly:\alt{vigorous, heroic, mighty}\pa the \lord, mighty in battle.%%
  
  \pa \verse{24:9} O gates, lift up your heads!\pa Be lifted up, O everlasting doors!\pa The King of Glory shall come in!%%
  
  \pa \verse{24:10} Who is He, this King of Glory?\pa The \lord\ of Hosts,\pa He is the King of Glory. Selah!%%
\end{inparaenum}

  \heading{45}{Either an erotic psalm (unlikely\ed{``Trust me, this is not erotic.'' ---Professor Ricks, 2015-01-28}) or a psalm of marriage, a wedding hymn, and symbolically theological~--- the groom is more beautiful than the sons of men~--- he shall gird himself in glory and majesty~--- God's throne is forever and always~--- the groom is anointed~--- his name will always be remembered}

{\noindent\textit{\small Note: The king (groom) is being addressed in verses~1--10, the bride (princess) in verses~11--16, and the king again in verses~17--18.}}

\begin{inparaenum}
  {\noindent\verse{45:1} To the music director:\halot{xxxx}{in title at the beginning of a psalm (55~times)\dots\ or at the end of \haref{Hab}{3}{19}; uncertain meaning; traditionally ``for the director of the music,'' Septuagint substantive \Greek{εἰς τὸ τέλος} ($\rightarrow$\thinspace\Hebrew{נֶצַח}), Targum infinitive \Hebrew{לְשַׁבָּחָא} in glorification} \textit{On the Lilies}.\halot{xxxx}{in the titles of Psalms: \dots no certain meaning for this has yet been found,\dots\ suggestions include\dots: $\alpha$) \dots an instruction concerning the tune; $\beta$) \dots either flowers that were placed on or in fron of the Ark (perhaps as a gift of allegiance), or, if \Hebrew{עֵדוּת} means offering) ``the lilies must have been offered as flowers which were to be used in some way to obtain a divine answer to a question that had been presented to a deity'';\dots\ $\gamma$)\dots the title indicated that it was originally a love-song which was subsequently used with wider relevance; $\delta$) Glasser\dots\ takes {\Hebrew{שַׁנִּים}\hspace*{0em}\Hebrew{(וֺ)}\hspace*{0em}\Hebrew{שֹׁ}} etc.\ from Akkadian \textit{\v su\v s\v su} ``one-sixth'', and here it would mean ``a six-stringed instrument''\dots. Deciding which of these suggestions is best must be left open, but the fourth one ($\delta$) has least to support it. $\dagger$ [every Biblical reference quoted]} For\footnotemark\ the sons of Korah.\ed{Could easily be a different Korah than the infamous one in Numbers~16.} An instruction~--- a song of love.\halot{xxxx}{(for \Hebrew{יְדִידוֺת}) \textbf{love} \haref{Ps}{45}{1}. $\dagger$}\ed{Although plural, \Hebrew{יְדִידֹת} means ``love,'' not ``love(r)s,'' and is undisputed (as given by the $\dagger$).}\ca{\septuagint\symmachus\ sg, l c pc Mss \Hebrew{דוּת}\hspace*{0em}-- vel \Hebrew{דֻת}\hspace*{0em}-- cf \aquila\ \Greek{προσφιλίας}}{The Septuagint and Symmachus' Greek translation of the Old Testament give this in singular, read with a few manuscripts [give different endings with a] \textit{waw} or a \textit{qubuts}, compare Aquila ``marriage''}}%%
  \alt{By, With}%%
  
  \pvab{\verse{45:2} My heart is moved\footnotemark\ by\footnotemark\ a pleasant thing.}{I will say my verses\footnotemark\ to the king.}%%
  \alt{stirred up}%%
  \alt{with}%%
  \alt{what I've done, my doings/deeds/words}%%
  
  \pvca{My tongue is a stylus of an experienced\footnotemark\ scribe.\footnotemark}%%
  \alt{skilled, ready}%%
  \alt{writer}%%
  
  \pvab{\verse{45:3} You are fairer than the children of men.\footnotemark}{Favor\footnotemark\ is poured out into your lips.}%%
  \ed{``Shocked, on the other hand, by these revolting fancies, there were many who held that Jesus, in His earth;y features reflected the charm and beauty of David, His great ancestor; and St.\ Jerome and St.\ Augustine preferred to apply to Him the words of Psalm~xlv.\ 2, 3, `Thou art fairer than the children of men'\thinspace'' (\textit{The Life of Christ}, Farrar, quoting Augustine \textit{in Ep.\ Joh.}, tract.\ ix.\ 9).}%%
  \alt{Grace}%%
  
  \pvca{Therefore, God\footnotemark\ will blessed you forever.}%%
  \ca{1 \Hebrew{יהוה}}{one has the Tetragrammaton}%%
  
  \pvab{\verse{45:4} Gird your sword on the\footnotemark\ thigh,\footnotemark\footnotemark\ O hero.}{Gird your majesty\footnotemark\ and your grandeur\footnotemark~--- \verse{45:5} have success in your glory.\footnotemark}%%
  \ca{\septuagint\symmachus\peshitta\targum\ + suff 2 sg}{The Septuagint, Symmachus' Greek translation of the Old Testament, the Peshitta, and the Targum all add a second singular pronominal suffix [thus, ``\textit{your} thigh'']}%%
  \lit{upper thigh}%%
  \ed{This is the part that could, because this is a wedding hymn, make this erotic. However, this is more probably about enthronement.}%%
  \alt{splendor, height}%%
  \alt{splendor, glory}%%
  \ca{huc tr \Hebrew{׃}}{hither transpose a \textit{sof pasuq} (symbol for end of verse)}%%
  
  \pvca{Ride for the sake\footnotemark\ of truth and humility\footnotemark\ and righteousness.}%%
  \lit{on the word}%%
  \halot{xxxx}{(one who understands himself to be) \textbf{low}, \textbf{humble}, \textbf{gentle} (before God)}%%
  
  \pvbb{May your right hand teach you awe-inspiring\footnotemark\ things.\footnotemark}{\verse{45:6} \footnotemark Your arrows are sharp.\footnotemark\ People fall beneath you}%%
  \alt{amazing, wonderful, terrifying}%%
  \halot{xxxx}{king's fearful deeds}%%
  \ca{huc tr \super{c--c}}{hither transpose ``People fall under you.''}%%
  \ca{\septuagint\ + \Greek{δυνατέ}, ins \Hebrew{הַגִּבּוֺר}}{The Septuagint adds ``be powerful,'' insert ``mighty''}%%
  
  \pvca{in the midst\footnotemark\ of the king's enemies.}%%
  \lit{heart}%%
  
  \pvab{\verse{45:7} Your throne, O God, is forever\footnotemark\ and\footnotemark\ always.}{A staff of Your kingdom is a staff of order.\footnotemark\footnotemark}%%
  \ca{pc Mss et var sec Odonem \Hebrew{לְע׳} cf \septuagint\aquila\theodotion}{A few manuscripts and variant readings according to Odo [Bishop of Beauvais] [``-nem'' is the Latin accusative ending] add ``to/for,'' compare the Septuagint, Aquila, and Theodotion's Greek translation of the Old Testament}%%
  \ed{See further in Appendix~\ref{app:waw-voweling}}%%
  \alt{righteousness, regulation, uprightness, justice}%%
  \lit{A staff of order is a staff of Your kingdom.}%%
  
  \pvba{\verse{45:8} You love righteousness and hate\footnotemark\ wickedness.}%%
  \alt{are unable to put up with}%%
  
  \pvbb{Therefore, God\footnotemark\ (your God) has anointed you}{with oil of joy\footnotemark\ more than your friends.\footnotemark}%%
  \ca{Ms \targum\ \Hebrew{יהוה}}{a manuscript and the Targum use the Tetragrammaton}%%
  \alt{exultation}%%
  \alt{companions, associates, fellows}%%
  
  \pvba{\verse{45:9} Myrrh\footnotemark\ and aloes,\footnotemark\ \footnotemark cinnamon-flowers,\footnotemark\ all your clothes\footnotemark~---}%%
  \halot{xxxx}{\textbf{myrrh}, resin of \textit{Commiphora abessinica}}%%
  \halot{xxxx}{\textbf{aloes} (aromatic wood), \textit{Aloexyllon Agallochum} \& \textit{Aquilaria Agallocha}}%%
  \ca{pc Mss Vrs \Hebrew{וּק׳}}{a few manuscripts, all or most manuscripts, have ``and''}%%
  \halot{xxxx}{\textbf{cassia}, \textbf{cinnamon-flowers} (dried for incense)}%%
  \ed{\dots have been scented with}%%
  
  \pvbb{from ivory palaces, stringed instruments\footnotemark\ have given you joy.\footnotemark}{\verse{45:10} The kings' daughters\footnotemark\ stand firm among your nobles.\footnotemark}%%
  \ed{``stringed instruments from ivory palaces''?}%%
  \alt{made you glad/joyful/rejoice}%%
  \ca{\peshitta\ \textit{brt mlk'}, l \Hebrew{הַמֶּלֶךְ} \Hebrew{בַּת}}{the Peshitta has ``daughters of the king,'' read ``a daughter of the king''}%%
  \halot{xxxx}{\textbf{rare}\dots\ \textbf{precious stone}\dots precious (building-)stones\dots \textbf{costly}, \textbf{valuable}\dots \textbf{noble}}%%
  
  \pvca{A queen\footnotemark\footnotemark\ is\footnotemark\ at your right hand in gold from Ophir.}%%
  \halot{xxxx}{traditionally \textbf{queen}\dots\ but suggested `favorite of harem'. $\dagger$}%%
  \alt{consort}%%
  \ed{Some translations give ``stands at,'' but this verb is not present in L.}%%
  
  \pvab{\verse{45:11} Listen, O daughter, and see. Listen.\footnotemark}{Forget your people and your father's house.\footnotemark}%%
  \lit{Incline your ear.}%%
  \ed{When women come from other lands to marry, they give up their now-foreign traditions.}%%
  
  \pvba{\verse{45:12} The king\footnotemark\ will long for\footnotemark\ your beauty because he is your lord!---}%%
  \ca{prb dl}{probably to be deleted [making it ``he will long for''?]}%%
  \alt{crave}%%
  
  \pvbb{worship\footnotemark\ him.\footnotemark\footnotemark\ \verse{45:13} The daughter of Tyre with an offering,}{the rich people will flatter\footnotemark\ your face.\footnotemark}%%
  \alt{bow down before}%%
  \ca{1 frt \Hebrew{לָךְ} \Hebrew{וָה}\hspace*{0em}--- et cj c 13}{one manuscript has ``he shall worship you,'' and it [this thought] connects with verse~13}%%
  \ca{\septuagint\ 3 pl}{the Septuagint gives this in third person plural [i.e., ``you shall worship them'']}%%
  \alt{appease}%%
  \ed{in reference to her beauty; easy on the eyes}%%
  
  \pvbb{\verse{45:14} The princess\footnotemark\footnotemark\ is glorious within~---}{\footnotemark her clothes are woven\footnotemark}%%
  \ca{prb dl m cs}{probably deleted because of the meter case}%%
  \lit{daughter of the king}%%
  \ed{works better as a separate stich}%%
  \ed{the root (\Hebrew{שׁבּץ}) means to ``weave in patterns''}%%
  
  \pvbb{with gold.\footnotemark\ \verse{45:15} \footnotemark She will be brought\footnotemark\ before the king in brightly colored clothes.\footnotemark}{The virgins\footnotemark\ and her friends\footnotemark\ after her shall be brought in to you.\footnotemark}%%
  \ed{reworked because Hebrew grammar is non-idiomatic for English. This literally says ``are woven with gold / her clothes.''}%%
  \ca{15.\ 16 crrp, metrum inc}{verses 15 and 16 corrupted, meter uncertain}%%
  \halot{xxxx}{as a bride}%%
  \lit{fabric of a variety of colors.}%%
  \halot{xxxx}{\textbf{virgin}:---1.\ mature girl `whom no man has known'}%%
  \ed{more probably just a ``young woman'' with no particular regard for whether she's a virgin}%%
  \ca{prb dl}{probably deleted}%%
  \ca{2 Mss \Hebrew{לָהּ}, \missing\ \septuagint\super{min}\ \peshitta; prp \Hebrew{הֹלְכֹת} (hpgr)}{two manuscripts have ``to her,'' but this is missing from the Septuagint (codices minusculis scripti) and the Peshitta. It has been proposed that there is missing a verb meaning ``to bring in'' (by haplography [the characters present are \Hebrew{לך} and it's been proposed that a second \Hebrew{כ} is missing which would make \Hebrew{הלכת}])}%% xxxx make sense of this haplography
  
  \pvba{\verse{45:16} They are brought in with joy and rejoicing. They enter the palace of the king.\footnotemark}%%
  \ed{although in the \textsc{bhs} it appears that ``of the king'' is on a separate line, this is just a typesetting problem and it should not be its own stich.}%%
  
  \pvab{\verse{45:17} Instead of your fathers it shall be your sons.\footnotemark}{You shall make\footnotemark\ them\footnotemark\ princes throughout the whole world.}%%
  \ie{your posterity will succeed your fathers}%%
  \ed{\Hebrew{מוֹ}\hspace*{0em}--- is the poetic equivalent of \Hebrew{הֶן}\hspace*{0em}--- and \Hebrew{הֶם}\hspace*{0em}---}%%
  \alt{appoint}%%
  
  \pvab{\verse{45:18} I will cause your name\footnotemark\ to be remembered\footnotemark\ throughout all generations.\footnotemark}{Therefore, the people shall praise You forever and always.\footnotemark}%%
  \ie{the king's name}%%
  \lit{(cohortative) let me mention your name}%%
  \alt{for endless generations.}%%
  \ca{prb dl alterutrum m cs}{either one or the other [either ``[t]herefore'' or ``and always''] should be deleted because of the meter case}%%
\end{inparaenum}

  \heading{46}{God is our refuge and strength~--- His mighty power enumerated~--- be still and come to know God}

\begin{inparaenum}
  \noindent\verse{46:1} To the music director.\cf{\vref{Ps}{45}{1}} For the sons of Korah. For the \Hebrew{עֲלָמוֺת}.\halot{xxxx}{unexplained term in performance} A song.\smallskip%%
  
  \pb \verse{46:2} God is our refuge and strength,\pa well proven as a help in distresses.%%
  
  \pb \verse{46:3} Therefore, we will not fear though the earth moves\pa and the mountains shake in the midst\lit{heart} of the seas.%%
  
  \pb \verse{46:4} The waters shake and foam;\pa the mountains shake as it swells.%
  \ca{ins \Hebrew{יעקב} \Hebrew{אלהי} \Hebrew{משגב־לנו} \Hebrew{עמנו} \Hebrew{צבאות} \Hebrew{יהוה} sec 8.12}{insert ``The \lord\ of Hosts is with us. The God of Jacob is our refuge'' according to verses~8 and~12 [like a chorus]}
  \pa Selah!%%
  
  \pc \verse{46:5} The river,%
  \halot{xxxx}{river around city of God \haref{Ps}{46}{5}}
  its streams, make%
  \ed{given in piel which can be causative}
  the city of God rejoice:\pa the holy dwelling-places of the Most High.%%
  
  \pb \verse{46:6} God is in the midst of it: it will not be shaken.\pa God will help at dawn.\lit{at the dawn of the morning.}%%
  
  \pb \verse{46:7} The nations raged and the kingdoms tottered.\pa He speaks and the earth melts.%%
  
  \pb \verse{46:8} The \lord\ of Hosts is with us.\pa The God of Jacob is our refuge.\pa Selah!%%
  
  \pc \verse{46:9} Come and see the works of the \lord%
  \ca{mlt Mss \septuagint\super{A\textit{L}}\peshitta\ \Hebrew{אֱלֹהִים} ut 66,5}{multiple Hebrew manuscript codices, the Septuagint (\textit{codex Alexandrinus and textus Graecus ex recensione Luciani}), and the Peshitta have ``God,'' as in \vref{Ps}{66}{5}}
  \pa who has set desolations in the earth.%%
  
  \pb \verse{46:10} He makes war cease to the end of the earth.%%
  
  \pc He smashes the bows and cuts up the spears.\pa He burns chariots%
  \ca{\septuagint\ \Greek{καὶ θυρεούς} = \Hebrew{וַעֲגִלוֺת} cf \targum}{the Septuagint has ``shields'' (equivalent to the Hebrew word for the same), compare the reading found in the Targum}
  with fire.%%
  
  \pb \verse{46:11} Be still and know that I am God.\pa I am exalted among the nations. I am exalted in the earth.%%
  
  \pb \verse{46:12} The \lord\ of Hosts is with us.\pa The God of Jacob is our refuge.\pa Selah!%%
\end{inparaenum}

  \heading{47}{Praise God~--- He rules over all}

\begin{inparaenum}
  \noindent\verse{47:1} To the music director.\cf{\vref{Ps}{45}{1}} A psalm for the sons of Korah.\smallskip%%
  
  \pb \verse{47:2} All the people, clap your hands!\pa Shout to God%
  \ca{\missing\ pc Mss}{missing in a few manuscripts}
  with a shout of joy!%%
  
  \pb \verse{47:3} Because the \lord, the Most High,\alt{is exalted} is awe-inspiring,\pa a great king over the whole earth.%%
  
  \pb \verse{47:4} He subjugates the people beneath us\pa and the people%
  \halot{xxxx}{1.\ \textbf{people} (ethnic community) (archaic word) \haref{Gn}{25}{23} \& often;---2.\ \textbf{people} (in general, French \textit{on}) \hadagger \haref{Pr}{11}{26}}
  under out feet.%%
  
  \pb \verse{47:5} He has chosen our%
  \ca{\septuagint\peshitta\ suff 3 sg}{the Septuagint and Peshitta have a third singular pronominal suffix [i.e., ``his heritage'']}
  heritage%
  \halot{xxxx}{(inalienable) \textbf{hereditary possession}, \textbf{heritage}, acquired by individual or family by conquest or inheritance, both property (i.e., land and buildings) and (movable) goods}
  for us,\pa the pride%
  \alt{loftiness, arrogance}
  \ed{other translations give ``excellency''}
  of Jacob whom He loves.\pa Selah!%%
  
  \pc \verse{47:6} God has gone up amidst shouting,\pa the \lord\ amidst\lit{with} the sound of a shofar.%%
  
  \pb \verse{47:7} Praise%
  \alt{sing, play an instrument [for all uses of ``praise'' in this verse]}
  God!%
  \ed{understood: in song}
  \ca{pc Mss \Hebrew{לֵא׳}, Ms \septuagint\ \Hebrew{לֵאלֹהֵנוּ}}{a few manuscripts have the preposition \textit{to}, and a manuscript and the Septuagint have ``to our God''}
  Praise!%
  \ca{\peshitta\ \textit{b\v{s}wb\d{h}'} in splendore}{the Peshitta says ``in splendor'' [instead of ``Praise!'']}%%
  \pa Praise our King! Praise%
  \ca{\missing\ 2 Mss \peshitta}{missing in two manuscripts and the Peshitta}%%
  
  \pb \verse{47:8} because God is the king%
  \ca{mlt Mss + \Hebrew{עַל־}}{multiple manuscripts add ``on'' [making it ``God is king over the whole earth,'' ``of'' simply being understood by context]}
  of the whole earth.\pa Praise with insight.%
  \halot{xxxx}{unclarified musical term in Psalms; suggesting: cultic-song; passage for learning; wisdom-song put to music; in contrast or opposition to \textit{ma\'sk\^il} `with insight,' verbal root \textit{\'skl} hiphil participle}%%
  \halotu{xxxx}{Septuagint \Greek{συνέσεως} [prudence], \Greek{εἰς σύνεσιν} [in prudence], Vulgate \textit{intellectus} [understanding], Jerome \textit{eruditio} [learning], \Hebrew{טָבָא} \Hebrew{שׂכְלָא}\hspace*{0em}/\hspace*{0em}\Hebrew{סִ}; technical term but meaning unclear: ``cult song'' (Kittel), ``memory passage'' (Maag 193f), wisdom song performed to music\dots\ ---in contrast or opposition to \Hebrew{מַשְׂכִּיל} with insight, I \Hebrew{שׂכל} hiphil participle \hadagger}%%
  
  \pb \verse{47:9} God rules over the nations,%
  \ca{2 Mss \septuagint\super{RAa1} + \Hebrew{כָּל־}}{two manuscripts and the Septuagint (codex Veronensis, codex Alexandrinus, xxxx) add ``whole/all/entire'' [i.e., ``over all the nations'']}
  \pa God%
  \ca{\missing\ Ms, frt dl}{missing in a manuscript, perhaps deleted [so simply ``He sits on\dots'']}
  sits on his holy throne.%%
  
  \pb \verse{47:10} the nobles\alt{noblemen, generous, willing, noble} are gathered%
  \ed{The niphal form of \Hebrew{אסף} means ``to gather'' when the verb is intransitive (as it is here). Were this verb transitive it could mean ``to conspire against'' or ``to unite against.''}
  \pa with%
  \ed{prepositions in poetry are generally understood through context}
  \ca{prp \Hebrew{אָהֳלֵי} \Hebrew{עִם}}{proposed ``with the tent of''}
  the people%
  \ca{\septuagint(\peshitta) \Greek{μετά} = \Hebrew{עִם}; read prb \Hebrew{עַם} \Hebrew{עִם} (hpgr)}{the Septuagint and Peshitta have ``with'' [instead of ``people'' (\Hebrew{עַם})]; read ``with the people'' (haplography)}
  of the God of Abraham.%%
  
  \pc Because the protection\alt{shield, refuge (metaphorically)} of the earth is God's.\pa He is greatly exalted.%%
\end{inparaenum}

  % ps66:5, compare 46:9 (already referenced the other way)
  \heading{82}{God will judge the earth~--- care for the poor and the needy}

\begin{inparaenum}
  {\noindent\verse{82:1} A psalm of Asaph:}%%
  
  \pvbb{God stands in the council of El~---}{He judges among the gods.}%%
  
  \pvbb{\vn{82:2} How long\footnotemark\ will you judge unrighteously?}{How long will you accept\footnotemark\ the face of the wicked? Selah!}%%
  \fntlit{Until when}%%
  \fntalt{lift up}%%
  
  \pvab{\vn{82:3} Judge the\footnotemark\ poor and the fatherless;}{bring justice to the humble\footnotemark\ and the poor.}%%
  \fntca{ex 4; l prb \Hebrew{דָּךְ}}{from 4; probably read ``your poor''}%%
  \fnthalot{xxxx}{(one who understands himself to be) \textbf{low}, \textbf{humble}, \textbf{gentle} (before God)}%%
  
  \pvab{\vn{82:4} Rescue the poor and the needy~---}{deliver them from the wicked's control.\footnotemark}%%
  \fntalt{hand, power}%%
  
  \pvbc{\vn{82:5} They don't know. They don't understand.\footnotemark}{They walk in darkness.}{All of the foundations of the earth waver.\footnotemark}%%
  \fntalt{They know nothing. They understand nothing.}%%
  \fntalt{stagger, move, totter, shake.}%%
  
  \pvab{\vn{82:6} I said, ``You are gods~---}{all of you are children of the Most High.}%%
  
  \pvab{\vn{82:7} You will die like men.}{You will fall like one of the rulers.\footnotemark}%%
  \fntalt{princes.}%%
  
  \pvbb{\vn{82:8} Rise, O God. Judge the earth}{because You will take possession the whole earth.''}%%
\end{inparaenum}

  \heading{93}{The Lord is exalted~--- He is eternal~--- even the rivers praise Him}

\begin{inparaenum}
  \pa \verse{93:1}%
  \ca{\septuagint\ pr \Greek{εἰς τὴν ήμέραν τοῦ προσαββάτον, ὄτε κατῴκισται ἡ γῆ}$\cdot$ \Greek{αἾνος ᾠδῆς τῷ Δαυιδ}}{the Septuagint inserts the following before everything else: ``In the day of xxxx, xxxx xxxx Earth$\cdot$ xxxx xxxx xxxx David''}
  The \lord, clothed in majesty,\alt{loftiness, excellency} reigns.\pa The \lord\ has clothed Himself, He has girded Himself with strength.%%
  
  \pb The earth\alt{continents, world} is established, it shall not be moved.\alt{will not stagger.}\pa \verse{93:2} Your throne is established from eternity:~---\pa You are from eternity.\alt{forever.}%%
  
  \pa \verse{93:3} The rivers have been lifted up, O \lord.\pa The rivers have lifted up their voices.\alt{sound.}\pa The rivers have lifted up their breaking waves.%
  \ca{\missing\ \septuagint*}{[this last bit] missing in the Septuagint (textus Graecus originalis)}
  \halot{xxxx}{\textbf{pounding} (of waves) \haref{Ps}{93}{3}. $\dagger$}%%
  
  \pa \verse{93:4} Mightier than the sound of the waves,\alt{many waters,}\pa than the mighty surf,\lit{breakers of the sea}\pa is the \lord\ on high.%
  \halot{xxxx}{\textbf{height} (i.e. place on high): --- 1.\ \textbf{elevation of ground} \haref{2K}{19}{23}, \textit{r\=o'\v s m\superit{e}r\^om\^im} \haref{Pr}{8}{2}; --- 2.\ highly placed location, \textbf{place on high} \haref{Is}{22}{16}; =~exalted \haref{Je}{17}{12}; --- 3.\ \textbf{upward} (adverb accusative) \haref{2K}{19}{22}; --- 4.\ \textbf{high social position} \haref{Jb}{5}{11}; --- 5.\ moral: \textit{mimm\=ar\^om} from a high position, loftily \haref{Ps}{73}{8}; --- 6.\ \textit{m\=ar\^om} =~heaven: a)~windows of heaven \haref{Is}{24}{18}, host of heaven \haref{Is}{24}{21}; b)~home of God (singular and plural) \haref{Is}{33}{5}; c)~to heaven \textit{lamm\=ar\^om} \haref{Is}{38}{14}, from heaven \textit{mimm\=ar\^om} \haref{2S}{22}{17}.}%%
  
  \pa \verse{93:5} Your statutes\alt{testimonies} are very sure.\alt{reliable, stable, firm}\pa Your house is befitting to holiness,\alt{Holiness is to your beautiful house,}\pa O \lord, forever.\lit{for length of days.}%%
\end{inparaenum}

  \heading{97}{Praise God~--- He is greater than the pagan gods}

\begin{inparaenum}
  \pvab{\verse{97:1} The \textsc{Lord} reigns. Let the earth be glad,}{let the many islands\footnotemark\footnotemark\ be joyful.\footnotemark}%%
  \halot{xxxx}{the islands and coasts of the Mediterranean are for the Old Testament the extremes of the western world.}%%
  \alt{Phoenicians}%%
  \alt{rejoice.}%%
  
  \pvab{\verse{97:2} Clouds and darkness surround Him~---}{righteousness and judgment are the support of His throne.}%%
  
  \pvab{\verse{97:3} Fire goes before Him}{and devours\footnotemark\ round about His\footnotemark\ enemies.}%%
  \alt{scorches}%%
  \ca{pc Mss \Hebrew{סְבִיבָיו}}{a few manuscripts have ``round about His'' [justifying the present (understood) rendering]}%%
  
  \pvab{\verse{97:4} His lightning has lit the world~---}{the earth saw and xxxx-root?.\footnotemark}%%
  \halot{xxxx}{xxxx}%%
  
  \pvab{\verse{97:5} The mountains melted like wax\footnotemark before the \textsc{Lord},\footnotemark}{from before\footnotemark\ the Lord of the whole earth.}%%
  \halot{xxxx}{always metaphor of melting}%%
  \ca{prb dl}{[``before the \textsc{Lord}''] should probably be deleted}%%
  \ca{\missing\ \peshitta}{[``from before'' is] missing in the Peshitta}%%
  
  \pvab{\verse{97:6} The heavens announce His righteousness,}{all the people see His glory.}%%
  
  \pvac{\verse{97:7} All the idol worshipers are ashamed~---}{those who boast in pagan gods:~---\footnotemark}{bow\footnotemark\ to Him, all you gods.\footnotemark\footnotemark}%%
  \halot{xxxx}{always contemptuously as nonentities, idols}%%
  \ca{\septuagint\peshitta\ imp}{the Septuagint and Peshitta have this in imperative form}%%
  \ca{\septuagint(\peshitta) \Greek{οἱ ἄγγελοι αὐτοῦ}}{the Septuagint (and Peshitta) have ``angels''}%%
  \ed{This is typical behavior for the Septuagint to avoid any reference to the anthropomorphic nature of God.}%%
  
  \pvac{\verse{97:8} Zion has heard and rejoiced,}{the daughters of Judah are joyful,}{on account of Your judgments, O \textsc{Lord},}%%
  
  \pvab{\verse{97:9} because You, O \textsc{Lord},\footnotemark\ are Most High over all the earth;}{You have been greatly exalted over all the gods.}%%
  \ca{\missing\ Ms, dl m cs}{missing in a manuscript, should be deleted to preserve the meter case}%%
  
  \pvac{\verse{97:10} Lovers of the \textsc{Lord}, hate evil:}{He protects the souls of His devout,}{He delivers them from the power of the wicked.}%%
  
  \pvab{\verse{97:11} Light is sown\footnotemark\ for the righteous,}{and joy for the righteous\footnotemark\ of heart.}%%
  \ca{1 c Ms Vrs \Hebrew{זָרַח} ut 112,4}{one with all or most translations has ``to shine'' [``has shone'' or ``is shining''] as in \vref{Ps}{112}{4}}%%
  \alt{just, upright}%%
  
  \pvab{\verse{97:12} Rejoice, O righteous ones, in the \textsc{Lord}!}{Praise His holy name!\footnotemark}%%
  \lit{remembrance, memory}%%
\end{inparaenum}

  \heading{110}{Christ will sit at the Father's right hand and will have the Melchizedek priesthood}\cf{Appendix~\ref{app:psalm-110}}

\begin{inparaenum}
  \noindent\verse{110:1} A psalm of David:\smallskip%%

  \pb The \textsc{Lord} declared to my lord:%
  \cf{Appendix~\ref{app:adonai}}
  \ed{Or any king, not necessarily David. Since this is a messianic psalm, it directed at Christ.}
  \pa ``Sit at my right hand%%

  \pb until I put your enemies\pa as your footstool.\lit{a footstool for your feet.}%%

  \pa \verse{110:2} Your mighty scepter\pa the \textsc{Lord} will send from Zion:\pa rule in the midst of your enemies.%%

  \pa \verse{110:3} Your people will voluntary gifts\footnotemark\pa in the day of your strength;%%
  \alt{free-will offering}%%

  \pb in royal robes%
  \alt{holy splendor}
  \ca{prb l c \fragheb\ mlt Mss \symmachus\ Hier \Hebrew{בהררי}}{probably read with the Cairo Genizah and multiple manuscripts from Symmachus' Greek translation of the Old Testament and Hieronymus give \Hebrew{הרר} [instead of \Hebrew{הדר}]}
  \pa from the womb, from the morning light,%
  \halot{xxxx}{(reddish) \textbf{(light before) dawn}}%%
  \pa to you the dew%
  \ca{\missing\ \septuagint; prp \Hebrew{כְּטַל}}{missing in the Septuagint; probably ``as the dew''}
  \alt{light rain}
  of your youth.%%

  \pa \verse{110:4} The \textsc{Lord} has sworn\pa and will not have a change of heart:%%

  \pb You are forever a priest\pa after the Order of Melchizedek.%%

  \pb \verse{110:5} The Lord%
  \ca{\fragheb\ mlt Mss \Hebrew{יהוה}}{multiple manuscripts from the Cairo Genizah have the Tetragrammaton}
  at your right hand\pa will beat kings to pieces\alt{smite} in the day of His anger.%%

  \pa \verse{110:6} He will judge among the nations;\pa He will fill {the nations with}%
  \ed{Understood from parallel in first hemistich.}
  corpses;\pa He will beat the heads in pieces\pa over the great earth.%%

  \pa \verse{110:7} He will drink from the brook by the road,\pa therefore he will lift up the head.%
  \ca{\fragheb\ \Hebrew{ראשׁי}, 2 Mss \peshitta\ \Hebrew{רֹאשׁוֺ}, 3 Mss + \Hebrew{הללויה} (2 Mss om in 111,1)}{The Cairo Genizah contains ``my head,'' two manuscripts of the Peshitta have ``his head,'' and three manuscripts include ``Hallelujah'' (two of which omit it in \vref{Ps}{111}{1})}%%
\end{inparaenum}

  
  % \book{Isaiah}{\Hebrew{ישעה}}
  % {\noindent\textit{\small See Appendix~\ref{app:isaiah} for more information on the book of Isaiah.}}
  % \heading{1}{Few in Israel remain faithful to the Lord~--- the Lord rejects their sacrifices and feasts~--- repentance proclaimed~--- Zion to be redeemed in the latter days}

\begin{inparaenum}
  \verse{1:1} %%
  \verse{1:2} %%
  \verse{1:3} %%
  \verse{1:4} %%
  \verse{1:5} %%
  \verse{1:6} %%
  \verse{1:7} %%
  \verse{1:8} %%
  \verse{1:9} %%
  \verse{1:10} %%
  \verse{1:11} %%
  \verse{1:12} %%
  \verse{1:13} %%
  \verse{1:14} %%
  
  \pvbb{\vn{1:15} And when you spread your hands,\mpfootnotemark}{I hid my eyes from you.}%%
  \footnotetext{i.e., to pray}%%
  
  \pvcb{Because you pray a lot,\mpfootnotemark}{I will not hear.\mpfootnotemark}%%
  \addtocounter{footnote}{-1}%%
  \footnotetext{lit., multiply prayers}%%
  \stepcounter{footnote}%%
  \footnotetext{alt., Your prayers are nothing to me.}%%
  
  \pvcb{Your hands are full of blood.}{\vn{1:16} Wash and purify yourselves.}%%
  
  \pvcb{Turn away from doing evil}{before my eyes.}%%
  
  \pvcb{Cease to do evil.}{\vn{1:17} Learn to do good.}%%
  
  \pvcb{Seek judgment.}{Make the oppressed happy.}%%
  
  \pvcb{Judge the fatherless;\mpfootnotemark}{learn to do good.}%%
  \footnotetext{orphans}%%
  
  \pvbb{\vn{1:18} I pray thee, come and let us reason together,''}{saith the \textsc{Lord}.}%%

  \pvcb{``If your sins are as scarlet,}{as snow they shall be white.}%%

  \pvcb{If they are blood\mpfootnotemark\ red,\mpfootnotemark}{they shall be as wool.}%%
  \addtocounter{footnote}{-1}%%
  \footnotetext{earth}%%
  \stepcounter{footnote}%%
  \footnotetext{The dye that was used back then was permanent. The cloth could fade, but would never again be truly white.}%%
  
  \pvbb{\vn{1:19} If you're willing and hearken,}{you shall consume\mpfootnotemark\ the good of the land.}%%%
  \footnotetext{alt., eat of}%%
  
  \pvbb{\vn{1:20} And if you refuse and rebel,}{the sword shall consume you}%%
  
  \pvca{for the mouth of the \textsc{Lord} hath proclaimed it.\mpfootnotemark}%%
  \footnotetext{alt., so spoken.}%%
  
  \verse{1:21} %%
  \verse{1:22} %%
  \verse{1:23} %%
  \verse{1:24} Therefore, thus saith the \textsc{Lord} of Hosts, The Mighty One of Israel:\footnote{The one in Israel who is mighty}%%

:``Ah, now I will be relieved\footnote{eased} of mine adversaries: I am avenged of mine enemies.%%
  \verse{1:25} Lest I turn my hand on thee\footnote{I will turn my hand back on thee}%%

:I will purify thine dross~--- I will turn aside all thine tin.\footnote{Tin \emph{is} useful. It is used to make brass (a copper and tin alloy). One of the symbolisms here is that although tin is useful, the Lord has a greater plan in mind for each of us. Therefore, we need to listen to Him and do as He commands although we may think that what we are doing in lieu of obeying is important and useful.}%%
  \verse{1:26} I will restore\footnote{return} thine judges as at first And thy counselors as in the beginning.%%

:After this though shalt be called \textit{A City of Righteousness: A Faithful City}.%%
  \verse{1:27} Zion is redeemed through judgment~--- Also those who are returned\footnote{rescued ones, captives}\footnote{i.e., they are also redeemed} in righteousness.%%
  \verse{1:28} The sinners and transgressors are destroyed together; Those forsaking the \textsc{Lord} are consumed.%%
  \verse{1:29} You are ashamed of the oaks\footnote{Idols used for fertility worship.} That you've desired.%%

:And you're confused because of the groves\footnote{gardens} That you've chosen.%%
  \verse{1:30} For you are as an oak Whose leaf is fading%%

:And as a grove That hath no water.%%
  \verse{1:31} The strong shall be as tow\footnote{Synonymous to oakum (n): Loose fiber from untwisted rope, used esp. to caulk wooden ships.} And its maker as spark.%%

:They shall burn together: None shall quench them.''%%
\end{inparaenum}

  % \heading{2}{Isaiah sees in vision the latter day temple, the gathering of Israel, the Millennium~--- the proud to be humbled at the Savior's Second Coming}

\begin{inparaenum}
  \verse{2:1} The thing that Isaiah the son of Amoz foresaw concerning Judah and Jerusalem.%%
  %% : represents a tab. There are no colons in this chapter.
  \verse{2:2} ::And in the last days:the mountain of the \textsc{Lord}'s house will be established%%
  
  ::in the tops of the mountains.:It shall be lifted up above the hills.%%
  
  ::All nations shall flow\footnote{Like a river} unto it.\verse{2:3} Tons of people will walk to it and say,%%
  
  ::``Come, and let us ascend unto the \textsc{Lord}'s mountain~--- :to the house of the God of Jacob.%%
  
  ::He will teach us of His ways.:We will walk in His paths\footnote{Theologically it should be ``path''}%%
  
  ::For the law\alt{teaching, instruction. Traditionally rendered ``law''} goes forth from Zion:and the word of the \textsc{Lord} from Jerusalem.''%%
  \verse{2:4} He shall judge among the nations;: He's arbitrates between many people.%%
  
  ::They'll forge ploughshares from their swords:and pruning hooks from their spears.%%
  
  ::One nation shall not lift its sword against another,:neither shall they learn warfare anymore.%%
  \verse{2:5} ::Come, O house of Jacob, that we may walk in the \textsc{Lord}'s light.%%
  \verse{2:6} :For you have left your people~--- the house of Jacob~--- to themselves%%
  
  :because they've been filled\footnote{Possibly missing ``with superstition.'' i.e., ``filled with superstition from the east.'' There is no expressly-stated object} from the east~--- :they're sorcerers like the Philistines.%%
  
  :They please themselves\footnote{Or ``clasp hands'' or ``make sufficient''} with foreigner's\footnote{Outsiders of Israel~--- foreigners, infidels, pagans, etc.} children.\footnote{This is to be understood in a sexual context}%%
  \verse{2:7} :Their land is full of silver and gold~--- :there is no end to their treasures.%%
  
  :Their land is full of horses~--- :there is no end to their chariots.%%
  \verse{2:8} :Their land is full of idols%%
  
  :They bow down before the work of their own hands~--- :that which their fingers have made.%%
  \verse{2:9} The low shall be bowed down; the haughty humbled. :Don't forgive them.%%
  \verse{2:10} Enter into a boulder, :hide in the dust,%%
  
  :from before the \textsc{Lord}'s face and the glory of His majesty.
  \verse{2:11} %%
  \verse{2:12} %%
  \verse{2:13} %%
  \verse{2:14} %%
  \verse{2:15} %%
  \verse{2:16} \cf{Appendix \ref{app:isa-2-16}}Upon the ships of Tarshish\ed{The ships going to Tarshish. Either in Asia Minor (where Paul was from) or in present-day Spain.} :and upon the ships all the beautiful vessels.\footnote{excellent ships}%%
  \verse{2:17} %%
  \verse{2:18} %%
  \verse{2:19} %%
  \verse{2:20} %%
  \verse{2:21} %%
  \verse{2:22} %%
\end{inparaenum}

  % \heading{6}{Throne theophany and divine council~--- seraphs praise the Lord~--- Isaiah proclaims his sins and their cause~--- he is forgiven of his sins~--- the Lord calls Isaiah to proclaim the gospel}

\begin{inparaenum}
    \verse{6:1} In the year King Uzziah died, I saw the Lord sitting on a high and exalted throne, His flowing skirt filled the temple.%%
    \verse{6:2} Seraphs stood right above Him. Each had six wings: they each covered their faces with two, their feet with two, and with two they flew.%%
    \verse{6:3} They proclaimed to each other, and said,\smallskip%%
    
    \pvcb{``Holy, holy, holy is the \textsc{Lord} of Hosts.}{The whole earth is full of His glory.''}%%
    
    {\noindent\verse{6:4} The threshold's foundations moved about at the sound of the crier; the house was filled with smoke.}%%
    \verse{6:5} I said:\smallskip%%
    
    \pvcb{``Woe is me because I am undone}{on account that I'm a man of unclean lips;}%%
    
    \pvca{because I live among people with unclean lips;}%%
    
    \pvca{because my eyes have seen the King, the \textsc{Lord} of Hosts.''}%%
    
    {\noindent\verse{6:6} One of the seraphs flew to me. In his hand he had a glowing coal which he'd taken with tongs from the altar.}%%
    \verse{6:7} He touched me on my mouth,\lit{He caused it to touch my lip}\ed{understood: with it} and said,\smallskip%%
    
    \pvcb{``This has indeed touched your lips:}{your guilt is taken away and your sins expiated.\footnotemark}%%
    \fnted{\Hebrew{רָז}, from Persian and attested in Aramaic, is equivalent to the Hebrew \Hebrew{סוֺד}, \Hebrew{סוֺד} originally meaning ``a divine council,'' ``a council in Heaven,'' or ``the decision of a divine council.'' The Latin came to mean ``secret'' because the decision of the divine council was a secret, cf.\ \vref{Amos}{3}{7}.}%%
    
    \pvac{\vn{6:8} I heard the voice of the Lord, saying,}{``Whom shall I send?}{Who shall go for us?''}%%
    
    \pvac{And I said, ``I'm here.\footnotemark\ Send me.''}{\vn{6:9} And He said,}{``Go. You shall say to this people:}%%
    \fnted{In Abraham's time this could simply be rendered as ``Yes.''}%%
    
    \pvcb{`Pay strict attention, yet you don't understand.}{Watch very carefully, yet you don't know.'\footnotemark}%%
    \fnted{given in the prophetic perfect tense}%%
    
    \pvbb{\vn{6:10} Fatten the heart of this people,}{make their ears heavy, blind their eyes,\footnotemark}%%
    \ed{with pitch}%%
    
    \pvcb{lest they see with their eyes and hear with their ears,}{their hearts\footnotemark\ consider, and they repent and be healed.''\footnotemark}%%
    \fnted{It's wonderful how this language exemplifies that they're united in their rebellion against the Lord.}%%
    \fntlit{they turn back and there is a healing for them.''}%%
    
    \pvac{\vn{6:11} And I said,}{``How long, Lord?''}{and He said,}%%
    
    \pvcb{``Until the cities}{are desolated without inhabitant.\footnotemark}%%
    \fnted{This is literally ``Until wasted / cities without inhabitant.'' It had to be reshuffled so it was idiomatic with SVO~English.}%%
    
    \pvcb{Until\footnotemark\ houses are without men}{and the ground becomes a desolation.}%%
    \fnted{repeated}%%
    
    \pvbb{\vn{6:12} Until\footnotemark\ the \textsc{Lord} has distanced man,}{and the forsakeness is great in the midst\footnotemark\ of the land.}%%
    \fnted{repeated}%%
    \fntlit{heart}%%
    
    \pvbb{\vn{6:13} And yet a tenth}{shall repent\footnotemark\ and shall be burned}%%
    \fntie{a remnant shall return}%%
    
    \pvcb{as the terebinth\footnotemark\ and the oak}{which\footnotemark\ leave a stump\footnotemark\ when they are cut down\footnotemark~---}%%
    \fnted{a small tree that used to be a source of turpentine}%%
    \fntca{prp \Hebrew{אֲשֵׁרָה}}{proposed to be ``Asherah'' [making it ``as the terebinth and the oak~--- Asherah\dots'']}%%
    \fnthalot{xxxx}{earlier ``rootstock'' impossible; either ``bare trunk'' after burning of branches, or ``new growth''}%%
    \fntca{\qumran\super{a} \Hebrew{במָה}}{1QIsa\super{a} has (from \textsc{halot}) ``4.\ (cultic) \textbf{high place}\dots\ associated with pagan worship and cultic prostitution''}%%
    
    \pvca{so shall the holy seed\footnotemark\ be a stump.''\footnotemark\footnotemark}%%
    \fntca{\missing\ \septuagint}{[``as the terebinth\dots\ the holy seed''] is missing in the Septuagint}%%
    \fnted{understood: in the land.''}%%
    \fntca{dl}{[this line] should be deleted}%%
\end{inparaenum}

  % \heading{7}{Syria, Ephraim, and the son of Remaliah conspire against the house of David~--- xxxx}

\begin{inparaenum}
  \verse{7:1} In the days of Ahaz, son of Jotham, son of Uzziah king of Judah, that Rezin the king of Syria and Pekah, the son of Remaliah, king of Israel, went up to Jerusalem to wage war against it, but he was not able to\ca{\qumran\super{a}\septuagint\peshitta\vulgate\ et \caref{2~R}{16}{5} \Hebrew{יָכְלוּ}}{Qumran, the Septuagint, the Peshitta, the Vulgate, and \vref{2~Ki}{16}{5} have ``they were not able to''} endure against it.\ca{\missing\ \caref{2~R}{16}{5}}{[``against it'' is] missing in \vref{2~Ki}{16}{5}}%%
  \verse{7:2} Is it told to the house of David, saying, ``Syria is in cahoots with Ephraim.'' xxxx%%
  \verse{7:3} The \textsc{Lord} said to Isaiah, ``Please go and meet Ahaz~--- you and your son Shear-jashub~--- to the end of the aqueduct\alt{conduit} of the upper pool, to the highway of the fuller's field.''%%
  \verse{7:4} Say to him, ``Take heed and be quiet! Don't let your heart be timid before these two, smoking firebrands\ed{used to stir a fire} because of the fierce anger of Rezin, Syria, and the son of Remaliah.%%
  \verse{7:5} Because that Syria counseled evil upon you, Ephraim and the son of Remaliah,\ie{all three counseling against the house of David} saying,%%
  \verse{7:6} `We're going up to Judah and shall vex\alt{harass, rouse} it: we shall rend it unto ourselves and cause a king to reign in its midst~--- the son of Tabeal.'\smallskip%%
  
  \pvab{\verse{7:7} Thus said the Lord \textsc{God},}{`It shall neither stand nor come to pass!'}%%
  
  \pvab{\verse{7:8} For Syria's head is Damascus;}{Damascus' head is Rezin.}%%
  
  \pvbb{Ephraim will be a broken people}{within sixty-five years.}%%
  
  \pvab{\verse{7:9} Ephraim's head is Samaria;}{Samaria's head is the son of Remaliah.}%%
  
  \pvbb{If you don't believe}{you shall surely not be established.''}%%
  
  \verse{7:10} The \textsc{Lord} again spoke to Ahaz, saying,%%
  \verse{7:11} ``Ask the \textsc{Lord} your God for a sign: ask to the depths of Sheol, raise it on high.''\ie{you can ask for whatever you want: full range}%%
  \verse{7:12} And Ahaz said, ``I will neither ask nor try\alt{test, task} the \textsc{Lord}.''%%
  \verse{7:13} So He said, ``Please hear,\alt{Hear now} O house of David: is wearying man such a small thing\alt{so trivial} that you will also weary my God?%%
  \verse{7:14} Therefore, the Lord Himself shall give you a sign: a young woman\ed{at least a woman of sexual maturity; it's from the Septuagint that we get the rendering ``virgin'' although it has that meaning in Hebrew. The Greek \Greek{παρθένος} and Latin \textit{virgo}, a woman who has not known man. In Hebrew, a young woman or maiden, whether or not she's known man} shall conceive and bring forth a son and name him Immanuel.\ed{this might not be Messianic}%%
  \verse{7:15} He shall eat butter and honey and thereby know to refuse evil and fixate on\alt{choose} good.%%
  \verse{7:16} For before the child knows to refuse the evil and choose the good, the land you're vexed with\alt{which you fear} shall be forsaken because of its two kings.%%
  \verse{7:17} The \textsc{Lord} will bring upon you, your people, and your father's house days which have not come since the day when Ephraim turned from Judah~--- even the king of Assyria.\smallskip%%
  
  \pvaa{\verse{7:18} In that day}%%
  
  \pvbb{the \textsc{Lord} will hiss for a fly}{that is in the extremity of the brooks of Egypt}%%
  
  \pvba{and for a bee that is in the land of Assyria.}%%
  
  \pvab{\verse{7:19} They've come. They're all rested}{in the desolate valleys and in the holes of the rocks,}%%
  
  \pvba{in the thorn bushes and in all the pastures.}%%
  
  \pvab{\verse{7:20} In that day the Lord will shave}{with a razor~--- which is hired from beyond the river\footnotemark\ by the king of Assyria~---}%%
  \ie{the Euphrates}%%
  
  \pvbb{the head, the hair of the legs\footnotemark}{and also the beard will he take away in that day.}%%
  \alt{feet}%%
  
  \pvab{\verse{7:21} In that day}{a man will nourish a calf and two sheep;}%%
  
  \pvab{\verse{7:22} from the abundance of milk yielded}{he shall eat butter.}%%
  
  \pvbb{For butter and honey shall be eaten}{by everyone who's left in the midst of the land.}%%
  
  \pvaa{\verse{7:23} In that day,}%%
  
  \pvbb{every place in which there were}{a thousand vines at a thousand silver pieces}%%
  
  \pvba{shall become briers and thorns.}%% What?
  
  \pvab{\verse{7:24} They shall come thither with bows and arrows}{because all the land has become briers and thorns.}%%
  
  \pvbb{\verse{7:25} All the mountains which have been cultivated with a hoe,}{the fear of brier and thorn shall not come there.}%%
  
  \pvba{It shall be for the sending forth of oxen and the treading of sheep.}%%
\end{inparaenum}

  % \heading{8}{xxxx}

\begin{inparaenum}
  \verse{8:1} The \textsc{Lord} said to me, ``Take a huge tablet and write on it with a man's engraving tool: 'Maher-shalal-hash-baz.'\thinspace''\ie{Hurry to the spoils}%%
  \verse{8:2} I called reliable witnesses to witness for me: Uriah the priest and Zechariah son of Jeberechiah\ed{Uriah from 2~Kings, a priest who served under Ahaz; Jeberechiah is otherwise unknown} xxxx%%
  \verse{8:3} I came near to the prophetess\ed{a possible reason for referring to Isaiah's wife as a ``prophetess'' is that she bore the word of the Lord (i.e., the child that she bore was named by God)} and she conceived and gave birth to a son. The \textsc{Lord} said to me, ``Name him Maher-shalal-hash-baz.''%%
  \verse{8:4} Because before the boy knows to say, ``My father'' and ``My mother,'' the property\alt{riches, army, power} of Damascus and the spoil of Samaria shall be taken before the king of Assyria.%%
  
  \verse{8:5} %%
  
  \verse{8:6} %%
  \verse{8:7} %%
  \verse{8:8} %%
  
  \verse{8:9} %%
  \verse{8:10} %%
  
  \verse{8:11} %%
  \verse{8:12} %%
  \verse{8:13} %%
  \verse{8:14} %%
  \verse{8:15} %%
  
  \verse{8:16} %%
  \verse{8:17} %%
  \verse{8:18} %%
  
  \verse{8:19} %%
  \verse{8:20} %%
  \verse{8:21} %%
  \verse{8:22} %%
\end{inparaenum}

  % \heading{22}{xxxx}

\begin{inparaenum}
  \verse{22:1} %%
  \verse{22:2} %%
  \verse{22:3} %%
  \verse{22:4} %%
  \verse{22:5} %%
  \verse{22:6} %%
  \verse{22:7} %%
  \verse{22:8} %%
  
  \pvab{\vn{22:9} The breaches in the City of David~---}{you've seen them, they are many;}%%
  
  \pvba{You gather the waters of the lower pool;}%%
  
  \pvab{\vn{22:10} You've counted the houses of Jerusalem}{and torn down the houses to make the wall inaccessible;}%%
  
  \pvab{\vn{22:11} You've made a reservoir between the two walls}{for the old pool's water.}%%
  
  \pvbb{You didn't trust in its Maker,}{neither did you see its distant Creator.}%%
  
  \verse{22:12} %%
  \verse{22:13} %%
  \verse{22:14} %%
  \verse{22:15} %%
  \verse{22:16} %%
  \verse{22:17} %%
  \verse{22:18} %%
  \verse{22:19} %%
  \verse{22:20} %%
  \verse{22:21} %%
  \verse{22:22} %%
  \verse{22:23} %%
  \verse{22:24} %%
  \verse{22:25} %%
\end{inparaenum}

  % \heading{33}{xxxx}

\begin{inparaenum}
  \verse{33:1} %%
  
  \verse{33:2} %%
  \verse{33:3} %%
  \verse{33:4} %%
  \verse{33:5} %%
  \verse{33:6} %%
  
  \verse{33:7} %%
  \verse{33:8} %%
  \verse{33:9} %%
  \verse{33:10} %%
  \verse{33:11} %%
  \verse{33:12} %%
  
  \pvdb{\vn{33:13} You who are far away, listen to what I've done.}{You who are near, know my greatness.}%%
  
  \verse{33:14} %%
  \verse{33:15} %%
  \verse{33:16} %%
  \verse{33:17} %%
  \verse{33:18} %%
  \verse{33:19} %%
  \verse{33:20} %%
  \verse{33:21} %%
  \verse{33:22} %%
  \verse{33:23} %%
  \verse{33:24} %%
\end{inparaenum}

  % \heading{36}{Rabshakeh comes to besieged Jerusalem, blasphemes against the Lord, implores the people to not listen to King Hezekiah}

\begin{inparaenum}
  \verse{36:1} In the fourteenth year of King Hezekiah,\ie{his reign} king of Judah, the Assyrian king, Sennacherib, came up against all the fortifications of Judah and conquered them.%%
  \verse{36:2} The king of Assyria sent Rabshakeh\ed{See further in Appendix~\ref{app:rabshakeh}}\ed{This is either his literal name or it should be rendered ``(his) great commander.''} with a strong force from Lachish to Jerusalem unto King Hezekiah. He stood by the aqueduct of the upper pool on the highway of the fuller's field.%%
  \verse{36:3} He then came unto Eliakim, son of Hilkijah (who's over the household affairs), Shebna the scribe, and Joah the son of Asaph (the clerk).%%
  \verse{36:4} Rabshakeh said to them, ``Please say to Hezekiah, `Thus says the great king, the king of Assyria, ``What is this confidence in which you trust?%%
  \verse{36:5} You say, `It's just lip service,'\lit{a word for the lips} but there is strength for war. Now, who do you rely on that you have revolted against me?\ie{who do you trust that you feel comfortable in revolting against me?}%%
  \verse{36:6} You trust in the staff of a broken reed~--- on Egypt~--- whom if a man leans it goes into his hand and pierces it: such is Pharaoh, king of Egypt, to his allies.%%
  \verse{36:7} If you say to me, `We rely upon the \textsc{Lord} our God'~--- is it not he whose high places and altars Hezekiah has removed, saying to Judah and Jerusalem, `You shall worship before this altar'?''%%
  \verse{36:8} And now, I pray thee, make a bargain that my master, the king of Assyria, and I will give you up to two thousand horses if you're able to put riders on them.\ed{He knows they don't have that many riders and is taunting them with the Assyrian's superiority.}%%
  \verse{36:9} How then will you turn away the face of the least of my master's servants when you trust in Egypt for chariots and horsemen?%%
  \verse{36:10} And now, have I come up against this land without the \textsc{Lord} to destroy it? The \textsc{Lord} said to me, ``Go up against this land and destroy it.''\thinspace'\thinspace''%%
  \verse{36:11} Eliakim, Shebna, and Joah said to Rabshakeh, ``Please speak to your servants in Aramaic\ed{Aramaean, Syriac}\lit{Syrian language} because we can understand it. But don't speak to us in Hebrew\lit{the Jewish language} while in earshot of the people on the wall.''%%
  \verse{36:12} And Rabshakeh said, ``Did my master send me to speak these words to just you and your master and not to the men who sit on the wall so that they can eat their own feces and drink their own piss with you?''\ed{Remember that they are besieged.}%%
  \verse{36:13} Rabshakeh stood and cried in Hebrew with a loud voice\ed{So that all the people can hear and understand him, hence why they wanted him to speak Aramaic. This is a pretty diplomatically dick move.} and said, ``Hear the words of the great king, the king of Assyria!%%
  \verse{36:14} Thus says the king, `Don't let Hezekiah deceive you; he is not able to deliver you.%%
  \verse{36:15} And don't let Hezekiah make you trust in the \textsc{Lord}, saying, ``The \textsc{Lord} shall certainly deliver us. He will not let this city be given into the hand of the king of Assyria.''%%
  \verse{36:16} Don't listen to Hezekiah.' Thus says the king of Assyria, `Make a deal with me: come out to me and everyone shall eat of his own vine and his own fig tree and drink from their own cisterns%%
  \verse{36:17} until I come and take you away to a land like your own land~--- a land of corn and wine, a land of bread and vineyards.%%
  \verse{36:18} Don't let Hezekiah persuade you, saying, ``The \textsc{Lord} will deliver us.'' Have any of the Gentile gods delivered their land from the hand of the king of Assyria?%%
  \verse{36:19} Where are the gods of Hamath and Arpad? Where are the gods of Sepharvaim? Have they delivered Samaria from out of my hand?%%
  \verse{36:20} Who are they among the gods of the land that have delivered their own land from out of my hand that the \textsc{Lord} should deliver Jerusalem from out of my hand?''%%
  \verse{36:21} They kept silent and did not answer him a word for the king's commandment was, saying, ``Don't answer him.''%%
  \verse{36:22} Eliakim (the son of Hilkijah) who is in charge of the house, Shebna the scribe, and Joah the son of Asaph the clerk came in to Hezekiah with rent clothes and reported to him the words of Rabshakeh.%%
\end{inparaenum}

  %
  % \book{Jeremiah}{\Hebrew{ירמיה}}
  % \heading{1}{The Lord foreordains Jeremiah to be a prophet~--- He comforts him and blesses him to fulfill his calling~--- xxxx}

\begin{inparaenum}
  \verse{1:1} The words of Jeremiah, son of Hilkiah, of the priests who were in Anathoth in the land of Benjamin;%%
  \verse{1:2} to whom the word of the \lord\ came in the days of Josiah, son of Amon, king of Judah, in the thirteenth year of his reign.%%
  \verse{1:3} xxxx\smallskip%%
  
  \pa \verse{1:4} The word of the \lord\ came to me, saying,%%
  
  \pa \verse{1:5} ``Before I formed%
  \ed{The verb used here, \Hebrew{יָצַר}, is defined in \textsc{halot} as generally meaning ``\textbf{form}, \textbf{shape}.'' However, when the subject is God, it means ``\textbf{create}, \textbf{form}'' but is noted as being an ``older, concrete synonym of \textit{b\=ar\=a'}''; since Jeremiah comes so late in the Hebrew period, it is doubtful, from a language standpoint, that the Lord is talking about creating. Additionally, from a theological and biological standpoint, it would be inconsistent to say that the Lord created Jeremiah in his mother's womb. This could only work poetically, not literally.}
  you in the womb, I knew you.\pa Before you came out of the womb, I sanctified you.%% xxxx: define the three types of ``knowing''
  
  \pa I made you a prophet to the nations.''%
  \ca{\septuagint\superit{C}\super{a1} sg}{the Septuagint (textus Graecus in genere Catenarum traditus a1) has this in singular [i.e., ``to the nation'']}
  \smallskip%%
  
  \noindent\verse{1:6} I said, ``Ah, \lord\ God! I can't speak because I'm a child.''%%
  
  \verse{1:7} The \lord\ said to me,\smallskip%%
  
  \pb ``Don't say, `I'm a child,'%
  \ca{pc Mss \septuagint\peshitta\ + \Hebrew{כִּי}}{a few Hebrew manuscript codices and the Septuagint and Peshitta add ``because'' [e.g., ``Don't say [it's] because I'm a child'']}%%
  
  \pb because you will go to everyone I will send you to,\pa and because\ed{repeated} you'll say everything I command you.%%
  
  \pa \verse{1:8} Don't be afraid of them\lit{of their faces}\pa because I am with you to deliver you,''%%
  
  \pa declares the \lord.\pa \verse{1:9} The \lord\ reached out%
  \lit{stretched out His hand}
  \ca{\septuagint\ + \Greek{πρός με}}{the Septuagint adds ``to me'' [i.e., The \lord\ stretched out His hand to me'']}
  and touched my mouth. And the \lord\ said to me,%%
  
  \pb ``I've put\alt{given} my words in your mouth.\pa \verse{1:10} Look, I've entrusted you this day%%
  
  \pa over the nations and kingdoms to tear up\alt{uproot, pull up, drive out} and tear down,\alt{break up, demolish, destroy} to destroy and to demolish, to build and to plant.''\smallskip%%
  
  \verse{1:11} %%
  \verse{1:12} %%
  
  \verse{1:13} %%
  \verse{1:14} xxxx\smallskip%%
  
  \pb xxxx\pa xxxx%%
  
  \pa \verse{1:15} xxxx\pa xxxx\pa xxxx%%
  
  \pb xxxx\pa xxxx%%
  
  \pb xxxx\pa xxxx%%
  
  \pa \verse{1:16} xxxx\pa xxxx%%
  
  \pb xxxx\pa xxxx%%
  
  \pb \verse{1:17} xxxx\pa xxxx%%
  
  \pa xxxx\pa xxxx%%
  
  \pa \verse{1:18} xxxx\pa xxxx%%
  
  \pb xxxx\pa xxxx%%
  
  \pb xxxx\pa xxxx%%
  
  \pa \verse{1:19} xxxx\pa xxxx%%
\end{inparaenum}

  % \heading{3}{xxxx}

\begin{inparaenum}
  \verse{3:1} %%
  \verse{3:2} %%
  \verse{3:3} %%
  \verse{3:4} %%
  \verse{3:5} %%
  
  \verse{3:6} %%
  \verse{3:7} %%
  \verse{3:8} %%
  \verse{3:9} %%
  \verse{3:10} %%
  
  \verse{3:11} %%
  \verse{3:12} %%
  \verse{3:13} %%
  \verse{3:14} %%
  \verse{3:15} %%
  \verse{3:16} And it shall be that when you have multiplied and become fruitful in the land: in those days (this is the declaration\alt{utterance, revelation} of the \textsc{Lord}) they shall no more say, ``The Ark of the Covenant of the \textsc{Lord},'' neither shall it come to mind, nor shall they remember it, nor give heed unto it~--- it shall not be done anymore.%%
  \verse{3:17} %%
  
  \verse{3:18} %%
  \verse{3:19} %%
  \verse{3:20} %%
  \verse{3:21} %%
  \verse{3:22} %%
  \verse{3:23} %%
  \verse{3:24} %%
  \verse{3:25} %%
\end{inparaenum}

  % \heading{31}{xxxx}

\begin{inparaenum}
  \verse{31:1} %%
  
  \verse{31:2} %%
  \verse{31:3} %%
  \verse{31:4} %%
  \verse{31:5} %%
  \verse{31:6} %%
  
  \verse{31:7} %%
  \verse{31:8} %%
  \verse{31:9} %%
  
  \verse{31:10} %%
  \verse{31:11} %%
  \verse{31:12} %%
  \verse{31:13} %%
  \verse{31:14} %%
  
  \verse{31:15} %%
  
  \verse{31:16} %%
  \verse{31:17} %%
  \verse{31:18} %%
  \verse{31:19} %%
  \verse{31:20} %%
  \verse{31:21} %%
  \verse{31:22} %%
  
  \verse{31:23} %%
  \verse{31:24} %%
  \verse{31:25} %%
  \verse{31:26} %%
  
  \verse{31:27} %%
  \verse{31:28} %%
  \verse{31:29} %%
  \verse{31:30} %%
  
  \verse{31:31} Listen, the day is coming,'' says the \lord, ``in which I shall make a new covenant with the houses of Israel and Judah.%%
  \verse{31:32} Not like the covenant I made with your fathers in the day I seized hold on this land in order to lead them out of the land of Egypt where\lit{there} they broke my covenant\ed{xxxx: check wording here} even though I ruled over them,'' says the \lord.%%
  \verse{31:33} ``Because of this, I will make a covenant with the people of Israel,'' says the \lord. ``I will given my law to them:\lit{I will place my law inside them} I will write it on their hearts: I shall be their God and they shall be my people.%%
  \verse{31:34} They no longer teach their neighbors and brothers, saying, `Become acquainted with the \lord.' They\lit{Because they} shall all know me, from the least to the greatest,'' says the \lord. ``I will pardon their iniquities and no longer mention their sins.''%%
  
  \verse{31:35} %%
  \verse{31:36} %%
  
  \verse{31:37} %%
  
  \verse{31:38} %%
  \verse{31:39} %%
  \verse{31:40} %%
\end{inparaenum}

  % \heading{32}{xxxx}

\begin{inparaenum}
  \verse{32:1} %%
  \verse{32:2} %%
  \verse{32:3} %%
  \verse{32:4} %%
  \verse{32:5} %%
  
  \verse{32:6} %%
  \verse{32:7} %%
  \verse{32:8} %%
  \verse{32:9} %%
  \verse{32:10} %%
  \verse{32:11} %%
  \verse{32:12} %%
  \verse{32:13} %%
  \verse{32:14} %%
  \verse{32:15} %%
  
  \verse{32:16} %%
  \verse{32:17} %%
  \verse{32:18} %%
  \verse{32:19} %%
  \verse{32:20} %%
  \verse{32:21} %%
  \verse{32:22} %%
  \verse{32:23} %%
  \verse{32:24} %%
  \verse{32:25} %%
  \verse{32:26} %%
  \verse{32:27} %%
  \verse{32:28} %%
  \verse{32:29} %%
  \verse{32:30} %%
  \verse{32:31} %%
  \verse{32:32} %%
  \verse{32:33} %%
  \verse{32:34} %%
  \verse{32:35} %%
  
  \verse{32:36} Now, therefore, thus says the \textsc{Lord}, the God of Israel, concerning this city of which you all say, ``It's been given to the king of Babylon by force, famine, and pestilence'':%%
  \verse{32:37} ``Listen to me! I am gathering them out of all the lands where I have driven them in My anger, fury, and great wrath. I will bring them back home and let them live here.%%
  \verse{32:38} They have been\alt{proven themselves to be} my people so I will be their God.%%
  \verse{32:39} I have given them one heart and one way: to respect me all their lives~--- for their good and for the good of\understood\ their children.%%
  \verse{32:40} I will make an everlasting covenant with them that I won't withhold from doing good to them. I will instill my fear in their hearts so that they won't turn their backs on me.%%
  \verse{32:41} I rejoice over them when they do good. With all My heart and soul, I have planted them in this land of truth.''%%
  
  \verse{32:42} %%
  \verse{32:43} %%
  \verse{32:44} %%
\end{inparaenum}

  % \heading{33}{xxxx}

\begin{inparaenum}
  \verse{33:1} %%
  \verse{33:2} %%
  \verse{33:3} %%
  
  \verse{33:4} Therefore, thus says the \textsc{Lord}, the God of Israel, regarding the houses of this city of the kings of Judah which are broken down because of the siege ramps\halot{xxxx}{xxxx} and the sword:%%
  \verse{33:5} ``They are coming to way with the Chaldeans to fill them with the dead bodies I've slain in Mine anger and fury. Because of all this wickedness, I have hidden my face from this city.%%
  \verse{33:6} Look, I will xxxx xxxx and xxxx. I will heal them and reveal to them an abundance of peace and truth.%%
  \verse{33:7} I will free the captives of Judah and Israel. I will build them up as they were at first.%%
  \verse{33:8} I will cleanse them from all their iniquities which they have sinned against me. I will also pardon all their iniquities which they have sinned and transgressed against me.%%
  \verse{33:9} It shall be to me a joyous name of praise and beauty to all the nations of the earth who shall hear\ed{Does this apply to only the nations who hear of this goodness, or does it apply to all people because everyone will hear of this goodness?} of the good I do to them. However, they shall also fear and tremble because of all the good and peace that I do to you.''%%
  
  \verse{33:10} Thus says the \textsc{Lord}, ``It is again heard in this place of which you say, `It is waste: devoid of man and beast;' in the cities of Judah and in the desolated streets of Jerusalem which are devoid of man, inhabitant, and beast:%%
  \verse{33:11} An exultant voice, even a joyous voice, the voice of the bride and groom, saying, `Thanks to the \textsc{Lord} of Hosts, for the \textsc{Lord} is good. For His everlasting kindness to those who bring thanks to the house of the \textsc{Lord}. For I will turn back the land's captivity and make it as it used to be as at first,'' says the \textsc{Lord}.%%
  
  \verse{33:12} %%
  \verse{33:13} %%
  
  \verse{33:14} ``Listen. The days are coming,'' confirms the \textsc{Lord}, ``when I shall perform xxxx the good deed that I said to Israel and Judah's family.%%
  \verse{33:15} In those days, at that time, I will cause a righteous branch to grow up from David who shall produce judgment and righteousness in the world.%%
  \verse{33:16} In those days, Judah shall be saved and shall dwell in safety. It shall be called \textit{The \textsc{Lord} is our righteousness}.''%%
  
  \verse{33:17} For thus says the \textsc{Lord}: ``It shall never be taken away from David that someone sit upon the throne of the family of Israel.%%
  \verse{33:18} Neither shall it be taken from the priests and Levites to offer up burnt-offerings, to make gifts, nor to do daily sacrifices.''%%
  
  \verse{33:19} The \textsc{Lord}'s words came to Jeremiah, saying, ``xxxx''%%
  \verse{33:20} ``Thus says the \textsc{Lord}: `If you break my covenant during the day or during the nighttime so that xxxx\halot{xxxx}{xxxx \Hebrew{לְבִלְתִּי}} there's not day and night when there's supposed to be.\lit{in this season.}%%
  \verse{33:21} Additionally, my covenant will be broken with my servant David such that one of his sons shall not sit on his throne. xxxx%%
  \verse{33:22} I will multiply the progeny of my servant David and my Ministers the Levites as unnumbered as the hosts of Heaven and as unmeasurable as the sand of the sea.'\thinspace''%%
  
  \verse{33:23} The word of the \textsc{Lord} to Jeremiah, saying,%%
  \verse{33:24} ``Have you not considered what this people have spoken? They say, `The \textsc{Lord} has rejected the two families that He has chosen.' They despise my people and shall no longer be considered a nation before them.''%%
  
  \verse{33:25} Thus says the \textsc{Lord}: ``If my morning and evening covenant\ie{a covenant that is effectual both day and night~--- an eternal covenant.} is not, then the statues of Heaven and Earth shall not stand.%%
  \verse{33:26} Furthermore, I will reject\alt{cast away, refuse} the descendants of Jacob and my servant David. I will not\ed{xxxx is this supposed to be negated?} take his children to be rulers over the children of Abraham, Isaac, and Jacob because I will return them from captivity and have mercy on them.''%%
\end{inparaenum}

  %
  % \book{Ezekiel}{\Hebrew{יחזקאל}}
  % \heading{16}{xxxx}

\begin{inparaenum}
  \verse{16:1} %%
  \verse{16:2} %%
  \verse{16:3} %%
  \verse{16:4} %%
  \verse{16:5} %%
  \verse{16:6} %%
  \verse{16:7} %% Begin quote needs to be satisfied before verse 8
  \verse{16:8} I passed by you and looked at you, and your time was a time\alt{season} of love. I spread my skirt\halot{xxxx}{\textbf{skirt} of garment} over you to cover your nakedness.\cf{\vref{Ruth}{3}{2--}} I swore an oath and entered into a covenant with you,'' declared the Lord\ca{\missing\ \septuagint*, add}{missing in the Septuagint (textus Graecus originalis), added} \textsc{God}. ``You have become mine.\ie{have become my wife.}%%
  \verse{16:9} I washed you in water. I washed your blood from off you. I anointed you with oil.\ed{ritualistic and ethical cleansing}%%
  \verse{16:10} I clothed you with embroidery. I put leather\halot{xxxx}{suggested \textbf{porpoise}, \textit{Tursiops tursis Fabr.}, or kind of \textbf{dolphin}, thus \textit{`\^or ta\d{h}a\v s} \haref{Nu}{4}{6} etc.\ would be dolphin-skin, so plural \haref{Ex}{25}{5}, and \textit{`\^ or ta\d{h}a\v s} \missing\ \textit{ta\d{h}a\v s} for such skin \haref{Nu}{4}{25}; in any event, a leather (perhaps imported) of fine quality.} sandals on you, girded you with fine\alt{white} linens, and covered you with silk\alt{costly, precious}\ed{sacred material in antiquity} garments.%%
  \verse{16:11} I adorned you with jewelery. I put bracelets on your wrists\lit{arms} and a necklace on your neck,%%
  \verse{16:12} a ring in your nose, earrings on your ears, and a beautiful crown on your head.%%
  \verse{16:13} You were adorned in gold and silver. Your clothing was linen and silk and embroidery. You ate fine flour, honey, and oil. You became exceedingly beautiful. You prospered into a kingdom.%%
  \verse{16:14} Your fame spread among the nation on account of your beauty because my giving you\understood\ splendor was perfect,'' a declaration of the Lord \textsc{God}.%%
  \verse{16:15} ``Yet you trusted in your beauty. You went a-whoring\alt{became a prostitute} because of your renown. You fornicated\ed{not a passive verb} with all who passed by. So it was to him.\ie{Your beauty became his.}%%
  \verse{16:16} %% double quote
  \verse{16:17} %%
  \verse{16:18} %%
  \verse{16:19} %%
  \verse{16:20} %%
  \verse{16:21} %%
  \verse{16:22} %%
  \verse{16:23} %%
  \verse{16:24} %%
  \verse{16:25} %%
  \verse{16:26} %%
  \verse{16:27} %%
  \verse{16:28} %%
  \verse{16:29} %%
  \verse{16:30} %%
  \verse{16:31} %%
  \verse{16:32} %%
  \verse{16:33} %%
  \verse{16:34} %%
  \verse{16:35} %%
  
  \verse{16:36} %%
  \verse{16:37} %%
  \verse{16:38} %%
  \verse{16:39} %%
  \verse{16:40} %%
  \verse{16:41} %%
  \verse{16:42} %%
  \verse{16:43} %%
  \verse{16:44} %%
  \verse{16:45} %%
  \verse{16:46} %%
  \verse{16:47} %%
  \verse{16:48} %%
  \verse{16:49} %%
  \verse{16:50} %%
  
  \verse{16:51} %%
  \verse{16:52} %%
  \verse{16:53} %%
  \verse{16:54} %%
  \verse{16:55} %%
  \verse{16:56} %%
  \verse{16:57} %%
  \verse{16:58} %%
  
  \verse{16:59} %%
  \verse{16:60} %%
  \verse{16:61} %%
  \verse{16:62} %%
  \verse{16:63} %%
\end{inparaenum}

  % \heading{37}{xxxx}

\begin{inparaenum}
  \verse{37:1} %%
  \verse{37:2} %%
  \verse{37:3} %%
  \verse{37:4} %%
  \verse{37:5} %%
  \verse{37:6} %%
  \verse{37:7} %%
  \verse{37:8} %%
  \verse{37:9} %%
  \verse{37:10} %%
  \verse{37:11} %%
  \verse{37:12} %%
  \verse{37:13} %%
  \verse{37:14} %%
  \verse{37:15} The word of the \lord\ came to me, saying,%%
  \verse{37:16} ``Son of man, take one stick and write \textit{For Judah, the children of Israel, and his companions} on it. Then take another stick and write \textit{For Joseph, the branch of Ephraim, and his companions, the whole house of Israel} on it.%%
  \verse{37:17} Bring them together as one stick because they shall be one in your hand.%%
  \verse{37:18} %% end double quote
  \verse{37:19} %%
  \verse{37:20} %%
  \verse{37:21} %%
  \verse{37:22} %%
  \verse{37:23} %%
  \verse{37:24} %%
  \verse{37:25} %%
  \verse{37:26} %%
  \verse{37:27} %%
  \verse{37:28} %%
\end{inparaenum}

  % \heading{38}{xxxx}

\begin{inparaenum}
  \verse{38:1} %%
  \verse{38:2} %%
  \verse{38:3} %%
  \verse{38:4} %%
  \verse{38:5} %%
  \verse{38:6} %%
  \verse{38:7} %%
  \verse{38:8} %%
  \verse{38:9} %%
  
  \verse{38:10} %%
  \verse{38:11} %%
  \verse{38:12} %%
  \verse{38:13} %%
  
  \verse{38:14} %%
  \verse{38:15} %%
  \verse{38:16} %%
  
  \verse{38:17} Thus says my master, the \textsc{Lord}: ``Aren't you he of whom I spake in past times through the hands of my servants, the prophets of Israel, who prophesied for years and bade them in those days that I would bring you against them?%%
  
  \verse{38:18} It shall be that in that day, when Gog comes against the land of Israel,'' the Lord \textsc{God} confirms, ``that My face will show how angry I am.%%
  \verse{38:19} In my zeal and my fiery wrath I have spoken: `There will be a great commotion throughout the land of Israel in that day.%%
  \verse{38:20} All the fish in the ocean, the birds in the sky, the beasts in the field, everything creeping along, all the men on the earth shall shake\alt{quake} at my presence. The mountains shall be thrown down, the cliffs shall fall, every wall shall fall to the earth.'%%
  \verse{38:21} I will proclaim through all the mountains\lit{my mountains. Since everything is the Lord's, \textit{the} mountains works just fine.} that a sword shall be brought against him,'' confirms the Lord \textsc{God}. ``Every man's sword shall be brought against his brother.%%
  \verse{38:22} I will judge\alt{plead} him through pestilence\alt{\textit{with him} or \textit{for him} through/with pestilence} and bloodshed, an abundance of rain and hailstorms, fire and brimstone, will I rain on him and the people who are with him.%%
  \verse{38:23} I have boasted, sanctified, and made Myself known in the eyes of many nations: thall shall know that I am the \textsc{Lord}.''%%
\end{inparaenum}

  %
  % \book{Daniel}{\Hebrew{דניאל}}
  % \heading{2}{Nebuchadnezzar has a dream and calls in the magicians and Chaldeans to interpret it~--- he refuses to tell them what the dream was because they will lie to him about its interpretation~--- Daniel is called in and interprets the king's dream~--- Nebuchadnezzar praises the God of Heaven~--- Daniel is promoted and made great in the land}

\begin{inparaenum}
  \verse{2:1} In the second year of Nebuchadnezzar's reign, Nebuchadnezzar dreamed\ed{``sexual, then general'' (\textsc{halot}). The context here is not clear if this was a sexual or general dream, although the chapter later tells us that the dream is not sexual in nature.} multiple\understood\ dreams and his mind\alt{temper, disposition, spirit} was disturbed and he couldn't sleep.\lit{his sleep was gone from him.}%%
  \verse{2:2} The king ordered the soothsayer-priests, the conjurers, the sorcerers, and the Chaldeans to be called in\understood\ to tell\alt{to give an opinion; this is unlikely as the king would most likely have had them killed for merely opining.} the king what his dreams meant.\lit{tell the king his dreams.}\alt{expound the king's dreams to the king.} So they came in and stood before the king.%%
  \verse{2:3} The king said to them, ``I have dreamed a dream and my mind is troubled to understand\alt{find out information about, perceive, know, come to understand} the dream.''%%
  \verse{2:4} The Chaldeans spoke to the king in Aramaic:\ed{The \textsc{bhs} here contains a horizontal break, replicated here. However, the critical apparatus (xxxx) states ``prb (probably) add (added/addition), prp (it has been proposed) \Hebrew{וַיֹּאמְרוּ}.''} ``O King! Live forever! Tell your servants the dream and we'll make its\understood\ interpretation known to you.''%%
  \verse{2:5} The King answered and said to the Chaldeans, ``The spoken word is promulgated by me. If you don't let me know\alt{communicate it to me} the dream and its interpretation, you will be dismembered\lit{made into limbs} and your houses will be pulled down as punishment.\halot{xxxx}{either (houses) shall be turned into public privy [restrooms], or pulled down as punishment; lit., garbage-heap, heap of ruins and debris}%%
  \verse{2:6} If you let me know the dream and its interpretation, you'll receive gifts and presents\alt{gifts} and great honor\alt{majesty} from before me. So\lit{Therefore. ``So'' is more casual and I'm picturing Nebuchadnezzar, harsh at first, becoming more casual in an attempt to butter up the Chaldeans and get them to do his bidding. However, it's perfectly likely that he remains majestic and non-casual.} let me know the dream and its interpretation.''%%
  \verse{2:7} They answered the second time and said, ``If\understood\ the king tells\lit{says to} us the dream,\lit{Let the king tell the dream to his servants} we'll let you know its interpretation.''%%
  \verse{2:8} The king answered and said, ``\emph{I}\ed{Emphasized because \Hebrew{אֲנָה} is also present.} certainly\alt{surely, yes} know that the time \emph{you}'re buying~--- because \lit{that}you see that the word is \ed{Understood: already}promulgated by me~---%%
  \verse{2:9} that if you don't let me know what the dream means\understood, there is only one royal decree\alt{judgment} on you because you've come to a decision to speak a lying and mischievous\halot{xxxx}{unsure if this is the actual root.} thing before me~--- until\alt{when, as soon as} the time is changed: so\lit{therefore} tell me the dream! Then I will know that you can make the interpretation known to me.''%%
  \verse{2:10} The Chaldeans answered before the king and said, ``There isn't a man in the world who is able to make the king's matter known. Therefore, there isn't a king or chief, mighty and powerful, who has required\alt{asked} such a thing from any magician, conjurer, or Chaldean.%%
  \verse{2:11} The thing that the king asks for\understood\ is difficult. There isn't anyone else\lit{another} who can make it known before the king except for the gods, but they don't dwell in the flesh.''\lit{whose dwelling is not in the flesh.}%%
  \verse{2:12} Because of this, the king became angry and super\lit{very, much, many} furious and ordered that\understood\ all the wise men of Babylon be slain.%%
  \verse{2:13} The order\lit{official decree} went out that the wise men were to be killed, so they requested Daniel and his companions to kill them.%%
  
  \verse{2:14} Then Daniel answered Arioch, the chief of the king's executioners,\alt{bodyguard. Executioner seems more in line with the context of people seeking Daniel's life.}~--- who had gone out to kill the wise men of Babylon~--- with counsel and good sense.\alt{understanding}%%
  \verse{2:15} He answered and said to the king's officer Arioch, ``Why is the decree from before the king so harsh?''\alt{severe} Arioch then made the reason why\lit{the thing, the matter} known to Daniel.%%
  \verse{2:16} So Daniel went in and requested that he be given a postponement from the king, that\understood\ he might show the interpretation to the king.%%
  
  \verse{2:17} Then Daniel went to his house and made this\understood\ thing known to his companions Hananiah, Mishael, Azariah.%%
  \verse{2:18} They requested compassion from before the God of the Heavens concerning the secret~--- that Daniel and his companions not be killed along with the wise men of Babylon.%%
  \verse{2:19} The secret was then revealed\alt{disclosed} to Daniel in a nighttime vision, and Daniel then blessed the God of the Heavens.%%
  \verse{2:20} Daniel answered and said,%%
  
  \pvbb{Let the name of God}{be blessed for ever and ever}%%
  
  \pvba{Because wisdom and might are His!}%%
  
  \pvad{\vn{2:21} He alters}{times and seasons.}{He removes kings}{and appoints kings.}%%
  
  \pvbb{He gives wisdom to the wise}{and knowledge to the insightful.\footnotemark}%%
  \fntlit{knowledge/understanding to the knowledgeable [with] insight.}%%
  
  \pvab{\vn{2:22} He reveals deep\footnotemark\ and hidden things.}{He knows what's in the darkness,}%%
  \fnthalot{xxxx}{xxxx i.e., impenetrable}%%
  
  \pvba{and the light dwells with him.}%%
  
  \pvaa{\vn{2:23} You, God of my fathers, I praise\footnotemark\ and praise!}%%
  \fnted{\textsc{ylt} and \textsc{darby} give ``thanking''; \textsc{halot} only gives ``praising.''}%%
  
  \pvba{You have given me wisdom and might.}%%
  
  \pvbb{And now, You've caused me to know what we've requested from You~---}{You've made the king's matter known to us.}%%
  
  \noindent\verse{2:24} Therefore,\lit{Because of this} Daniel went up to Arioch~--- whom the king had appointed to kill the wise men of Babylon~--- he went and said the following\lit{thus} to him: ``The wise men of Babylon: don't kill them. Bring me before the king and I'll show the king the interpretation.''%%
  
  \verse{2:25} Then Arioch brought Daniel in before the king in a hurry, and he\ie{Arioch} said the following to him: ``I have found a man\halot{xxxx}{adult male} among the sons of the exiles of Judah who can make the interpretation known to the king.''%%
  \verse{2:26} The king answered and said to Daniel, whose name is Belteshazzar, ``Are you able to make the dream that I've seen and its interpretation known to me?''%%
  \verse{2:27} Daniel answered before the king and said, ``The secret that the king asks for is unable to be shown to the king by the wise men, conjurers, magicians, astrologers.\alt{haruspices: people trained in a kind of divination (haruspicy) that inspects the livers and other entrails of sacrificed animals.}%%
  \verse{2:28} However,\lit{but, yet} there is a God in the Heavens who can reveal secrets and make known to King Nebuchadnezzar what is to happen in the last\lit{end of} days. Your dream and the visions of your head on your bed are these:%%
  
  \verse{2:29} You, O King, your thoughts on your bed regarding\understood\ what should happen\lit{come up} and what should\understood\ be after this: He who reveals secrets has made known to you what is to be.%%
  \verse{2:30} This secret has been revealed to \emph{me}, not because\understood\ of any wisdom that I have\lit{that is in me} more than any living thing, but in order that the interpretation be made known to the king: so that you can understand the thoughts of your heart.%%
  \verse{2:31} O King, you looked and, whoa~--- a great statue! That statue was great and had extraordinary complexion. It stood before you and its appearance was frightening.%%
  \verse{2:32} This statue's head was good gold; its chest\lit{breasts} and arms were silver; its stomach\lit{belly} and upper thigh were copper;\alt{bronze}%%
  \verse{2:33} its legs, iron; its feet part iron, part clay.%%
  \verse{2:34} You saw until\alt{up to (probably including)}\ed{It's possible that the king only saw until this and that everything preceding this was shown to Daniel in order to prove to the king that Daniel was sent from God; then, Daniel saw and related the rest of the dream to the king.} a stone was cut out\alt{quarried} without hands and it struck the statue's iron and clay feet and broke it into pieces.%%
  \verse{2:35} The iron, clay, brass, silver, and gold were then broken into small pieces and became as the chaff of the summer threshing floor and were carried away in the wind and no trace could be found of them. However, the stone that struck the image because a great mountain and filled the whole earth.%%
  \verse{2:36} This was the dream. We will now tell its interpretation before the king.%%
  \verse{2:37} You, O king, king of kings to whom the God of Heaven has given the kingdom, might, power, and glory.%%
  \verse{2:38} Wherever the sons of men dwell, He has given the beasts of the field and the birds in the air into your hand, and caused you to rule over all of them: you are the head of gold.%%
  \verse{2:39} Another, lower kingdom shall arise after you. And yet another (third) kingdom of bronze shall rule over all the earth.%%
  \verse{2:40} Then there shall be a fourth kingdom~--- strong as iron~--- which shall arise. It shall break to pieces and shatter all things because it's iron. And like iron, which crushes all these, it shall break and crush.%%
  \verse{2:41} The part made of potter's clay with part iron feet and toes which you saw, it shall be a divided kingdom. But some of the firmness of iron shall be in it~--- as you saw~--- iron mixed with miry clay.%%
  \verse{2:42} The partially iron and partially clay toes of the feet mean that the kingdom shall be part strong and part brittle.%%
  \verse{2:43} The iron mixed with miry clay which you saw, they shall mix offspring, but they will not hold together: just as iron does not mix with clay.%%
  \verse{2:44} In those kings' days, the God of Heaven will set up\alt{establish} a kingdom and a sovereignty which shall never be destroyed nor left to another people. It shall break all of those kingdoms in pieces and bring them\understood\ to an end. It shall stand\alt{live} forever.%%
  \verse{2:45} Just as you saw a stone that was cut out of\alt{from} the mountain, but not by hands,\alt{without hands} and it broke the iron, bronze, clay, silver, and gold in pieces, a great God has made known to the king what shall be hereafter: the dream is certain and its interpretation is sure.''\alt{accurate.''}%%
  
  \verse{2:46} Then King Nebuchadnezzar fell on his face and paid homage to Daniel and commanded that an offering of incense be offered to him.%%
  \verse{2:47} The king answered and said to Daniel, ``Truly your God is a God of gods, a lord of kings, and a revealer of mysteries, for you have been able to reveal this mystery!''%%
  \verse{2:48} The king then made Daniel great, gave him many great gifts, and made him a ruler over the whole province of Babylon and chief prefect over all the wise men of Babylon.%%
  \verse{2:49} Daniel requested the king that he appoint Shadrach, Meshach, and Abednego over the affairs of the province. And Daniel was in the court of the king.%%
\end{inparaenum}

  % \heading{3}{King Nebuchadnezzar sets up a golden statue and commands everyone to pay homage to it whenever the stringed instruments play~--- Shadrach, Meshach, and Abednego do not pay homage to the statue and are tattled on by some Chaldean men~--- King Nebuchadnezzar commands them to pay homage to the golden statue, lest they be thrown into the fiery, burning furnace~--- they boldly refuse while testifying of God, are thrown in, and are miraculously saved~--- xxxx}

\begin{inparaenum}
  \verse{3:1} King Nebuchadnezzar made a golden statue sixty~cubits\ed{90~ft (27.43~m)} tall and six~cubits\ed{9~ft (2.74~m)} wide, and he stood it up in the Dura plain in the province of Babylon.%%
  \verse{3:2} King Nebuchadnezzar sent to assemble the Apharsachites,\ed{title of Babylonian officials} the prefects,\alt{governors} the governors,\halot{xxxx}{of Babylonian and Persian empires} the counselors, the treasurers, the judges, the police officers,\alt{magistrates} and all the provincial administrators to come to the dedication of the statue that King Nebuchadnezzar had set up.%%
  \verse{3:3} Then, the Apharsachites, the prefects, the governors, the counselors, the treasurers, the judges, the police officers, and all the provincial administrators gathered\alt{assembled} to the dedication of the image that King Nebuchadnezzar had set up, and they stood before the statue that Nebuchadnezzar had set up.%%
  \verse{3:4} The herald cried out with force: ``They say to you: `O people, nations, and languages!%%
  \verse{3:5} As soon as\alt{When}\lit{At the time} you hear the sound of the horn and the pipe and the kithara\halot{xxxx}{i.e., a kind of lyre or lute} and the lyre\halot{xxxx}{a type of lyre, evidently triangular, with four strings and a bright tone.} and the harp\halot{xxxx}{a three-cornered stringed instrument with sounding board.} and the harmony\ed{``Last in list of musical instruments, most say \textbf{bagpipe}, but others say `concord, harmony'\thinspace'' (\textsc{halot}). Since the next item in the list is ``all sorts of stringed music,'' it seems more likely that this be the concord or harmony since they're stringed.} and all sorts of stringed music, you shall fall down and pay homage to the golden statue that King Nebuchadnezzar had set up.%%
  \verse{3:6} Whoever doesn't fall down and pay homage in that moment shall be thrown into\lit{into the midst of} the fiery, burning furnace.'\thinspace''%%
  
  \verse{3:7} Then, at that time, when all the people heard the sound of the horn and the pipe and kithara and the lyre and the harp and all sorts of stringed music, all the people, nations, and languages fell down and paid homage to the golden statue that King Nebuchadnezzar had set up.%%
  
  \verse{3:8} Then, at that time, Chaldean men drew near and slandered\lit{eat pieces of} the Jews.%%
  \verse{3:9} They answered and said to King Nebuchadnezzar, ``O King, live forever!%%
  \verse{3:10} You, O King, have given an order that everyone who hears the sound of the horn and the pipe and kithara and the lyre and the harp and the harmony and all sorts of stringed music shall fall down and pay homage to the golden image~---%%
  \verse{3:11} that whoever doesn't fall down and pay homage shall be thrown into the fiery, burning furnace.%%
  \verse{3:12} There are Jewish men whom you've appointed\ed{The Aramaic includes \Hebrew{יָתְהוֺן} which marks this verb as an accusative.} over the work of the province of Babylon: Shadrach, Meshach, and Abednego. These men do not care about you, O King, or take you into consideration. They do not serve your gods. And the golden statue you've set up, they do not pay homage to it.''\understood%%
  
  \verse{3:13} Nebuchadnezzar, in a furious rage, commanded that Shadrach, Meshach, and Abednego should be brought in, and those men were brought in before the king.%%
  \verse{3:14} And answering, Nebuchadnezzar said to them, ``Shadrach, Meshach, and Abednego: is it true that you do not serve my gods nor worship the golden image which I have set up?%%
  \verse{3:15} Now, if you're ready, when you hear the sound of the horn, pipe, lyre, trigon, harp, bagpipe, and every sound of music, fall down and worship the image which I have set up. However, if you do not immediately worship, you shall be cast in the the midst of the burning, fiery furnace. And who is `God' that he shall deliver you out of my hands?''%%
  \verse{3:16} Shadrach, Meshach, and Abednego answered and said to King Nebuchadnezzar, ``We have no need to answer you in this matter.%%
  \verse{3:17} But\understood if it happens that way, our God~--- whom we serve~--- is able to save\alt{rescue} us from the fiery, burning furnace: He will rescue us from out of your hand, O King.%%
  \verse{3:18} But even\understood\ if not, let it be known to you, O King, we will not serve your gods. We will not pay homage to the golden statue that you've set up.''%%
  
  \verse{3:19} Then, Nebuchadnezzar was filled with fury\alt{rage, but I like the alliteration.} and his expression\halot{xxxx}{his features}\lit{the features of his face} was changed against\alt{concerning, with regard to} Shadrach, Meshach, and Abednego. He answered and ordered that they should heat the furnace seven times hotter\lit{more} than it should be\halot{xxxx}{it is customary to be} heated.%%
  \verse{3:20} He ordered the mightiest men in his army to tie up Shadrach, Meshach, and Abednego, and throw them into the fiery, burning furnace.%%
  \verse{3:21} These men then tied up their garments\alt{trousers or cloaks}\halotu{xxxx}{xxxx \Hebrew{סַרְבָּל} for commentaries discussing further alternatives} and their garments\alt{coat or trousers} and their caps and their garments and threw them\alt{imposed a tax on them} into the midst of the fiery, burning furnace.%%
  \verse{3:22} And because of the fact that the king's order was so\understood\ severe\alt{harsh} and the furnace was so extraordinarily hot, the men who took up Shadrach, Meshach, and Abednego were killed by the flame of the fire.%%
  \verse{3:23} %%
  
  \verse{3:24} %%
  \verse{3:25} %%
  
  \verse{3:26} %%
  \verse{3:27} %%
  \verse{3:28} %%
  \verse{3:29} %%
  \verse{3:30} %%
  
  \verse{3:31} %%
  \verse{3:32} %%
  \verse{3:33} %%
\end{inparaenum}

  %
  % \book{Obadiah}{\Hebrew{עבדיה}}
  % \heading{1}{xxxx}

\begin{inparaenum}
    \hfil\verse{1:1} Obadiah's vision:\hfil%%
    
    \smallskip\pvaa{Thus says the Lord \textsc{God} to\footnotemark\ Edom:}%% The \smallskip needs to be here and not on the previous line, otherwise it messes up the centering of the title.
    \fntalt{concerning}%%
    
    \pvbb{``We have heard\footnotemark\ a report from the \textsc{Lord},}{a messenger has been sent out throughout the nations.}%%
    \fnted{The critical apparatus makes frequent comparison to Jer~49, but is not cited here. The reader is encouraged to study the two chapters side-by-side, specifically \vref{Jer}{49}{9--15} as they correspond to \vref{Obad}{1}{1--6}.}%%
    
    \pvba{Get up! Let's go against her\footnotemark\ to battle.}%%
    \fntie{Edom}%%
    
    \pvab{\vn{1:2} I've made you small among the nations~---}{you are greatly despised.}%%
    
    \pvaa{\vn{1:3} The arrogance\footnotemark\ of your heart has tricked\footnotemark\ you.\footnotemark}%%
    \fnthalot{xxxx}{\textbf{presumption}}%%
    \fntalt{deceived}%%
    \fntlit{lifted you up.}%%
    
    \pvbb{O dweller in the clefts in the rock,}{his dwelling-place is lofty.}%%
    
    \pvbb{He says in his heart,}{`Who will bring me down to earth?'}%%
    
    \pvab{\vn{1:4} Although you soar like the eagle,\footnotemark}{and though you place your nest among the stars,}%%
    \fntalt{vulture}%%
    
    \pvbb{I will bring you down from there,''}{declares the \textsc{Lord}.}%%
    
    \pvbb{\vn{1:5} ``If robbers come in to you~---}{if plunderers come in by night~---}%%
    
    \pvcb{(how you've been cut off!)}{will they not steal enough for themselves?}%%
    
    \pvcb{If gleaners\footnotemark\ come to you,}{won't they leave gleanings?}%%
    \fntalt{grape harvesters}%%
    
    \pvbb{\vn{1:6} How Esau has been searched!}{his treasures are sought out!}%%
    
    \pvbb{\vn{1:7} You've been sent out to the border}{by all of your allies.\footnotemark}%%
    \fntlit{the men of your treaty.}%%
    
    \pvcb{They've forgotten you. The men of your peace}{have eaten\footnotemark}%%
    \fnted{The stichs have been slightly reworked. It is literally ``They've forgotten you. They have eaten [next stich] the men of your peace\dots.''}
    
    \pvcb{your bread.}{They've placed a trap\footnotemark\ under you.}%%
    \fnthalot{xxxx}{\textbf{(man-)trap}}%%
    
    \pvca{There is no understanding in him.}%%
    
    \pvab{\vn{1:8} On that day, shall I not,''}{declares the \textsc{Lord},}%%
    
    \pvbb{``destroy the wise men out of Edom,}{and understanding from the mountain of Esau?}%%
    
    \pvab{\vn{1:9} Your mighty men, O Teman, shall be struck down}{in order that every man in the mountain of Esau shall be cut off}%%
    
    \pvbb{by being murdered.\footnotemark\ \vn{1:10} Because of violence against your brother Jacob,}{shame shall cover you, and you shall be cut off forever.}%%
    \fntalt{slaughtered.}%%
    
    \pvab{\vn{1:11} In the day of your standing on the other side,}{in the day when non-Israelites took his property captive,}%%
    
    \pvbb{when\footnotemark\ foreigners entered his gate,}{and they cast lost on Jerusalem,}%%
    \fnted{understood}%%
    
    \pvba{even you were one of them!}%%
    
    \pvbb{\vn{1:12} But you looked on the day of your brother,\footnotemark}{on the day of his estrangement;}%%
    \fntca{prp \Hebrew{בְאחיך}}{it has been proposed to be ``on your brother''}%%
    
    \pvcb{neither did you rejoice over the children of Judah}{in the day of their going astray,}%%
    
    \pvcb{nor make your mouth great}{in the day of distress,}%%
    
    \pvbb{\vn{1:13} nor come through the gate of My people}{in the day of their disaster,\footnotemark}%%
    \fnthalot{xxxx}{(final) \textbf{disaster}}%%
    
    \pvcb{nor did you, even you, look on its wickedness}{is the day of its disaster,}%%
    
    \pvcb{nor did you send against its outer wall}{in the day of its disaster,}%%
    
    \pvbb{\vn{1:14} nor stand on the crossroads}{to cut off its fugitives,}%%
    
    \pvcb{nor delivered its survivors}{in the day of its distress,}%%
    
    \pvbb{\vn{1:15} because the day of the \textsc{Lord} is near}{on all the nations.}%%
    
    \pvcb{As you've done, it's done to you:}{your reward will return on your own head.}%%
    
    \pvab{\vn{1:16} Because as you've drunk on My holy mountain,}{so shall all the nations drink unceasingly\footnotemark~---}%%
    \fntca{mlt Mss \Hebrew{סָבִיב}}{multiple Hebrew manuscripts have ``all around'' [instead of ``unceasingly/continually'']}%%
    
    \pvbb{they shall drink and slurp,\footnotemark}{they shall be like they were note.}%%
    \fntca{prp \Hebrew{ונָעו}}{it has been proposed to be ``and stumble/teeter/shake/tremble''}%%
    
    \pvac{\vn{1:17} On mount Zion}{there's deliverance}{and holiness; [why would it be ``there is holiness''?]\footnotemark}%%
    \fntca{add?}{[this hemistich] added?}%%
    
    \pvbb{the house of Jacob shall possess}{their possessions.}%%
    
    \pvab{\vn{1:18} The house of Jacob shall be a[?] fire,}{the house of Joseph shall be a flame,}%%
    
    \pvbb{the house of Esau shall be stubble:\footnotemark}{they shall burn among them and consume them.}%%
    \fntalt{straw}%%
    
    \pvbb{There will be no survivors}{in the house of Esau}%%
    
    \pvba{because the \textsc{Lord} has spoken.}%%
    
    \pvbb{\vn{1:19} The south [subject?] shall possess mount Esau [make consistent with this one] and the lowlands of the Philistines,}{they shall possess}%%
    
    \pvba{the field\footnotemark\ of Ephraim and the field of Samaria and of Benjamin\footnotemark\ and Gilead.}%%
    \fntca{\septuagint\ \Greek{τὸ ὄρος}, frt l \Hebrew{הַר}}{the Septuagint has ``the mountain,'' perhaps read ``mount''}%%
    \fntca{xxxx}{xxxx}%%
    
    \pvba{\vn{1:20} The captives of this force of the children of Israel (that are of the Canaanites) are to[?] Zarephath,}%%
    
    \pvba{and the captives of Jerusalem who were in Sepharad shall possess the cities of the south.}%%
    
    \pvbb{\vn{1:21} And saviors shall go up on Mount Zion}{to judge the mountain of Esau,}%%
    
    \pvca{and the kingdom shall be the \textsc{Lord}'s.}%%
\end{inparaenum}

  %
  % \book{Jonah}{\Hebrew{יונה}}
  % \heading{1}{xxxx}

\begin{inparaenum}
  \verse{1:1} The word of the \lord\ to Jonah\ed{Jonah is different than most Hebrew names in that it is not a sentence, just a word: ``love.''} son of Amitta, saying,%%
  \verse{1:2} ``Get up and go to the great city Nineveh and proclaim against them\lit{it; however, it's referring to the people and not the city, the city not being capable of committing sin.} because their wickedness has come up before Me.''%%
  \verse{1:3} So Jonah got up  and fled from before the face of the \lord. He went down to Joppa and found a ship bound for Tarshish. He paid his fare and went down into the ship in order to go with them to Tarshish from before the face of the \lord.%%
  
  \verse{1:4} The \lord\ cast a great wind, and there was a great tempest on the sea and the ship was about to be broken up.%%
  \verse{1:5} The sailors were afraid and each man cried to his god. They cast their goods that were in the ship into the sea to make it light of them. And Jonah went down to the lower part of the vessel, lied down, and was fast asleep.%%
  \verse{1:6} The skipper\lit{chief of the company} approached him and said to him, ``What's with you, sleeping one? Get up, call on your god,\ed{Theologically ``God,'' but that's probably not what the skipper would have said.} and perhaps god will bear us in mind that we don't perish.''%%
  \verse{1:7} And each man said to his neighbor, ``Come, let's cast lots that we may know on whose account this evil has befallen us.'' So they cast lots and the lot fell on Jonah.%%
  \verse{1:8} So they said to him, ``Please tell us on what account this evil is upon us. What's your occupation? Where are you from? What is your nationality, seeing as you're not of this people?''\ie{What's your ethnicity?}%%
  \verse{1:9} He said to them, ``I am a Hebrew and I worship the \lord, the God of Heaven who made xxxx the sea and the dry land.''%%
  \verse{1:10} And the men were terribly afraid\lit{feared a great fear} xxxx and said to him, ``What have you done?''\lit{What is this that you've done?} because the men knew that he was fleeing from before the face of the \lord\ xxxx%% because he'd told them before\understood
  \verse{1:11} They said to him, ``What should we do to you so that the sea may be calm for us? because the sea is more and more tempestuous xxxx.''%%
  \verse{1:12} He said to them, ``Lift me up and cast me into the sea so that the sea will calm down for you because I know that it is because of me that this great tempest is on you.''%%
  \verse{1:13} The men rowed to bring the ship\understood\ back to dry land and weren't able to because the sea was more and more tempestuous\alt{was storming over them, continued to rage} against them.%%
  \verse{1:14} They cried to the \lord\ and said, ``Please, \lord, do not let us die for this man's life. Do not lay innocent blood on us because You, O \lord, as You've done as You've pleased.''%%
  \verse{1:15} So they lifted up Jonah and cast him into the sea and the sea stopped raging.%%
  \verse{1:16} The men terribly feared the \lord\ and offered a sacrifice and made vows.%%
\end{inparaenum}

  % \heading{2}{Jonah is swallowed by a great fish~--- he prays to the Lord and repents~--- Jonah is forgiven and spit out onto the dry land}

\begin{inparaenum}
  \verse{2:1}%
  \ed{In English versions, this is \vref{Jon}{1}{17} and the following verses are \vref{Jon}{2}{1--10}. This version follows the Hebrew numbering scheme.}
  The \textsc{Lord} had prepared%
  \halot{xxxx}{xxxx}
  a great fish%
  \halot{xxxx}{xxxx}
  to swallow Jonah, and Jonah was in the belly of the fish for three days and three nights.%%
  \verse{2:2} Jonah prayed to the \textsc{Lord} his God from the belly of the fish,%%
  \verse{2:3} and said,\smallskip%%
  
  \pd ``I cried because of my afflictions\alt{in my despair/distress}\pa and the \textsc{Lord} answered me.%%
  
  \pd I cried for help from the belly of Sheol\pa and He heard my voice.%%
  
  \pc \verse{2:4} You cast me into the deep~--- into the midst of the sea~---\pa and the rushing%
  \halot{xxxx}{xxxx deep?}
  waters surrounded me.%%
  
  \pd All the billows and waves\pa passed over me%%
  
  \pc \verse{2:5} and I said, `I am cast\pa out of Your sight,%%
  
  \pd yet I will again look\pa at Your holy temple.'\lit{the temple of Your holiness.}%%
  
  \pc \verse{2:6} The waters encompassed me to my neck,\pa the deep sea engulfed me.%%
  
  \pd The reeds%
  \halot{xxxx}{xxxx seaweed?}
  were wrapped around my head\pa \verse{2:7} I went down to the base of the mountains.%%
  
  \pd The land, its bars\ie{prison}\pa came up to me forever.%%
  
  \pd But You brought up my life from destruction,\pa O \textsc{Lord} my God.%%
  
  \pc \verse{2:8} When my soul was faint\pa I remembered the \textsc{Lord}.%%
  
  \pd My prayer came to You,\pa to Your holy temple.%%
  
  \pc \verse{2:9} xxxx\pa xxxx%%
  
  \pc \verse{2:10} With a voice of thanksgiving\pa I will sacrifice to You.%%
  % xxxx Let me sacrifice to you with a voice of thanksgiving?
  
  \pd xxxx\pa xxxx''\smallskip%%
  
  \verse{2:11} The \textsc{Lord} spoke to the fish and it vomited\alt{spit} Jonah onto the dry land.%%
\end{inparaenum}

  % \heading{3}{Jonah preaches in Nineveh~--- the people believe and repent~--- the Lord forgives them}

\begin{inparaenum}
  \verse{3:1} The word of the \textsc{Lord} came to Jonah a second time, saying,%%
  \verse{3:2} ``Get up and go to the great city, Nineveh, and preach the words\lit{preaching} I will tell you.''%%
  \verse{3:3} So Jonah got up and went to Nineveh according to the word of the \textsc{Lord}. Now, Nineveh was a huge city xxxx three days' journey across.\understood%%
  \verse{3:4} %%
  \verse{3:5} The men of the city of Nineveh, from greatest to least, believed in God and called for a fast and put on sackcloth.%%
  \verse{3:6} And the matter came to the king of Nineveh and he arose from his throne and removed his royal robe and put on sackcloth and sat in ashes.%%
  \verse{3:7} At the behest of the king and his courtiers, a proclamation was made and it was published throughout Nineveh, saying, ``Don't let people, animals, and flocks taste anything. Don't let them feed or drink water.%% xxxx
  \verse{3:8} %%
  \verse{3:9} xxxx''%%
  \verse{3:10} %%
\end{inparaenum}

  % \heading{4}{Jonah is angry with the Lord's decision to spare Nineveh~--- he leaves town to pout~--- the Lord causes a gourd to grow and provide shade for him, then destroys the gourd to teach him a lesson~--- Jonah is still upset}

\begin{inparaenum}
  \verse{4:1} Jonah was really angry, and it displeased him.%%
  \verse{4:2} He prayed to the \textsc{Lord} and said, ``O \textsc{Lord}, is this not my word while I was in my own xxxx that I was going to flee to Tarshish xxxx%% xxxx fix this whole verse
  \verse{4:3} Now, \textsc{Lord}, please take my life\alt{spirit} from me because death would be better than my life.''%%
  
  \verse{4:4} The \textsc{Lord} said, ``Is doing good displeasing to you?''%%
  \verse{4:5} Jonah went out from the city and sat on the east of the city and made a booth for himself there. And he sat under it in the shade until he could see what was going to happen in the city.%%
  \verse{4:6} And the \textsc{Lord} God prepared a gourd\halot{xxxx}{xxxx}\ed{castor bean plant} and caused it to come up over\alt{placed it over} Jonah to be a shade over his head,\alt{to give shade to his head} and Jonah greatly rejoiced\lit{rejoiced with great joy} because of the gourd.%%
  \verse{4:7} God prepared a worm when it xxxx the next day and it struck the gourd that it whithered.%%
  \verse{4:8} When the sun came up, God prepared a scorching, east wind; and the sun beat on Jonah's head and he wrapped himself up and requested his soul to die, and said, ``It is better for me to die than to live.''\lit{My death is better than life.}%%
  \verse{4:9} God said to Jonah, ``Do you like being angry about the gourd?'' He said, ``I do well to be angry, even unto death!''%%
  \verse{4:10} The \textsc{Lord} said, ``You have pitied the gourd which you didn't toil\alt{work} for, neither did you nourish it that one night was and perished the next night.%%
  \verse{4:11} Have I not pity on the great city Nineveh in the which are more than 120\thinspace000~people who cannot discern between their right hand and their left hand, and also many cattle?''%%
\end{inparaenum}

  %
  % \book{Zephaniah}{\Hebrew{צפניה}}
  % \heading{1}{xxxx}

\begin{inparaenum}
  \verse{1:1} %%
  \verse{1:2} :::I will totally destroy everything :from off the face of the land :dictates the \lord.%%
  \verse{1:3} :::I will destroy man and beast, :the birds in the sky, :and the fish in the sea.%%
  
  :::The stumbling blocks (the wicked) XXXX%%
  \verse{1:4} %%
  \verse{1:5} %%
  \verse{1:6} %%
  \verse{1:7} %%
  \verse{1:8} %%
  \verse{1:9} %%
  
  \verse{1:10} %%
  \verse{1:11} %%
  \verse{1:12} %%
  \verse{1:13} %%
  \verse{1:14} %%
  \verse{1:15} %%
  \verse{1:16} %%
  \verse{1:17} %%
  \verse{1:18} %%
\end{inparaenum}

  %
  % \book{Haggai}{\Hebrew{חגי}}
  % \heading{1}{The Lord reproves the people for having not rebuilt the temple, but rather their own homes~--- they obey the Lord and work on the temple}

\begin{inparaenum}
  \verse{1:1} In the second year of King Darius,\ed{ca.\ 522~\textsc{b.c.}} in the sixth month, on the first day of the month, the word of the \textsc{Lord} came by the hand of the prophet Haggai to Zerubbabel son of Shealtiel, governor of Judah, and to Joshua son of Jehozadak, the high priest, saying:%%
  \verse{1:2} ``Thus says the \textsc{Lord} of Hosts, saying: `This people has said, ``The time has not come, the time that the house of the \textsc{Lord} will be built!''\thinspace'\thinspace''%%
  
  \verse{1:3} The word of the \textsc{Lord} through\lit{by the hand of} the prophet Haggai, saying:\smallskip%%
  
  \pvcb{\verse{1:4} ``Is it time for you~--- you~---}{to live in your paneled houses\footnotemark}%%
  \ed{probably one that is nicely finished on the outside, but probably on the inside as well}%%
  
  \pvda{while this house lies in ruin?''}%%
  
  \pvca{\verse{1:5} And so now, thus says the \textsc{Lord} of Hosts:}%%
  
  \pvda{``Take heed:\footnotemark}%%
  \ed{For variant renderings, see Appendix~\ref{app:take-heed}.}%%
  
  \pvcb{\verse{1:6} you've sewn much, but brought in little.}{You eat, but you're not satiated.}%%
  
  \pvdb{You drink, but don't get drunk.\footnotemark}{You get dressed, but you're not warm.}%%
  \ed{This seems to be the Lord's equivalent of ``go big or go home.'' The people aren't doing good~--- they're not even doing bad well. It seems to be that the Lord just wants them to commit: if you're going to be good, be good! If you're going to be bad, be bad! There is no efficacy in anything they do.}%%
  
  \pvdb{He who hires himself out for money, hires himself out}{for a bag that has holes in it.''}\bigskip%%
  
  \verse{1:7} Thus says the \textsc{Lord} of Hosts:\smallskip%%
  
  \pvda{Take heed.\footnotemark}%%
  \ca{add? cf 5}{[Was this stich] added? Compare verse~5}%%
  
  \pvcc{\verse{1:8} Go up to the mountain}{and bring wood\footnotemark}{and build a house}%%
  \lit{a tree}%%
  
  \pvdc{that I may be pleased with it}{and be glorified,''}{says the \textsc{Lord}.}%%
  
  \pvcb{\verse{1:9} ``Turning about\footnotemark\ much, but it came to little;}{you brought it home and blew on it.}%%
  \alt{``Paying attention to, Worrying about}%%
  
  \pvdb{Why?''}{declares the \textsc{Lord} of Hosts.}%%
  
  \pvdb{``Because My house is in ruins}{and every man is running to his house.}%%
  
  \pvcb{\verse{1:10} Therefore, the heavens have withheld their dew\footnotemark\ from over you}{and the land has withheld its produce.}%%
  \alt{light rain}%%
  
  \pvcb{\verse{1:11} I proclaimed\footnotemark\ a drought on the land,}{the mountains, the grain,}%%
  \alt{called for}%%
  
  \pvdb{the new wine,\footnotemark\ the pure oil,\footnotemark}{and on\footnotemark\ that which the ground brings forth~---}%%
  \ed{probably freshly squeezed; not fermented or only recently starting to ferment}%%
  \ca{add?}{[``the oil''] added?}%%
  \ca{ins c mlt Mss Vrs \Hebrew{־כָּל־}}{multiple Hebrew manuscripts and all or most translations insert ``all''}%%
  
  \pvdb{on mankind, on beasts,}{and on all the exertions\footnotemark\ of the hands.''}\bigskip%%
  \alt{labors, toils}%%
  
  \verse{1:12} Zerubbabel son of Shealtiel and the high priest, Joshua son of Jehozadak, and all the remnant of the people heeded the voice of the \textsc{Lord} their God and the words of the prophet Haggai as the \textsc{Lord} their God had sent them, and the people feared before the \textsc{Lord}.%%
  \verse{1:13} Haggai, the messenger of the \textsc{Lord}, in the \textsc{Lord}'s messages to the people, spoke, saying,\hspace*{2em} ``\thinspace`I am with you,' declares the \textsc{Lord}.''%%
  \verse{1:14} The \textsc{Lord} stirred\alt{raised} up the spirit of the governor of Judah, Zerubbabel son of Shealtiel, and the spirit of the high priest, Joshua son of Jehozadak, and the spirit of all the remnant of the people: they came in\ed{This preposition could easily be rendered ``on'' which would greatly change the meaning of this sentence.} and did work in the house of the \textsc{Lord} of Hosts, their God.%%
  
  \pvca{\verse{1:15} In the twenty-fourth day of the sixth month in the second year of King Darius.\footnotemark}%%
  \ca{huc ins \caref{}{2}{15--19}}{hither insert \vref{Hag}{2}{15--19}}%%
\end{inparaenum}

  % \heading{2}{The Lord encourages Zerubbabel, Joshua, and the people~--- He asks Haggai to question the priests concerning the Law~--- the Lord chooses Zerubbabel}

\begin{inparaenum}
    \noindent\verse{2:1} \ed{This verse does not start a new paragraph, rather it is a continuation of the previous chapter.}In the seventh month,\understood\ on the twenty-first day of the month, the word of the \textsc{Lord} came by the hand of Haggai the prophet, saying:%%
    \verse{2:2} ``Please speak to the governor of Judah, Zerubbabel son of Shealtiel, and the high priest, Joshua son of Jehozadak, and the remnant of the people, saying:\smallskip%%
    
    \pvca{\vn{2:3} `Who among you is left}%%
    
    \pvdb{who saw this house}{in its former glory?}%%
    
    \pvdb{And how do you see}{it now?\footnotemark}%%
    \fntalt{How do you look at it now?}%%
    
    \pvdb{Is it like nothing}{in your eyes?}%%
    
    \pvca{\vn{2:4} Now, be strong, Zerubbabel,' declares the \textsc{Lord}.\footnotemark}%%
    \fntca{add?}{[``declares the \textsc{Lord}''] added?}%%
    
    \pvdb{`Be strong, Joshua son of Jehozadak,}{the high priest;}%%
    
    \pvdb{be strong, all the people of the land,'}{declares the \textsc{Lord}.\footnotemark}%%
    \fntca{add?}{[``Joshua\dots\ \textsc{Lord}''] added?}%%
    
    \pvdb{`Work, because I am with you,'}{declares the \textsc{Lord} of Hosts.\footnotemark}%%
    \fntca{add?}{[``of Hosts''] added?}%%
    
    \pvca{\vn{2:5} \footnotemark The thing\footnotemark\ I covenanted with you when you came out of Egypt}%%
    \fntca{prb l \Hebrew{זֹאת} \Hebrew{הַבְּרִית}, al dl \Hebrew{את} = \Hebrew{(צב)אות}}{probably read ``this covenant,'' others delete \textit{\d tbaot} and leave \textit{et}}%% xxxx fix the latinization here
    \fntca{\missing\ \septuagint*, add}{[this stich] missing in the Septuagint (textus Graecus originalis) and was added}%%
    
    \pvdb{and My spirit remains in your midst~---}{don't be afraid.'\thinspace''}\bigskip%%
    
    \pvcb{\vn{2:6} For thus says the \textsc{Lord} of Hosts:}{``In just a little bit\footnotemark}%%
    \fntlit{``Yet once, it's little,}%%
    
    \pvdb{I will shake the heavens and the earth,}{the sea and the dry land.}%%
    
    \pvcb{\vn{2:7} I will shake all the nations,}{and the desire of the nations shall come in.}%%
    
    \pvdb{I will fill this house with glory,''}{says the \textsc{Lord} of Hosts.}%%
    
    \pvcb{\vn{2:8} ``The silver is Mine, the gold is Mine,''}{declares the \textsc{Lord} of Hosts.}%%
    
    \pvcc{\vn{2:9} ``The glory of this last\footnotemark\ house is greater}{than the first,''}{says the \textsc{Lord} of Hosts.}%%
    \fnted{moved ``last'' from the second hemistich to here for idiomaticy}%%
    
    \pvdb{``I will create\footnotemark\ peace in this place,''}{declares the \textsc{Lord} of Hosts.}%%
    \fntlit{give}%%
    
    \verse{2:10} On the twenty-fourth day\understood\ of the ninth month\understood\ in the second year of Darius, the word of the \textsc{Lord} came by the hand of Haggai the prophet, saying:%%
    \verse{2:11} ``Thus says the \textsc{Lord} of Hosts: `Please inquire\ed{This is direction being given to Haggai, not to the people through Haggai.} of the priests concerning\understood\ the Law, saying:%%
    \verse{2:12} ``If someone is carrying holy flesh in the fold\lit{skirt} of his garment and he touches bread with his skirt~--- or a boiled dish or wine or oil or any food~--- shall it become holy?'' And the priest will answer and say, ``No.''%%
    \verse{2:13} And Haggai will say, ``If someone who is cultically unclean shall touch any of these, will they become unclean?'' And the priest will answer and say, ``It shall be unclean.''%%
    \verse{2:14} And Haggai will answer and say,\smallskip%%
    
    \pvdc{``So is this people}{and so is this nation}{before Me,'' declares the \textsc{Lord}.}%%
    
    \pvdc{``So is everything they do with their hands}{and that which they offer there:}{it is unclean.}%%
    
    \pvaa{\vn{2:15} And now,\footnotemark}%%
    \fntca{tr 15--19 post 1,15a}{transpose verses~15--19 after \vref{Hag}{1}{15}}%%
    
    \pvab{please take heed}{from this day on}%%
    
    \pvab{before a stone was laid on a stone}{in the house of the \textsc{Lord}. \vn{2:16} When\footnotemark\footnotemark}%%
    \fntlit{From their being}%%
    \fntalt{From that time}%%
    
    \pvab{they come to a sheath of twenty,}{there are only ten;\footnotemark}%%
    \fnted{Meaning that when they try to produce something, they will only produce half; when they try to sanctify things, they will only be unclean.}
    
    \pvab{when they come to the wine vat to get fifty,}{there will only be twenty.}%%
    
    \pvab{\vn{2:17} I have smitten you with blight and mildew}{and hail, all the work of your hands.}%%
    
    \pvab{None of you are with Me,''}{declares the \textsc{Lord}.}%%
    
    \pvab{\vn{2:18} ``Please take heed}{from today onward}%%
    
    \pvaa{from the twenty-fourth day of the ninth month, from the day the temple of the \textsc{Lord} was founded: take heed.}%%
    
    \pvac{\vn{2:19} Is the seed\footnotemark\ still in the grain-pit?\footnotemark\footnotemark}{Until the vine, the fig, and the pomegranate,\footnotemark}{and the olive tree do not produce:\footnotemark\footnotemark}%%
    \fntca{prb l \Hebrew{םִגְרָע} deminutio, al ins \Hebrew{נִגְרַע}}{probably read ``diminished'' [probably ``Is the grain-pit yet diminished?''], while others insert ``diminish'' [i.e., ``Is the seed still diminished in the grain-pit?'']}%%
    \fntalt{storage chamber?}%%
    \fnted{This is a classic example of a non-idiomatic translation used in vogue: many English translations render this as ``barn'' even though the concept of a barn is anachronistic.}%%
    \fntca{add?}{[Was this hemistich] added?}%%
    \fntlit{bear}%%
    \fntalt{bring forth}%%
    
    \pvaa{from this day I will bless you.''}%%
    
    \pvba{\vn{2:20} The word of the \textsc{Lord} came to Haggai on the twenty-fourth day of the month,}%%
    
    \pvaa{saying: \vn{2:21} ``Speak to the governor of Judah, Zerubbabel, saying:}%%
    
    \pvcb{`I am shaking}{the heavens and the earth.w}%%
    
    \pvbb{\vn{2:22} I will overthrow the throne of the kingdoms;}{I will exterminate the strength of the kingdoms\footnotemark\ of the nations.}%%
    \fntca{prb dl, var lect}{[``of the kingdoms''] is probably deleted, [or is a] variant reading}%%
    
    \pvcc{I will overthrow the chariots and their riders:}{their horses and riders shall come down,}{each by his brother's sword.\footnotemark}%%
    \fntca{add?}{[this hemistich] added?}%%
    
    \pvcb{\vn{2:23} In that day,' declares the \textsc{Lord} of Hosts, `I will take you,}{Zerubbabel, son of Shealtiel, My servant,' the \textsc{Lord} declares,\footnotemark}%%
    \fntca{add?}{[``son of Shealtiel'' and ``the \textsc{Lord} declares''] added?}%%
    
    \pvcb{`and I will make you like a seal\footnotemark}{because I have chosen you,'}%%
    \fnthalot{xxxx}{for identification}%%
    
    \pvda{declares the \textsc{Lord} of Hosts.''}%%
\end{inparaenum}

  %
  % % xxxx In Malachi, add a reference to Appendix on yada for whatever verse talks about marriage
  % \book{Malachi}{\Hebrew{מלאכי}}
  % \heading{1}{The Lord continues to love His people who in turn reject Him~--- the Lord enumerates the ways in which His people have rejected Him~--- they offer impure sacrifices}

\begin{inparaenum}
    \verse{1:1} The pronouncement\lit{burden}\alt{oracle, prophecy} of the \textsc{Lord} to Israel by the hand of Malachi:\smallskip%%
    
    \pvab{\vn{1:2} ``\thinspace`I have loved you,' says the \textsc{Lord}.}{Yet you say, `How have you loved us?'}%%
    
    \pvbd{`Isn't Esau Jacob's brother?'}{A declaration of the \textsc{Lord}.\footnotemark}{`I have loved Jacob}{\vn{1:3} and hated Esau.}%%
    \fntca{dl m cs}{deleted to preserve the meter case}%%
    
    \pvbb{I have made his mountains desolate\footnotemark}{and his inheritance [shall be/will go?] to the jackals\footnotemark\footnotemark\ of the desert.'}%%
    \fntalt{deserted}%%
    \fnthalot{xxxx}{\textit{Canis aureus}}%%
    \fntca{frt l \Hebrew{נָתַתִּי}, prp \Hebrew{נְוֺת} cf \peshitta}{perhaps read ``I will give,'' it has been proposed to be ``xxxx'' compare the Peshitta}%%
    
    \pvab{\vn{1:4} But Edom says, `We have been decimated,\footnotemark}{but we will return and rebuild the ruined sites,'\footnotemark}%%
    \fntalt{shattered}%%
    \fntalt{desolate places}%%
    
    \pvbb{thus says the \textsc{Lord} of Hosts:}{`They shall build and I will destroy.'}%%
    
    \pvbb{They've called\footnotemark\ them `a territory of wickedness,'}{`a people whom the \textsc{Lord} has cursed forever.'}%%
    \fnted{This is not ``They've called \emph{to}'' because there is no expressly stated subject.}%%
    
    \pvab{\vn{1:5} Your eyes have seen, and you shall say,}{`The \textsc{Lord} is exalted beyond the border of Israel.}%%
    
    \pvbb{\vn{1:6} A son honors his\footnotemark\ father,}{a servant his master.}%%
    \fnted{understood}%%
    
    \pvcb{But if I am a father,}{where is My honor?}%%
    
    \pvcb{And if I am a master,}{where is My fear?\footnotemark}%%
    \fntalt{the reverence toward Me?}%%
    
    \pvcb{The \textsc{Lord} of Hosts says to you,}{``Priests have despised My name,''}%%
    
    \pvcb{yet you say,}{``How have we despised Your name?''}%%
    
    \pvbb{\vn{1:7} You have offered on My altar}{polluted\footnotemark\ bread.}%%
    \fnthalot{xxxx}{(cultic) \textbf{pollution}}%%
    
    \pvcb{Yet you say,}{``How have we polluted you?''}%%
    
    \pvcb{In that you say,\footnotemark\ ``The altar of the \textsc{Lord}}{is despised.''\footnotemark}%%
    \fntlit{By your saying, By your having said, By saying}%%
    \fntalt{contemptible.'}%%
    
    \pvbc{\vn{1:8} If you bring}{the blind\footnotemark to sacrifice~---}{there is no wrong in this.\footnotemark}%%
    \fntie{a blind animal}%%
    \fntlit{isn't it evil?}%%
    
    \pvcc{If you bring}{the lame\footnotemark\ or ill~---}{there is no wrong in this.}%%
    \fnted{could refer to an animal}%%
    
    \pvcc{Please offer it to the governor:\footnotemark}{will he be pleased with you, or will he accept you?'}{says the \textsc{Lord} of Hosts.\footnotemark}%%
    \fnthalot{xxxx}{a rather vague title}%%
    \fntlit{lift up your face?'}%%
    \fntca{dl m cs}{delete [this hemistich] to preserve the meter case}%%
    
    \pvbb{\vn{1:9} `And now, please, appease\footnotemark}{God\footnotemark\ and He will graciously provide for us.}%%
    \fntalt{flatter, entreat}%%
    \fntlit{the face of God}%%
    
    \pvcc{This has been from your hand:\footnotemark}{will He accept your offering?'}{says the \textsc{Lord} of Hosts.\footnotemark}%%
    \fntie{You did this:}%%
    \fntlit{will He lift up from your faces?'}%%%
    \fntca{dl}{delete [this hemistich]}%%
    
    \pvbb{\vn{1:10} `Who's even among you who would close the doors?\footnotemark}{You would not kindle on My altar without cause.\footnotemark}%%
    \fntie{of the temple?}%%
    \fntalt{in vain.}%%
    
    \pvcc{I have no pleasure in you,'}{says the \textsc{Lord} of Hosts.\footnotemark}{`I do not want an offering\footnotemark\ from you\footnotemark}%%
    \fntca{add}{[this hemistich] added}%%
    \fntalt{oblation, sacrifice, allegiance}%%
    \fntlit{your hand.}%%
    
    \pvbb{\vn{1:11} \footnotemark because from sunrise to sunset}{My name is great among the nations.}%%
    \fntca{11--13 prb add}{verses~11--13 probably added}%%
    
    \noindent In every place, incense and a clean\halot{xxxx}{\textbf{(cultically) clean}} offering are brought in to My name, because My name is great among the nations,' says the \textsc{Lord} of Hosts.%%
    \verse{1:12} `But you are profaning it by saying, ``The table of the Lord is impure;\alt{polluted, desecrated} its fruit, its food,\ca{prp \Hebrew{כֻלּוֺ}}{it has been proposed to be ``all of his''} is contemptible.''\ca{prb l \Hebrew{וְנִבְזֶה} (\Hebrew{ניבו} ortum ex dttg \Hebrew{נב}) cf \peshitta\targum}{probably read ``and is contemptible (``its fruit'' arising from dittography of \textit{nb}), compare the Peshitta and Targum}%%
    \verse{1:13} You say, ``What a bother!'' and become enraged\lit{sniff} at it,'\ca{1 \Hebrew{אותִי} (Tiq soph)}{1 has ``at Me'' (the Tiqqun sopherim)} says the \textsc{Lord} of Hosts. `You've brought in stolen goods, the lame, and the sick, and you bring in your offering. Should I accept\alt{be please with} this from your hand?' says the \textsc{Lord}.\ca{frt ins c pc Mss \septuagint\peshitta\super{W} \Hebrew{צְבָאוֺת}}{perhaps insert ``of Hosts'' with a few Hebrew manuscripts, the Septuagint, and Peshitta}%%
    
    \verse{1:14} \ca{14~add}{verse~14 added}`Cursed is the deceiver who has\lit{when there's} a male\halot{xxxx}{\textbf{male animal}, especially \textbf{ram}} in his flock and vows\ie{that it's clean} and\ed{understood: then} sacrifices a defective thing\halot{xxxx}{\textbf{defect}} to the Lord,\ca{prp c mlt Mss \Hebrew{לַיהוָה} vel \Hebrew{לִי}}{it has been proposed to be the Tetragrammaton in multiple Hebrew manuscripts} because I am a great king,' says the \textsc{Lord} of Hosts, `and My name is revered among the nations.'\thinspace''%%
\end{inparaenum}

  % \heading{2}{The people have fallen away from the Lord through transgression~--- they have forsaken God (the wife of their youth\ed{It's difficult to say if this chapter is literal or figurative. However, it's always best to assume that it's symbolic and find the meaning there, and then to figure out if it's literal. If this is purely symbolic, Israel is being called to repentance for forsaking the Lord in their youth and serving other gods. If this is literal, Israelites dealt unfaithfully with their covenant wives, divorced them, and married foreign women. Either way, the message is of faithfulness to spouse and God.}\ed{In ancient Israel there are three elements of marriage: contract, consummation, and celebration. The contract is between the parents of the prospective groom and the prospective bride. Getting out of an engagement could be a very sticky affair. This was tradition from the period before Moses and continued through the end of the monarchic period. See further in Appendix~\ref{app:covenants-in-antiquity}.}) for foreign gods~--- they call good evil and evil good}\ed{\S~--- This has been a really spiritually powerful chapter for me to translate.}\ed{``Malachi~2 is difficult because it doesn't make much sense.'' ---Professor Ricks, 2015-02-11}

\begin{inparaenum}
  \pvac{\verse{2:1} Now, to you}{is this commandment,}{O priests:}%%
  
  \verse{2:2} ``If you don't listen, if you don't pay attention,\lit{lay it to heart} to give glory to My name,'' says the \textsc{Lord} of Hosts, ``I will send a\lit{the} curse\ed{for offering impure sacrifices} on you. I will curse your blessings. I have also cursed them because you're not paying attention.%%
  
  \pvbb{\verse{2:3} I will rebuke\footnotemark\footnotemark\ your posterity}{I've spread dung\footnotemark\footnotemark\ on your faces,}%%
  \alt{reproach}%%
  \ca{prb l \Hebrew{גֹדֵעַ} cf \septuagint\ \Greek{ἀφορίζω} = \Hebrew{גרע}}{probably read ``cut off,'' compare the Septuagint which has ``excommunicate'' = ``shave/diminish/take away''}%%
  \halot{xxxx}{\textbf{contents of the stomach} (usually of ruminant [even-toed, hoofed mammals that chew the cud regurgitated from their first stomach, comprising the cattle, sheep, antelopes, deer, giraffes, and their relatives] animals); (other: \textbf{dung})}%% Definition taken from Google's "define" command for ruminant, ungulate, and rumen.
  \ed{the intestines (which may contain dung) that are removed from an animal when it's sacrificed}%%
  
  \pvcb{the dung of your festivals,\footnotemark}{and it shall take you away with it.}%%
  \ca{gl, dl}{glossed, to be deleted [this hemistich]}%%
  
  \pvbc{\verse{2:4} You will know}{that\footnotemark\ I have sent}{this commandment to you,}%%
  \ca{\septuagint\ + \Greek{ἐγώ}}{the Septuagint adds ``I'' [understood in Hebrew]}%%
  
  \pvcb{that My covenant may continue with Levi,''}{says the \textsc{Lord} of Hosts.}%%
  
  \pvbb{\verse{2:5} ``My covenant that's been with him}{is one of life and peace.}%%
  
  \pvcb{I've made him fear\footnotemark\ so he fears Me}{so he reverences My name.}%%
  \ca{frt l \Hebrew{וְהַמּורא}; exc vb?}{perhaps read ``the fear''; have words been dropped out?}%%
  
  \pvbb{\verse{2:6} The law of truth was in his mouth;}{perversity\footnotemark\ was not found on\footnotemark\ his lips.}%%
  \alt{wickedness, iniquity}%%
  \alt{in}%%
  
  \pvcb{He walked with Me in peace and uprightness}{and turned many away from iniquity.}%%
  
  \pvcb{\verse{2:7} \footnotemark Because the priest's lips should be careful with\footnotemark\ knowledge~---}{they seek the Law from his lips}%%
  \ca{7~prb add}{verse~7 was probably added}%%
  \ed{This word (\Hebrew{שׁמר}) is the same word in the phrase ``\textit{keep} the commandments''; it has a sense of keeping, obeying, or preserving, often with the connotation of doing so with great care.}%%
  
  \pvca{because he is the messenger of the \textsc{Lord} of Hosts.}%%
  
  \pvbb{\verse{2:8} You've turned out of the way}{and caused many to stumble in the Law.}%%
  
  \pvcb{You've corrupted the covenant of Levi,''}{says the \textsc{Lord} of Hosts.\footnotemark}%%
  \ed{\S~--- This is beautiful and powerful parallelism.}
  
  \pvbb{\verse{2:9} ``Additionally, I have made you contemptible}{and abased before all the people}%% xxxx punctuation?
  
  \pvcb{My palm that you're not keeping My way}{and show partiality\footnotemark\ to those\footnotemark\footnotemark\ in the Law.}%%
  \halot{xxxx}{show partiality, favoritism to(-ward) \haref{Ma}{2}{9}}%%
  \lit{mouths}%%
  \ed{probably ``those interpreting or administering''}%%
  
  \pvab{\verse{2:10} Do we not all have one father?}{Hasn't one God made us?}%%
  
  \pvbb{Why is a man betrayed by\footnotemark\ his brother}{by profaning the covenant of our fathers?}%%
  \alt{deal faithlessly against}%%
  
  \pvac{\verse{2:11} Judah has dealt faithlessly.}{An abomination has been carried out\footnotemark}{against Israel and Jerusalem\footnotemark}%%
  \alt{made in}%%
  \ca{prb dl, var lect}{probably deleted [or] a variant reading [so it would be either ``against Israel'' or ``against Jerusalem,'' not both]}%%
  
  \pvca{\footnotemark because Judah has profaned the holy thing of the \textsc{Lord} in that He's loved and married the daughter of a foreign god.\footnotemark}%%
  \ca{prb add}{probably added [referring to the entirety from this mark through verse~12]}%%
  \ie{a foreigner, not of their faith}%%
  
  \pvba{\verse{2:12} The \textsc{Lord} will cut off from the tent of Jacob the man who does this, \Hebrew{עֵר}\footnotemark\footnotemark\ and answers,}%%
  \ca{prp \Hebrew{עֵד} cf Ms \septuagint\ (\Greek{ἕως} = \Hebrew{עַד})}{it's been proposed to be ``witness/testimony.'' And compare a manuscript of the Septuagint which has ``until''}%%
  \ed{This word is utterly perplexing. The critical apparatus helps shed some light, but not much.}%%
  
  \pvca{even he who offers a sacrifice to the \textsc{Lord} of Hosts.}%%
  
  \pvaa{\verse{2:13} Here's another thing you should do:\footnotemark\footnotemark}%%
  \lit{You do this a second time:}%%
  \ca{prb add cf 11/12\super{c--c}, al dl sol \Hebrew{שׁנית}}{probably added, compare verses~11 and~12; others add only ``twice''}%%
  
  \pvbc{you cover, with tears,}{the throne of the \textsc{Lord},}{with weeping and groaning,}%%
  
  \pvbb{He no longer regards your offering\footnotemark}{or to receive\footnotemark\ with favor from your hand.}%%
  \lit{from where more face of the sacrifice}%%
  \alt{take}%%
  
  \pvac{\verse{2:14} You've asked, `Why?'}{Because the \textsc{Lord} has witnessed}{between you and the wife of your youth\footnotemark}%%
  \ed{Is ``the wife of your youth'' referring to a false god that was served in one's younger days?}%%
  
  \pvbb{that you've been unfaithful with her.}{Yet she is your companions, the wife of your covenant!}%%
  
  \pvaa{\verse{2:15} \footnotemark He didn't make one.\footnotemark\ The remainder of the Spirit was his. What's the one? He is seeking the posterity}%%
  \ca{prb add}{probably added [this line and the next line]}%%
  \ed{This could also be rendered, although a few too many gaps are being filled in for me to be comfortable with this, ``Did not one [God] make [us]?''}
  
  \pvca{of\footnotemark\ God. Be careful with your spirit.}%%
  \alt{from}%%
  
  \pvea{\footnotemark\footnotemark Don't treat the wife of your youth unfaithfully}%%
  \ca{exc hemist praecedens (vel compl)?}{has one (or have several) preceding hemistichs been dropped out?}%%
  \ed{The typesetting here is strange indeed: the previous line is a half stich, then this line starts on a new line but aligned with the preceding stich.}%%
  
  \pvac{\verse{2:16} because I hate divorce,''\footnotemark\footnotemark}{says the \textsc{Lord},}{the God of Israel.\footnotemark}%%
  \lit{sending away,''}%%
  \ed{Divorce, while ``hated'' by God was sometimes necessary, as was the case with Ezra and his foreign wife.}%%
  \ca{add?}{added? [``says the \textsc{Lord}, the God of Israel'']}%%
  
  \pvbb{``He will cover his garments with violence,''}{says the \textsc{Lord} of Hosts.}%%
  
  \pvca{``Take care of yourselves\footnotemark\ and don't deal unfaithfully.\footnotemark\footnotemark}%%
  \lit{``Be careful with your spirit, ``Take care within your spirit}%%
  \ie{with the wife of your covenant.}%%
  \ca{prb add cf 15b$\alpha$}{[this line] probably added, compare verse~15}%%
  
  \pvab{\verse{2:17} You've wearied the \textsc{Lord} with your words,}{yet you say, `How\footnotemark\ have we wearied Him?'\footnotemark}%%
  \lit{In what}%%
  \alt{You?'}%%
  
  \pvbb{In your saying, `Everyone who does evil}{is good in the eyes of the \textsc{Lord}~---}%%
  
  \pvbb{He takes pleasure in them!'}{Or in saying,\footnotemark\ `Where is the God of judgment?'\thinspace''\footnotemark}%%
  \ed{repeated}%%
  \ed{This whole idea here of ``sin vigorously because God will forgive'' is wearing to the Lord.}%%
\end{inparaenum}

  % \heading{3}{The Lord will send a messenger to prepare His way~--- His people will be refined and purified~--- men rob God by offering poor sacrifices~--- they call good evil and evil good~--- the Lord's remembers the righteous}

\begin{inparaenum}
  \pvab{\vn{3:1} ``I will send My messenger}{and He shall prepare the\footnotemark\ way before Me.}%%
  \fntlit{a}%%
  
  \pvbb{The Lord whom you seek}{shall suddenly enter His temple.\footnotemark}%%
  \fnted{reversed the order of these stichs for idiomaticy}%%
  
  \pvca{The angel of the covenant in whom you delight~---He cometh,'' says the \textsc{Lord} of Hosts.\footnotemark}%%
  \fntca{prb add}{[this line] probably added}%%
  
  \pvab{\vn{3:2} ``But who shall endure\footnotemark\ the day of His coming?}{Who shall be standing when He appears?}%%
  \fntalt{abide, live through}%%
  
  \pvbb{For He is like refiner's fire}{and fuller's lye.\footnotemark}%%
  \fntalt{soap; the primary ingredient being lye}%%
  
  % Verse 3 should not be indented.
  \verse{3:3} For He shall sit as a refiner and purifier of silver and He shall purify the children of Levi and purge them like gold and silver. And they shall be an offering unto the \textsc{Lord} in righteousness.%%
  \verse{3:4} The offering of Judah and Jerusalem shall be a sweet savor unto the \textsc{Lord} even as in days of old~--- as in previous years.\smallskip%%
  
  \pvab{\vn{3:5} I will draw near to you for judgment.}{I have been a hasteful\footnotemark\ witness}%%
  \fnthalot{xxxx}{xxxx}%%
  
  \pvbb{against the sorcerers and adulterers,}{against those who swear falsely,}%%
  
  \pvbb{and against those who monetarily extort their employees\footnotemark~---}{widows, orphans, those who turn away the resident aliens,\footnotemark}%%
  \fntlit{oppressors of their hireling's wages}%%
  \fnted{the most exposed (or least protected) groups of people}%%
  
  \pvbb{and those who don't fear Me,''}{says the \textsc{Lord} of Hosts.}%%
  
  \pvab{\vn{3:6} ``For I, the \textsc{Lord}, have not changed.}{And you, children of Jacob, have not been consumed.}%%
  
  \pvab{\vn{3:7} Even from the days of your fathers have you shied away\footnotemark\ from My statutes}{and not taken heed.\footnotemark}%%
  \fntalt{turned away/aside}%%
  \fntca{frt ins \Hebrew{מִשְׁמַרְתִּי} cf 14}{perhaps insert ``to my commands,'' compare verse~14}%%
  
  \pvbb{Turn your back on Me and I will turn My back on you,''}{says the \textsc{Lord} of Hosts.}%%
  
  \pvbb{``Yet you say,}{`How do we turn away?'\footnotemark}%%
  \fntca{add?}{[this line] added?}%%
  
  \pvab{\vn{3:8} Will a man deceive\footnotemark\footnotemark\ God?}{Because you have been deceiving Me.}%%
  \fnthalot{xxxx}{\textbf{deceive} \haref{Ma}{3}{8f} \haref{Pr}{22}{23}; or perhaps \textbf{rob}}%%
  \ed{The critical apparatus gives \Hebrew{עקב} (grasp by the heel, cheat) instead of \Hebrew{קבע} (deceive, rob) for all instances in this verse and verse~9. \textsc{halot} says that this is ``perhaps metathesis of \Hebrew{קבע} to avoid assonance of \Hebrew{יַעֲקֹב}.''}%%
  
  \pvbb{But you say, `How have we been deceiving you?'}{By tithes and offerings.\footnotemark}%%
  \fntca{frt l \Hebrew{וּבַתּרומה} \Hebrew{בַּמּעשׂר} cf \peshitta\targum\vulgate, sed cf 9\super{b--b}}{perhaps read ``in tithes and in offerings,'' compare the Peshitta, Targum, and Vulgate, but compare 9\super{b--b} [see following]}%%
  
  \pvac{\vn{3:9} You are cursed with a curse,}{and you rob Me,}{even this whole nation.\footnotemark}%%
  \fntca{add? prb l \Hebrew{כלו} \Hebrew{הֲגַם} et tr ad fin 8}{added? probably read ``Will they also all this?'' and transposed to the end of verse~8}%%
  
  \pvac{\vn{3:10} Bring all of the tithes}{into the storehouse:}{there is food in My house.}%%
  
  \pvbb{Please, try Me on this,''}{says the \textsc{Lord} of Hosts,}%%
  
  \pvbb{``If I will not open the windows of heaven for you}{and empty out upon you a blessing \textit{ad infinitum}.}%%
  
  \pvab{\vn{3:11} I will rebuke the devourer for your sake.}{He shall not destroy the fruit of your land,}%%
  
  \pvbb{neither shall the vine of your the field be barren,''}{says the \textsc{Lord} of Hosts.}%%
  
  \pvab{\vn{3:12} ``All of the gentile nations shall call you blessed}{because you shall be a desirable\footnotemark\ land,''}%%
  \fntalt{delightful}%%
  
  \pvaa{says the \textsc{Lord} of Hosts.}%%
  
  \pvab{\vn{3:13} ``Your words have been harsh\footnotemark\ against Me,''}{says the \textsc{Lord}.\footnotemark}%%
  \fntalt{strong}%%
  \ca{prb ins \Hebrew{צְבָאוֺת} cf \septuagint\superit{L}}{probably insert ``of Hosts,'' compare the Septuagint (textus Graecus ex recensione Luciani)}%%
  
  \pvbb{``Yet you say,}{`How have we spoken against You?'}%%
  
  \pvab{\vn{3:14} ``You've said, `It is vain to serve God.}{And what profit is there in keeping His charges}%%
  
  \pvbb{and walking mournfully\footnotemark}{before the \textsc{Lord} of Hosts?}%%
  \fntlit{walking like mourners}%%
  
  \pvab{\vn{3:15} We now declare}{the proud happy.\footnotemark}%%
  \fntalt{arrogant, proud, haughty, insolent}
  
  \pvbb{Moreover, those who work iniquity\footnotemark\ prosper;\footnotemark}{and also, those who tempt God escape.'\thinspace''}%%
  \fntalt{do wickedness}%%
  \fntlit{are built up, we build up}%%
  
  \pvac{\vn{3:16} Then those who feared the \textsc{Lord} spoke}{one to another,}{and the \textsc{Lord} observed and heard it.}%%
  
  \pvbb{And a scroll of remembrance was written before Him}{for those who reverenced the \textsc{Lord} and pondered\footnotemark\ on His name.}%%
  \fntalt{consider, reflect on}%%
  
  \pvac{\vn{3:17} ``They shall be Mine,''}{says the \textsc{Lord} of Hosts,}{``in the day I shall make them a peculiar\footnotemark\ treasure.}%%
  \fnted{see further in Appendix~\ref{app:peculiar}}%%
  
  \pvbb{And I shall show mercy on them, just like a man shows mercy}{on his son who serves him.}%%
  
  \pvab{\vn{3:18} You will return and differentiate between those who are righteous and those who are wicked~---}{between those who serve God and those who don't serve Him.''}%%
  
  \pvad{\vn{3:19} \footnotemark Because the day comes,}{burning as a furnace,\footnotemark}{when all the proud}{and all who work wickedness shall be stubble.\footnotemark}%%
  \fnted{In English translations, this is the beginning of chapter~4.}%%
  \fntca{\septuagint* + \Greek{καὶ φλέζει αὺτούς}}{the Septuagint (textus Graecus originalis) adds ``xxxx''}%%
  \fntalt{straw.}%%
  
  \pvbb{``The day is coming when He shall burn them up,''}{says the \textsc{Lord} of Hosts.}%%
  
  \pvbb{``It shall leave them neither}{root nor branch.}%%
  
  \pvab{\vn{3:20} ``Risen to you who fear My name,}{the Sun\footnotemark\footnotemark\ of Righteousness has risen with healing in His wings.}%%
  \fnted{In Egyptian iconography, there is an image of a beneficent sun god with wings. This is likely the same kind of imagery.}%%
  \fnted{We cannot ascribe word play (i.e., sun/son) here because it doesn't exist in Hebrew (i.e., \Hebrew{שֶׁמֶשׁ}/\Hebrew{בן}).}%%
  
  \pvbb{And you shall go forth and grow up}{as calves of the stall.}%%
  
  \pvab{\vn{3:21} You shall tread down the wicked for they shall be ashes}{under the soles of your feet}%%
  
  \pvbb{In the day that I shall do this,''}{says the \textsc{Lord} of Hosts.}%%
  
  \pvab{\vn{3:22} \footnotemark``Remember the Law of Moses, My servant,}{that I commanded him,}%%
  \fntca{22~add; \septuagint* tr~22 post~24}{verse~22 added; the Septuagint (textus Graecus originalis) transposes verse~22 after verse~24}%%
  
  \pvbb{In Horeb, for all Israel~---}{The judgments and statutes.}%%
  
  \pvab{\vn{3:23} \footnotemark I will send you}{Elijah the prophet}%%
  \fntca{23sq add}{verse~23 [and?] following add[ed/ition?]}%%
  
  \pvbb{Before the coming of the day of the \textsc{Lord},}{which is great and dreadful.\footnotemark}%%
  \fntalt{awe-inspiring, terrifying.}%%
  
  \pvab{\vn{3:24} \footnotemark He shall turn the fathers' hearts to the children,}{and the children's hearts to their father,}%%
  \fntca{23sq add}{verse~23 [and?] following add[ed/ition?]}%%
  
  \pvbb{otherwise I will come and smite}{the earth with a curse.''\footnotemark}%%
  \fntalt{destruction.''}%%
\end{inparaenum}

  %
  % ~\clearpage\thispagestyle{empty}
  %
  % \appendix
  % % Rework header and section title formatting.
  % \fancyhead[LE,RO]{APPENDIX}
  % \fancyhead[LO,RE]{\thepage}
  % \fancyhead[CH]{} % Removes book name from center head.
  % \titleformat{\chapter}{\large}{}{0em}{\uppercase{Appendix \thechapter~--- #1}}
  % % \titlespacing*{\chapter}{0pt}{0pt}{12pt}
  % \renewcommand*\thesection{\arabic{section}}
  % \titleformat{\section}{\scshape}{}{0em}{Excursus \thesection~--- #1}
  % \titleformat{\subsection}{}{}{0em}{\textsc{#1}}
  % \titleformat{\subsubsection}{}{}{0em}{\textsc{#1}}
  % \chapter{Additional notes}\thispagestyle{fancy}
  % \section{Notes on Psalm 110}\label{app:psalm-110}
Psalm 110 deals with the endowment of a king in ancient Israel. Kings and priests were in different orders of Priesthood, kings having a higher order. What is interesting about this is that the differences in the rites for becoming a king or a priest are not that different.

It is assumed that David wrote this psalm before he was crowned (i.e., when Saul was king). As a side note, one remarkable thing about David is that he was consistently obedient to the crown: he served the position, not the person.

The Aaronic, or Levitical, Priesthood was known anciently as \Hebrew{כְּהֻנָּה}.%
\footnote{from the root \Hebrew{כהנ}, meaning ``priest,'' and \Hebrew{כֹּהֵן}, meaning ``priesthood of the Levitical order.''}
This priesthood was for Aaron and his descendants: ``And Aaron and his sons shalt thou appoint that they may attend to their priest's office'' (Numbers 3:10, \textsc{darby}). However, the higher priesthood, \Hebrew{דִּבְרָה},\footnote{As found in Psalm 110:4 as \Hebrew{מַלְכִּי־צֶֽדֶק עַל־דִּבְרָתִי}} was given to prophets and kings. The assumption is that all who reigned in Israel had this latter order of priesthood.

This higher authority allowed those in its possession to enter the Holy Place and the Holy of Holies without particular regard to worthiness, as compared to those of the Aaronic order who had to be ritualistically and ethically clean, among other prerequisites. However, an interesting story is found in 2~Chronicles chapter~26 where king Uzziah (the ruler at the time of Isaiah) assumed that he had this authority and walked into the Holy of Holies.\footnote{While it is not explicitly stated in 2~Chronicles~26 that he entered the Holy of Holies, it is stated that he went to burn incense before the Lord~--- something that is done in the Holy of Holies.} He was struck with leprosy.\footnote{Most likely some skin disease (\Hebrew{צְרוּעָ}) and not necessarily leprosy. \textsc{halot} states that it is ``not leprosy: leucodermia \& related diseases.''} One possible explanation for this is that Uzziah was king of Judah, not Israel, and this may not have been sufficient for him to be of the higher order of priesthood.

To abstract \Hebrew{כֹּהֵן} we follow pe'ullah (\Hebrew{פְּעֻלָּה}) and get \Hebrew{כְהֻנָה}. In much the same way we can take \Hebrew{כֹּמֶר} (a non-Levitical (non-Israelite) priest\footnote{This is a non-attested form.}) to get the abstracted form \Hebrew{כְּמֹרָה} (Cumorah), ``non-Levitical priesthood'' (possibly meaning Melchizedek priesthood).

  % \clearpage
  % \section{Notes on Isaiah}\label{app:isaiah}
Most of the prophetic writings are either in judgment or hope for the people; they are not usually neutral. One theory postulates that there are two Isaiahs: one comprising Isaiah~1--39 and dealing with judgment, the other Isaiah~40--66 and dealing with hope (and possibly a third comprising just 56--66). However, there is plenty of hope in the first section and plenty of judgment in the second section which may controvert this theory.

\subsection{Isaiah 2:16}\label{app:isa-2-16}
Sailing in antiquity was a dangerous activity~--- it was taking your life in your hands. Sailors would go as close to the coast as possible so that if the ship breaks up (cf. Jonah 1:4) they could live through the incident by swimming to shore.

The \textit{Hebrew and Aramaic Lexicon of the Old Testament} by Koehler and Baumgartner defines \Hebrew{שְׂכִיָּה} as follows:
\begin{quote}
  Ug. \textit{\underline{t}kt} (Gordon \textit{Textbook} \S19:2680; Aistleitner 2862; Driver \textit{Myths}\super{2} 160a; cf. Fisher \textit{Parallels} 2: p. 8 entry 5) < Eg. \textit{\'skty} ship (Erman-G. 4:315), see Lambdin \textit{Loan Words} 154f; Ellenbogen \textit{Foreign Words} 154; cf. also Wildberger BK 10:94: \textbf{ship} \haref{Is}{2}{16}. \hadagger
\end{quote}

The following is from Ellenbogen's \textit{Foreign Words in the Old Testament}:
\begin{quotation}
  {\noindent\Hebrew{שכיות} -- Ships\hfill Egyptian -- \underline{\textit{\'sk.tj}} \includegraphics[scale=1.1]{images/egt-ship} (ship)}
  
  {\noindent Isa. 2:16 -}
  
  The LXX translation of the phrase \Hebrew{החמדה כל־שכיות ועל}, which runs: \Greek{ἐπὶ πᾶσαν θέαν πλοίων κάλλους}, is very remarkable; it incorporates both what seems to be the real meaning of the term \Hebrew{שכיות}, namely, \Greek{πλοιον} ``ship,'' and also what is required by the traditional understanding of the word, namely \Greek{θέᾶ} ``sight, view.'' The Vulgate renders \Hebrew{שכיות} by \underline{visus},\footnote{Prob. ``views''} and the Peshitta by \Syriac{ܕܘܩܐ} ``view.'' Rashi has a note saying that \Hebrew{שכיות} refers to palaces whose floors are paved with marble mosaics. This seems to be purely conjectural, and possibly based on the Targum's rendering of the term (\Hebrew{בירנתא} ``castle'').
  
  The Verses Isa. 2:13, 14, 15, and 17 contain parallelisms, so it would appear reasonable to assume that Verse 2:16 also contains a parallelism that would match the word \Hebrew{אניות} ``ships,'' so Budde-Begrich (\underline{ZATW} 49, p.198) are apparently right in seeing in Hebrew \Hebrew{שכיות} the borrowed Egyptian term \underline{\textit{\'sk.tj}}, a word which is attested from the days of the New Kingdom. [See Erman-Grapow, \underline{WB} IV, p.315.] The Egyptian derivation of \Hebrew{שכיות} was also accepted by Albright (\underline{Bertholet Festschrift}, p.5) who mentions that H.L. Ginsberg identifies Ugaritic \underline{\underline{t}}\underline{kt} with Hebrew \Hebrew{שכיות}.
\end{quotation}

Note that the Septuagint allows for the rendering found in the King James Version.

\subsection{Rabshakeh}\label{app:rabshakeh}
Rabshakeh was possibly an ex-Jew who was sent to Jerusalem because he spoke Hebrew. Some justification for this conclusion comes from the fact that he refers to the Lord by His ineffable name. abshakeh has a hatred within him that is common in people who have left their faith. In Isaiah~36:15, he gives an example of something that Hezekiah would say, but says it in a truly Hebrew manner.

  % \clearpage
  % \input{Appendix/aramaic-in-the-ot}
  % \clearpage
  % \section{Covenants in Antiquity}\label{app:covenants-in-antiquity}
\subsection{\Hebrew{ידע}~--- to know}
% Split into enumerate
The verb \Hebrew{ידע} has three senses in BH. First, it means ``to know (a fact).''; for instance, ``To know the time.'' Second, it is used with a sexual connotation (to have sexual relations). Lastly, it is used in a covenantal sense~--- to enter a covenant (or treaty) with someone. Examples of this usage include:
\begin{itemize}
  \item ``And there arose a new king over Egypt, who did not \emph{know} Joseph'' (Exodus 1:8, \textsc{darby}, emphasis added). In other words, a king came to succession who had not covenanted with Joseph: ``I didn't know him, so all bets are off.''
  \item ``Before I formed thee in the belly I \emph{knew} thee'' (Jeremiah 1:4, \textsc{darby}, emphasis added).
  \item ``[A]nd then will I avow unto them, I never \emph{knew} you'' (Matthew 7:23, \textsc{darby}, emphasis added).
  \item ``\dots if thou art God, wilt thou make thyself known unto me, and I will give away all my sins to \emph{know} thee'' (Alma 22:18, emphasis added).
\end{itemize}

BH does not have a sense of knowing a person, like the French \textit{conna\^\i tre}. The closest to that sense is \Hebrew{נכר} which means, in the hiphil, ``to be acquainted.''

% This should be in a different section.
In ancient Israelite marriages, covenants were made to God, \emph{not} to the other person. Therefore, the breaker of the covenant must answer to God.

\subsection{Oath taking syntax}
The syntax of oath taking:

``I will not give your grain any longer as food for your enemies'' is literally ``\emph{If} I give your food to your enemies \emph{and} [understood: you will kill me].''

``If I don't do this, may my throat be slit just as the throat of this animal.''

cf.\ Alma 46:22--24. ``Preserved'' is a Muslim, not an Israelite, tradition.

\subsection{Apodictic and casuistic law}
Exodus 20 and on is all part of a covenant making ceremony.

Apodictic law is the ``thou shalt not.'' Casuistic law contains situations in which things happen (e.g., if someone does such and such, then such will happen).

  % \clearpage
  % \input{Appendix/misc-hebrew}
  % \clearpage
  % \section{Tabernacles and temples}\label{app:tabernacle}
Numbers 10:11 contains the phrase \Hebrew{הָעֵדֻת מִשְׁכַּן} which is difficult to translate. The word \Hebrew{מִשְׁכַּן} is classically rendered \textit{Tabernacle}, but Koehler-Baumgartner says the following:
\begin{quote}
  \textbf{dwelling-place, home} of Y.\footnote{meaning ``the \lord''}
\end{quote}
It can also mean tomb, sanctuary (especially the central sanctuary of Israel while in the desert), or Tabernacle. The word \Hebrew{הָעֵדֻת} is classically rendered \textit{testimony}, but Koehler-Baumgartner says this about it:
\begin{quote}
  \textbf{warning signs, reminders, urgings}
\end{quote}
Making an idiomatic rendering of this proves difficult because the sense of ``the home of the \lord'' is important, but also stating that it is a home that is to serve as a reminder or an urging (most likely to be righteous).

In Numbers 12, the term \textit{tent} is interchangeable with \textit{Tabernacle}. It is also this way in most of the Pentateuch.

\subsection{Temple rituals in antiquity}\label{app:sins-transgressions}
In Leviticus, there are gradations of wrongdoing:
\begin{itemize}
  \item \Hebrew{פֶּשַׁע} intentional wrongdoing
  \item \Hebrew{חֵטְא} unintentional wrongdoing
\end{itemize}

The tabernacle complex also has gradations of purification. If the sin is unintentionally, or unknowingly committed, the sinner need only up until the entrance of the holy place (meaning they must purify at the laver and at the altar). In cases where the community has unintentionally sinned, the sacrifice must be made in the Holy Place. Finally, intentional sins (whether by an individual or the community) can only be redeemed by the high priest on the Day of Atonement in the Holy of Holies.

  % \clearpage
  % \section{Names of the Lord}\label{app:names-of-the-lord}
\subsection{The Tetragrammaton}
\subsubsection{Etymology}
To help prevent people from speaking the ineffable name of God, the Tetragrammaton (\Hebrew{יהוה}) was sometimes voweled with the same vowel points as \Hebrew{אֲדֹנָי} (Adonai). More commonly it is written \Hebrew{יְהוָה}.

The English transliteration is Jehovah, as follows:
\begin{center}
  \Hebrew{יהוה} > hwhy = yhwh > jhwh > jhvh > Jehovah
\end{center}
The \Hebrew{י} was substituted for a \textit{j} because \textit{y} is a weak phoneme. The \Hebrew{ו}, anciently pronounced as \textit{w}, was changed to the stronger \textit{v}. Thus did these changes make Jehovah out of the original Yahweh.

% xxxx: add information on voweling and how it's written in Hebrew with the same vowel pointings for whatever and how that's how we also say it in Anglish. I know this is in paragraph 1, but work on it.

\subsubsection{Sanctity}
Ezra is one of the founders of modern Judaism~--- an ethnic religion practiced by a monotheistic people, the Judahites. It is the only surviving ethnic religion in the world besides Hinduism. This religion observed the sanctity of the Tetragrammaton by not pronouncing it. This practice became so strong that individuals who pronounced it would be put to death. The Romans allowed the Jews to stone those who uttered the ineffable name. Christ may very well have pronounced this when He declared, ``Before Abraham was \textsc{I am}'' (John 8:58).

This practice continued through the 5\super{th} and 6\super{th} centuries~\textsc{a.d.} It is still considered sacrilegious by very observant Jews, so much so that even \Hebrew{אֱלוֺהִים} is pronounced by Hasidic Jews as \Hebrew{אֱלוֺקִים}. In writing, ``G-d'' or ``G\_{}d'' is substituted for ``God.'' However, this practice is not peculiar to Judaism as Parley P.\ Pratt would write ``G-d'' when quoting blasphemers.

When reading, observant Jews will say ``Adonai'' (master, master of the universe, master of an individual) or ``Hashem'' (lit., the name) in place of Yahweh.

Some Orthodox Jews will not write \Hebrew{יהוה} because things containing this cannot be burned, erased, or destroyed. For this reason there is a place in the synagogue known as the \Hebrew{גְּנִיזָה} (genizah, pl.\ genizot) which is used for the disposal of sacred writings. Unfortunately, many \Hebrew{גְּנִיזָוֺת} were destroyed in the Middle Ages by fire (either arson or accidental). Fortunately, the \Hebrew{גְּנִיזָה} in the ben-Asher synagogue in Cairo, Egypt (built in the 10\super{th} century \textsc{a.d.}) has never burned. The Damascus scroll, likely part of the Dead Sea Scrolls, was found there because some of the documents belonging to the Qumran community were left there.

Sometimes, especially in the Dead Sea Scrolls, the Tetragrammaton is rendered in the Paleo-Hebrew: \textphnc{hwhy}.

It is shocking how often the people of Israel, especially in the Pentateuch, blaspheme against the Lord and His servants while using the (according to them) ineffable name of God. Although the third commandment refers to swearing an oath, the higher law incorporates using His name with respect and reverence (cf.~D\&C\,107:4), something the rebellious Israelites clearly disobeyed, possibly helping justify their terrible punishments.

\subsection{Ahman}
Not much is known on this name. What follows is only preliminary work\footnote{From Professor Stephen D.\ Ricks in Heb~432R (Biblical Hebrew Syntax), Summer term 2013.} and is in no way meant to be authoritative.

Ahman, possibly related to \Hebrew{אֹמֵן}, meaning ``to be true,'' ``to be faithful,'' or ``to be realized.'' May also refer to ``faith'' or ``veracity.''

Could possibly be related to the Book of Mormon name \Hebrew{אֹמְנִי} (Omni) which means either ``faith in me'' or ``my faith,'' \Hebrew{אֹמְנִי} being either a subjective or objective genitive.

\subsection{Different vowelings of \textit{Adonai}}\label{app:adonai}
\Hebrew{לַאדֹנִי} is directed towards a human, \Hebrew{לַאדֹנָי} is directed to deity.

  % \clearpage
  % \input{Appendix/idioms}
  %
  % % Rework section formatting so "Excursus #" is not present.
  % \titleformat{\section}{\scshape}{}{0em}{#1}
  % \chapter{Dead Sea Scrolls}\thispagestyle{fancy}
  % \section{4Q521~--- Messianic Apocalypse}\label{dss:4q521}

\begin{inparaenum}
  {\noindent\verse{4q521:1} For the] heavens and the earth shall listen to His\ed{The Lord's Messiah?} Messiah,}%%
  
  {\noindent\verse{4q521:2} and everyone who is in them shall not stray\alt{turn} from the commandments of the holy ones.}%%
  
  {\noindent\verse{4q521:3} You who seek the Lord, strengthen yourselves in His service.}\ed{There is a \textit{vacat} here, indicating a space left empty by the scribe.}%% http://members.bib-arch.org/publication.asp?PubID=BSBA&Volume=18&Issue=6&ArticleID=15
  
  {\noindent\verse{4q521:4} Everyone who hopes in their heart, won't you find the Lord in this?}%%
  
  {\noindent\verse{4q521:5} For the Lord will choose\alt{consider} the pious and call the righteous by name.}%%
  
  {\noindent\verse{4q521:6} His spirit will rest on\alt{hover over} the humble;\alt{poor;} He will rejuvenate the faithful by His power.}%%
  
  {\noindent\verse{4q521:7} He shall glorify\alt{exalt} the pious on the throne of the kingdom of eternity.}%%
  
  {\noindent\verse{4q521:8} He shall free the prisoners, restore sight to the blind, straighten the be[nt\dots.(?)]}%%
  
  {\noindent\verse{4q521:9} For[ev]er I will cling to those who have hope, and in His loving-kindness He shall [\ ]}%%
  
  {\noindent\verse{4q521:10} The fruit[s?] of a good deed shall not be delayed for anyone.}%%
  
  {\noindent\verse{4q521:11} Glorious things which had never been done before, the Lord shall do just as He's s[aid\dots.(?)]}%%
  
  {\noindent\verse{4q521:12} For He shall heal the wounded,\lit{(probably mortally) pierced ones} He shall bring the dead to life,\ed{possibly denotes resurrection, but definitely denotes supernatural powers} He shall bring good tidings to the poor.}%%
  
  {\noindent\verse{4q521:13} He will provide suff[iciently] for the p[oo]r, He will guide the cast outs, and He will invite the hungry to a banquet.}%%
  
  {\noindent\verse{4q521:14} He [will lead the uprooted] and knowledge and sm[oke ?]}%%
  
  {\noindent\verse{4q521:15} I will(?) [}%%
\end{inparaenum}

  % % \clearpage
  %
  % % Let the book breath.
  % ~\clearpage\thispagestyle{empty}
  % ~\clearpage\thispagestyle{empty}
\end{document}
