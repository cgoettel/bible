% xelatex (use Makefile)
\documentclass[twoside,twocolumn,9pt]{extbook} % extbook allows for smaller font sizes.

% Packages that don't necessary belong elsewhere.
\usepackage{mathtools} % For underlining. This seems like a bad joke. There's got to be a better way to underline (especially one that doesn't involve loading a massive package).
\usepackage{textcomp} % For copyleft symbol.

% Page layout options
% \usepackage[paperwidth=4.6in, paperheight=6in, margin=1cm]{geometry} % Small size, like my personal scriptures.
\usepackage[paperwidth=5.75in, paperheight=8.5in, top=1in, bottom=1in, left=0.5in, right=0.5in]{geometry} % The size of Virginia Tufte's Syntax as Style
\setlength{\topmargin}{-0.7in} % Relative to 1in default
\setlength{\headsep}{0.1cm} % Distance from header to top of text

\usepackage{multicol} % For copyright page. Possibly for poetry.

% Hyphen penalty (higher number more greatly discourages hyphenation, up to 10000).
% 4000 seems to work fine because it only allows hyphenation in extreme conditions (and doesn't require having to use ~ all the time). 8000 was previously being used but anything with a badness over 5000 is ugly to the naked eye, so 4000 is now being used (5000 wasn't enough because it allowed things in the range 4800--5200).
% Never use 10000.
\sloppy\hyphenpenalty=4000

% %%%%%%%%%%%%%%%%%%
%     APPENDICES
% %%%%%%%%%%%%%%%%%%
\usepackage[toc,page]{appendix}

% %%%%%%%%%%%%%%%%%%%%%%%%%%%
%     CHAPTER and SECTION
% %%%%%%%%%%%%%%%%%%%%%%%%%%%
\usepackage[explicit]{titlesec}

% Options for reconfiguring book (\chapter{}) headings.
\titleformat{\chapter}[display]{\centering\normalfont\Huge}{}{0ex}{#1}

% Spacing around book (\chapter{}) headers.
\titlespacing*{\chapter}{0pt}{0pt}{7ex}

% Options for reconfiguring chapter (\section{}) headings.
\titleformat{\section}{\centering}{}{1em}{#1}

% Options for reconfiguring preface (\subsection*{}) headings.
\titleformat{\subsection}{\large\bfseries}{}{1em}{#1}

% Shorter way of declaring a chapter with both a Hebrew and an English name.
\newcommand\chap[2]{%
    \chapter[#1\hfill#2~~~]{#1}% ~ added for spacing in ToC.
    \renewcommand\chaptername{#2}%
    \thispagestyle{empty}% Forces no page number in cfoot.
}

% More changes happen within the actual document (below). This is because the Appendix needs different styles.

% %%%%%%%%%%%%%%%%%%%%%%%%
%     CHAPTER HEADINGS
% %%%%%%%%%%%%%%%%%%%%%%%%
\newcommand\heading[1]{%
    \vspace{-0.5em}%
    {\small\emph{#1.}}%
    \vspace{0.5em}%
}

% %%%%%%%%%%%%%%%%%
%     ENUMERATE
% %%%%%%%%%%%%%%%%%
% NOTE: Do not \usepackage{paralist} because it conflicts with options set for enumitem.

% Changes spacing inside enumerate environment. TODO: Change to center.
\usepackage{enumitem}
\setenumerate{leftmargin=0em, itemindent=2.2em, listparindent=-1em, nolistsep}
% itemindent = distance from left margin to start of label

% Center aligns number.
\SetLabelAlign{center}{\hfil#1\hfil}

% Changes enumerate label to be a number without a period following it.
\renewcommand\labelenumi{\arabic{enumi}}

% %%%%%%%%%%%%%
%     FONTS
% %%%%%%%%%%%%%
\usepackage{xltxtra}

% Egyptian hieroglyphs
\usepackage{graphicx}

% Greek
\newfontfamily{\sblg}[Script=Greek,Scale=1.2]{SBL Greek}
\newcommand\Greek[1]{{\sblg #1}}

% Hebrew
\newfontfamily{\sblh}[Script=Hebrew,Scale=1.2]{SBL Hebrew}
\newcommand\Hebrew[1]{{\sblh #1}}

% Hoefler text
% \usepackage{fontspec}
% \setromanfont[Mapping=tex-text]{Hoefler Text}
% \fontspec[Mapping=tex-text, Ligatures={Common, Rare, Historic}]{Hoefler Text}
% \newfontinstance\scshape[Letters=SmallCaps, Numbers=Uppercase]{Hoefler Text}

% Phoenician
\usepackage{phoenician}

% Syriac
\newfontfamily{\sbls}[Script=Syriac]{Estrangelo (V1.1)}
\newcommand\Syriac[1]{{\large\sbls #1}}

% %%%%%%%%%%%%%%%%%
%     FOOTNOTES
% %%%%%%%%%%%%%%%%%
% Resets footnote numbering for each page (perpage) and adds the symbol option to number footnotes according to verse.
% Allows footnotes to be separated by a comma (multiple).
% Hang and flushmargin remove the indentation.
\usepackage[symbol,perpage,multiple,hang,flushmargin]{footmisc}

% Makes footnotes lowercase, italic alpha instead of Arabic numerals.
\renewcommand{\thefootnote}{\textit\alph{footnote}}

% Force footnotes to stay on the same page and not bleed over. Long footnotes should be placed in the appendix.
\interfootnotelinepenalty=10000

% %%%%%%%%%%%%%%%%%%%%%%%%%
%     HEADER and FOOTER
% %%%%%%%%%%%%%%%%%%%%%%%%%
% Header options for dictionary style headers.
\usepackage{fancyhdr}
\pagestyle{fancy}
\usepackage{fix2col} % Fixes potential problems with marks on two column pages.
\usepackage{substr}

% Extracts book name and chapters from rightmark and leftmark.
\newcommand\bookname{\BeforeSubString{^}{\leftmark}}
\newcommand\rightchap{\BehindSubString{^}{\rightmark}}
\newcommand\leftchap{\BehindSubString{^}{\leftmark}}
\fancyhead[LE]{~\rightchap} % The "~" is for cushioning.
\fancyhead[RO]{\leftchap~}
\fancyhead[LO,RE]{~\thepage~}
\fancyhead[CE]{\bookname} % Puts book name (e.g., Genesis, Exodus) in center header on even pages.
\fancyhead[CO]{\chaptername} % Puts Hebrew book name in center header on odd pages.
\cfoot{} % Removes page number from center foot.

% %%%%%%%%%%%%%%%%%%%%%%%%%
%     TABLE OF CONTENTS
% %%%%%%%%%%%%%%%%%%%%%%%%%
\AtBeginDocument{\addtocontents{toc}{\protect\thispagestyle{empty}}} % Removes page numbering from TOC.
\usepackage[toc]{multitoc} % Two column TOC.
\setcounter{tocdepth}{0} % So TOC only displays chapters (books).
\def\numberline#1{} % Removes numbering from TOC but not from sections (chapters).

% %%%%%%%%%%%%%%%%%%%%%%%%%%%%%
%     VERSES and REFERENCES
% %%%%%%%%%%%%%%%%%%%%%%%%%%%%%
% Patch enumi counter to always start with 1.
\usepackage{etoolbox}
\newcounter{myc}[enumi]
\patchcmd{\enumi}{}{\setcounter{myc}{1}}{}{}

\renewcommand\verse[1]{%
    \markboth{#1}{#1}% Sets up header. Stores #1 as both \leftmark and \rightmark.
    \label{#1}% Reference name.
    \item% Print verse number.
}

% POETRY
% Note: Don't use smaller text because then the (entire) book of Isaiah will be tiny.
\newcommand\pverse[2]{%
    \markboth{#1}{#1}% Sets up header. Stores #1 as both \leftmark and \rightmark
    \label{#1}% Reference name.
    \item {\small #2}% Print verse number and contents.
}

% For referencing a verse.
\newcommand\vref[1]{\hyperref[#1]{#1}}

% %%%%%%%%%%%%
%     YEAR
% %%%%%%%%%%%%
\usepackage{datetime}
\newdateformat{justtheyear}{\THEYEAR}

% %%%%%%%%%%%%%%%%
%     HYPERREF
% %%%%%%%%%%%%%%%%
% hyperfootnotes must be turned to 'false' for commas to appear in instances of multiple footnotes.
% \usepackage[xetex, hyperfootnotes=false, colorlinks=false, pdfborder={0 0 0}, pdftitle={The Holy Bible}, pdfauthor={Colby Goettel}, pdfproducer={pdfLaTeX}, pagebackref, pdfpagemode=None]{hyperref}

% %%%%%%%%%%%%%%%%
%     DOCUMENT
% %%%%%%%%%%%%%%%%
\begin{document}
    \frontmatter
    \thispagestyle{empty}
\begin{titlepage}
    \begin{center}
        {~}
        
        \vspace{10pt}
        
        {\large T$\:$H$\:$E}
        
        \vspace{10pt}
        
        {\Huge \textbf{\textsc{H$\;\:$O$\;\:$L$\;\:$Y$\;\:$ $\;\:$B$\;\:$I$\;\:$B$\;\:$L$\;\:$E}}}
        
        \vfill
    \end{center}
\end{titlepage}

    \clearpage\thispagestyle{empty}
    \begin{titlepage}
    \begin{center}
        {T$\:$H$\:$E}
        
        \vspace{10pt}
        
        {\Huge \textbf{\textsc{H$\;\:$O$\;\:$L$\;\:$Y$\;\:$ $\;\:$B$\;\:$I$\;\:$B$\;\:$L$\;\:$E}}}
        
        \vspace{35pt}
        
        {C$\,$O$\,$N$\,$T$\,$A$\,$I$\,$N$\,$I$\,$N$\,$G$\,$ $\,$T$\,$H$\,$E}
        
        \vspace{10pt}
        
        {\huge Old Testament}
        
        \vspace{100pt}
        
        {Translated from the}
        
        \vspace{5pt}
        
        {\large \emph{Biblia Hebraica Stuttgartensia}}
        
        \vspace{5pt}
        
        {with inspiration drawn from}
        
        \vspace{5pt}
        
        {\emph{Darby English}, \emph{Louis Segond}}
        
        \vspace{5pt}
        
        {and \emph{Young's Literal Translation}}
        
        \vfill
        
        {\large Colby Goettel}
        
        \vspace{2.5pt}
        
        {\small Provo, UT}
    \end{center}
\end{titlepage}

    \clearpage\thispagestyle{empty}
    {\small
\vfil

{\noindent\textbf{\textcopyleft\thinspace\justtheyear\today\ Colby Goettel}}

\vfil

{\noindent This project is open source and hosted on GitHub:}
\begin{center}
    \texttt{https://github.com/cgoettel/bible/}
\end{center}

\vfil

This file is part of Colby Goettel's translation of the \emph{Biblia Hebraica Stuttgartensia} (``this project'').

This project is free software: you can redistribute it and/or modify it under the terms of the GNU General Public License as published by the Free Software Foundation, either version 3 of the License, or (at your option) any later version.

This project is distributed in the hope that it will be useful, but WITHOUT ANY WARRANTY; without even the implied warranty of MERCHANTABILITY or FITNESS FOR A PARTICULAR PURPOSE.  See the GNU General Public License for more details.

You should have received a copy of the GNU General Public License along with this project. If not, see \texttt{http://www.gnu.org/licenses/}.

\vfil

Typeset in \XeTeX

10/12 \texttimes\ 23
}

    \clearpage\thispagestyle{empty}
    ~\bigskip

~\bigskip

{\noindent To Dad}

\medskip

{\noindent \textit{for inspiring my love of the Old Testament}}

    \clearpage\thispagestyle{empty}
    \onecolumn
\chapter*{Preface}
\subsection*{The Bible}
The Bible is the word of God so far as it is translated correctly. Personally, the belief that the Bible is infallible shows a clear lack of understanding and education since the Bible is rife with poor translations, mistranslations, and even typos. However, having the Bible in as good of condition as we have it today is a miracle~--- one for which I thank the Lord.

\subsection*{Translation philosophy}
Growing up with the King James Version of the Holy Bible was a two-edged sword: on one hand, it's a beautifully written and well-accepted version; on the other hand, it's a poetic translation. Personally, non-idiomatic translations show a lack of understanding on the translator's part as to how language works. Poetic translations are difficult to render, but read beautifully; however, they are non-intuitive and therefore not properly suited for most audiences. Therefore, this translation is a rather idiomatic translation with few liberties taken.

\subsection*{The Tetragrammaton}
The Tetragrammaton (lit., a word having four letters) is the holy name of God, written \Hebrew{יהוה}. In Orthodox Hebrew culture it is unlawful for this word to be uttered by man but once a year by the High Priest on the Day of Atonement in the Holy of Holies. Traditionally, the Tetragrammaton is rendered \textsc{Lord} (in small caps). This tradition has been adhered to in this edition except in the case of \Hebrew{יְהֹוָה אֲדֹנָי}\footnote{Ketiv. Qere ``adonai elohim.''} where it is usually rendered as ``the Lord \textsc{God}.''\footnote{To avoid rendering it as ``Lord \textsc{Lord}.''} See further in Appendix~\ref{app:names-of-the-lord}.

\subsection*{Textual basis}
This text was translated from the \emph{Biblia Hebraica Stuttgartensia}. Inspiration for this translation was taken from the Darby English Bible and Young's Literal Translation. The lexicons used were \emph{The Brown-Driver-Briggs Hebrew and English Lexicon}, Holladay's \emph{A Concise Hebrew and Aramaic Lexicon of the Old Testament}, and Koehler and Baumgartner's \emph{Hebrew and Aramaic Lexicon of the Old Testament}.

\subsection*{Footnotes and appendix}
Footnotes are used to show alternate renderings and to provide historical, symbolical, and other, expository notes. An appendix appears in the back of the book and contains notes too long for inclusion in footnotes.

\subsection*{Abbreviations}
alt. = alternatively

lit. = literally % Do this as a table maybe? Check how the BHS does it.

pl. = plural

BH = Biblical Hebrew

Fr. = French

Aram. = Aramaic

Sp. = Spanish
\twocolumn

    {\pagestyle{empty}\cleardoublepage}
    \tableofcontents
    \clearpage\thispagestyle{empty}
    \mainmatter
    % openrightfalse forces TeX to not generate a blank page between chapters.
    \makeatletter\@openrightfalse
    \chap{Genesis}{\Hebrew{בראשית}}
    \section{1}\label{Genesis 1}
\heading{God creates the world~--- the various acts of the creation enumerated~--- man and woman created in God's image~--- dominion of the earth given to man}
\begin{enumerate}[align=center]
    \verse{1:1} In the beginning, God\footnote{It is not ``the Gods'' because every verb is conjugated for the third masculine singular, not plural. \Hebrew{אֱלוֺהִים} is the plural of majesty for God. Theologically, Christ created the Universe under the direction of the Father. Although He had help throughout the planning and construction phases, the honor and glory go to Him and the Father solely, not the rest of the Gods that assisted.} created\footnote{This verb, \Hebrew{בּרא}, means \emph{to create}. It does not carry with it the notion of \emph{ex nihilo} creation, but rather to organize. This can only be done by Deity~--- mortals cannot \Hebrew{בּרא}.} the Heavens and Earth.%%
    \verse{1:2} The earth was formless and void~--- darkness moved upon the face of the deep, and the Spirit of God moved upon the face of the waters.%%
    \verse{1:3} God said, ``Let there be light!'' And there was light.%%
    \verse{1:4} And God saw the light that\footnote{for} it was good, so God divided the light from the darkness.%%
    \verse{1:5} And God called the light, Day; and the darkness, Night. And there was an evening and a morning: the first day.%%
    \verse{1:6} And God said, ``Let there be an expanse in the midst of the waters: let it separate the waters.''\footnote{lit., the waters from the waters.}%%
    \verse{1:7} So God made the expanse. And it separated between the waters which are under the expanse and the waters which are above the expanse~--- and thus it was.%%
    \verse{1:8} And God called the expanse, Heaven. And there was an evening and a morning: the second day.%%
    \verse{1:9} God said, "Collect the waters under Heaven unto one place, and let the dry land be appear\footnote{seen}~--- and thus it was.%%
    \verse{1:10} And God called the dry land, Earth; and the collection of waters He called, Seas.%%
    \verse{1:11} God said, ``Let Earth yield tender grass, seed producing herbs, and fruit trees yielding fruit after their kind (the seed of which is in them) on Earth'': and thus it was.%%
    \verse{1:12} So Earth brought forth grass, seed producing herbs after its kind, and trees yielding fruit (the seed being\footnote{which} in them) after their kind~--- and God saw that it was good.%%
    \verse{1:13} And there was an evening and a morning: the third day.%%
    \verse{1:14} God said, "Let there be lights in the expanse of Heaven to separate\footnote{divide} the day from the night. Let them be for signs and for seasons, for days and for years,%%
    \verse{1:15} for\footnote{let them be for} lights in the expanse of Heaven to illuminate\footnote{give light to} Earth": and thus it was.%%
    \verse{1:16} So God made the two great lights: the greater\footnote{great} light to rule the day and the lesser\footnote{small} light (and the stars) to rule the night;%%
    \verse{1:17} and God placed them in the expanse of Heaven to illuminate Earth,%%
    \verse{1:18} to rule during the day and night, and to separate the light from the darkness~--- and God saw that it was good.%%
    \verse{1:19} And there was an evening and a morning: the fourth day.%%
    \verse{1:20} God said, ``Let the waters teem with life\footnote{teeming, living creatures} and let fowls fly on the earth and before the Heavens.''%%
    \verse{1:21} God created the great sea monsters and every living, teeming creature\footnote{lit., soul} which are innumerable in the waters after their kind and all the winged birds after their kind. And God saw that it was good.%%
    \verse{1:22} God blessed them, saying, ``Be fruitful and multiply. Fill the waters in the sea and let the birds multiply in the earth.''%%
    \verse{1:23} And there was an evening and a morning: the fifth day.%%
    \verse{1:24} God said, ``Let living souls come forth from the earth\footnote{Let the earth bring forth living souls} after their kind, wild animals,\footnote{cattle, animals} reptiles,\footnote{small animals, creeping things} and the wild, untamed animals\footnote{\Hebrew{חַיָּה} rarely means a single animal. It means ``animals, untamed animals, water or land animals, or wild, predatory animals.''} of the earth, after their kind.'' And thus it was.%%
    \verse{1:25} God made the wild, untamed animals of the earth after their kind, the wild animals after their kind, the ground reptiles after their kind. And God saw that it was good.%%
    \verse{1:26} God said, ``Let Us make man in Our image and according to Our likeness. Give them dominion over the fish of the sea, the birds in the sky, the wild animals~--- over the whole earth. And give them dominion\footnote{Verb repeated for idiomatic rendering.} over all the reptiles which creep upon the earth.''%%
    \verse{1:27} God created man in His image. In the image of God created He him. Male and female created He them.%%
    \verse{1:28} God blessed them and He\footnote{lit., God} said to them, ``Be fruitful. Multiply. Replenish the earth. Subdue\footnote{subjugate} it. Have dominion over the fish of the sea, the birds in the sky, and on all life that moves on the earth.''%%
    \verse{1:29} And God said, ``Look, I have given you every seed-bearing herb in the whole world and every tree which has tree-producing seeds. These shall be your food.''\footnote{lit., These shall be food to you.}%%
    \verse{1:30} Every wild animal on the earth, every bird in the sky, all the reptiles in the world~--- in which is a living soul~--- and every green herb: these shall be for food.\footnote{lit., for food} And thus it was.%%
    \verse{1:31} God saw everything that He had made and it was good, very good. And there was an evening and a morning: the sixth day.%%
\end{enumerate}

    % Genesis 10:25 --- Peleg and the countries being split up probably refers to the nations and not the tectonic plates.
    \section{Genesis 17}\label{Genesis 17}
\heading{The Lord commands Abram to be perfect~--- continuation of Abrahamic covenant~--- name changed to Abraham~--- Canaan given to Abraham~--- covenant of circumcision set forth~--- Sarai changed to Sarah~--- Isaac promised~--- covenant to continue through him~--- Abraham and his household are circumcised}
\begin{enumerate}[align=center]
    \verse{Genesis^17:1} Abraham was 99 years old when the \textsc{Lord} appeared to him. He said, ``I am God Almighty\footnote{Omnipotent~--- pl. for violence XXXX=[check unabridged Kohler-Baumgartner]}~--- walk before Me and be perfect.%
    \verse{Genesis^17:2} I will give\footnote{set up, establish} My covenant between us and I will greatly multiply you.''%
    \verse{Genesis^17:3} Abraham fell on his face as God spoke with him, saying,%
    \verse{Genesis^17:4} ``Pay attention. My covenant is with you~--- you shall be like a father of a multitude of nations.%
    \verse{Genesis^17:5} You shall no longer be called Abram, but rather Abraham, for I have made you a father of a multitude of nations.%
    \verse{Genesis^17:6} You shall be exceedingly fruitful. You shall become many nations. Kings will be among your posterity.\footnote{lit., Kings will come from you.}%
    \verse{Genesis^17:7} I will establish a covenant between us and also between your offspring.\footnote{progeny} It shall be an everlasting covenant: to be the God of you and your children.%
    \verse{Genesis^17:8} I will give you and your descendants\footnote{lit, seed after thee} the land of your sojournings and the land of Canaan as an everlasting possession for I have become their God.''%
    \verse{Genesis^17:9} God said to Abraham, ``You and the generations that follow shall honor this covenant.%
    \verse{Genesis^17:10} This is My covenant, between Me and you and your progeny,\footnote{lit., your seed after you} that you shall observe carefully: every male shall be circumcised.''%
    \verse{Genesis^17:11} You shall circumcise the flesh of your foreskin as a sign of the covenant between Me and you.%
    \verse{Genesis^17:12} Every male in your generations shall be circumcised when they are eight days old: those born in a house and those, not your offspring (the children of a stranger), that are bought with money.%
    \verse{Genesis^17:13} Those born in a house and those bought with your money shall certainly be circumcised. My covenant shall be an everlasting covenant in your flesh.%
    \verse{Genesis^17:14} The uncircumcised male, the flesh of whose foreskin is uncircumcised, shall be cut off\footnote{Great word choice} from My people: he has broken My covenant.%
    \verse{Genesis^17:15} God said to Abraham, ``Your wife will no longer be known as Sarai, but Sarah.%
    \verse{Genesis^17:16} I will bless her. Additionally, I will give you a son from her. I will bless her that she shall become a nation~--- people's kings will be among her posterity.''%
    \verse{Genesis^17:17} And Abraham fell on his face and laughed. He said in his heart, ``Shall a son be born to a hundred year old man? And shall a ninety year old woman bear him?''\footnote{More lit., Shall one be born to an hundred year old? Shall a daughter who is ninety bear?}%
    \verse{Genesis^17:18} Abraham said to God, ``O that Ishmael might live in Your presence!''%
    \verse{Genesis^17:19} %
    \verse{Genesis^17:20} %
    \verse{Genesis^17:21} %
    \verse{Genesis^17:22} %
    \verse{Genesis^17:23} %
    \verse{Genesis^17:24} %
    \verse{Genesis^17:25} %
    \verse{Genesis^17:26} %
    \verse{Genesis^17:27} %
\end{enumerate}

    \section{Genesis 22}\label{Genesis 22}
\heading{Abraham commanded to sacrifice his only son, Isaac~--- Abraham and Isaac both submit to God's will~--- continuation of Abrahamic covenant~--- Rebekah born to Bethuel}
\begin{enumerate}
    \verse{Genesis^22:1} After these things, God tested Abraham. He said to him, ``Abraham'' and Abraham replied, ``Yes?''\footnote{He says ``\Hebrew{הִנֵּֽנִי}'' which means anything from ``Behold!'' to ``I am here'' to ``Pay attention.'' It's a pretty all inclusive word with ``Yes?'' being an appropriate, idiomatic response.}%
    \verse{Genesis^22:2} He said, ``Please take your son Isaac~--- your only son~--- whom you love,\footnote{This is poignantly repetitive.} and get you yonder\footnote{lit., ``go for thyself''} to the land of Moriah. You shall raise him up as a burnt offering upon one of the mountains which I shall tell you.''%
    \verse{Genesis^22:3} So Abraham rose early in the morning and saddled his donkey. He took two of his young servants and his son Isaac with him. He chopped up some wood for the burnt offering, rose, and went to the place where God told him.%
    \verse{Genesis^22:4} On the third day, Abraham looked up\footnote{lifted up his eyes} and saw the place afar off.\footnote{``from afar,'' but this is written from his perspective}%
    \verse{Genesis^22:5} Abraham said to his young servants, ``Stay here with the donkey while the boy and I go off, worship, and return.''%
    \verse{Genesis^22:6} Abraham took the wood for the burnt offering and had his son Isaac carry it.\footnote{lit, placed it on his son Isaac} He took some fire, a knife, and the both of them went off together.%
    \verse{Genesis^22:7} Isaac said to Abraham, ``Dad?'' and he responded, ``Yes, my son?'' \footnote{``And Isaac said'' has been removed to help the flow.}``I see the fire and the wood, but where is the lamb for the burnt offering?''%
    \verse{Genesis^22:8} Abraham replied, ``My son, God Himself will provide a lamb for the burnt offering'' and they went on together.%
    \verse{Genesis^22:9} They came to the place which God had before told them and Abraham built an altar and arranged the wood. Then he bound Isaac and placed him on the wood on the altar.%
    \verse{Genesis^22:10} And Abraham took the knife in his hand to slay his son.%
    \verse{Genesis^22:11} The messenger of the \textsc{Lord} called to him from the heavens and said, ``Abraham! Abraham!'' and he said, ``Yes?''%
    \verse{Genesis^22:12} \footnote{``He said''}``Don't slay\footnote{lit., put forth your hand} the boy, neither do anything to him because now I know that you fear God~--- you have not withheld your son~--- your only son~--- from me.''%
    \verse{Genesis^22:13} %
    \verse{Genesis^22:14} %
    \verse{Genesis^22:15} %
    \verse{Genesis^22:16} ``%
    \verse{Genesis^22:17} I will richly bless. As the stars of the heavens and as the sand of the seashore will I greatly multiply your posterity. Your descendants shall possess the gate of their enemies.%
    \verse{Genesis^22:18} Because you have hearkened to my words, through your posterity shall all the nations of the earth be blessed.''%
    \verse{Genesis^22:19} %
    \verse{Genesis^22:20} %
    \verse{Genesis^22:21} %
    \verse{Genesis^22:22} %
    \verse{Genesis^22:23} %
    \verse{Genesis^22:24} %
\end{enumerate}

    % \heading{31}{xxxx}

\begin{inparaenum}
  \verse{31:1} %%
  \verse{31:2} %%
  \verse{31:3} %%
  \verse{31:4} %%
  \verse{31:5} %%
  \verse{31:6} %%
  \verse{31:7} %%
  \verse{31:8} %%
  \verse{31:9} %%
  \verse{31:10} %%
  \verse{31:11} %%
  \verse{31:12} %%
  \verse{31:13} %%
  \verse{31:14} %%
  \verse{31:15} %%
  \verse{31:16} %%
  \verse{31:17} %%
  \verse{31:18} %%
  \verse{31:19} %%
  \verse{31:20} %%
  \verse{31:21} %%
  \verse{31:22} %%
  \verse{31:23} %%
  \verse{31:24} %%
  \verse{31:25} %%
  \verse{31:26} %%
  \verse{31:27} %%
  \verse{31:28} %%
  \verse{31:29} %%
  \verse{31:30} %%
  \verse{31:31} %%
  \verse{31:32} %%
  \verse{31:33} %%
  \verse{31:34} %%
  \verse{31:35} %%
  \verse{31:36} %%
  \verse{31:37} %%
  \verse{31:38} %%
  \verse{31:39} %%
  \verse{31:40} %%
  \verse{31:41} %%
  \verse{31:42} %%
  \verse{31:43} %%
  \verse{31:44} %%
  \verse{31:45} %%
  \verse{31:46} %%
  \verse{31:47} \dots Jegar-sahadutha\ed{Aramaic (\Hebrew{שָׂהֲדוּתָא יְגַר}): keep of storms of witnessing}\dots%%
  \verse{31:48} %%
  \verse{31:49} %%
  \verse{31:50} %%
  \verse{31:51} %%
  \verse{31:52} %%
  \verse{31:53} %%
  \verse{31:54} %%
  \verse{31:55} %%
\end{inparaenum}

    % Whenever you get to the coat of many colors: \Hebrew{כְּתֹבֶת פּסִּים} \Greek{χιτῶνα ποικίλον} is rendered as \emph{coat of many colors}. Rabbis tell us that this is more probably a long sleeve garment with markings. Additionally, this is a \emph{hapax legomenon}.
    % 

\heading{49}{XXXX}

\begin{enumerate*}[mode=unboxed]
    \verse{49:1}{``}%%
    \verse{49:2}{}%%
    \verse{49:3}{}%%
    \verse{49:4}{}%%
    \verse{49:5}{}%%
    \verse{49:6}{}%%
    \verse{49:7}{}%%
    \verse{49:8}{~~To Judah~--- your brothers will praise you. Your hand will be on your enemies' necks.
    
    ~Your siblings}%%
    \verse{49:9}{}%%
    \verse{49:10}{}%%
    \verse{49:11}{}%%
    \verse{49:12}{}%%
    \verse{49:13}{}%%
    \verse{49:14}{}%%
    \verse{49:15}{}%%
    \verse{49:16}{}%%
    \verse{49:17}{}%%
    \verse{49:18}{}%%
    \verse{49:19}{}%%
    \verse{49:20}{}%%
    \verse{49:21}{}%%
    \verse{49:22}{}%%
    \verse{49:23}{}%%
    \verse{49:24}{}%%
    \verse{49:25}{}%%
    \verse{49:26}{}%%
    \verse{49:27}{}%%
    \verse{49:28}{}%%
    \verse{49:29}{}%%
    \verse{49:30}{}%%
    \verse{49:31}{}%%
    \verse{49:32}{}%%
    \verse{49:33}{}%%
\end{enumerate*}

    \chap{Exodus}{\Hebrew{שמות}}
    \section{Exodus 19}\label{exodus:19}
\heading{Israel to be a peculiar treasure, a kingdom of priests, a holy nation~--- they are sanctified~--- the Lord appears in Sinai}
\begin{enumerate}[align=center]
    \verse{Exodus^19:1} In the third month since the departure of the children of Israel from the land of Egypt,\footnote{in that day} they came into the Sinai wilderness.%
    \verse{Exodus^19:2} They left Rephidim, came into the Sinai wilderness, and camped there\footnote{in the Sinai wilderness.} before the mountain.%
    \verse{Exodus^19:3} And Moses went up to God, and the \textsc{Lord} called unto him from the mountain, saying, "Thus shalt thou say unto the children of Jacob, and tell unto the sons of Israel:%
    \verse{Exodus^19:4} XXXX%
    \verse{Exodus^19:5} And now, if thou shalt truly obey me and keep\footnote{observe} my commandments, thou shalt be\footnote{to me} a treasure{Refers to a temple treasure. This has covenantal implications.} from among all people\footnote{above all people} for all the earth is mine.%
    \verse{Exodus^19:6} And thou\footnote{\emph{pl.}} shalt be\footnote{to me} a kingdom of priests and an holy nation. These words shalt thou say unto the children of Israel."%
    \verse{Exodus^19:7} And Moses entered in\footnote{i.e., to the camp} and called unto the elders of the people and laid before them all these things that the \textsc{Lord} had commanded him.%%
    \verse{Exodus^19:8} Then all the people answered together, saying, ``All that the \textsc{Lord} hath said we will do.'' So Moses returned what the people said\footnote{the words of the people} to the \textsc{Lord}.%
    \verse{Exodus^19:9} The \textsc{Lord} said unto Moses, "I will come unto thee in a thick cloud so that the people may hear my words when I shall speak with thee. And they shall believe thee forever. Then Moses told the words of the people unto the \textsc{Lord}.%
    \verse{Exodus^19:10} And the \textsc{Lord} said unto Moses, "Go to the people and sanctify them today and tomorrow. Let them wash their clothes,%
    \verse{Exodus^19:11} for on the third day, the \textsc{Lord} will descend in the eyes of all the people before Mount Sinai.%
    \verse{Exodus^19:12} And thou shalt set bounds around the people XXXX to say, "Hearken to them XXXX in the mountain and anyone touching its edge\footnote{i.e., of the mountain} will surely be put to death.%
    \verse{Exodus^19:13} A hand will not touch it\footnote{the mountain} or he\footnote{whomever touches it} will surely be stoned or shot:\footnote{with arrows} whether beast or man, it shall not live while the trumpet sounds\footnote{continues} as they approach the mountain.%
    \verse{Exodus^19:14} So Moses went down from the mountain to the people and sanctified\footnote{consecrated, made holy} the people. And they washed their clothes.\footnote{As a way of preparing themselves to go up into the mountain.}%
    \verse{Exodus^19:15} And he said unto the people, ``Prepare yourselves: for three days do not approach a woman.''%
    \verse{Exodus^19:16} On the third morning, when it was morning, there were noises and flashes of lightning, a thick cloud over the mountain, and a very powerful sound of a trumpet:\footnote{One used for ritualistic purposes.} and all the people in the camp were afraid.%
    \verse{Exodus^19:17} Moses brought the people out from the camp to meet God and they stationed themselves at the base of the mountain.%
    \verse{Exodus^19:18} And Mount Sinai, all of it, was smoking because the \textsc{Lord} descended on it in fire. And the smoke went up like the smoke of a furnace, and the whole mount shook.%
    \verse{Exodus^19:19} The sound of the trumpet grew continually stronger\footnote{more and more powerful} while Moses spoke and God answered him.\footnote{with a voice}%
    \verse{Exodus^19:20} And the \textsc{Lord} descended upon the summit of Mount Sinai and\footnote{\dots the \textsc{Lord}\dots} called unto Moses.\footnote{at the summit} Moses went up.%
    \verse{Exodus^19:21} The \textsc{Lord} said unto Moses, ``Go down and solemnly charge the people lest they break through to see the \textsc{Lord} and many of them perish.''\footnote{Not because of some great iniquity, merely because of the glory of the \textsc{Lord} and His inherent power.}%
    \verse{Exodus^19:22} And also the priests... XXXX%
    \verse{Exodus^19:23} Moses said to the \textsc{Lord}, ``The people will not be able to come up to Mount Sinai because thou hast charged us, saying, `Make a border around the mount and sanctify it.'$\,$''%
    \verse{Exodus^19:24} The \textsc{Lord} saith unto him, ``Go down — you and Aaron with you — and the priests will come up. Let not the people break through to go up to the \textsc{Lord} lest His glory break out against them.''%
    \verse{Exodus^19:25} So Moses went down to the people and he spoke unto them.%
\end{enumerate}

    \heading{20}{The Decalogue~--- Israel commanded to bear witness that the Lord has spoken~--- altars of unhewn stone are to be built~--- sacrifices performed thereon}

\begin{inparaenum}
    \verse{20:1} And God spake all these words unto them, saying,%%
    
    \verse{20:2} \textsc{Preface.} ``I am the \textsc{Lord} your God who brought you out of the land of Egypt, from the house of captivity.\footnote{servitude, bondage, slavery}%%
    \verse{20:3} \textsc{i.}\footnote{There are varied approaches to numbering the commandments. The Philonic tradition is used here.} Never\footnote{The Hebrew negator \Hebrew{לוֹא} is used here. It is used when someone in authority is speaking to an inferior. When Moses speaks, he uses \Hebrew{אַל}~--- a word that is spoken between equals.} shalt thou have other gods besides me.\footnote{lit., my face. The Greek rendering is used herein.}\footnote{In the BHS there is no \textit{sof pasuq} (\Hebrew{׃}). This could possibly be used to argue the Philonic tradition.}%%
    \verse{20:4} \textsc{ii.} Never shalt thou make for yourselves graven images,\footnote{idols} neither any image that is in the heavens above, nor in the earth, nor beneath the earth, nor in the waters beneath the earth.%%
    \verse{20:5} And thou shalt not bow down to them, neither shalt thou worship them: for I, the \textsc{Lord} your God, am a jealous God and will seek retribution unto the third and fourth generation of them that hate me,%%
    \verse{20:6} but showing kindness\footnote{keeping my covenant} with\footnote{unto} those who love me\footnote{lit., my lovers} and\footnote{to those who} keep my commandments.%%
    
    \verse{20:7} \textsc{iii.} Never shalt thou use\footnote{take, lift up} the name of the \textsc{Lord} thy God with vain intent\footnote{in vain, with vanity, to/with no good purpose} for the \textsc{Lord} will not hold him innocent\footnote{guiltless} who uses His name with vain intent.\footnote{The real meaning here is to not take an oath in the name of God and not intend to keep it.}%%
    
    \verse{20:8} \textsc{iv.} Remember the Sabbath day to sanctify it.\footnote{consecrate, make it holy. The notion of \emph{making} the Sabbath day holy is more powerful than merely \emph{keeping} it holy for the responsibility then rests upon us to be an holy nation.}%%
    \verse{20:9} Six days shalt thou labor and do all thy work,%%
    \verse{20:10} but the seventh day, the Sabbath of the \textsc{Lord} thy God, never shalt thou do any work: neither thee, nor thy son, nor thy daughter, nor thy male or female servant,\footnote{lit., nor his/thy manservant, nor his/thy maidservant} nor thy beast, nor thy stranger that is within thy gates:%%
    \verse{20:11} for it took six days for the \textsc{Lord} to make the heavens and the earth and all that is upon the face thereof, and on the seventh day he rested. Therefore, the \textsc{Lord} blessed the Sabbath day and consecrated it.%%
    
    \verse{20:12} \textsc{v.} Take thy father and thy mother seriously\footnote{make their words heavy, honor them} so that thy days may be lengthened upon the land the \textsc{Lord} thy God giveth\footnote{Referring to the Promised Land that they have yet to inherit. It is rendered in the participle form thereby showing an ongoing action.} thee.%%
    
    \verse{20:13} \textsc{vi.} Never shalt thou murder.\footnote{It is not ``kill.'' The root that appears in the BHS (\Hebrew{רצח}) has behind it the idea of malicious forethought.}%%
    
    \verse{20:14} \textsc{vii.} Never shalt thou commit adultery.%%
    
    \verse{20:15} \textsc{viii.} Never shalt thou steal.%%
    
    \verse{20:16} \textsc{ix.} Never shalt thou answer falsely.\footnote{bear false witness/testimony}%%
    
    \verse{20:17} \textsc{x.} Never shalt thou desire\footnote{covet} thy neighbor's house, neither\footnote{\dots shalt thou desire\dots} thy neighbor's wife, nor his male or female servant, nor his ox, nor his male donkey, nor anything that is thy neighbors.''%%
    
    \verse{20:18} Then all the people were witnesses to the thunder,\footnote{lit., His voice} lightning, the sound of the trumpet, and the smoke of the mount. And they were witnesses and removed themselves.\footnote{In other words, they recognized the power and glory of God and stood back so as to not be consumed by His almighty power.}%%
    \verse{20:19} They then said to Moses, ``Speak on our behalf that we hear, and let Him not speak with us lest we die.''%%
    \verse{20:20} So Moses said unto the people, ``Do not be afraid,\footnote{Fear not} because in order to test thee, God is coming; and in order that thy reverence for Him be before you, that you don't sin.''%%
    \verse{20:21} The people stood back as Moses approached the thick cloud where God was.%%
    
    \verse{20:22} The \textsc{Lord} said to Moses, ``Thus shalt thou say unto the sons of Israel: `You have seen that I have spoken with you from the heavens.%%
    \verse{20:23} Never shalt thou make of me gods of gold or silver for yourselves.%%
    \verse{20:24} Thou shalt make for me an altar of earth and shalt offer unto me a burnt offering and a peace offering.%%
    \verse{20:25} But if you make an altar of stones to me, thou shalt not build it of hewn stones\footnote{Lest to be confused with an idol or graven image.} nor\footnote{lit., Never shalt thou} fashion those stones with tools: if thou wieldest thine tool\footnote{A metal instrument or tool. Not really a sword, although that is the word used in the BHS.} and lay it upon it\footnote{i.e., the altar} thou wilt defile it.%%
    \verse{20:26} Thou shalt not ascend on the steps to my altar in order that thy nakedness be not revealed on this altar.'\thinspace''%%
\end{inparaenum}

    \chap{21}

\heading{The Lord's law of servants, marriage, and the death penalty~--- eye for an eye, tooth for a tooth~--- damage caused by oxen}

\begin{enumerate*}[mode=unboxed]
    \verse{21:1} These are the judgments\footnote{laws} that you shall set before them.\footnote{Lit., ``their face''}%%
    \verse{21:2} If you acquire a Hebrew slave\footnote{male slave, servant. This person is subject to Jewish jurisprudence as evidenced by the use of the word \Hebrew{עבד}} six years, he shall serve thee; in the seventh year he shall go free without having to pay.\footnote{Like indentured servanthood.}%%
    \verse{21:3} If he comes alone, alone shall he go out. If he is married,\footnote{The husband/lord of a woman.} his wife shall go with him.%%
    \verse{21:4} If his master giveth him a wife and she bears sons or daughters to him, their\footnote{her} offspring\footnote{children} shall belong to the master and the man shall go forth by himself.%%
    \verse{21:5} If the servant actually says, ``I love my master, my wife, my children~--- I will not be freed.''%%
    \verse{21:6} Then his master will bring him\footnote{cause him to approach} to the presence of God, to the door, and his master shall pierce his ear with an awl and he shall be his\footnote{the master's} slave forever.%%
    \verse{21:7} If a man sells his daughter to be a handmaid, then she shall not go out as the bondsmen do.%%
    \verse{21:8} However, if she is unacceptable to her buyer\footnote{lit., master} (who has taken her for himself), then he shall let her be ransomed. However, because he has dealt with her unfairly, he shall not have power to sell her to foreigners.%%
    \verse{21:9} If he designates\footnote{assigns, but not necessarily \textit{marry}. \textit{Kohler-Baumgartner} simply says of \Hebrew{יִיעָדֶנָּה} (\Hebrew{יעד}) in the qal: designate, assign a woman} her to his son, then it shall be done to her according to the law of daughters.%%
    \verse{21:10} If he takes another for himself, then her food, clothing, and right to motherhood\footnote{alt., marital intercourse. Right to motherhood is more fitting contextually.}\footnote{How does this cover consent? Is that an issue in the ancient world?} shall not be taken away.%%
    \verse{21:11} But if he doesn't do these three things for her, she shall go free without money.%%
    \verse{21:12} Anyone who strikes\footnote{smite, (deal a) blow} a man to death shall certainly be killed.%%
    \verse{21:13} However, if he has not lain in wait, but God has delivered him\footnote{i.e., the killed} into his hand, then I will appoint a place where he can flee.%%
    \verse{21:14} If a man schemes\footnote{acts presumptuously} against his neighbor to kill him by deceit,\footnote{design, scheme, subtlety} you shall take him from my altar to kill him.%%
    \verse{21:15} He who strikes his father or his mother\footnote{\dots he\dots} shall surely be put to death.%%
    \verse{21:16} Anyone who kidnaps\footnote{steals} someone\footnote{to sell them} and he is caught with the man still in his possession:\footnote{power} he shall surely be put to death.\footnote{In other words, kidnapping is a capital offense.}%%
    \verse{21:17} He who curses\footnote{reviles, insults~--- but harsher than we tend to think} his father or his mother shall sure be put to death.%%
    \verse{21:18} And if men quarrel\footnote{fight} and one man hits another (whether with a stone or his fists), but not to kill him~--- merely to put him to bed~---%%
    \verse{21:19} if he rises and walks about outside\footnote{\textit{Koehler-Baumgartner} specifies this as the area outside a house.} on his staff,\footnote{For the sick.} then the smiter shall be declared exempt from punishment.\footnote{alt., acquitted of relevant charges} Only, he shall pay for his lost time until he is thoroughly healed.%%
    \verse{21:20} If a man strikes\footnote{hits} his male or female servant with a staff\footnote{rod} and that person dies under his hand then he will surely be punished in proxy for him.\footnote{or her.}%%
    \verse{21:21} But if a day or two goes by and he\footnote{i.e., the slave} gets up, he\footnote{the master} will not be punished for it is his\footnote{Understood: own.} property.%%
    \verse{21:22} If men fight and strike a pregnant woman so that she has a miscarriage,\footnote{Lit., ``her children go forth.''} but there is no further harm, he will surely be fined according to what the husband deems right, and he\footnote{the other man} will give according to what is assessed.%%
    \verse{21:23} If there is a serious injury or death, you shall take life in place of\footnote{for} life:%%
    \verse{21:24} an eye for an eye, a tooth for a tooth, a hand for a hand, a foot for a foot,\footnote{This is significantly more merciful than people let on. In other ancient civilizations (e.g., Hammurabi's code) there were different punishments depending on the person's social status.}%%
    \verse{21:25} burning for burning, wound for wound, strike for strike.%%
    \verse{21:26} If a man strikes the eye of his male or female servant and knocks it out,\footnote{Lit., destroys it} he will let him go free in penance for his eye.%%
    \verse{21:27} If the tooth of a male or female servant is knocked out, he must let them go free because of their tooth.%%
    \verse{21:28} If an ox\footnote{``Bull'' can be supplied for every following instance of ``ox'' in this chapter.} gores a man or a woman and that person dies, the ox will surely be stoned to death, his flesh shall not be eaten, and the owner shall not be liable.%%
    \verse{21:29} If the ox has been known to gore in the past and this was known\footnote{made known} to its master(s), and his master didn't keep him under guard, and it kills a man or a woman, the ox will be stoned and his master will be put to death as well.%%
    \verse{21:30} If a payment is laid upon him, he shall pay\footnote{give} a redemption\footnote{ransom} for his life according to everything that is laid upon him.%%
    \verse{21:31} Whether it runs down\footnote{alt., butts, thrusts, gores} a son or a daughter, then you shall do to it according to this judgment.%%
    \verse{21:32} If the ox runs down a male servant or a handmaid, then he shall pay the master thirty pieces of silver\footnote{Thirty shekels} and the ox shall be stoned.%%
    \verse{21:33} If a man uncovers\footnote{opens} or digs a pit, doesn't properly cover it,\footnote{Probably meant ``at night.''} and a donkey or an ox falls in,%%
    \verse{21:34} then the owner of the pit shall make it right: he shall give money to the owner\footnote{of the animal} and the dead animal shall be his.%%
    \verse{21:35} If a man's bull strikes the bull of his neighbor\footnote{associate} and he dies, then he shall sell the live ox and half the money and the dead ox.%%
    \verse{21:36} If it is known that the bull has previously run people down\footnote{lit., is known to run people down} and its owner has not kept watch of him, then the owner will have to compensate the owner of the dead ox and the dead ox will belong to him.\footnote{i.e., the one who paid.}%%
    \verse{21:37} If a man should steal an ox or a sheep\footnote{lamb} and slaughters\footnote{use it for food} or sells it, he shall recompense\footnote{make peace} with five oxen for the ox, or four sheep for the lamb.%%
\end{enumerate*}

    \section{Exodus 23}\label{Exodus 23}
\heading{Laws relating to honesty and conduct given, especially as relating to peer pressure~--- sabbatical year expounded~--- three feasts set forth~--- an angel will guide Israel~--- Canaanite nations will slowly be driven out}
\begin{enumerate}[align=center]
    \verse{Exodus^23:1} You shouldn't bear fraudulent hearsay. Don't extend your hand to the wicked to be a violent witness.%
    \verse{Exodus^23:2} Don't follow the multitude for evil; don't testify concerning a strife, to go after the multitude to turn others aside.%
    \verse{Exodus^23:3} Don't treat the helpless with distinction in their dispute.\footnote{alt., case, lawsuit}%
    \verse{Exodus^23:4} If you encounter your enemy's ox or donkey wandering,\footnote{lit., going back and forth} then you shall definitely bring it back to him.%
    \verse{Exodus^23:5} If you see your hater's donkey lying down under its burden, then you shall stop it\footnote{alt., leave it alone} from leaving. You shall certainly set it free.\footnote{alt., abandon it}%
    \verse{Exodus^23:6} Don't turn aside the judgment of the poor in their dispute.%
    \verse{Exodus^23:7} %
    \verse{Exodus^23:8} %
    \verse{Exodus^23:9} %
    \verse{Exodus^23:10} %
    \verse{Exodus^23:11} %
    \verse{Exodus^23:12} %
    \verse{Exodus^23:13} %
    \verse{Exodus^23:14} %
    \verse{Exodus^23:15} %
    \verse{Exodus^23:16} %
    \verse{Exodus^23:17} %
    \verse{Exodus^23:18} %
    \verse{Exodus^23:19} %`` somewhere here or above
    \verse{Exodus^23:20} I will send a messenger\footnote{i.e., a heavenly messenger} before you to guard you in the way and to bring you to the place I have prepared.%
    \verse{Exodus^23:21} Keep his face before you and hear his words. Do not anger him because he will not forgive your sins because my name is in his midst.%
    \verse{Exodus^23:22} For if ye shall surely hearken unto his voice and do all that I have said: I will be an enemy to thy enemy and a foe to thy foe.%
    \verse{Exodus^23:23} For my messenger shall go before you and bring you to the Amorites and the Hittites and the Perezites and the Canaanites and the Jebusites; and I will annihilate\footnote{cut them off, efface} them.%
    \verse{Exodus^23:24} You shall not bow down to their gods, nor shall you worship\footnote{serve} them. You shall not do as they do, but you shall overthrow them and shall smash their stone images to pieces.%
    \verse{Exodus^23:25} You shall serve the \textsc{Lord} thy God and He will bless your food\footnote{bread} and water. I will remove sickness from among you.%
    \verse{Exodus^23:26} There shall not be a woman who miscarries, neither is sterile\footnote{infertile, barren} in your land. And you shall live a full life.\footnote{Lit., I will complete the number of thy days.}%
    \verse{Exodus^23:27} I will send forth reverence\footnote{fear} of me before thee and confuse all the people that come out against thee. I will make all thy enemies flee\footnote{Lit., give to you all your enemies' backs}.%
    \verse{Exodus^23:28} I shall send hornets before thee and it shall drive out the Hivites, the Canaanites, and the Hittites from before thee.%
    \verse{Exodus^23:29} I will not drive them out before you in one year lest the land become desolate and the beasts of the field multiply and they will not be able to be controlled.%
    \verse{Exodus^23:30} Little by little will I drive them out from before you until you become\footnote{are} fruitful enough to take possession of the land.%
    \verse{Exodus^23:31} I will set your boundaries from the Reed Sea\footnote{Red Sea} to the sea of the Philistines, and from the desert to the river;\footnote{It's difficult to tell if it's the Jordan or the Tigris.} for I will place the inhabitants of the land in your hand and you will drive them out from before yourselves.%
    \verse{Exodus^23:32} You shall not make\footnote{cut} a covenant with them, neither with their gods.%
    \verse{Exodus^23:33} They shall not dwell\footnote{live} in your land lest they cause you to deviate\footnote{miss the mark, sin, transgress} from\footnote{against} me in that you serve their gods: that you become a snare to yourselves.''%
\end{enumerate}

    \heading{24}{Israel, by covenant, accept the Lord~--- Moses, Aaron, Nadab, Abihu, and the Seventy see God~--- the Lord calls Moses to the mountain to receive the stone tablets}

\begin{inparaenum}
    \verse{24:1} And He saith unto Moses: ``Go up unto the \textsc{Lord}~--- you, Aaron, and seventy of the elders of Israel~--- and you shall bow down\footnote{i.e., to worship} from afar.%%
    \verse{24:2} And Moses will approach the \textsc{Lord} alone. The people\footnote{i.e., Aaron and the Seventy} will not go up with him.''%%
    \verse{24:3} And Moses came and told the people all the words and judgments of the \textsc{Lord}; and all of the people answered with one voice, saying, ``All the words that the \textsc{Lord} hath spoken we will do.''%%
    \verse{24:4} Moses wrote all the words of the \textsc{Lord}. Then, he rose early in the morning, and built an altar at the foot of the mountain. He also made\footnote{built} twelve memorial stones\footnote{pillars} for the twelve tribes of Israel.%%
    \verse{24:5} And he sent lads\footnote{young men, servants} of the children of Israel and they offered burnt offerings and sacrificed sacrifices, peace offerings, and bulls.\footnote{oxen}%%
    \verse{24:6} Then Moses took half of the blood and placed it in bowls;\footnote{basins} and half\footnote{i.e., the other half} of the blood he sprinkled\footnote{cast} on\footnote{at} the altar.%%
    \verse{24:7} He took the scroll of the covenant and he read\footnote{it} out in the ears\footnote{hearing} of the people and they said, ``Everything that the \textsc{Lord} hath said we will do.''\footnote{hear, obey}%%
    \verse{24:8} And Moses took the blood and he sprinkled\footnote{cast} it upon the people and said, ``This is the blood of the covenant which the \textsc{Lord} hath made with you concerning all these things.''%%
    \verse{24:9} Then Moses, Aaron, Nadab, Abihu, and seventy elders of Israel went up%%
    \verse{24:10} and saw the God of Israel. Under His feet was a slab of sapphire as bright as heaven.%%
    \verse{24:11} He did not lay His hand on the eminent children of Israel. They saw God and they ate and they drank.\footnote{xxxx: reference to the sacrament?}%%
    \verse{24:12} The \textsc{Lord} said to Moses, ``Come up to me in the mountain\footnote{lit., mountain-ward} and be there. I will give you stone tablets, the Law and the Commandments that I have written for their instruction.''%%
    \verse{24:13} Moses rose up~--- and Joshua, his minister, too~--- and Moses went to the mountain of God.%%
    \verse{24:14} To the elders he said, ``Wait here\footnote{lit., in this place} for us until we return for you. And look, Aaron and Hur are here\footnote{Understood} with you, and whoever has any problems, bring it up with them.''\footnote{lit., approach them with it}%%
    \verse{24:15} Moses went up into the mountain, and the cloud covered the mountain.%%
    \verse{24:16} The glory of the \textsc{Lord} settled on Mount Sinai, and the cloud covered the mountain for six days. On the seventh day, He called to Moses from the thick\footnote{lit., midst} of the cloud.%%
    \verse{24:17} In the people's view, the appearance of the \textsc{Lord} in His glory was like fire consuming the peak of the mountain.%%
    \verse{24:18} Moses came into the thick of the cloud and climbed up the mountain. And Moses was in the mountain for forty days and forty nights.%%
\end{inparaenum}

    \chap{Numbers}{\Hebrew{במדבר}}
    \section{Numbers 5}\label{Numbers 5}
\heading{How to deal with lepers~--- repentance necessary for forgiveness~--- dealing with women accused of immorality}
\begin{enumerate}
    \verse{Numbers^5:1} %
    \verse{Numbers^5:2} %
    \verse{Numbers^5:3} %
    \verse{Numbers^5:4} %
    \verse{Numbers^5:5} %
    \verse{Numbers^5:6} %
    \verse{Numbers^5:7} %
    \verse{Numbers^5:8} %
    \verse{Numbers^5:9} %
    \verse{Numbers^5:10} %
    \verse{Numbers^5:11} %
    \verse{Numbers^5:12} %
    \verse{Numbers^5:13} %
    \verse{Numbers^5:14} %
    \verse{Numbers^5:15} %
    \verse{Numbers^5:16} %
    \verse{Numbers^5:17} %
    \verse{Numbers^5:18} %
    \verse{Numbers^5:19} %
    \verse{Numbers^5:20} %
    \verse{Numbers^5:21} %
    \verse{Numbers^5:22} %
    \verse{Numbers^5:23} %
    \verse{Numbers^5:24} %
    \verse{Numbers^5:25} %
    \verse{Numbers^5:26} %
    \verse{Numbers^5:27} %
    \verse{Numbers^5:28} %
    \verse{Numbers^5:29} This is the law concerning those suspected of infidelity (when the wife goes astray while married to her husband and is defiled).%
    \verse{Numbers^5:30} %
    \verse{Numbers^5:31} Then the man is free from wrong-doing, but the woman will pay for her iniquity.%
\end{enumerate}

    \heading{6}{Law of the Nazarite~--- Aaronic blessing}

\begin{enumerate*}[mode=unboxed]
    \verse{6:1} The \textsc{Lord} spoke to Moses, saying,%%
    \verse{6:2} ``Speak with the people of Israel and say to them, `If a man or woman vows the oath of the Nazarite, to consecrate themselves to the \textsc{Lord},%%
    \verse{6:3} then they\footnote{Lit., \textit{he}, but \textit{they} is gender neutral which agrees better with verse 2.} shall not drink beer\footnote{KB: intoxicating drink, evid\@. a kind of \textbf{beer}.} or vinegar beer,\footnote{xxxx: Archaic way of saying ``wine''?} neither shall they drink grape-juice or grape-extract or eat dried or fresh grapes.%%
    \verse{6:4} %%
    \verse{6:5} %%
    \verse{6:6} %%
    \verse{6:7} %%
    \verse{6:8} %%
    \verse{6:9} %%
    \verse{6:10} %%
    \verse{6:11} %%
    \verse{6:12} %%
    \verse{6:13} %%
    \verse{6:14} %%
    \verse{6:15} %%
    \verse{6:16} %%
    \verse{6:17} %%
    \verse{6:18} %%
    \verse{6:19} %%
    \verse{6:20} %%
    \verse{6:21} %%
    \verse{6:22} God spake unto Moses, saying,%%
    \verse{6:23} ``Speak unto Aaron and his sons, saying, `Thus shall you bless the children of Israel, saying unto them,%%
    \verse{6:24} ``May the \textsc{Lord} bless and preserve you.%%
    \verse{6:25} May the \textsc{Lord} cause His face to shine upon you and show you favor.%%
    \verse{6:26} May the \textsc{Lord} lift His face toward you and may He give peace unto you.''\,'%%
    \verse{6:27} They shall put my name upon the children of Israel and I will bless them.''%%
\end{enumerate*}

    \heading{10}{Trumpets to be used in calling and assembling Israel~--- cloud removes from camp~--- Israel sets forth in their orders~--- the Ark of the Covenant goes before the people}

\begin{enumerate*}[mode=unboxed]
    \verse{10:1} The \textsc{Lord} spoke to Moses, saying,%%
    \verse{10:2} ``Make two silver trumpets\footnote{KB: long, straight instrument of metal for signaling.}\footnote{Not to be confused with a \textit{shofar}.} for yourselves. You shall make them of hammered\footnote{or embossed} metalwork. They shall serve you when the assembly is gathered and when the order of departure for the camp is given.%%
    \verse{10:3} When they blow them, the whole company shall gather to you at the entrance of the meeting tent.%%
    \verse{10:4} If they blow with one, then the princes, the heads of thousands, shall gather to you.%%
    \verse{10:5} When you've blown an alarm, the camps which are encamped eastward shall break camp and march on.%%
    \verse{10:6} When you blow the alarm the second time, the camps which are encamped southward shall set forward. They shall blow an alarm when they set forward.%%
    \verse{10:7} In the assembling of the congregation of people, you shall blow the horn, but you shall not shout.%%
    \verse{10:8} The descendants of Aaron the High Priest shall blow the trumpets. They shall be an eternal, diving statute for you.%%
    \verse{10:9} If you go to war in your land against the enemy who is being hostile toward\footnote{alt., in a state of conflict with} you, then you shall blow the trumpets and you shall be remembered by\footnote{alt., in the sight of} the \textsc{Lord} your God and be saved from your enemies.%%
    \verse{10:10} In the day that you rejoice, in your assemblies and in your new moons, you shall blow the trumpets because of and in addition to\footnote{\Hebrew{עַל}, here, carries with it both of these meanings.} your burnt offerings and your communional peace sacrifices. These\footnote{lit., they} shall be a reminder for you \footnote{xxxx: Check KB on \Hebrew{לִפְנֵי}.}of your God: I, the \textsc{Lord}, am your God.%%
    \verse{10:11} And in the second year, in the second month, on the twentieth of the month, that the cloud was taken up from the dwelling place of the reminders.\footnote{See Appendix~\ref{}}%%
    \verse{10:12} The children of Israel set forth on their journey from the wilderness of Sinai and the cloud rested in the wilderness of Paran.%%
    \verse{10:13} And the first went not before the face of the \textsc{Lord} by the hand of Moses.%%
    \verse{10:14} The banner of the camp of Judah set forth first off, according to their ranks; and over its host was Nahshon the son of Amminadab.%%
    \verse{10:15} Over the host of the tribe of the children of Issachar: Nethaneel the son of Zuar.%%
    \verse{10:16} Over the host of the tribe of the children of Zebulun: Eliab the son of Helon.%%
    \verse{10:17} The utensils of the tabernacle were taken down, and the sons of Gershon and the sons of Merari~--- who carried the tabernacle and its accouterments~--- set forth.%%
    \verse{10:18} The banner of the camp of Reuben traveled according to their ranks; and over its host was Elizur the son of Shedeur.%%
    \verse{10:19} Over the host of the tribe of the children of Simeon: Shelumiel the son of Zurishaddai.%%
    \verse{10:20} Over the host of the tribe of the children of Gad: Eliasaph the son of Deuel.%%
    \verse{10:21} The carriers of the Holy Place\footnote{Inferred: the Holy of Holies as well.}~--- the Kohathites~--- went forth; and they would be the ones to set up the tabernacle at their\footnote{i.e., the Israelites} coming to a particular spot.%%
    \verse{10:22} The banner of the camp of the children of Ephraim traveled according to their ranks; and over its host was Elishama the son of Ammihud.%%
    \verse{10:23} Over the host of the tribe of the children of Manasseh: Gamaliel the son of Pedahzur.%%
    \verse{10:24} Over the host of the tribe of the children of Benjamin: Abidan the son of Gideoni.%%
    \verse{10:25} The banner of the camp of the children of Dan set forth, which was the rear of all the camps throughout their armies: and over its rank was Ahiezer the son of Ammishaddai.%%
    \verse{10:26} Over the host of the tribe of the children of Asher: Pagiel the son of Ocran.%%
    \verse{10:27} Over the host of the tribe of the children of Naphtali: Ahira the son of Enan.%%
    \verse{10:28} Such were the journeys of the children of Israel according to their ranks when they set forth.''%%
    \verse{10:29} And Moses said to Hobab,\footnote{Jethro} the son of Reuel the Midianite, the father-in-law of Moses, ``We are setting off to the same place that the \textsc{Lord} said, `I will give to you.' Come with us and we will treat you kindly because the \textsc{Lord} hath dealt kindly with Israel.''%%
    \verse{10:30} But he said, ``I will not go; rather, I will go unto mine own land and to mine own home country.''%%
    \verse{10:31} And he\footnote{Moses} said, ``Please don't leave me because you know how we should encamp in the desert~--- you shall be our eyes.%%
    \verse{10:32} ``If you will go with us, it will be good to you in the same measure as the \textsc{Lord} will be\footnote{was. Some slight eisegesis here, but I feel the rendering is more accurate with an eternal perspective. However, it works perfectly well with ``was'' because the Israelites have already been delivered from Egypt.} with us.''%%
    \verse{10:33} So they set out from the mountain of the \textsc{Lord} a journey of three days, and the Ark of the Covenant of the \textsc{Lord} went before them three days to seek out a place of rest for them.%%
    \verse{10:34} And the \textsc{Lord}, as a cloud, went over them by day when they journeyed from the camp.%%
    \verse{10:35} And when the Ark went forth, Moses said, ``Rise up, O \textsc{Lord}, and let thine enemies be scattered; yea, let thine haters flee from before thy face.''%%
    \verse{10:36} And when it rested he said, ``Return, O \textsc{Lord}, to the tens of thousands of Israelites.''\footnote{10,000 is the largest root describing numbers in Biblical Hebrew.}%%
\end{enumerate*}

    \chap{11}

\heading{The Lord destroys the rebels with fire~--- Israel complains about manna~--- Moses cannot take the burden alone~--- the Lord commands Moses to call the Seventy~--- meat will be given until it is loathsome~--- Seventy are called and chosen~--- Eldad and Medad prophesy~--- the people lust~--- many are destroyed by a plague}

\begin{enumerate*}[mode=unboxed]
    \verse{11:1} And the people complained that which is evil\footnote{complained bitterly, complained and it was evil} in the ears of the \textsc{Lord}, and the \textsc{Lord} heard it and was angry;\footnote{His anger was kindled} and the \textsc{Lord}'s fire burned against them and consumed the outer part of the camp.%%
    \verse{11:2} Then the people complained to Moses, so Moses prayed unto the \textsc{Lord} and the fire was quenched.%%
    \verse{11:3} He called the name of the place Taberah for the fire of the \textsc{Lord} burned there among them.%%
    \verse{11:4} And the crowd that was in its midst\footnote{among them} as well as the children of Israel lusted and cried again, saying, ``Who shall give us flesh to eat?%%
    \verse{11:5} We remember the fish that we would eat in Egypt for no cost;\footnote{free fish, but not freely eat. That connotes something different.} and the cucumbers, melons, leeks, onions, and garlic~---%%
    \verse{11:6} now our souls are dried up for there is nothing that we can see besides manna!''%%
    \verse{11:7} The manna was like coriander\footnote{Sp. \textit{cilantro}. However, this probably refers to the fruit (seed), not the leaves, and is referring to the taste, not its color.} and was the color of bdellium.\footnote{A tree resin varying from yellow to green, but usually a brown color.}%%
    \verse{11:8} The people roamed and they gathered the manna and they ground with millstones, beat in a mortar, and cooked,\footnote{boiled in a pot} or made round loafs. Its taste was like the taste of a cake with olive oil.%%
    \verse{11:9} When the dew descended on the camp at night the manna would likewise descend upon it.%%
    \verse{11:10} Moses heard the people crying according to their tribes\footnote{each with about five generations}~--- every man at the opening of his tent~--- and the anger of the \textsc{Lord} was great. And Moses was displeased.%%
    \verse{11:11} Moses said to the \textsc{Lord}, ``Why have you afflicted your servant? How have I not found favor in your sight in that you lay the burden of all this people on me?%%
    \verse{11:12} Have I conceived all this people? Have I given birth to them? You say to me, `Carry them in your bosom as a nursing father the infant,' to the land which you swore by oath to their fathers.%%
    \verse{11:13} %%
    \verse{11:14} I am not able to bear all this people alone because they are too heavy for me.%%
    \verse{11:15} If you deal thusly with me, just kill me~--- please~--- if I have found favor in your eyes, that I may not dwell upon my displeasure.''%%
    \verse{11:16} The \textsc{Lord} said unto Moses, ``Gather seventy men for me among the elders of Israel whom you know to be elders and overseers of the people. Take them unto the meeting tent\footnote{part of the tabernacle complex} and cause them to stand there with you.%%
    \verse{11:17} I will come down and converse with you there. I will take from the spirit that is upon you\footnote{the burden of the people; delegation of priesthood responsibility} and shall put it upon them. They shall bear the load with you~--- you shall not bear it alone.%%
    \verse{11:18} %%
    \verse{11:19} %%
    \verse{11:20} ''%%
    \verse{11:21} ``%%
    \verse{11:22} Shall we slay all the flock and cattle for them? That would suffice. Or should we gather all the fish in the sea for them? That would be enough to satisfy them.''%%
    \verse{11:23} The \textsc{Lord} said unto Moses, ``Is the hand of the \textsc{Lord} shortened? You shall see whether or not My word shall come to pass.''%%
    \verse{11:24} Moses went forth and he spoke the \textsc{Lord}'s words unto the people. He gathered up seventy of the elders of the people and caused them to stand surrounding the tent.\footnote{probably the tabernacle}%%
    \verse{11:25} The \textsc{Lord} went down in the cloud and spoke unto them. He took of the spirit that was on Moses and he conferred it on the seventy elders and they prophesied without ceasing.%%
    \verse{11:26} Two men remained in the camp. One was named Eldad and the other Medad. They were in the register, but did not go to the tent. When the spirit rested upon them they went into the camp and prophesied.%%
    \verse{11:27} And a youth ran and told Moses, saying, ``Eldad and Medad are prophesying in the camp.''%%
    \verse{11:28} Joshua, the son of Nun and the servant of Moses, answered saying, ``My lord, Moses, restrain them!''%%
    \verse{11:29} But Moses said unto him, ``Are you jealous on my behalf? I wish that all the \textsc{Lord}'s people were prophets and that the \textsc{Lord} would so place His spirit upon them all.''%%
    \verse{11:30} So Moses gathered himself and the elders of Israel%%
    \verse{11:31} and the spirit of the \textsc{Lord} went forth and drove quails from the sea who passed over and fell on the camp: a day's journey this way and about a day's journey that way round about the camp: ten cubits above the face of the earth.%%
    \verse{11:32} The people arose all that day, that night, and the morrow to gather quail. The least gathered ten homers\footnote{An ancient Hebrew measure of capacity, equal to ten ephahs or ten baths, and approximately equal to ten or eleven bushels. About eight gallons} and they spread out for themselves a place surrounding the camp.%%
    \verse{11:33} While the flesh was still in between their teeth, before is was consumed, the anger of the \textsc{Lord} was kindled against the people and the \textsc{Lord} smote the people with a great plague.%%
    \verse{11:34} He called the name of that place Kibroth-hattaavah\footnote{Lit., burial place of the lusters} because there they buried the people who had lusted.%%
    \verse{11:35} The people traveled from this place to Hazeroth and they stayed there.%%
\end{enumerate*}

    \heading{12}{Miriam and Aaron complain against Moses~--- Miriam is cursed with leprosy~--- Moses prays and Miriam is healed}

\begin{enumerate*}[mode=unboxed]
    \verse{12:1} Miriam and Aaron spoke against Moses because of the Kushite\footnote{Possibly Midianite} woman whom he married; for he had taken a Kushite woman in marriage.%%
    \verse{12:2} They said, ``Has the \textsc{Lord} only spoken through Moses? Hasn't He also spoken through us?'' And the \textsc{Lord} heard it.%%
    \verse{12:3} Now, the man (Moses) was incredibly unassuming\footnote{meek, humble}~--- more than any other man on the face of the earth.\footnote{In other words, he is too humble to fight his own battles. Instead, he lets the Lord fight for him.}%%
    \verse{12:4} The \textsc{Lord} spoke suddenly unto Moses, Aaron, and Miriam, ``The three of you shall go forth unto the tent of meeting.'' So the three of them went forth.%%
    \verse{12:5} Then the \textsc{Lord} went down in the pillar of a cloud and stood in the door of the tent. He called for Aaron and Miriam and they both came out.%%
    \verse{12:6} He said, ``Heed my words: if you have a prophet the \textsc{Lord} will reveal Himself unto him in a vision. I will speak unto him in a dream.%% xxxx: poetry through verse 8
    \verse{12:7} Is it not true that Moses is My servant? That of all My house he is faithful?%%
    \verse{12:8} I speak to him face to face: visibly. Not in riddles. He has seen the form of the \textsc{Lord}. Why weren't you hesitant to speak against my servant Moses?''%%
    \verse{12:9} The \textsc{Lord} grew angry with them and He left.%%
    \verse{12:10} The cloud turned away from the tent\footnote{The Tabernacle. See Appendix~\ref{app:tabernacle}.} and Miriam was leprous\footnote{What is mentioned here as ``leprosy'' is a curable skin condition. See Appendix~\ref{app:psalm-110}.} as snow. Aaron turned to face Miriam and she was leprous.
    \verse{12:11} Moses said, ``My \textsc{Lord}, please do not lay this sin upon us that we foolishly sinned.%%
    \verse{12:12} %% '' NOTES FROM ORIGINAL TRANSLATION: stillborn. much like the appearance of the skin disease, typically rendered as ``leprosy.'' The idea of a malformed stillborn.
    \verse{12:13} Moses cried unto the \textsc{Lord}, saying, ``Oh God, please heal her!''\footnote{\Hebrew{נָא} is used twice to emphasize Moses' plea.}%%
    \verse{12:14} So the \textsc{Lord} said to Moses, ``If her father had spat in her face, would she not be in disgrace for seven days? Let her be removed from the rest of the people and after that\footnote{Understood} she shall be received.''\footnote{We don't know why nothing happened to Aaron. Perhaps it's because he merely gave in to peer pressure (as he was wont to do).}%%
    \verse{12:15} Miriam was confined outside the camp for seven days, and the people did not travel until she was received back into the camp.%%
    \verse{12:16} Then the people up and left Hazeroth and camped in the desert of Paran.%%
\end{enumerate*}

    \chap{Deuteronomy}{\Hebrew{דברים}}
    \heading{5}{Moses restates the Decalogue~--- he recounts its giving}

\begin{inparaenum}
    \verse{5:1} %%
    \verse{5:2} %%
    \verse{5:3} %%
    \verse{5:4} %%
    \verse{5:5} %%
    \verse{5:6} \textsc{Preface.} I am the \textsc{Lord} your God. I brought you out from the land of Egypt~--- from the house of captivity.%%
    \verse{5:7} \textsc{i.} There will be no other gods in preference to me.%%
    \verse{5:8} \textsc{ii.} Never make for yourselves graven images, or any image,\footnote{an idol, picture, image, or any likeness} in the heavens above, neither that is in the earth nor the waters underneath the earth.%%
    \verse{5:9} Never bow down to them or worship\footnote{serve} them because I, the \textsc{Lord} your God, am a jealous God: I visit the punishment\footnote{iniquity} unto the third and fourth generation of my haters%%
    \verse{5:10} Showing mercy unto thousands who love me and obey my commandments.%%
    \verse{5:11} \textsc{iii.} Never take the name of the \textsc{Lord} thy God with vain purpose for the \textsc{Lord} shall not hold him guiltless who uses His name to vain purpose.%%
    \verse{5:12} \textsc{iv.} Keep the Sabbath day holy, just as the \textsc{Lord} your God has commanded you.%%
    \verse{5:13} You shall work and perform all your responsibilities\footnote{tasks, errands} in six days,%%
    \verse{5:14} but the seventh day is the Sabbath of the \textsc{Lord} your God: you shouldn't work, neither your son, daughter, handmaid, ox, donkey, any of your cattle, nor an outsider who's living in Israel. This so that your manservant and maidservant may rest just like you.%%
    \verse{5:15} Remember when you were slaves in the land of Egypt and the \textsc{Lord} your God brought you from there with a strong hand and an outstretched arm.\footnote{So that He can show the way; fight our battles} Therefore the \textsc{Lord} your God has commanded you to keep the Sabbath day.%%
    \verse{5:16} \textsc{v.} Take your mom and dad seriously just as the \textsc{Lord} your God has commanded you. This that your days may be lengthened and in order that it may be well for you upon the land which the \textsc{Lord} your God is going to give you.%%
    \verse{5:17} \textsc{vi.} Don't murder.%%
    \verse{5:18} \textsc{vii.} Don't commit adultery.%%
    \verse{5:19} \textsc{viii.} And don't steal.%%
    \verse{5:20} \textsc{ix.} Never give a vain witness against your neighbor.%%
    \verse{5:21} \textsc{x.} Don't desire your neighbor's wife or his house, field, bondman, handmaid, ox, donkey, or anything that's his.%%
    \verse{5:22} %%
    \verse{5:23} %%
    \verse{5:24} %%
    \verse{5:25} %%
    \verse{5:26} %%
    \verse{5:27} %%
    \verse{5:28} %%
    \verse{5:29} %%
    \verse{5:30} %%
    \verse{5:31} %%
    \verse{5:32} %%
    \verse{5:33} %%
\end{inparaenum}

    \heading{6}{xxxx}

\begin{inparaenum}
    \verse{6:1} ``These are the commandments, statutes, and legal decisions that the \textsc{Lord} your God has commanded me\footnote{Understood} to teach you, that you apply them in the land thither that you will pass through and\footnote{lit., to} possess.%%
    \verse{6:2} Fear the \textsc{Lord} your God so that you keep all of His statutes and commandments that I have command you and your children and your children's children. Keep them\footnote{Repeated for idiomaticity's sake.} all the days of your life so that your days may be prolonged.%%
    \verse{6:3} You, Israel, shall hear and observe to do that which is good for you in order that you may greatly multiply in the land flowing with milk and honey, even as your fathers' God, the \textsc{Lord}, has told you.%%
    \verse{6:4} Hear, O Israel! I, the \textsc{Lord} am God: there is only one \textsc{Lord}\footnote{lit., the \textsc{Lord} is one}.%%
    \verse{6:5} Thou shalt love the \textsc{Lord} thy God with all thy heart, with all thy soul, and with all thy being\footnote{strength}.%%
    \verse{6:6} These are the words\footnote{commands} that I am going to command you in your heart.%%
    \verse{6:7} You shall repeat them to your children. You shall talk about them when you sit in your home, when you go in the road, or when you rise\footnote{stand} up.%%
    \verse{6:8} You shall bind them as a sign upon your hand. They shall be as frontlets.%%
    \verse{6:9} Ye shall write them upon the lintel posts of your house and gate\footnote{i.e., any official entrance to your house, even doors to separate rooms: these words shall be upon them.}.%%
    \verse{6:10} It shall be, when the \textsc{Lord} your God brings you into the land that He swore to your fathers~--- to Abraham, Isaac, and Jacob~--- that He would give to you, the great and good cities that you didn't have to\footnote{Understood} build.%%
    \verse{6:11} Houses that you didn't fill, but which are full of everything good; hewed cisterns that you did not quarry; vineyards and olive orchards that you did not plant; of these\footnote{Understood} you have eaten and been satiated.%%
    \verse{6:12} Be careful lest you forget the \textsc{Lord} who brought you out from the land of Egypt, from the house of servitude.%%
    \verse{6:13} Reverence the \textsc{Lord} your God, serve Him, swear by His name.%%
    \verse{6:14} You shall not follow\footnote{alt., adhere to} other gods, the gods of the people who are round about you,%%
    \verse{6:15} because the \textsc{Lord} your God, who is\footnote{Understood} in your midst, is an envious God. Don't do this\footnote{Understood} lest the \textsc{Lord} your God becomes indignant with you and exterminates you from off the face of the land.%%
    \verse{6:16} Don't put the \textsc{Lord} your God to the test as you did\footnote{lit., put Him to the test} in Massah.%%
    \verse{6:17} %%
    \verse{6:18} %%
    \verse{6:19} %%
    \verse{6:20} %%
    \verse{6:21} %%
    \verse{6:22} %%
    \verse{6:23} %%
    \verse{6:24} %%
    \verse{6:25} %%
\end{inparaenum}

    \heading{25}{Rules about beating transgressors~--- Levirite law given~--- fair measurements to be used~--- commandments regarding how and why to deal with Amalek enumerated}

\begin{inparaenum}
    \verse{25:1} ``If there's a lawsuit between men and they come to\alt{decide upon, resort to} judgment, then they shall judge and pronounce the righteous not guilty and pronounce the transgressor guilty.%%
    \verse{25:2} If the transgressor is supposed to be beaten, the judge shall cause him to fall down and someone shall hit him in the judge's presence, sufficient by number for his transgression.%%
    \verse{25:3} He shall be smitten no more than forty times. They shall not add to this, otherwise, if they hit him more than these many strikes, then your brother shall be of low esteem in your eyes.%%
    
    \verse{25:4} You shouldn't muzzle an ox while it's threshing.\alt{Seems like something is missing here. This is very \textit{non sequitur}, although it is in a separate section than the surrounding verses.}%%
    
    \verse{25:5} When brothers dwell together\ed{This does not necessarily denote two men with the same parents, but rather male relatives. Same goes for whenever sister-in-law is said: it is simply a female relative.} and one of them dies and he has no son\lit{there is not a son to him}, his wife shall not go unto foreigners\lit{outside} to find a husband, but her brother-in-law will come to her and he shall take her unto himself as a wife. And thus\ed{``Thus'' is not in the verse, but helps the flow.} shall he perform the duty of a brother-in-law.%%
    \verse{25:6} He shall raise up for the deceased\lit{over the name of the deceased} the oldest child that she shall bear that his\ie{the deceased} name be not erased from Israel.\lit{The oldest child that she shall bear, he shall raise up for the deceased that his name be not erased from Israel.}%%
    \verse{25:7} And if the man is not inclined\alt{pleased} to take his sister-in-law then let her\lit{his sister-in-law} go up to the gate to the elders and say, ``My brother-in-law hath refused to raise up into his brother a name in Israel.''%%
    \verse{25:8} The elders of the city will call for him and he will stand and he shall say, ``I do not desire to take her.''%%
    \verse{25:9} Then his sister-in-law will approach him in the eyes\alt{presence} of the elders and she will remove his sandal from off his foot. And she shall spit in his face and say,\lit{and she will answer and say} ``So shall it be done\lit{it is done} to the man who will not build up the house of his brother.''%%
    \verse{25:10} And his name will be called \textit{The house of the man whose sandal was removed}.%%
    
    \verse{25:11} When men quarrel together\lit{one with another} and one of their wives comes near to deliver her husband from the guy who's hitting him,\lit{from the hand of his smiter} and she's stretched out her hand and grabbed\alt{taken hold of, seized} his shame,\ed{\textsc{halot} doesn't seem to have a rendering which means \textit{privates} or \textit{private parts}. The verb form, \Hebrew{בושׁ}, means \textit{to be ashamed} from which we surmise that the noun form refers to something of which a man would be ashamed, therefore his genitals.}%%
    \verse{25:12} then you shall cut off her hand. You shall not pity her.\lit{You shall not be troubled about (or look compassionately) in your opinion (understood: on her).}%%
    
    \verse{25:13} There shall not be any stones in your bag, even a great or a small stone.%%
    
    \verse{25:14} There shall not be an ephah, great or small, in your house.%%
    \verse{25:15} You shall have a complete and just stone. You shall have a complete and just ephah. This\understood\ in order that your days may be lengthened in the land that the \textsc{Lord} your God shall give you.%%
    \verse{25:16} Because anyone doing these things~--- anyone doing wickedness~--- is abhorred by the \textsc{Lord} your God.%%
    
    \verse{25:17} Remember what Amalek did to you in the road on your way out of Egypt,%%
    \verse{25:18} how he happened upon you in the way and seized and destroyed the rear-guard (all those who are unfit to travel\alt{stragglers} who were behind you); and you were weary and tired.\alt{extremely weary. Both \Hebrew{עָיֵף} and \Hebrew{יָגֵעַ} mean weary, but the latter also means tired.} But he did not fear God.%%
    \verse{25:19} Since the \textsc{Lord} your God has given you peace and quiet\ed{\Hebrew{נוח} means to give rest or quiet.} from all the enemies which surround you in the land that the \textsc{Lord} your God has given to you~--- a hereditary possession for you\understood\ to possess~--- that\understood\ the remembrance of Amalek shall be wiped out from under the heavens. You shall not forget it.%%
\end{inparaenum}

    \chap{Joshua}{\Hebrew{יהושע}}
    \section{Joshua 3}\label{Joshua 3}
\heading{xxxx}
\begin{enumerate}
    \verse{Joshua^3:1} %
    \verse{Joshua^3:2} %
    \verse{Joshua^3:3} %
    \verse{Joshua^3:4} %
    \verse{Joshua^3:5} %
    \verse{Joshua^3:6} %
    \verse{Joshua^3:7} %
    \verse{Joshua^3:8} %
    \verse{Joshua^3:9} %
    \verse{Joshua^3:10} %
    \verse{Joshua^3:11} %
    \verse{Joshua^3:12} %
    \verse{Joshua^3:13} %
    \verse{Joshua^3:14} During the people's journey from their tents to pass over the Jordan, the priests bore the ark of the covenant before the people.%
    \verse{Joshua^3:15} And when those bearing the ark came to the Jordan, the feet of the priests bearing the ark dipped into the extremity of the waters (the Jordan being full over all its banks all the days of harvest)%
    \verse{Joshua^3:16} and the flowing waters stood and rose up in a heap, far above the city Adam (which is beside Zaretan); and those going down by the sea of the plain~--- the Salt Sea~--- were completely cut off. And the people passed through to Jericho.%
    \verse{Joshua^3:17} And the priests bearing the ark of the covenant of the \textsc{Lord} stood established on the dry ground in the midst of the Jordan. And all Israel passed over on dry ground, even until the whole nation had completely passed over the Jordan.%
\end{enumerate}

    \chap{Judges}{\Hebrew{שופטים}}
    \section{Judges 7}\label{Judges 7}
\heading{xxxx}
\begin{enumerate}[align=center]
    \verse{7:1} %%
    \verse{7:2} And the \textsc{Lord} said unto Gideon, ``The people with thee are too great\footnote{many} for me to just give the Midianites into their hand, lest Israel vaunt themselves against me, saying, `My own hand hath saved me!'%%
    \verse{7:3} Now, please proclaim within earshot of the people, saying, `Whosoever is fearful and trembling, let him turn back and leave\footnote{go} early from Mount Gilead.''' And 22,000 of the people turned back, and there remained 10,000.%%
    \verse{7:4} And the \textsc{Lord} saith unto Gideon, ``Still, the people are too many. Bring them down to the water and I shall refine\footnote{test} them for you there. It shall be that he to whom I say to thee, `This man shall go with you,' shall go with you; and any that I say unto you, `This man shall not go,' shall not go.''%%
    \verse{7:5} So he brought the people down to the water. And the \textsc{Lord} said unto Gideon, ``Any who laps water with his tongue (like a dog), you shall set by himself. Likewise with those that kneel down to drink.''%%
    \verse{7:6} The number of those who lapped~--- putting their hands to their mouth~--- was 300 men; but all the rest of the people knelt down to drink water.%%
    \verse{7:7} And the \textsc{Lord} saith unto Gideon, ``I will deliver you with the 300 men who lapped; and I will give the Midianites into your hand. Let all the rest go unto their homes.''%%
    \verse{7:8} %%
    \verse{7:9} %%
    \verse{7:10} %%
    \verse{7:11} %%
    \verse{7:12} %%
    \verse{7:13} %%
    \verse{7:14} %%
    \verse{7:15} %%
    \verse{7:16} %%
    \verse{7:17} %%
    \verse{7:18} %%
    \verse{7:19} %%
    \verse{7:20} %%
    \verse{7:21} %%
    \verse{7:22} %%
    \verse{7:23} %%
    \verse{7:24} %%
    \verse{7:25} %%
\end{enumerate}

    \chap{Ruth}{\Hebrew{רות}}
    \heading{1}{Naomi's husband and sons die~--- one of her daughters-in-law, Orpah, leaves~--- the other, Ruth, promises to stay}

\begin{inparaenum}
    \verse{1:1} In the days when the judges judged there was a famine in the land. A man of Bethlehem-Judah went to sojourn in the field of Moab\footnote{Moabite territory or country}~--- he, and his wife, and his two sons.%%
    \verse{1:2} And the name of the man~--- Elimelech; and the name of his wife~--- Naomi; and the names of his two sons~--- Mahlon and Chilion: Ephrathites from Bethlehem-Judah. And they entered the field of Moab and they stayed\footnote{continued, sojourned} there.%%
    \verse{1:3} And Naomi's husband, Elimelech, died, and she was left with\footnote{and} her two sons.%%
    \verse{1:4} And they took to\footnote{raised up unto} themselves Moabite wives, the name of the one, Orpah; and the name of the other, Ruth. And they stayed there about ten years.%%
    \verse{1:5} Both Mahlon and Chilion died. And the woman was bereft of her two boys and her husband.%%
    \verse{1:6} And\footnote{Then} she rose up with her daughters-in-law to return to the land of Moab, for she had heard that in the land of Moab the \textsc{Lord} had visited His people by giving them bread.%%
    \verse{1:7} She went out from the place where she was with her two daughters-in-law and they went on the road to return to the land of Judah.%%
    \verse{1:8} And Naomi said to her two daughters-in-law to go return to their mothers' houses. ``The \textsc{Lord} will do kindly with you according as ye have done with the dead and myself.%%
    \verse{1:9} The \textsc{Lord} will give you rest in your husbands' houses.'' Then she kissed them, and they lifted up their voices and wept.%%
    \verse{1:10} But they said to her, ``Surely we will return to thy people.''%%
    \verse{1:11} But Naomi said, ``Return, my daughters. Why would you go with me? Do I yet have sons in my womb that they may be husbands to you?%%
    \verse{1:12} Return, my daughters, for I have grown too old to have a husband \footnote{I have grown old from having a man.}. If I were to say I have hope, and if I were to have a man tonight, and have sons,%%
    \verse{1:13} would you wait for them until they were grown and keep yourselves from a husband? Not so, my daughters, for it grieveth me greatly for your sakes that the hand of the \textsc{Lord} is against me.''%%
    \verse{1:14} They lifted up their voices and wept again. And Orpah kissed her mother-in-law, but Ruth clung to her.%%
    \verse{1:15} And she\footnote{Naomi} said, ``Your sister-in-law\footnote{The Hebrew, \Hebrew{בּמת}, denotes a relationship by marriage, not necessarily that they are sisters.} is returning to her people and to her gods. Return after your sister-in-law.''%%
    \verse{1:16} And Ruth said, ``Do not ask me to abandon you nor to turn from following thee: for wherever you go, will I go; and where you lodge, I will lodge; your people shall be my people; and your God shall be my God;%%
    \verse{1:17} and where thou diest, will I die and be buried there: the \textsc{Lord} do to me, and more also, if anything but death parts me and thee.''%%
    \verse{1:18} When she saw how bold\footnote{steadfastly minded} she\footnote{Ruth} was to go with her, she ceased speaking with her.%%
    \verse{1:19} And the two of them went to Bethlehem. And when they got to Bethlehem the city was abuzz about them, saying, ``Is this Naomi?''%%
    \verse{1:20} And she said unto them, ``Don't call me Naomi. Call me Mara, for the Almighty hath treated me harshly.%%
    \verse{1:21} I left full, and empty the \textsc{Lord} returned me. Why would you call me Naomi since the \textsc{Lord} hath afflicted me and the Almighty hath broken me in pieces?''%%
    \verse{1:22} And Naomi returned with Ruth the Moabitess, her daughter-in-law, with her to the land of Moab and they entered into Bethlehem in the beginning of barley harvest.%%
\end{inparaenum}

    \heading{2}{Ruth gathers gleanings of grain~--- Boaz asks about Ruth, gives her permission to glean, and has his servants make sure that there will be gleanings to gather}

\begin{inparaenum}
    \verse{2:1} Naomi got to know one of her husband's relatives, a heroic landowner from Elimelech's extended family whose name was Boaz.%%
    \verse{2:2} Ruth the Moabitess said to Naomi, ``Please, let me go to the field and I'll glean grain\footnote{lit., ears of grain} behind the man\understood\ in whose opinion I shall find grace.'' And she said to her, ``Go, my daughter.''%%
    \verse{2:3} So she went and she came and gleaned behind the harvesters. And it so happened that she came upon one of the fields allotted to Boaz (who is in Elimelech's extended family).%%
    \verse{2:4} Boaz came from Bethlehem and said to the harvesters, ``The \textsc{Lord} be with you!'' And they said to him, ``The \textsc{Lord} bless you!''%%
    \verse{2:5} Boaz said to his servant who was in charge of the harvesters, ``Who is this girl?''\footnote{\Hebrew{נערה}, in this context, refers to an unmarried girl (who is a virgin).}%%
    \verse{2:6} The servant who was in charge of the harvesters answered and said, ``The girl is a Moabitess who came back from the fields of Moab with Naomi.%%
    \verse{2:7} She has said, `Please let me gather gleanings. I have harvested ears of grain\footnote{KB: not sheaves; the stalks were cut off right under the ears.} behind the harvesters.' She's come and been here\footnote{lit., remained} from morning until now. She did\understood\ sit in the house a little.''%%
    \verse{2:8} So Boaz said to Ruth, ``Hello girl,\footnote{lit., my daughter} haven't you heard? Don't go gather gleanings in another field. Also, don't pass through here, but\footnote{lit., thus} stick with my girls.\footnote{alt., female servants}%%
    \verse{2:9} Let your eyes be on the field that is being harvested and go after them. I've commanded the servants to not hurt\footnote{alt., touch} you, haven't I? When you're thirsty, go to the vessels and drink from the water\understood\footnote{lit., what, that, which} the servants draw.''%%
    \verse{2:10} She fell on her face and bowed to the ground, and said to him, ``Why have I found grace in your eyes that you recognize me, \footnote{lit., I am; I being}a stranger?''%%
    \verse{2:11} And Boaz, answering, said unto her, ``I've been told about \emph{everything} you've done to your mother-in-law: how you left\footnote{abandoned} your father and your mother and the land of your birth, and you've come to a people you've never before known.%%
    \verse{2:12} May the \textsc{Lord} recompense your efforts and may your reward be full\footnote{complete} from the \textsc{Lord} God of Israel under whose wings you have come to seek refuge.''%%
    \verse{2:13} And she said, ``May I find grace\footnote{favor} in your eyes for you have been kind to us, and because you have been kind to your maidservant even though I am not like\footnote{among one of} your maidservants.''%%
    \verse{2:14} And Boaz said to her at mealtime, ``Come hither and eat bread and dip the morsel\footnote{broken pieces} in vinegar.'' And she stayed by the harvesters and he\footnote{refers to either Boaz or the harvesters. It is ambiguous.} offered\footnote{reached out, extended} her parched\footnote{roasted} grain and she ate, was satiated, and had some left over.%%
    \verse{2:15} And she stood up to go glean, and Boaz commanded his servants saying, ``Let her glean between the sheaves and do not bother\footnote{rebuke} her.%%
    \verse{2:16} And also draw out for her some bundles\footnote{sheaves} and leave and she shall gather. And rebuke her not.''%%
    \verse{2:17} And she gathered in the land\footnote{field} until eventide and she threshed that which she had gathered\footnote{gleaned}: about an ephah of barley.%%
    \verse{2:18} And she took it up and went into the city. And her mother-in-law saw her and she\footnote{Naomi} took that which she\footnote{Ruth} had gathered and she brought it out to her and she was satiated.%%
    \verse{2:19} And her mother-in-law said unto her, ``Where did you glean today? And where did you work? May her who helped thee be blessed.'' And she told her mother-in-law, ``The person with whom I worked today was Boaz.''%%
    \verse{2:20} And Naomi said to her daughter-in-law, ``May he be blessed of the \textsc{Lord} who did not abandon his loving-kindness toward the living nor toward the dead.'' And Naomi said to her, ``The man that is near to us, he is our levir\footnote{is among our levirs}\footnote{A levir was a tribal leader, avenger of blood, or a redeemer.}.''%%
    \verse{2:21} And Ruth the Moabitess said, ``He also said to me, `You shall stay with my workers until they have ended my harvests.'''%%
    \verse{2:22} And Naomi said to Ruth her daughter-in-law, ``My daughter, it is well that you go out with the maidservants and that they don't molest you in any other field.''%%
    \verse{2:23} And she stayed fast by the maidservants of Boaz to glean until the end of the barley and wheat harvests. And she stayed with her mother-in-law.%%
\end{inparaenum}

    \heading{3}{Naomi has Ruth entreat herself to Boaz, their levir~--- Boaz tells Ruth that there is another levir who has more immediate responsibility over her than he~--- he comes up with a plan to redeem her himself~--- she spends the night and he sends her off}

\begin{inparaenum}
    \verse{3:1} So Naomi, her mother-in-law, said to her, ``My daughter, rest that it may be well for you.%%
    \verse{3:2} Is not Boaz our kinsman among whose maidservants you were with? He is threshing barley at the threshing floor tonight.%%
    \verse{3:3} Wash and anoint thyself, put your clothes on, and get down to the threshing floor, but don't make thyself known to the man until he's done eating and drinking.%%
    \verse{3:4} And when he lies down you shall know the place where he lieth. And you shall enter\footnote{go in} and uncover his feet and lie down. And he shall tell you that which you shall do.''%%
    \verse{3:5} And she\footnote{Ruth} said unto her, ``All that thou hast said unto me I will do.''%%
    \verse{3:6} And she went down to the threshing floor and she did just about everything that her mother-in-law told her.%%
    \verse{3:7} And Boaz ate and drank and was happy. So he entered to lie down at the edge of the heap of grain. And she entered in quietly\footnote{secretly} and uncovered his feet and lied down.%%
    \verse{3:8} And night came, and the man was shocked and turned and there was a woman lying at his feet.%%
    \verse{3:9} And he said, ``Who is there?'' And she said, ``I am Ruth, thy handmaiden. Spread the hem of thy garment over thy maidservant for you have the right of a levirite.''%%
    \verse{3:10} And he said, ``Blessed be thou of the \textsc{Lord}, my daughter. You've done a finer act of kindness at the end than in the beginning in not going after the young men,\footnote{KB: fully grown, vigorous, still unmarried.} be they rich or poor.%%
    \verse{3:11} Now, my daughter, fear not\footnote{do not be afraid} for all I have said unto you, that will I do. All of my people think that thou art a virtuous woman.%%
    \verse{3:12} And now, truly I am a levir; however, there's a levir nearer than I.%%
    \verse{3:13} Stay the night. If he redeems you in the morning, great. If he's not inclined to redeem you, then as the \textsc{Lord} lives, I will redeem you. Lie down till the morning.''%%
    \verse{3:14} She laid at his feet till morning and got up before you can tell who's who.\footnote{lit., before you can tell a man from another.} He said, ``Don't let it be known that a woman came to the threshing floor.''%%
    \verse{3:15} He said, ``Give me the scarf from off your head.'' And he measured out for her six measures of barley. Then he\footnote{The Septuagint version can be translated as either ``he'' or ``she.'' The BHS leaves no such ambiguity and can only be rendered as ``he.''} went into the city.%%
    \verse{3:16} And she came to her mother-in-law and she\footnote{Naomi} said, ``How did it go, my daughter?'' And then she told her everything that the man had done for her,%%
    \verse{3:17} saying\footnote{Literally, ``and she said''}, ``He gave me six measures because he said to me, `Don't go to your mother-in-law empty-handed.'\thinspace''%%
    \verse{3:18} Then she said, ``Sit, my daughter, until you know how the thing will play out, for the man will not be quiet until he has finished the matter today.''%%
\end{inparaenum}

    \section{Ruth 4}\label{Ruth 4}
\heading{xxxx}
\begin{enumerate}
    \verse{Ruth^4:1} And Boaz went up the the gate and he sat down there. And the levirite\footnote{The text renders this in such a fashion as to not give away the man's identity out of respect to his posterity. It can be determined, though, that this man was the levirite.} was passing by, and he\footnote{Boaz} said, ``Turn away, sit here.'' And he turned aside and sat down.%
    \verse{Ruth^4:2} And he took ten men from among the elders of the city and told them, ``Sit down here.'' And they sat down.%
    \verse{Ruth^4:3} And the levir said to his kinsman\footnote{The levir}, ``Naomi~--- who is returned from the plains of Moab~--- is selling the land of our kinsman Elimelech.%
    \verse{Ruth^4:4} Now, let me speak to you\footnote{Lit., I will uncover thy ear} to acquire it before\footnote{in the presence} of those who are sitting here\footnote{by} and before the elders of my people. If you are going to redeem it, then redeem it. If he will not\footnote{It seems that at this point Boaz has turned and is addressing the people standing by.} redeem it, tell me so that I can know since there is no one besides you to redeem it\footnote{I am next after you}.'' And he said, ``I will redeem it.''%
    \verse{Ruth^4:5} And Boaz said, ``When you\footnote{In the day that you} acquire the property\footnote{land, field} from the hand of Naomi, you will acquire Ruth the Moabitess~--- the wife of the\footnote{aforementioned} deceased kinsman~--- to raise up the name of the deceased over\footnote{upon} his inheritance.''%
    \verse{Ruth^4:6} Then the levir\footnote{redeemer} said to him, ``I will not be able to redeem it for myself lest I dilute\footnote{mar} my inheritance. Redeem my right\footnote{i.e., the property that I would redeem} to thyself for I cannot redeem it.''%
    \verse{Ruth^4:7} (Now this was the manner of redemption in Israel in order to effect\footnote{concerning} redemption or exchange\footnote{of Levirite marriage or property} in order to establish anything. The man took off his sandal and gave it to his neighbor as a testimony to his neighbor.)%
    \verse{Ruth^4:8} The levir said to Boaz, ``Acquire it for thyself.'' And he took off his sandal\footnote{In Deuteronomy 25:5-10 we learn that the spurned woman~--- Ruth in this case~--- would have removed the man's sandal and then spit in his face.}.%
    \verse{Ruth^4:9} So Boaz said to the elders and to all of the people, ``Today you are all witnesses that I have acquired all that which belonged to Elimelech, Chilion, and Mahlon, from the hand of Naomi.%
    \verse{Ruth^4:10} And also Ruth the Moabitess~--- the wife of Mahlon~--- have I obtained\footnote{Lit., \dots for myself} to be my wife, to raise up the name of the deceased over his inheritance that the name of the deceased be not cut off from among his brethren and from the gate of his place\footnote{dwelling}: of this ye are witnesses this day.''%
    \verse{Ruth^4:11} Then all the people that were within the gates\footnote{city limits} and the elders said, ``We are witnesses\footnote{Witness!}. May the \textsc{Lord} make the wife come into thine house like Rachel and Leah who built\footnote{the two of whom built} the house of Israel\footnote{\S - Quite the thing to say especially considering there is no mention of the Patriarchs.}. Now, do thou worthily\footnote{act virtuously} in Ephratah and make a name for thyself in Bethlehem.%
    \verse{Ruth^4:12} And now let thy house be as the house of Pharez, whom Tamar bare for Judah, of the seed which the \textsc{Lord} will give thee of this young woman.''%
    \verse{Ruth^4:13} So Boaz took Ruth unto himself to be his wife. And he went in unto her and the \textsc{Lord} gave unto her conception\footnote{pregnancy} and she bare a son.%
    \verse{Ruth^4:14} And the women\footnote{female neighbors} said unto Naomi, ``Blessed be the \textsc{Lord} who hath not abandoned thee this day without a redeemer: that His name may be glorified in Israel.%
    \verse{Ruth^4:15} And He shall restore your soul\footnote{shall be a restorer of life unto thee} and sustain\footnote{support you in} your old age because your daughter-in-law, who loves you, hath given birth to him, and she is better to you than seven sons.''%
    \verse{Ruth^4:16} And Naomi took the child\footnote{boy} and laid him\footnote{Lit., it} on her bosom and became his nurse.%
    \verse{Ruth^4:17} And the women gave him a name, saying, ``A son is born to Naomi.'' And they called his name\footnote{named him} Obed. He is the father of Jesse the father of David.%
    \verse{Ruth^4:18} % [Use term: "sired"]
    \verse{Ruth^4:19} %
    \verse{Ruth^4:20} %
    \verse{Ruth^4:21} %
    \verse{Ruth^4:22} %
\end{enumerate}

Epilogue

The sum total of the verses of this manuscript: 75

And the end of it is Ruth the Moabitess said XXXXXXXX

There are two sections

    \chap{1 Samuel}{\Hebrew{א שמואל}}
    \heading{15}{Samuel commands Saul to annihilate the Amalekites~--- Saul and the people disobey God, spare Agag, and bring back some of the animals to sacrifice~--- Saul is sharply reprimanded by Samuel, denies any wrongdoing, and is stripped of his royalty}

\begin{inparaenum}
    \verse{15:1} And Samuel spake unto Saul, ``The \textsc{Lord} didst send me to anoint thee king over His people~--- over Israel~--- therefore, now, hearken unto the voice\footnote{sound} of the words of the \textsc{Lord}:%%
    \verse{15:2} Thus saith the \textsc{Lord} of Hosts, `I have reviewed\footnote{looked after} that which Amalek did to Israel, that which he placed for himself in the way going up out of Egypt.%%
    \verse{15:3} Go now, and smite Amalek and all that he hath; show no mercy\footnote{have no pity on them}. Put to death every man, woman, infant, suckling, ox, sheep, camel, and ass.'''%%
    \verse{15:4} So Saul summoned the people and inspected them in Telaim: 200,000 footmen and 10,000 men of Judah.%%
    \verse{15:5} Then Saul came unto the city of Amalek and waited in a valley.%%
    \verse{15:6} And Saul saith to the Kenite, ``Go, turn aside. Leave from the midst of Amalek lest I consume thee as well, for thou didst show kindness unto the children of Israel when they came up out of Egypt.'' So the Kenite turned aside from the midst of Amalek.%%
    \verse{15:7} Then Saul smote Amalek from Havilah to Shur (on the borders of Egypt).%%
    \verse{15:8} He caught Agag, king of Amalek, alive; but all the people were destroyed by the sword.%%
    \verse{15:9} But Saul~--- and the people as well~--- had pity on Agag, the best of the flocks, herds, garments, rams, and all that was beautiful and they would not destroy them.%%
    \verse{15:10} Then the word of the \textsc{Lord} came unto Samuel, saying,%%
    \verse{15:11} ``I'm sorry that I made Saul king: he hath forsaken Me in that he did not keep My commandments.'' And Samuel was grieved and cried unto the \textsc{Lord} all night.%%
    \verse{15:12} So Samuel rose early in the morning to meet Saul. And it was declared to Samuel, saying, ``Saul hath come in to Carmel and is setting up to himself a monument. Then he shall go round, pass over, and go down into Gilgal.''%%
    \verse{15:13} Samuel came to Saul, and Saul saith to him, ``Blessed be thou of the \textsc{Lord}; I have done what the \textsc{Lord} hath said.''%%
    \verse{15:14} So Samuel said, ``Why then do I hear bleating of sheep and the noise of a herd?''%%
    \verse{15:15} And Saul said, ``We brought them from Amalek because the people had pity on the best of the flocks and herds so that we could sacrifice them to the \textsc{Lord} thy God; the rest we destroyed.''%%
    \verse{15:16} Samuel said to Saul, ``Hold on,\footnote{Relax, wait} let me tell you what the \textsc{Lord} told me tonight.'' And Saul said, ``Go on.''%%
    \verse{15:17} Samuel said, ``When you were little\footnote{in your own eyes}, were you not made head of the tribes of Israel and anointed by the \textsc{Lord} to be king over Israel?%%
    \verse{15:18} And the \textsc{Lord} sent you on a journey and said, `Go and utterly destroy the Amalekites and all that they hath.'%%
    \verse{15:19} Why then did you not obey the voice of the \textsc{Lord}, but instead did fly on their spoil and do evil in the eyes of the \textsc{Lord}?''%%
    \verse{15:20} And Saul said to Samuel, ``I \emph{have} hearkened to the voice of the \textsc{Lord}. I went the way the \textsc{Lord} sent me, and I have brought Agag, king of Amalek, and destroyed the Amalekites.%%
    \verse{15:21} And the people took the spoil of the flocks and herds\footnote{Idiomatically rendered in definite plural}~--- the choicest\footnote{chief, first, best} part of the devoted things\footnote{i.e., that which should have been destroyed}~--- to sacrifice unto the \textsc{Lord} their God in Gilgal.''%%
    \verse{15:22} And Samuel saith:
    
:``Doth the \textsc{Lord} delight in sacrifices and burnt offerings So much as He delighteth\footnote{Poetic: therefore the verb's repetition is implied.} in hearkening to the voice\footnote{Implied: commandments} of the \textsc{Lord}?

:Listen! Obedience is better than sacrifice\footnote{Interesting Hebrew grammatical construct here. It is literally ``Obedience sacrifice good.'' That is how comparatives are done.}. Paying attention is more important than ram's fat.%%
    \verse{15:23} :Since rebellion is like the sin of witchcraft And stubbornness is like idolatry:

:Because you've rejected the word of the \textsc{Lord} He hath you from being king.%%
    \verse{15:24} Then Saul said to Samuel, ``I have sinned, for I transgressed the commandment of the \textsc{Lord}\footnote{Literally, I have passed over (or by) the command (or mouth) of the \textsc{Lord}} and thy words\footnote{Repetitious because they're the same in this instance.\footnote{cf. DyC 1:38} Shows a serious lack of understanding on Saul's part.} in that I showed reverence to\footnote{feared} the people by hearkening\footnote{listening} to their voice.\footnote{i.e., instead of the \textsc{Lord}}%%
    \verse{15:25} Please forgive me\footnote{Literally, Now, please bear my sin} and come with me as I bow down before the \textsc{Lord}.''%%
    \verse{15:26} Samuel said to Saul, ``No, I'm not going with you\footnote{Literally, I will not turn back with thee} because you've rejected the word of the \textsc{Lord} and because the \textsc{Lord} hath rejected you from being king over Israel.''%%
    \verse{15:27} Then Samuel turned around to go, but he\footnote{Saul} caught hold of his\footnote{Samuel's} robe's mantle\footnote{upper skirt} and it rent.%%
    \verse{15:28} And Samuel said to him, ``The \textsc{Lord} hath rent the Kingdom of Israel from thee today and given it to thy neighbor (who is better than thee).%%
    \verse{15:29} Additionally, the eminence\footnote{preeminence, perpetuity, everlastingness} of Israel\footnote{Is this a name-title of the \textsc{Lord}?} neither lies nor repents for He is not a man, that He is penitent.''\footnote{repents, is repentant}%%
    \verse{15:30} And he said, ``I have sinned. Come, now, and honor me before the elders of the people\footnote{This is horribly pretentious on Saul's part: if Samuel wouldn't bow down to the \textsc{Lord} with Saul, why in the world would he honor him before the elders of the people and all of Israel? It's like Saul doesn't think before he speaks.} and before Israel. Come with me as I bow down before the \textsc{Lord}.''%%
    \verse{15:31} So Samuel turned back after Saul as Saul bowed before the \textsc{Lord}.\footnote{I know, I'm as shocked as you.}%%
    \verse{15:32} And Samuel said, ``Bring Agag, king of Amalek, unto me.'' And Agag came in unto him delicately and said, ``Certainly the bitterness of death has past.''%%
    \verse{15:33} Samuel said, ``As thy sword hath bereaved women, even so shall thy mother be bereaved among women,'' and Samuel hewed Agag into pieces before the \textsc{Lord} in Gilgal.%%
    \verse{15:34} Samuel went to Ramath and Saul went to his house in Gibeah of Saul.%%
    \verse{15:35} Samuel never again came to see Saul, even until his death; nevertheless, Samuel mourned for Saul. And the \textsc{Lord} was sorry that he had made Saul king over Israel.%%
\end{inparaenum}

    \chap{2 Samuel}{\Hebrew{ב שמואל}}
    \heading{6}{xxxx}

\begin{inparaenum}
    \verse{6:1} David once again gathered every firstborn in Israel, 30\thinspace000.%%
    \verse{6:2} David and all the people that were with him got up and went from Baale-Judah\footnote{Known as Kiriath-jearim in 1~Chronicles~13\thinspace:\thinspace6.} to take up the Ark of God. God,\footnote{Understood} whose name has been invoked upon it~--- that name being ``The \textsc{Lord} of Hosts who sits between the cherubs.''%%
    \verse{6:3} The Ark of God rode on a new wagon.\footnote{KB: not a chariot.} They lifted it up from Abinadab's house in Gibeah. Uzzah and Ahio, Abinadab's sons, led the new wagon.%%
    \verse{6:4} %%
    \verse{6:5} %%
    \verse{6:6} They came unto Nachon's threshing floor and Uzzah put forth his hand\footnote{Understood} to the ark of God and took hold of it because the oxen stumbled.%%
    \verse{6:7} And the \textsc{Lord} was exceedingly wroth with Uzzah,\footnote{And the anger of the \textsc{Lord} burned against Uzzah} so God smote him there for his error that he died there before the ark of God.\footnote{Referring to this passage (v. 6-7), John Taylor said, ``The ark of God does not need steadying, especially by incompetent men without revelation and without knowledge of the kingdom of God and its laws.'' (\textit{The Gospel Kingdom}, 166)}%%
    \verse{6:8} %%
    \verse{6:9} %%
    \verse{6:10} %%
    \verse{6:11} %%
    \verse{6:12} %%
    \verse{6:13} %%
    \verse{6:14} %%
    \verse{6:15} %%
    \verse{6:16} %%
    \verse{6:17} %%
    \verse{6:18} %%
    \verse{6:19} %%
    \verse{6:20} %%
    \verse{6:21} %%
    \verse{6:22} %%
    \verse{6:23} %%
\end{inparaenum}

    \chap{1 Kings}{\Hebrew{א מלכים}}
    \section{17}\label{1 Kings 17}
\heading{Elijah prophesied no rain~--- Elijah stays with the widow of Zarephath~--- the widow's son dies~--- the Lord heals the lad}
\begin{enumerate}[align=center]
    \verse{17:1} %%
    \verse{17:2} %%
    \verse{17:3} %%
    \verse{17:4} %%
    \verse{17:5} %%
    \verse{17:6} %%
    \verse{17:7} %%
    \verse{17:8} And the word of the \textsc{Lord} came unto him, saying,%%
    \verse{17:9} ``Arise. Go unto Zarephath~--- that is, unto Zidon~--- and dwell there. Behold! I have there commanded a widow woman to sustain thee.''%%
    \verse{17:10} So he arose and went unto Zarephath, came unto the city's entrance, and lo! there was a widow woman gathering sticks. So he called unto her and said, ``Please bring me a little water in a vessel so I can drink.''%%
    \verse{17:11} So she went to bring it and he calleth after her, and said, ``Please bring me some bread in thy hand.''%%
    \verse{17:12} Then she said, ``As the \textsc{Lord} thy God liveth, I don't even have a cake, but only a handful of meal in a pitcher and a little oil in a dish. Behold, I am gathering two sticks so that I can go in, prepare it for myself and my son, so that we can eat it and then die.''%%
    \verse{17:13} Elijah said unto her, "Fear not: go and do as I've said, only make for me from thence a little cake first and bring it out to me. Then make for thee and thy son last.%%
    \verse{17:14} ``For thus saith the \textsc{Lord} God of Israel, `The pitcher of meal shall not be consumed and the oil dish shall not be lacking until the day the \textsc{Lord} makes it rain on the face of the land.' ''%%
    \verse{17:15} So she went and did as Elijah had said. And she ate~--- she, he, and her household~--- for days.%%
    \verse{17:16} And the pitcher of meal was not consumed, neither did the oil dish lack, even according to the word of the \textsc{Lord} that He had spoken by the hand of Elijah.%%
    \verse{17:17} After these things, the son of the woman (the mistress of the house) became sick. And his sickness was so severe that there was no breath left in him.%%
    \verse{17:18} And she said unto Elijah, ``What's this to me and thee, O man of God? You came to me to make me remember my iniquity and to kill my son!''%%
    \verse{17:19} So he said to her, ``Give me your son.'' And he took him from her bosom, and took him to the upper chamber where he was abiding, and laid him on the bed.%%
    \verse{17:20} Then he cried unto the \textsc{Lord} and said, ``O \textsc{Lord} my God, have you really brought evil upon the widow with whom I sojourn by killing her son?''%%
    \verse{17:21} He stretched himself upon the child thrice and cried to the \textsc{Lord} and said, ``O \textsc{Lord} my God, please let this child's soul return to him!''%%
    \verse{17:22} And the \textsc{Lord} heard the voice of Elijah and the child's soul returned and he lived.%%
    \verse{17:23} And Elijah took the child and brought him down from the upper room into the house and gave him to his mother. And Elijah said, ``Look, your son lives.''%%
    \verse{17:24} And the woman said to Elijah, ``By this I now know that thou art a man of God and that the word of the \textsc{Lord} in thy mouth is true.''%%
\end{enumerate}

    \section{18}\label{1 Kings 18}
\heading{xxxx}
\begin{enumerate}[align=center]
    \verse{18:1} %%
    \verse{18:2} %%
    \verse{18:3} %%
    \verse{18:4} %%
    \verse{18:5} %%
    \verse{18:6} %%
    \verse{18:7} %%
    \verse{18:8} %%
    \verse{18:9} %%
    \verse{18:10} %%
    \verse{18:11} %%
    \verse{18:12} %%
    \verse{18:13} %%
    \verse{18:14} %%
    \verse{18:15} %%
    \verse{18:16} %%
    \verse{18:17} It was so that when Ahaz saw Elijah, Ahaz said unto him, ``Are you the troubler\footnote{the one who causes trouble} of\footnote{in} Israel?''%%
    \verse{18:18} So he\footnote{Elijah} said, ``I have not troubled Israel, rather you and your father's house have in forsaking the commandments of the \textsc{Lord} and going after Baalim.\footnote{false gods. The definite article precedes the noun so ``Baalim'' is probably more accurate than ``false prophets.''}%%
    \verse{18:19} Send now and gather all of Israel unto me at Mount Carmel as well as the 450 prophets of Baal and the 400 prophets of Asherah who eat at Jezebel's table.''%%
    \verse{18:20} So Ahaz sent unto all the children of Israel and gathered all the prophets unto Mount Carmel.%%
    \verse{18:21} Elijah drew near to all of the people and said, ``How long will you stall between two opinions?\footnote{Lit., How long will you hop between two branches?} If the \textsc{Lord} is God, follow him; but if Baal, then follow him.'' And the people didn't say a thing.%%
    \verse{18:22} Then Elijah said unto the people, ``I alone remain a prophet of the \textsc{Lord}, yet the prophets of Baal are 450 men.%%
    \verse{18:23} Alright, have them\footnote{the false prophets} bring\footnote{Lit., let them give us} two bullocks: they shall choose one bullock for themselves, cut it in pieces, and place it on wood, but they shall not put any fire on it;\footnote{Lit., place no fire} and I shall prepare the other bullock, place it on wood, and not light it.%%
    \verse{18:24} You shall call on the name of your god and I will call on the name of the \textsc{Lord}\footnote{That's pretty powerful language (i.e., the Tetragrammaton) to be using around heathens.}. And thus it shall be that the god who answers by fire, He is God.'' And all the people answered and said, ``Sounds good.''\footnote{Lit., Good is the word.}\footnote{This brings up an interesting thought: either the prophets of Baal had before been able to call down fire from heaven or they knew that since they couldn't do it that Elijah couldn't either, and they would use that in the soon to come (or so they thought) argument.}%%
    \verse{18:25} And Elijah said unto the prophets of Baal, ``Choose a bullock and prepare it first because there are more of you.\footnote{you are the multitude, you are many} But don't put any fire on it, merely call on the name of your god.''\footnote{Lit., Call on the name of your god and (but) place no fire.}%%
    \verse{18:26} So they took the bullock given to them, prepared it, and called on the name of Baal from the morning until noon, saying, ``Baal, answer us!'' Yet there was no voice\footnote{sound} and no one answered. And they leapt on the altar they had made.%%
    \verse{18:27} And when it was midday, Elijah toyed with them and said, ``Call with a loud voice for he is a god and maybe he is meditating or busy or gone on an adventure.\footnote{journey} Perhaps he is asleep and just waking up!''%%
    \verse{18:28} So they cried with a loud voice and cut themselves (according to their ordinances) with swords and spears until their blood flowed.%%
    \verse{18:29} And when midday had passed, they prophesied until the time of the evening sacrifice, and yet there was neither a voice nor an answer nor any that regarded them.%%
    \verse{18:30} Then Elijah said unto all the people, ``Come here.'' So all the people drew near as he repaired the altar of the \textsc{Lord} that was broken down.%%
    \verse{18:31} And he took twelve stones (according to the number of the tribes of the sons of Jacob~--- unto whom the word of the \textsc{Lord} was, saying, ``Thy name shall be Israel.'')%%
    \verse{18:32} and built an altar from the stones in the name of the \textsc{Lord}. And he made a trench about the space of two measures of seed encircling the altar.%%
    \verse{18:33} And he arranged the wood, cut the bullock in pieces, and placed it on the wood.%%
    \verse{18:34} \footnote{V. 33 still in English translations.} And he said, ``Fill four pitchers of water and pour them on the burnt offering and on the wood.'' And he told them to do it a second time, and they did it a second time; and he told them to do it a third time, so they did it a third time.%%
    \verse{18:35} \footnote{Back to normal versing.} And the water encircled the altar insomuch that the trench was filled with water.%%
    \verse{18:36} And at eventide, Elijah the prophet drew near and said, ``\textsc{Lord} God of Abraham, Isaac, and Israel: let it be known today that Thou art God in Israel, that I am Thy servant, and that by Thy word I have done all of these things.%%
    \verse{18:37} Answer me, \textsc{Lord}. Answer me and this people shall know that Thou art the \textsc{Lord} God: that Thou hast turned back their heart.''\footnote{i.e., That they shall know that Thou, Lord, hast turned back their hearts. Not me. Not this miracle that Thou shalt perform. But that Thou has done this thing.}%%
    \verse{18:38} Then the fire of the \textsc{Lord} fell and consumed the burnt offering, wood, stones, and dust; it licked up the water that was in the trench.%%
    \verse{18:39} And all the people saw it, fell on their faces, and said, ``The \textsc{Lord} is God! The \textsc{Lord} is God!''%%
    \verse{18:40} Elijah said unto them, ``Seize the prophets of Baal~--- don't let one of them escape!'' So they seized them. Then Elijah brought them down to the brook\footnote{torrent} Kishon and slew them there.\footnote{That's 450 men. This doesn't even talk about the other 400 false prophets of Asherah who were probably slain as well.}%%
    \verse{18:41} Elijah said unto Ahab, ``Get up. Eat. Drink. You can hear a bunch of rain coming.''\footnote{There is a sound of an abundance of rain.}%%
    \verse{18:42} So Ahab went up to eat and drink. Elijah went up to the summit of Carmel, bowed down on the earth, and put his face between his knees.%%
    \verse{18:43} Then he said unto his servant, ``Go up\footnote{Go up where? They're at the summit.} and look at the sea.'' So he went up, looked, and said, ``There's nothing.''\footnote{I don't see anything.} So Elijah said, ``Go again seven times.''%%
    \verse{18:44} And on the seventh time he said, ``Behold! There is a cloud~--- small as a man's hand~--- rising up out of the sea!'' And he\footnote{Elijah} said, ``Go up and say to Ahab, `Prepare\footnote{i.e., thy chariot} and go down that the rains do not restrain thee.'''%%
    \verse{18:45} In the meantime the heavens blackened with clouds and wind, and there was a great pouring of rain. And Ahab got on his chariot and went to Jezreel.%%
    \verse{18:46} Thus the hand of the \textsc{Lord} was upon Elijah, and he girded up his loins and ran before Ahab to the border of Jezreel.%%
\end{enumerate}

    \chap{2 Kings}{\Hebrew{ב מלכים}}
    \section{2 Kings 2}\label{2 Kings 2}
\heading{xxxx}
\begin{enumerate}[align=center]
    \verse{2 Kings^2:1} %%
    \verse{2 Kings^2:2} %%
    \verse{2 Kings^2:3} %%
    \verse{2 Kings^2:4} %%
    \verse{2 Kings^2:5} %%
    \verse{2 Kings^2:6} %%
    \verse{2 Kings^2:7} %%
    \verse{2 Kings^2:8} %%
    \verse{2 Kings^2:9} %%
    \verse{2 Kings^2:10} %%
    \verse{2 Kings^2:11} %%
    \verse{2 Kings^2:12} %%
    \verse{2 Kings^2:13} %%
    \verse{2 Kings^2:14} %%
    \verse{2 Kings^2:15} %%
    \verse{2 Kings^2:16} %%
    \verse{2 Kings^2:17} %%
    \verse{2 Kings^2:18} %%
    \verse{2 Kings^2:19} %%
    \verse{2 Kings^2:20} %%
    \verse{2 Kings^2:21} %%
    \verse{2 Kings^2:22} %%
    \verse{2 Kings^2:23} And he went from there up to Bethel, and as he was going up on the highway, little boys came out of the city and mocked him, saying unto him, ``Go up, bald head! Go up, bald head!''%%
    \verse{2 Kings^2:24} And he turned back, faced them, and cursed them in the name of the \textsc{Lord}. And two mama bears came forth and out of the woods and tore 42 children among them.%%
    \verse{2 Kings^2:25} He then went from there up to Mount Carmel, and from there he returned to Samaria.%%
\end{enumerate}

    \heading{5}{Naaman, an Aram\ae{}an, beseeches Elisha to cure him of his skin disease~--- Elisha, through his servant, commands Naaman to wash seven times in the Jordan River~--- at first, he disbelieves, but after being prompted by his own servant, he obeys and is healed~--- Naaman tries to give a gift to Elisha, is refused~--- Elisha's servant accepts and is struck with a similar skin condition}

\begin{inparaenum}
  \verse{5:1} And Naaman the head of the king of Aram's\ie{Syria} army was a great and honorable man before the his superior officer\alt{lord, master} because through him the \textsc{Lord} delivered Aram. Yea, he was a mighty man of valor, but he was a leper.%%
  \verse{5:2} And the Aram\ae{}ans had gone forth in bands and captured a little maid from out of the land of Israel; and she waited upon Naaman's wife.%%
  \verse{5:3} And she said unto her maidservant, ``I wish that my lord was in the presence of the prophet in Samaria! Then he might be healed of his leprosy.''%%
  \verse{5:4} So he went and told his lord, saying, ``Thus and thus said the maid that is from the land of Israel.''%%
  \verse{5:5} The king of Aram said, ``Go! I will send a letter to the king of Israel.'' So he went and took 10~talents of silver,\ed{\textsc{halot} states that a talent is ``normally about 35~kg\thinspace=\thinspace75~lbs.'' Ten talents of silver would be equal to roughly 750~lbs (340~kg) of silver, or \$215,687.50 as of 2014-04-26.} 6\thinspace000~gold pieces, and 10~changes of clothes with him.\lit{in his hand; however, this is impossible with this amount of silver.}%%
  \verse{5:6} He brought the letter to the king of Israel, saying, ``Now, as this letter comes to you, see that I have sent my servant, Naaman, to you that he may cure your skin disease.''%%
  \verse{5:7} When the king of Israel had read the letter, he rent his garments and said, ``Am I God? Can I\understood\ kill and give life? Because this man\understood\ has sent me to heal a man of his skin condition. Please be concerned about this and see because he's presenting himself to me.''%%
  \verse{5:8} When Elisha, the man of God, had heard that the king of Israel had rent his garments, he sent to the king, saying, ``Why have you rent your clothes? Please let him come to me and he shall know that there's a prophet in Israel.''%%
  \verse{5:9} So Naaman came with his horse and chariot and stood in the doorway of Elisha's house.%%
  \verse{5:10} Elisha sent a messenger to him, saying, ``Go. Bathe in the Jordan River seven times and your skin will be clean\alt{pure, genuine, (cultically) clean} again.''\alt{turn back, return, revert}%%
  \verse{5:11} And Naaman became angry and left. He said, ``Look, I said that\understood\ he would definitely come out to see me and stand and call on the name of the \textsc{Lord} his God and wave his hand over the place\ie{the affected place} and heal the skin condition.%%
  \verse{5:12} Aren't the Abana\ed{Modernly, the Barada River, the main river of Damascus, Syria.} and Pharpar\ed{Modernly, either the A`waj or Taura River in Damascus, Syria.} rivers in Damascus better than all the waters of Israel? May I not\alt{Can't, Shouldn't} just\understood\ wash in them and be clean?'' And he turned and left in a fury.\alt{anger}%%
  \verse{5:13} His servant came near, spoke to him, said, ``My father,\halot{xxxx}{fatherly \textbf{protector}, honorable title: of one's elder, prophet, husband.} if\understood\ the prophet had commanded you to do some great thing, wouldn't you have done it? How much more then when he tells you, `Wash, and be clean?'\thinspace''%%
  \verse{5:14} He went down and dipped in the Jordan River seven times according to the words of the man of God. His skin became\alt{turned back, returned} like a young man's\lit{a young (little) boy (youth, young man). However, \Hebrew{קָטֹן} is redundant in an English rendering and \textit{young man} suffices.} skin. He was clean.%%
  \verse{5:15} He and his entire company returned to the man of God. He came and stood before him and said, ``Please look. I know that there is no god in the entire world except in Israel. Now, please receive\lit{take. This doesn't work well idiomatically.} a blessing from your servant.''%%
  \verse{5:16} He said, ``As the \textsc{Lord}, before whom I stand, lives, I cannot receive it.'' He strongly urged him to take it, but he refused.%%
  \verse{5:17} Naaman said, ``If not, then please let a mule team's load of earth be given to your servant because he\lit{your servant} shall make no more burnt offerings and sacrifices to other gods, only to the \textsc{Lord}.%%
  \verse{5:18} In this matter, may\understood\ the \textsc{Lord} forgive your servant. When my master comes to the house in Rimmon,\ed{A Syrian cult image, only mentioned in this verse, identified as Baal.} bows himself there, and leans on my hand as I bow in the house of Rimmon~--- when I bow in the house of Rimmon, may the \textsc{Lord} please forgive your servant in this matter.''%%
  \verse{5:19} He said to him, ``Go in peace,'' and he went away a little.%%
  
  \verse{5:20} Gehazi, the servant of Elisha the man of God, said, ``Look, my master has taken good care of Naaman, this Aram\ae{}an, not to receive from his hand what he'd brought. As the \textsc{Lord} lives, I will run after him and take something from him,''%%
  \verse{5:21} so Gehazi pursued\lit{pursued after} Naaman. And Naaman saw someone running after him and he got down from his chariot to meet him, and said, ``Is everything alright?''\lit{Is there peace?}%%
  \verse{5:22} And he said, ``Everything's alright.\lit{Peace, All is well.} My master has sent me, saying, `Pay attention to this right now.\lit{Look now this.} Two young men from Mount Ephraim, sons of the prophets, are coming to me. Please give them a talent of silver\ed{75~lbs, \$21,284.38 as of 2014-05-03} and two changes of clothes.'\thinspace''%%
  \verse{5:23} Naaman said, ``Agree to take two talents,''\ed{dual form} and he urged him and tied up two talents of silver in two bags,\halot{xxxx}{xxxx can't find this noun in either Lexicon} two changes of clothes, and gave it to two of his servants who bore it before him.%%
  \verse{5:24} When he came to the Ophel,\ed{A high, fortified part of a city, either in the City of David or in the Old City of Jerusalem. Sometimes refers to a place in Samaria, the ancient capital of the kingdom of Israel.} he took them out of their possession\lit{hand} and put them away in the house. He sent the men away and they left.%%
  \verse{5:25} He entered and stood before his master. And Elisha said to him, ``Where did you come from,\lit{Whence} Gehazi?'' And he replied, ``Your servant didn't go here nor there.''%%
  \verse{5:26} He said to him, ``My heart didn't go when the man turned from his chariot to meet you.  Is it time to take silver? To take clothes? Olive yards, vineyards, sheep, oxen, servants, maids?%%
  \verse{5:27} Naaman's skin condition shall cling\alt{cleave} to you and your posterity forever.'' So he went out from his presence, suffering from a skin eruption like snow.%%
\end{inparaenum}

    \heading{6}{Elisha miraculously causes an ax head to float~--- the Syrian king surrounds Elisha on a hill~--- Elisha's servant is afraid so Elisha prays for his eyes to be opened and the servant sees angelic hosts surrounding and protecting them~--- Samaria is besieged and the people suffer~--- the Syrian king commands Elisha's head to be struck from his body, but Elisha is forewarned}

\begin{inparaenum}
    \verse{6:1} The sons of the prophets said to Elisha, ``Please, the place that we've been living in before you is too narrow for us.%%
    \verse{6:2} Please let us go to the Jordan River\understood\ and every one will bring a beam\footnote{i.e., a framing beam} and there we'll make a place for ourselves where we can live.'' So he said, ``Go.''%%
    \verse{6:3} One of them\understood\ said, ``Please agree and go with your servants,'' and he said, ``I will go.''%%
    \verse{6:4} He went with them and they came to the Jordan River\understood\ and cut down the trees.%%
    \verse{6:5} As one of them\understood\ was felling the beam, the iron fell into the water, and he cried out and said, ``Ah!\footnote{KB: a cry for help} My master!'' he asked.%%
    \verse{6:6} The man of God said, ``Where did it fall?'' And he showed him the place. He cut a stick and threw it out there and made the iron float.%%
    \verse{6:7} He said, ``Pick it up.''\footnote{lit., Pick it up to you.} And he stretched out his hand and took it.%%
    
    \verse{6:8} The king of Aram\footnote{Syria} hath been fighting against Israel and took counsel with his servants, saying, ``My encamping is wherever.''\footnote{Seriously ``wherever.'' They use \Hebrew{פְּלֹנִי אַלְמֹנִי} which means ``whoever'' or ``whatever.'' It's the same usage as found in Ruth 4\thinspace:\thinspace1 to obfuscate who or what is being referred to.}%%
    \verse{6:9} The man of God sent unto the king of Israel, saying, ``Beware of passing through this place for the Aram\ae{}ans are coming down thence.''%%
    \verse{6:10} So the king of Israel sent to the place of which the man of God had told and warned him. And he\footnote{the king of Israel} stayed on guard. This happened not once, but twice.%%
    \verse{6:11} And the heart of the king of Aram was troubled because of this thing\footnote{i.e., these words}, so he called his servants and saith unto them, ``Will you not tell me which of us is for the king of Israel?''\footnote{i.e., Is there a double agent among us?}%%
    \verse{6:12} One of the servants said, ``None, my lord the king. However, Elisha, the prophet that is in Israel, tells the king of Israel the things you have spoken in private.''\footnote{in thy bedchamber.}%%
    \verse{6:13} Then he said, ``Go and see where he is. Then I will send for and fetch him.'' It was then told him, saying, ``He is in Dothan.''%%
    \verse{6:14} So he sent forth horses, chariots, and a great host, and they came by night and surrounded the city.%%
    \verse{6:15} And the man of God's servant arose early and went out, and lo! an army, horses and chariots, surrounded the city. Then his servant said unto him, ``My lord, what shall we do?''\footnote{lit., how do we do? Or, how will we do?}%%
    \verse{6:16} And he said, ``Don't be afraid~--- there are more with us than with them.''\footnote{alt., greater are they who are with us than they who are with them.}%%
    \verse{6:17} And Elisha prayed and said, ``\textsc{Lord}, I pray that thou wilt open his eyes and let him see.'' So the \textsc{Lord} opened the servant's eyes and he saw, and lo! the mountain was full of fiery horses and chariots surrounding Elisha.%%
    \verse{6:18} They went down to him and Elisha prayed to the \textsc{Lord} and said, ``Please smite this nation with blindness.'' And he smote them with blindness according to the word of Elisha.%%
    \verse{6:19} Elisha said to them, ``This is not the way. This is not the city. Follow me and I will bring you to the man whom you seek.'' And he led them to Samaria.%%
    \verse{6:20} When they entered Samaria, Elisha said, ``O \textsc{Lord}, open their eyes so that they can see!'' And the \textsc{Lord} opened their eyes and they saw. And they were in the midst of Samaria.%%
    \verse{6:21} When the king of Israel saw them, he said to Elisha, ``My father, shall I certainly smite them?''%%
    \verse{6:22} He said, ``Don't smite them. Would you smite those whom you've taken captive with your sword and your bow? Place bread and water before them so that they can eat and drink and return to their master.''%%
    \verse{6:23} He prepared many provisions for them, and they ate and drank. He sent them and they returned to their master and the Aram\ae{}an robbers\footnote{alt., military troops, raiding parties} never again came\footnote{lit., did no more continue to come} into the land of Israel.%%
    
    \verse{6:24} After this, Ben-hadad, king of Aram, assembled his entire army, went up, and besieged Samaria.%%
    \verse{6:25} There was a great famine in Samaria and they besieged it until a donkey head was worth\footnote{lit., was fattened to} silver\footnote{xxxx: it either says that a donkey head was fattened to (had accrued worth to the measure of) silver, or that a donkey head was worth (understood) fifty silver. The other translations give it as being worth eighty silver, but this seems to be not attested in the text.} and a fourth of a cav\footnote{alt. rendered \textit{cab}}\footnote{KB: a measure of capacity, about 1.5~L (1.33~qt).} of doves' dung\footnote{Some translations render this as ``seed pods'' (\textsc{niv}) or ``wild onions'' (\textsc{njb}), but the KB states that it is ``doves' dung'' (note that it is given as plural possessive where other translations render it as singular possessive). ``The Geneva Bible posits that the dung was used as a fuel for fire. Jewish historian Josephus suggested that dove's dung could have been used as a salt substitute. An alternative view is that `dove's dung' was a popular name for some other food, such as Star-of-Bethlehem or falafel. A third position, based on amending the Hebrew text, is that the passage actually refers to locust-beans, the fruit of the carob tree'' (Wikipedia). The Hebrew here is \Hebrew{חִרְייֹונִים} (the footnote gives \Hebrew{יוֺנִים חֲרֵי}), meaning ``doves' dung.''} was worth\understood\ five silver pieces.\footnote{Some translations render this as being five shekels, but there is no way to know for certain that the unit of measurement is a shekel; the Hebrew simply says ``five silver.''}%%
    \verse{6:26} As the king of Israel passed from one side of the wall to the other, a woman cried to him, saying, ``O King! Help\footnote{alt., save, aid} me,\understood\ my master!''%%
    \verse{6:27} He said, ``If the \textsc{Lord} doesn't come to your aid, from where\footnote{alt., with what} do I save you?\footnote{In other words, ``How am I supposed to save you?''} From the threshing floor? From the the wine press?''%%
    \verse{6:28} And the king said to her, ``What is troubling\understood\ you?''\footnote{lit., What to you? \textit{or} What is it to you?} And the woman replied, ``This woman said to me, `Give me your son. We'll eat him today and we'll eat my son tomorrow.'%%
    \verse{6:29} So we cooked\footnote{alt., boiled, roasted} my son and ate him. Then I said to her the next day, `Give me your son and we'll eat him,' but she hid her son.''%%
    \verse{6:30} When the king had heard the woman's words, he rent his garments. He passed from one side of the wall to the other and the people saw that he had sackcloth on his flesh within.%%
    \verse{6:31} He said, ``Thus does God do to me and more if Shaphat's son Elisha's head remain on him today!''%%
    \verse{6:32} Meanwhile, Elisha sat in his house and the elders sat with him. A man was sent before him, but before the messenger got to him, he said to the elders, ``Do you see that this murderer's son has sent someone\understood\ to remove my head? Watch for\understood\ when the messenger comes. Close the door and press him\footnote{KB: \textbf{crowd}, \textbf{press} s.one in a given direction} with the door. Isn't the sound of his master's feet following him?''%%
    \verse{6:33} While he was still\footnote{alt., yet} talking with them, the messenger came down to him and said, ``Hey! This evil is from the \textsc{Lord}. Why should I hope in\footnote{alt., wait on} \textsc{God} any more?''\footnote{alt., longer}%%
\end{inparaenum}

    \chap{2 Chronicles}{\Hebrew{ב הימים דברי}}
    \heading{26}{The people install Uzziah as king~--- Uzziah, like his father, was righteous and God made him prosper~--- he goes to war, becomes vain, and tempts God~--- he makes an illegal sacrifice and is cursed with a serious skin condition}

\begin{inparaenum}
    \verse{26:1} All the Judahites\lit{the people of Judah} took Uzziah, who was sixteen years old, and crowned\alt{made, installed} him king instead of his father, Amaziah.%%
    \verse{26:2} He built Eloth and restored it to Judah after the king lied with his fathers.\ie{after death}%%
    
    \verse{26:3} Uzziah was sixteen years old when he became king. He ruled in Jerusalem for fifty-two years. His mother's name was Jecholiah of Jerusalem.%%
    \verse{26:4} He did what was right in the eyes of \textsc{God} like everything that his father Amaziah had done.%%
    \verse{26:5} He worshiped\alt{sought, cared about, inquired regarding} God in the days of Zechariah (who understood\alt{perceived, paid attention to, considered, gave heed to, noticed} the visions of God). In the days he worshiped the \textsc{Lord}, God made him prosper.%%
    
    \verse{26:6} He went and fought\alt{do battle with, come to close quarters with} against the Philistines and made a breach in\alt{broke down} the wall of Gath, the wall of Jabneh, the wall of Ashdod. He built cities in Ashdod and Philistia.%%
    \verse{26:7} God came to his aid\alt{supported, helped} against the Philistines, the Arabians living in Gur-Baal, and the Maonites.%%
    \verse{26:8} The Ammonites gave a present to Uzziah and his name went to the entrance of Egypt because he became incredibly\lit{lifted way up, escalated. From \Hebrew{עלה} meaning to lift up.} strong.%%
    \verse{26:9} Uzziah built towers\halot{in a defensive wall} in Jerusalem by the corner gate, the valley gate, and at the corner, and he fortified\alt{strengthened} them.%%
    \verse{26:10} He built towers in the desert and hewed out\alt{quarried; other translations render this as ``dug,'' but \textsc{halot} only mentions stonework.} many wells because he had a lot of cattle, both in the lowland and in the plains;\alt{on the plateau, \Hebrew{מִישׁוֺר} meaning both plain and plateau.} he had\understood\ serfs\halot{peasant, not owning land, belonging to landlord} and wine-growers\alt{vinedresser} in the mountains and in Carmel because he loved growing stuff.%%
    
    \verse{26:11} Uzziah had an army\halot{alt., wealthy landowner, qualified, fit for military service; (large) landowner, obligated to military service and the furnishing of a certain number of men; then valiant man without regard to property; all the armed host of a people and a satrapy (a provincial governor in ancient Persia).} who was making war, going out by troop as commissioned\lit{listed by name} by the hand of Jeiel the scribe and Maaseiah the record-keeper and by the hand of Hananiah, the second after the king.%%
    \verse{26:12} The whole number of the head honchos\lit{chief fathers} of the elite troops was 2\thinspace600.%%
    \verse{26:13} With regard to the strength\alt{power; lit., hand} of the army,\alt{host} 307\thinspace500~warriors who made war with great ability\alt{power, capacity, strength, means} to support\alt{help, come to the aid of} the king against the enemies.%%
    \verse{26:14} For the entire army, Uzziah prepared shields, spears, helmets, scale-armor,\alt{coat of mail} bows, and slinging-stones for them.%%
    \verse{26:15} In Jerusalem, he made war machines\halot{(skillfully contrived) \textbf{war-machines}, spec. catapults}\ed{Not a ballista. The earliest ballistas were invented circa 400~\textsc{bc}.}~--- thought out\alt{invented} by technicians~--- to be placed\understood\ on the towers and the corners to shoot arrows and boulders.\lit{great stones} His name spread some distance\lit{out a distance} because he was extraordinarily helped  until he became strong.%%
    \verse{26:16} However, when he become strong, his heart was lifted up unto destruction. And he transgressed against the \textsc{Lord} his God by going unto the temple of the \textsc{Lord} and offering incense upon the altar of incense.%%
    \verse{26:17} And Azariah the priest went in after him with the priests of the \textsc{Lord}: eighty valiant men.%%
    \verse{26:18} And they withstood Uzziah the king and said unto him, ``This is not for thee, Uzziah, to burn incense unto the \textsc{Lord}; rather for the priests~--- the sons of Aaron~--- that are set apart\alt{consecrated, sanctified} to burn incense. Leave this holy place\lit{Go out from this sanctuary} for you have transgressed. And neither shall this be for thine honor from the \textsc{Lord} God.''%%
    \verse{26:19} Uzziah was wroth (and in his hand he had a censer of incense). And while he was angry at the priests, leprosy appeared\lit{rose up} in his forehead in the presence of the priests in the House of the \textsc{Lord} next to the altar of incense.%%
    \verse{26:20} Azariah, the chief priest, and all the priests looked upon him, and lo! he was leprous in his forehead. So they thrust him out from thence, yea, even he\ie{Uzziah} hurried out because the \textsc{Lord} has smitten him.%%
    \verse{26:21} Uzziah the king was leper unto his dying day~--- he dwelt in a separate house (being a leper) because he was cut off from the House of the \textsc{Lord}. And Jotham, his son, took over\lit{was over} the king's affairs,\lit{house} judging the people of the land.%%
    \verse{26:22} The rest\alt{remainder} of the history of Uzziah, the first and the last, was written by the prophet Isaiah, the son of Amoz.%%
    \verse{26:23} Uzziah lay with his fathers,\ie{in death} and they buried him with his fathers in the kings'\lit{that the kings have} cemetery,\lit{field of the graves} for the said, ``He had a skin condition.'' His son, Jotham, reigned in his stead.%%
\end{inparaenum}

    \heading{31}{The Israelites destroy all the false worship sites in their land~--- Hezekiah appoints priests to serve~--- the people are given goods from King Hezekiah and pay a tithe on their increase~--- the people, as a direct result of their tithing, now have enough to eat~--- }

\begin{inparaenum}
    \verse{31:1} When all this was finished, all of the Israelites who were found went out to the cities of Judah and smashed\footnote{alt., shattered; this doesn't work as well in context.} the standing-stone,\footnote{KB: usually an unhewn, upright stone for cult, burial-marking, or memorial purposes.} cut down the Asherah statues,\footnote{alt., shrines. The Hebrew simply says Asherah (lit.\ as \Hebrew{הָאֲשֵׁרִים}, the plural of \Hebrew{אֲשֵׁרָה}). In Ugarit, Asherah is a goddess, the wife of El and the mother of the gods (KB). Since it is in plural, it refers to something representing her, either statues, shrines, or something used to represent her, probably for cultic worship.} and tore down\footnote{alt., broke up, demolished} the high places\footnote{KB: associated with pagan worship and cultic prostitution.} and altars in all of Judah and Benjamin, in Ephraim and Manasseh, until they'd all been destroyed.\footnote{lit., until completion} All the children of Israel returned, everyone to their property\footnote{alt., landed property, city where they own land} in their cities.%%
    
    \verse{31:2} Hezekiah stationed\footnote{alt., appointed} the division of priests and Levites according to their divisions~--- everyone appointed, priests and Levites~--- for burnt offerings and peace offerings, to minister\footnote{alt., serve} and to praise\footnote{alt., give voice to (praise and thanksgiving)} in the gates of the camps of the \textsc{Lord}.%%
    
    \verse{31:3} The king apportioned\footnote{alt., counted out, divided in parts} from his property\footnote{alt., KB: goods (gained by work, not by purchase); goods (furnishings, gear, utensils); baggage-train; personal property, domain (of king)} for the morning and evening burnt offerings, for Sabbath-day\footnote{lit., Sabbath; more accurately ``\textit{Shabbat},'' the Jewish weekly sabbath observed from sundown Friday through sundown Saturday.} burnt offerings, for new moons, for appointed feasts, as it is written in the law of the \textsc{Lord}.%%
    \verse{31:4} He said to the people who lived in Jerusalem to give the priests' and Levites' portion in order that they may be strengthened in the law of the \textsc{Lord}.%%
    \verse{31:5} As the matter spread forth, the children of Israel multiplied the first-fruits of corn, new wine, oil, honey, and all the increase of the field: they brought in the tithe of the whole in abundance.%%
    \verse{31:6} The children of Israel and Judah (those dwelling in the cities of Judah) also brought forth a tithe of their herds and flocks,\footnote{``[T]heir'' is taken from context (unless they're presenting a tithe of things they do not own in which case they're following the poor example of Saul (cf.~1~Sam. 15\thinspace:\thinspace15)) and the plurality is given for idiomatic purposes.} as well as a tithe of the holy things that are consecrated unto the \textsc{Lord} their God: heaps and heaps were brought in.%%
    
    \verse{31:7} In the third month they began to lay the foundation of the heaps and finished in the seventh month.%%
    
    \verse{31:8} Hezekiah and the heads of the people\footnote{lit., heads. ``[O]f the people'' is understood, unless we're to assume that Hezekiah came in with a bunch of heads.} came in to see the heaps and to bless the \textsc{Lord} and His people Israel.%%
    
    \verse{31:9} Hezekiah asked\footnote{inquired at} the priests and the Levites about the heaps%%
    \verse{31:10} and Azariah, the chief priest (of the house of Zadok), spoke unto him and said, ``Since we began taking in the heave offerings to the House of the \textsc{Lord}, there's plenty to eat, be satisfied, and leave abundantly for the \textsc{Lord} hath blessed his people, and this is the surplus.''%%
    
    \verse{31:11} Hezekiah ordered\footnote{alt., commanded} them\understood\ to prepare\footnote{alt., ordered that [understood] the chambers be prepared} the hall\footnote{KB: a room on three walls of which were benches where worshipers ate sacrificial meal, fourth open to the courtyard.} of the House of the \textsc{Lord}, and they prepared it.\understood%%
    \verse{31:12} They faithfully brought in the heave offering and the tithe and the holy things; the leader over them was Conaniah the Levite and his brother, Shimei, was second in command.\understood%%
    \verse{31:13} Jehiel, Azaziah, Nahath, Asahel, Jerimoth, Jozabad, Eliel, Ismachiah, Mahath, and Benaiah were officers\footnote{alt., commissioners} under the authority\footnote{lit., hand} of Conaniah and his brother, Shimei, by King Hezekiah and Azariah the leader of the House of God's appointment.%%
    \verse{31:14} %%
    \verse{31:15} %%
    \verse{31:16} %%
    \verse{31:17} %%
    \verse{31:18} %%
    \verse{31:19} %%
    \verse{31:20} %%
    \verse{31:21} %%
\end{inparaenum}

    \chap{Ezra}{\Hebrew{עזרא}}
    \section{Ezra 4}\label{Ezra 4}
\heading{Rehum, Shimshai, and others conspire against Israel to halt the building of the temple~--- they write a letter to Artaxerxes who commands that the construction cease}
\begin{enumerate}[align=center]
    \verse{4:1} %%
    \verse{4:2} %%
    \verse{4:3} %%
    \verse{4:4} %%
    \verse{4:5} %%
    \verse{4:6} %%
    \verse{4:7} %%
    \verse{4:8} Rehum the commander and Shimshai the scribe wrote a letter to the king, Artaxerxes.%%
    \verse{4:9} Then Rehum the commander and Shimshai the scribe and all their companions~--- the judges, envoys, officials, secretaries, Urukites, Babylonians, Susaites (who are Elamites),%%
    \verse{4:10} and the rest of the people whom the great and noble Osnappar brought over and settled\footnote{Understood: \textit{them}} in the cities of Samaria; the rest on this side of the river, and so forth.%%
    \verse{4:11} Here is a copy of the letter which they sent to him~--- to Artaxerxes the king~--- the servants, the men on this side of the river, and so forth%%
    \verse{4:12} %% Check punctuation at end of 11.
    \verse{4:13} %%
    \verse{4:14} %%
    \verse{4:15} %%
    \verse{4:16} We make it known to the king that if this city is built and its walls finished, because of this you will have no portion on this side of the river.''%%
    \verse{4:17} The king sent a decree to Rehum the chief of report\footnote{commander} and to Shimshai the scribe and the rest of their companions who live in Samaria and the rest\footnote{Understood: \textit{of the people}} on the other side of the river: ``Peace and so on.%%
    \verse{4:18} The document which you sent to us has been interpreted and read to me.%%
    \verse{4:19} I have established a decree and they have investigated and found that this city, from the days of old, raises itself against the kings~--- rebellion and sedition are made therein.%%
    \verse{4:20} And xxxx xxxx kings over Jerusalem, mighty officers on all the other side of the river, to them is given toll, tribute, and customs.%%
    \verse{4:21} Now, make a decree to stop these men. This city will not be built until I make a decree.%%
    \verse{4:22} And be warned of doing this negligence: why should hurt come to the detriment of the kings?''%%
    \verse{4:23} Then, from the time that a copy of king Artaxerxes' letter was read before Rehum and Shimshai the scribe and their companions, they went in haste to Jerusalem against the Jews and stopped them by force.\footnote{With a strong arm.}%%
    \verse{4:24} The work of the house of God in Jerusalem ceased and remained stopped until the second year of king Darius of Persia's reign.%%
\end{enumerate}

    \section{5}\label{Ezra 5}
\heading{Haggai and Zechariah prophesy~--- Zerubbabel and others begin construction of the temple~--- Tatnai, Shethar-boznai, and others write a letter to King Darius asking for an order to stop the Jews}
\begin{enumerate}[align=center]
    \verse{5:1} Haggai the prophet and Zechariah the son of Iddo prophesied in the name of the God of Israel concerning the Jews in Judah and in Jerusalem.%%
    \verse{5:2} Then Zerubbabel, son of Shealtiel, and Jeshua, son of Jozadak, rose and began to rebuild the house of God in Jerusalem. And the prophets supported them.%%
    \verse{5:3} At that time, Tatnai, the governor on the other side of the river, came to them with Shethar-boznai and their companions; and they said to them, ``Who gave you orders to build this house\footnote{temple} and to finish this wall?''%%
    \verse{5:4} They then said thus unto them, ``What are the names of the men who are building this house?''%%
    \verse{5:5} The eye of their God had been upon the elders\footnote{hoar-headed} of the Jews and they had not caused them to stop until the matter went to Darius. They then sent back a letter concerning this.%%
    \verse{5:6} A copy of the letter that Tatnai the governor on the other side of the river, Shethar-boznai, and his companions the Apharsachites\footnote{title for an official} who are on the other side of the river sent unto Darius the king.%%
    \verse{5:7} In the letter they sent to him was written: ``Peace be unto Darius the king.%%
    \verse{5:8} Be it known to the king that we have gone to the province of Judah, to the great temple of God, and it is build with square stones, wood is placed in the walls, and this work is done speedily. This work prospers in their hand.%%
    \verse{5:9} Then we asked these elders~--- thus did we ask them: `Who hath made a decree for you for this house to be built and this wall to be completed?'%%
    \verse{5:10} Additionally, we asked them for their names, to make it known unto you, so we can write the names of their leaders.\footnote{the men who are at the head}%%
    \verse{5:11} Thus have they returned the word, saying, `We are the servants of the God of heaven and earth. We have been building the house for many years'\footnote{from before this year} and that a great king of Israel had built and finished it.%%
    \verse{5:12} But after that, they made the God of heaven angry and He delivered them into the hands of the Chaldean Nebuchadnezzar, king of Babylon, who demolished this house and removed the other people to Babylon.%%
    \verse{5:13} But in the first year of Cyrus, king of Babylon, Cyrus, the king, gave an order to build this house of God.%%
    \verse{5:14} Furthermore, the gold and silver vessels from the house of God that Nebuchadnezzar removed from the temple in Jerusalem and brought to the temple in Babylon, Cyrus, the king, removed them from the temple in Babylon and brought them to Sheshbazzar whom he had appointed as governor.%%
    \verse{5:15} He\footnote{Cyrus} said to him, ``Lift up these vessels and go and put them in the temple in Jerusalem. Let the house of God be build on its place.''%%
    \verse{5:16} Then did Sheshbazzar come and lay the foundations of the house of God in Jerusalem. From then until now it has been being built and is not finished.%%
    \verse{5:17} And now, if it be good to the king, let an investigation\footnote{search, inquiry} be made into the treasury of the house of the king in Babylon whether it be that Cyrus the king made a decree to build this house of God in Jerusalem. Let the king's will concerning this be sent to us.''%%
\end{enumerate}

    \chap{6}

\heading{Darius starts an investigation to find if sanctions were ever given to the Jews to build a temple~--- evidence is found that Cyrus gave sanctions through a decree~--- Darius states that this decree is still in effect~--- he commands that the Jews be provided with whatever they need to build the temple~--- the temple is completed and dedicated}

\begin{enumerate*}[mode=unboxed]
    \verse{6:1} King Darius then made a decree and they investigated in the records house where the treasures of Babylon were laid up.%%
    \verse{6:2} A scroll was found in Achmetha, a fortress in the province of Media, in the midst of which was written a record:%%
    \verse{6:3} ``In the first year of Cyrus, the king, King Cyrus made an order regarding the house of God in Jerusalem: let the house be built in the place where they are sacrificing sacrifices. Let its foundations be strongly laid: its height sixty cubits and its breadth sixty cubits.%%
    \verse{6:4} \footnote{Understood: \textit{with}}Three layers of square stones and a layer of new wood. Let the expense be charged\footnote{Lit., given} to the king's house.%%
    \verse{6:5} Additionally, the gold and silver of the house of God which Nebuchadnezzar removed from the Jerusalem temple and brought to Babylon, let it be returned to the Jerusalem temple: you shall put it in the house of God.%%
    \verse{6:6} And now, Tatnai, governor of the trans-Euphrates,\footnote{on the opposite side of the river} Shethar-boznai, and your companions the Apharsachites\footnote{officials} of the trans-Euphrates, stay away from this.\footnote{Lit., be ye far from thence}%%
    \verse{6:7} Leave behind the work of the house of God; let the governor of the Jews and the elders of Judah build this house of God in its place.%%
    \verse{6:8} From me I give an order regarding what you should do to the work of those elders of Judah and the building of this house of God: the riches of the king which are on the other side of the river, the expenses be speedily given to these men that their work cease not.%%
    \verse{6:9} What they need, both young bullocks, rams, and lambs, for burnt offerings to the God of heaven; wheat, salt, wine, and oil, according to the saying of the Levite priest who is in Jerusalem, let it be given to them daily without negligence%%
    \verse{6:10} so that they can offer sweet odors to the God of heaven and pray for the life of the king and his sons.%%
    \verse{6:11} From me I give a command to anyone\footnote{all mankind} who changes this decree: let wood be pulled out from their house and let them be impaled\footnote{in impaling let them be impaled} and their house shall be made a dunghill for this.%%
    \verse{6:12} God caused His name to dwell there. He overthrew all the kings and people who try and cause His hand to change and hurt this house of God in Jerusalem. I, Darius, have made a decree~--- let it be done with all diligence.''\footnote{Lit., let it be speedily done}%%
    \verse{6:13} Then Tatnai, the governor of the trans-Euphrates, Shethar-boznai, and their companions speedily did as King Darius had sent.%%
    \verse{6:14} The elders of the Jews were continuing to build and prosper because of the prophecies of Haggai the prophet and Zechariah the son of Iddo. Since the decree of the God of Israel, Cyrus, Darius, and Xerxes\footnote{It says Artaxerxes (\Hebrew{ְאַרְתַּחְשַׁ֖שְׂתְּא}), but historically this is more probably Xerxes} king of Persia, they built and finished.\footnote{Presumably understood: \textit{the house of God in Jerusalem}}%%
    \verse{6:15} This house was completed on the third day of the month Adar which was in the sixth year of the reign of King Darius,%%
    \verse{6:16} and the sons of Israel, the priests, Levites, and the rest of the sons of the captivity made a joyous dedication to this house of God.%%
    \verse{6:17} They brought one hundred bullocks, two hundred rams, four hundred lambs, and twelve young goats (for a sin offering for all Israel according to the number of tribes of Israel) near to this temple of God for its dedication.%%
    \verse{6:18} They put their priests in their divisions, the Levites in their courses over the work of God that is in Jerusalem according to the writing in the scroll of Moses.%%
    \verse{6:19} %%
    \verse{6:20} %%
    \verse{6:21} %%
    \verse{6:22} %%
\end{enumerate*}

    \heading{7}{Genealogy of Ezra~--- }

\begin{inparaenum}
    \verse{7:1} After these things, during the reign of Artaxerxes, king of Persia, Ezra the son of Seraiah, son of Azariah, son of Hilkiah,%%
    \verse{7:2} son of Shallum, son of Zadok, son of Ahitub,%%
    \verse{7:3} son of Amariah, son of Azariah, son of Meraioth,%%
    \verse{7:4} son of Zerahiah, son of Uzzi, son of Bukki,%%
    \verse{7:5} son of Abishua, son of Phinehas, son of Eleazar, son of Aaron the chief priest;%%
    \verse{7:6} this Ezra came up from Babylon, a ready scribe in the Mosaic Law that the \textsc{Lord} God of Israel gave. The king gave him his every request when the hand of the \textsc{Lord} his God was upon him.%%
    
    \verse{7:7} In the seventh year of King Artaxerxes' reign, some of the children of Israel, some of the priests, Levites, singers, gatekeepers, and temple slaves,\footnote{alt., bondsmen; lit., those who are donated} went up to Jerusalem.%%
    \verse{7:8} %%
    \verse{7:9} %%
    \verse{7:10} %%
    
    \verse{7:11} %%
    
    \verse{7:12} Artaxerxes, king of kings, to Ezra the priest, a scribe of the law of the perfect God of heaven,\footnote{Horribly ambiguous: \textit{perfect} could modify \textit{law}, \textit{God}, or \textit{heaven}. Most probably God.} and so forth:%%
    \verse{7:13} ``From me has a declaration been made to all of the people of Israel in my kingdom: any priest or Levite who is willing to go to Jerusalem with them shall go.%%
    \verse{7:14} Because that from before the king and his seven counselors you are sent to investigate concerning the Jews and Jerusalem with the law of God which is in your hand.%%
    \verse{7:15} To bring the silver and gold which the king and his counselors willingly offered to the God of Israel whose tabernacle\footnote{The root of this word, \Hebrew{xxxx}, means ``a dwelling place.''} which is in Jerusalem.%%
    \verse{7:16} All the silver and gold you find in all the provinces of Babylon with the donations\footnote{Free-will offerings} of the people and of the priests shall be freely offered\footnote{donated} to the house of their God which is in Jerusalem.%%
    \verse{7:17} Therefore, you shall speedily buy with this money\footnote{silver and gold. Think \textit{argent}} bullocks, rams, lambs, and their presents and libations, and bring them to the altar which is in the house of their God in Jerusalem.%%
    \verse{7:18} That which is good to you and your brethren with the rest of the silver and gold to do according to the will of your God, that shall you do.%%
    \verse{7:19} The vessels which are given to you for the service of your God's house~--- completely finish it before the God of Jerusalem.%%
    \verse{7:20} The remainder of the needful things of your God's house which have fallen to you to give, give them\footnote{understood} from the treasure house of the king.%%
    \verse{7:21} From me, I, Artaxerxes the king, an order is made to all treasures of the trans-Euphrates that all the requirements of Ezra the priest, scribe of the law of the God of heaven, let them be diligently done%%
    \verse{7:22} unto one hundred talents of silver, one hundred cords of wheat, one hundred baths of wine, and one hundred baths of oil. And salt without writing.\footnote{i.e., permission}%%
    \verse{7:23} All that is from the decree of the God of heaven, let it be done with zeal\footnote{with eagerness} to the house of the God of heaven: for why should there be wrath upon the kingdom of the king and his sons?%%
    \verse{7:24} We are informing you that there is no officer to lift a tribute or tax upon all the priests, Levites, singers, gate keepers, sanctuary servants, and servants of this house of God.%%
    \verse{7:25} You, Ezra, according to the wisdom of your God which is in your hand, place judges and magistrates who will judge all the people of the trans-Euphrates and all who know the laws of your God; and unto those who don't know, teach them.%%
    \verse{7:26} All who will not do the law of your God and the law of the king with exactness, let judgment come upon the, whether to death, banishment, or a fine of riches and a bond.''%%
    
    \verse{7:27} Blessed be the \textsc{Lord} God of our fathers%% xxxx this is Ezra speaking. Quotes?
    \verse{7:28} %%
\end{inparaenum}

    \chap{Nehemiah}{\Hebrew{נחמיה}}
    \heading{6}{xxxx}

\begin{inparaenum}
    \verse{6:1} And when Sanballat, Tobiah, Geshem the Arabian, and the rest of our enemies had heard\footnote{When it was heard by} that I had built the wall and that there was no breach in it (also, up until that time\footnote{until then} I had not set up the gates)%%
    \verse{6:2} that Sanballat and Geshem sent unto me, saying, ``Come, and we shall meet together in the villages of\footnote{in} the valley of Ono,'' but they were thinking of doing me wrong.\footnote{thinking to do evil unto me.}%%
    \verse{6:3} So I sent messengers unto them, saying, ``I am doing a great work so I cannot come down. Why should the work cease\footnote{\textit{sabbath} is used here. That is the concept being conveyed.} while I leave it and come down to you?''%%
    \verse{6:4} So they sent unto me~--- as I've said\footnote{according to this word}~--- four times, and I told them likewise.%%
    \verse{6:5} So Sanballat sent his servant unto me~--- as I've said~--- the fifth time with an open letter in his hand,%%
    \verse{6:6} and it was written therein: ``It hath been heard among the nations, and Gashmu hath said, `You and the Jews are thinking of rebelling. Therefore, you are building the wall and hast been a king unto them. Even according to these words!%%
    \verse{6:7} Moreover, thou hast appointed prophets who\footnote{to} call for thee in Jerusalem, saying, ``A king is in Judah!''' Now, it hath been heard by the king, even according to these words. Come now and let us reason together.''%%
    \verse{6:8} So I sent unto him, saying, ``No, it hath not been as thou hast said\footnote{according to these words that thou hast said} because you've made this up\footnote{devised} from thine own heart.%%
    \verse{6:9} For all of them are making us afraid, saying, `Their hands are too weak\footnote{feeble} for the work~--- that's why it's not done.' O God, strengthen my hands!''\footnote{Some slight eisegesis to get ``O God'' into this clause, but other translations seem to agree with me. Logic: why would he be asking his enemies to strengthen his hands?}%%
    \verse{6:10} Then I entered the house of Shemaiah, son of Delaiah (son of Mehetabeel), who was shut up,\footnote{locked in, under house arrest, not going outside} and he said,%

    \pvcb{``Let us meet at the House of God,}{inside the temple.}%

    \pvcb{So we shall shut the doors of the temple}{because they are coming to slay thee.}%

    \pvca{Even by night are they coming to slay thee.''}\\*%%
    \verse{6:11} And I said, ``Would a man like me flee?\footnote{A man such as I (or, like me)~--- would he flee?} And who, like me, would go in unto the temple to\footnote{and} live? I will not go.''\footnote{go in}%%
    \verse{6:12} And I perceived that lo! God had not sent him because he had prophesied this word against me: Tobiah and Sanballat had hired him.\footnote{That's some serious transgression of taking the name of the Lord in vain.}%%
    \verse{6:13} He was hired to frighten me.\footnote{Therefore was he hired, that I might be afraid} That I might do so: sin~--- that I might have something to go off of to spread an evil report so that they could reproach me.%%
    \verse{6:14} My God, remember Tobiah and Sanballat according to these, their works. And also the prophetess Noadiah and the rest of the prophets who would have made me afraid.%%
    \verse{6:15} So the wall was finished in fifty-two days: on the twenty-fifth of Elul.%%
    \verse{6:16} And when all of our enemies had heard it, and when all the nations round about us were afraid (and much cast down in their own perception) for they knew that this work was brought to pass by our God.%%
    \verse{6:17} Additionally, in those days, the noblemen of Judah sent a lot of letters to Tobiah, so those who are with Tobiah came to them.%%
    \verse{6:18} For there are many in Judah sworn to him: \footnote{for}he hath a son-in-law of Shechaniah, son of Arah; his son Johanan hath married the daughter of Meshullam, son of Berechiah.%%
    \verse{6:19} They also spoke before me of his goodness and then reported my words to him. So Tobiah sent letters to make me afraid.%%
\end{inparaenum}

    % psalm 2: Kingship/temple kind of psalm
    % Psalm 15:4:
    % footnote c, Greek "to his neighbor, fellow companion"
    % footnote d, Perhaps you could insert the writing "and does not turn away from (that) evil \Hebrew{בְּרָע}" [in other words, what was already put in explanatory brackets]
    % \heading{110}{Christ will sit at the Father's right hand and will have the Melchizedek priesthood~--- see further in Appendix~\ref{app:psalm-110}}

\begin{inparaenum}
  {\noindent\verse{110:1} A psalm of David:}%%
  
  \pvbb{The \textsc{Lord} declared to my lord:\footnotemark\footnotemark}{``Sit at my right hand}%%
  \ed{See further in Appendix~\ref{app:adonai}}%%
  \ed{or a king; not necessarily David. This is a messianic psalm so this is directed at Christ.}%%
  
  \pvbb{until I put your enemies}{as your footstool.\footnotemark}%%
  \lit{a footstool for your feet.}%%
  
  \pvac{\verse{110:2} Your mighty scepter}{the \textsc{Lord} will send from Zion:}{rule in the midst of your enemies.}%%
  
  \pvab{\verse{110:3} Your people will voluntary gifts\footnotemark}{in the day of your strength;}%%
  \alt{free-will offering}%%
  
  \pvbc{in royal robes\footnotemark\footnotemark}{from the womb, from the morning light,\footnotemark}{to you the dew\footnotemark\footnotemark\ of your youth.}%%
  \alt{holy splendor}%%
  \ca{prb l c \fragheb\ mlt Mss \symmachus\ Hier \Hebrew{בהררי}}{probably read with the Cairo Genizah and multiple manuscripts from Symmachus' Greek translation of the Old Testament and Hieronymus give \Hebrew{הרר} [instead of \Hebrew{הדר}]}%%
  \halot{xxxx}{(reddish) \textbf{(light before) dawn}}%%
  \ca{\missing\ \septuagint; prp \Hebrew{כְּטַל}}{missing in the Septuagint; probably ``as the dew''}%%
  \alt{light rain}%%
  
  \pvab{\verse{110:4} The \textsc{Lord} has sworn}{and will not have a change of heart:}%%
  
  \pvbb{You are forever a priest}{after the Order of Melchizedek.}%%
  
  \pvbb{\verse{110:5} The Lord\footnotemark\ at your right hand}{will beat kings to pieces\footnotemark\ in the day of His anger.}%%
  \ca{\fragheb\ mlt Mss \Hebrew{יהוה}}{multiple manuscripts from the Cairo Genizah have the Tetragrammaton}%%
  \alt{smite}%%
  
  \pvad{\verse{110:6} He will judge among the nations;}{He will fill {the nations with}\footnotemark\ corpses;}{He will beat the heads in pieces}{over the great earth.}%%
  \ed{Understood from parallel in first hemistich.}%%
  
  \pvab{\verse{110:7} He will drink from the brook by the road,}{therefore he will lift up the head.\footnotemark}%%
  \ca{\fragheb\ \Hebrew{ראשׁי}, 2 Mss \peshitta\ \Hebrew{רֹאשׁוֺ}, 3 Mss + \Hebrew{הללויה} (2 Mss om in 111,1)}{The Cairo Genizah contains ``my head,'' two manuscripts of the Peshitta have ``his head,'' and three manuscripts include ``Hallelujah'' (two of which omit it in \vref{Ps}{111}{1})}%%
\end{inparaenum}
 % Add reference to Appendix on psalm 110
    \chap{Isaiah}{\Hebrew{ישעה}}
    {\noindent\heading{See Appendix~\ref{app:isaiah} for more information on the book of Isaiah}}
    \section{Isaiah 1}\label{Isaiah 1}
\heading{Few in Israel remain faithful to the Lord~--- the Lord rejects their sacrifices and feasts~--- repentance proclaimed~--- Zion to be redeemed in the latter days}
\begin{enumerate}
    \verse{Isaiah^1:1} %
    \verse{Isaiah^1:2} %
    \verse{Isaiah^1:3} %
    \verse{Isaiah^1:4} %
    \verse{Isaiah^1:5} %
    \verse{Isaiah^1:6} %
    \verse{Isaiah^1:7} %
    \verse{Isaiah^1:8} %
    \verse{Isaiah^1:9} %
    \verse{Isaiah^1:10} %
    \verse{Isaiah^1:11} %
    \verse{Isaiah^1:12} %
    \verse{Isaiah^1:13} %
    \verse{Isaiah^1:14} %
    \verse{Isaiah^1:15} %
    \verse{Isaiah^1:16} %
    \verse{Isaiah^1:17} %
    
    :Judge the fatherless;\footnote{orphans} Learn to do good.%
    \verse{Isaiah^1:18} I pray thee, come and let us reason together,'' Saith the \textsc{Lord}.%

:``If your sins are as scarlet, As snow they shall be white.%

:If they are blood\footnote{earth} red, They shall be as wool.\footnote{The dye that was used back then was permanent. The cloth could fade, but would never again be truly white.}%
    \verse{Isaiah^1:19} If you're willing and hearken You shall consume\footnote{eat of} the good of the land.%
    \verse{Isaiah^1:20} And if you refuse and rebel The sword shall consume you:%

:For the mouth of the \textsc{Lord} hath proclaimed it.\footnote{so spoken.}%
    \verse{Isaiah^1:21} %
    \verse{Isaiah^1:22} %
    \verse{Isaiah^1:23} %
    \verse{Isaiah^1:24} Therefore, thus saith the \textsc{Lord} of Hosts, The Mighty One of Israel:\footnote{The one in Israel who is mighty}%

:``Ah, now I will be relieved\footnote{eased} of mine adversaries: I am avenged of mine enemies.%
    \verse{Isaiah^1:25} Lest I turn my hand on thee\footnote{I will turn my hand back on thee}%

:I will purify thine dross~--- I will turn aside all thine tin.\footnote{Tin \emph{is} useful. It is used to make brass (a copper and tin alloy). One of the symbolisms here is that although tin is useful, the Lord has a greater plan in mind for each of us. Therefore, we need to listen to Him and do as He commands although we may think that what we are doing in lieu of obeying is important and useful.}%
    \verse{Isaiah^1:26} I will restore\footnote{return} thine judges as at first And thy counselors as in the beginning.%

:After this though shalt be called \emph{A City of Righteousness: A Faithful City}.%
    \verse{Isaiah^1:27} Zion is redeemed through judgment~--- Also those who are returned\footnote{rescued ones, captives}\footnote{i.e., they are also redeemed} in righteousness.%
    \verse{Isaiah^1:28} The sinners and transgressors are destroyed together; Those forsaking the \textsc{Lord} are consumed.%
    \verse{Isaiah^1:29} You are ashamed of the oaks\footnote{Idols used for fertility worship.} That you've desired.%

:And you're confused because of the groves\footnote{gardens} That you've chosen.%
    \verse{Isaiah^1:30} For you are as an oak Whose leaf is fading%

:And as a grove That hath no water.%
    \verse{Isaiah^1:31} The strong shall be as tow\footnote{Synonymous to oakum (n): Loose fiber from untwisted rope, used esp. to caulk wooden ships.} And its maker as spark.%

:They shall burn together: None shall quench them.''%
\end{enumerate}

    \section{Isaiah 2}\label{Isaiah 2}
\heading{Isaiah sees in vision the latter day temple, the gathering of Israel, the Millennium~--- the proud to be humbled at the Savior's Second Coming}
\begin{enumerate}[align=center]
    \verse{Isaiah^2:1} The thing that Isaiah the son of Amoz foresaw concerning Judah and Jerusalem.%
    % : represents a tab. There are no colons in this chapter.
    \verse{Isaiah^2:2} ::And in the last days:the mountain of the \textsc{Lord}'s house will be established%
    
    ::in the tops of the mountains.:It shall be lifted up above the hills.%
    
    ::All nations shall flow\footnote{Like a river} unto it.\verse{Isaiah^2:3} Tons of people will walk to it and say,%
    
    ::``Come, and let us ascend unto the \textsc{Lord}'s mountain~--- :to the house of the God of Jacob.%
    
    ::He will teach us of His ways.:We will walk in His paths\footnote{Theologically it should be ``path''}%
    
    ::For the law\footnote{teaching, instruction. Traditionally rendered ``law''} goes forth from Zion:and the word of the \textsc{Lord} from Jerusalem.''%
    \verse{Isaiah^2:4} He shall judge among the nations;: He's arbitrates between many people.%
    
    ::They'll forge ploughshares from their swords:and pruning hooks from their spears.%
    
    ::One nation shall not lift its sword against another,:neither shall they learn warfare anymore.%
    \verse{Isaiah^2:5} ::Come, O house of Jacob, that we may walk in the \textsc{Lord}'s light.%
    \verse{Isaiah^2:6} :For you have left your people~--- the house of Jacob~--- to themselves%
    
    :because they've been filled\footnote{Possibly missing ``with superstition.'' i.e., ``filled with superstition from the east.'' There is no expressly-stated object} from the east~--- :they're sorcerers like the Philistines.%
    
    :They please themselves\footnote{Or ``clasp hands'' or ``make sufficient''} with foreigner's\footnote{Outsiders of Israel~--- foreigners, infidels, pagans, etc.} children.\footnote{This is to be understood in a sexual context}%
    \verse{Isaiah^2:7} :Their land is full of silver and gold~--- :there is no end to their treasures.%
    
    :Their land is full of horses~--- :there is no end to their chariots.%
    \verse{Isaiah^2:8} :Their land is full of idols%
    
    :They bow down before the work of their own hands~--- :that which their fingers have made.%
    \verse{Isaiah^2:9} The low shall be bowed down; the haughty humbled. :Don't forgive them.%
    \verse{Isaiah^2:10} Enter into a boulder, :hide in the dust,%
    
    :from before the \textsc{Lord}'s face and the glory of His majesty.
    \verse{Isaiah^2:11} %
    \verse{Isaiah^2:12} %
    \verse{Isaiah^2:13} %
    \verse{Isaiah^2:14} %
    \verse{Isaiah^2:15} %
    \verse{Isaiah^2:16} \footnote{See Appendix \ref{app:isa-2-16}}Upon the ships of Tarshish\footnote{Going to Tarshish. Either in Asia Minor (where Paul was from) or in present-day Spain} :and upon the ships all the beautiful vessels.\footnote{excellent ships}%
    \verse{Isaiah^2:17} %
    \verse{Isaiah^2:18} %
    \verse{Isaiah^2:19} %
    \verse{Isaiah^2:20} %
    \verse{Isaiah^2:21} %
    \verse{Isaiah^2:22} %
\end{enumerate}

    \section{Isaiah 36}\label{Isaiah 36}
\heading{xxxx}
\begin{enumerate}[align=center]
    \verse{Isaiah^36:1} In the fourteenth year of King Hezekiah,\footnote{his reign} king of Judah, the Assyrian king, Sennacherib, came up against all the fortifications of Judah and conquered them.%
    \verse{Isaiah^36:2} The king of Assyria sent Rabshakeh\footnote{This is either his literal name or it should be rendered ``(his) great commander.''} with a strong force from Lachish to Jerusalem unto King Hezekiah. He stood by the aqueduct of the upper pool on the highway of the fuller's field.%
    \verse{Isaiah^36:3} He then came unto Eliakim, son of Hilkijah (who's over the household affairs), Shebna the scribe, and Joah the son of Asaph (the chronicler).%
    \verse{Isaiah^36:4} Rabshakeh said to them, ``Please say to Hezekiah, `Thus says the great king~--- the king of Assyria~--- ``What is this confidence in which you trust?%
    \verse{Isaiah^36:5} You say, `It's just lip service,'\footnote{Lit., a word for the lips} but there is strength for war. Now, who do you rely on that you have revolted against me?\footnote{i.e., who do you trust that you feel comfortable in revolting against me?}%
    \verse{Isaiah^36:6} You trust in the staff of a broken reed (Egypt) on whom if a man loans it goes into his hand and pierces it: such is Pharaoh, king of Egypt, to his allies.%
    \verse{Isaiah^36:7} % '' ' ''
    \verse{Isaiah^36:8} %
    \verse{Isaiah^36:9} %
    \verse{Isaiah^36:10} %
    \verse{Isaiah^36:11} %
    \verse{Isaiah^36:12} %
    \verse{Isaiah^36:13} %
    \verse{Isaiah^36:14} %
    \verse{Isaiah^36:15} %
    \verse{Isaiah^36:16} %
    \verse{Isaiah^36:17} %
    \verse{Isaiah^36:18} %
    \verse{Isaiah^36:19} %
    \verse{Isaiah^36:20} %
    \verse{Isaiah^36:21} %
    \verse{Isaiah^36:22} %
\end{enumerate}

    \chap{Jeremiah}{\Hebrew{ירמיה}}
    \heading{3}{xxxx}

\begin{inparaenum}
  \verse{3:1} %%
  \verse{3:2} %%
  \verse{3:3} %%
  \verse{3:4} %%
  \verse{3:5} %%
  
  \verse{3:6} %%
  \verse{3:7} %%
  \verse{3:8} %%
  \verse{3:9} %%
  \verse{3:10} %%
  
  \verse{3:11} %%
  \verse{3:12} %%
  \verse{3:13} %%
  \verse{3:14} %%
  \verse{3:15} %%
  \verse{3:16} And it shall be that when you have multiplied and become fruitful in the land: in those days (this is the declaration\alt{utterance, revelation} of the \textsc{Lord}) they shall no more say, ``The Ark of the Covenant of the \textsc{Lord},'' neither shall it come to mind, nor shall they remember it, nor give heed unto it~--- it shall not be done anymore.%%
  \verse{3:17} %%
  
  \verse{3:18} %%
  \verse{3:19} %%
  \verse{3:20} %%
  \verse{3:21} %%
  \verse{3:22} %%
  \verse{3:23} %%
  \verse{3:24} %%
  \verse{3:25} %%
\end{inparaenum}

    \chap{Daniel}{\Hebrew{דניאל}}
    \heading{2}{xxxx~--- Daniel interprets the king's dream~--- Nebuchadnezzar praises the God of Heaven~--- Daniel is promoted and made great in the land}

\begin{inparaenum}
    \verse{2:1} In the second year of Nebuchadnezzar's reign, Nebuchadnezzar dreamed\footnote{KB: sexual, then general. The context here is not clear if this was a sexual or general dream.} multiple\understood\ dreams and his mind\footnote{alt., temper, disposition, spirit} was disturbed and he couldn't sleep.\footnote{lit., his sleep was gone from him.}%%
    \verse{2:2} The king ordered the soothsayer-priests, the conjurers, the sorcerers, and the Chaldeans to be called in\understood\ to tell\footnote{alt., to give an opinion; this is unlikely as the king would most likely have had them killed for merely opining.} the king what his dreams meant.\footnote{lit., tell the king his dreams; alt., expound the king's dreams to the king.} So they came in and stood before the king.%%
    \verse{2:3} The king said to them, ``I have dreamed a dream and my mind is troubled\footnote{alt., disturbed} to understand\footnote{alt., find out information about, perceive, know, come to understand} the dream.''%%
    \verse{2:4} The Chaldeans spoke to the king in Aramaic:\footnote{The BHS here contains a horizontal break, replicated here. However, the Masorah states ``prb (probably) add (added/addition), prp (it has been proposed) \Hebrew{וַיֹּאמְרוּ}.''}\hspace*{4em}``O King! Live forever! Tell your servants the dream and we'll make its\understood\ interpretation known to you.''%%
    \verse{2:5} The King answered and said to the Chaldeans, ``The spoken word is promulgated by me. If you don't let me know\footnote{alt., communicate it to me} the dream and its interpretation, you will be dismembered\footnote{lit., made into limbs} and your houses will be pulled down as punishment.\footnote{KB: either (houses) shall be turned into public privy [restrooms], or pulled down as punishment; lit., garbage-heap, heap of ruins and debris}%%
    \verse{2:6} If you let me know the dream and its interpretation, you'll receive gifts and presents\footnote{alt., gifts} and great\footnote{alt., much, many, very} honor\footnote{alt., majesty} from before me. So\footnote{lit., Therefore. ``So'' is more casual and I'm picturing Nebuchadnezzar, harsh at first, becoming more casual in an attempt to butter up the Chaldeans and get them to do his bidding. However, it's perfectly likely that he remains majestic and non-casual.} let me know the dream and its interpretation.''%%
    \verse{2:7} They answered the second time and said, ``If\understood\ the king tells\footnote{lit., says to} us the dream,\footnote{lit., Let the king tell the dream to his servants} we'll let you know its interpretation.''%%
    \verse{2:8} %%
    \verse{2:9} %%
    \verse{2:10} %%
    \verse{2:11} %%
    \verse{2:12} %%
    \verse{2:13} %%
    
    \verse{2:14} %%
    \verse{2:15} %%
    \verse{2:16} %%
    
    \verse{2:17} %%
    \verse{2:18} %%
    \verse{2:19} %%
    \verse{2:20} %%
    \verse{2:21} %%
    \verse{2:22} %%
    \verse{2:23} %%
    \verse{2:24} %%
    
    \verse{2:25} %%
    \verse{2:26} %%
    \verse{2:27} %%
    \verse{2:28} %%
    
    \verse{2:29} %%
    \verse{2:30} %%
    \verse{2:31} %%
    \verse{2:32} %%
    \verse{2:33} %%
    \verse{2:34} %%
    \verse{2:35} xxxx such that no trace of them could be found. However, the stone that struck the image because a great mountain and filled the whole earth.%%
    \verse{2:36} This was the dream. We will now tell its interpretation before the king.%%
    \verse{2:37} You, O king, king of kings to whom the God of Heaven has given the kingdom, might, power, and glory.%%
    \verse{2:38} Wherever the sons of men dwell, He has given the beasts of the field and the birds in the air into your hand, and caused you to rule over all of them: you are the head of gold.%%
    \verse{2:39} Another, lower kingdom shall arise after you. And yet another (third) kingdom of bronze shall rule over all the earth.%%
    \verse{2:40} Then there shall be a fourth kingdom~--- strong as iron~--- which shall arise. It shall break to pieces and shatter all things because it's iron. And like iron, which crushes all these, it shall break and crush.%%
    \verse{2:41} The part made of potter's clay with part iron feet and toes which you saw, it shall be a divided kingdom. But some of the firmness of iron shall be in it~--- as you saw~--- iron mixed with miry clay.%%
    \verse{2:42} The partially iron and partially clay toes of the feet mean that the kingdom shall be part strong and part brittle.%%
    \verse{2:43} The iron mixed with miry clay which you saw, they shall mix offspring, but they will not hold together: just as iron does not mix with clay.%%
    \verse{2:44} In those kings' days, the God of Heaven will set up a kingdom and a sovereignty which shall never be destroyed. It shall not be left to another people. It shall break in pieces and bring to an end all of those kingdoms. Yea, it shall live forever.%%
    \verse{2:45} Just as you saw a stone that was cut from the mountain, but not by hands, and it broke the iron, bronze, clay, silver, and gold in pieces, a great God has made known to the king what shall be hereafter: the dream is certain and its interpretation is sure.''%%
    
    \verse{2:46} Then King Nebuchadnezzar fell on his face and paid homage to Daniel and commanded that an offering of incense be offered to him.%%
    \verse{2:47} The king answered and said to Daniel, ``Truly your God is a God of gods, a lord of kings, and a revealer of mysteries, for you have been able to reveal this mystery!''%%
    \verse{2:48} The king then made Daniel great, gave him many great gifts, and made him a ruler over the whole province of Babylon and chief prefect over all the wise men of Babylon.%%
    \verse{2:49} Daniel requested the king that he appoint Shadrach, Meshach, and Abednego over the affairs of the province. And Daniel was in the court of the king.%%
\end{inparaenum}

    \section{Daniel 3}\label{Daniel 3}
\heading{xxxx}
\begin{enumerate}[align=center]
    \verse{Daniel^3:1} %
    \verse{Daniel^3:2} %
    \verse{Daniel^3:3} %
    \verse{Daniel^3:4} %
    \verse{Daniel^3:5} %
    \verse{Daniel^3:6} %
    \verse{Daniel^3:7} %
    \verse{Daniel^3:8} %
    \verse{Daniel^3:9} %
    \verse{Daniel^3:10} %
    \verse{Daniel^3:11} %
    \verse{Daniel^3:12} %
    \verse{Daniel^3:13} Nebuchadnezzar, in a furious rage, commanded that Shadrach, Meshach, and Abednego should be brought in, and those men were brought in before the king.%
    \verse{Daniel^3:14} And answering,\footnote{Answering what?} Nebuchadnezzar said to them, ``Shadrach, Meshach, and Abednego: is it true that you do not serve my gods nor worship the golden image which I have set up?%
    \verse{Daniel^3:15} Now, if you're ready, when you hear the sound of the horn, pipe, lyre, trigon, harp, bagpipe, and ever sound of music, fall down and worship the image which I have set up. However, if you do not immediately worship, you shall be cast in the the midst of the burning, fiery furnace. And who is `God' that he shall deliver you out of my hands?''%
    \verse{Daniel^3:16} Shadrach, Meshach, and Abednego answered and said to King Nebuchadnezzar, ``We have no need to answer you in this matter.%
    \verse{Daniel^3:17} %
    \verse{Daniel^3:18} %
    \verse{Daniel^3:19} %
    \verse{Daniel^3:20} %
    \verse{Daniel^3:21} %
    \verse{Daniel^3:22} %
    \verse{Daniel^3:23} %
    \verse{Daniel^3:24} %
    \verse{Daniel^3:25} %
    \verse{Daniel^3:26} %
    \verse{Daniel^3:27} %
    \verse{Daniel^3:28} %
    \verse{Daniel^3:29} %
    \verse{Daniel^3:30} %
\end{enumerate}

    % In Malachi, add a reference to Appendix on yada for whatever verse talks about marriage
    \chap{Obadiah}{\Hebrew{עבדיה}}
    \heading{1}{xxxx}

\begin{inparaenum}
  \hfil\verse{1:1} Obadiah's vision:\hfil%%
  
  \smallskip\pvaa{Thus says the Lord \textsc{God} to\footnotemark\ Edom:}%% The \smallskip needs to be here and not on the previous line, otherwise it messes up the centering of the title.
  \alt{concerning}%%
  
  \pvbb{``We have heard\footnotemark\ a report from the \textsc{Lord},}{a messenger has been sent out throughout the nations.}%%
  \ed{The critical apparatus makes frequent comparison to Jer~49, but is not cited here. The reader is encouraged to study the two chapters side-by-side, specifically \vref{Jer}{49}{9--15} as they correspond to \vref{Obad}{1}{1--6}.}%%
  
  \pvba{Get up! Let's go against her\footnotemark\ to battle.}%%
  \ie{Edom}%%
  
  \pvab{\verse{1:2} I've made you small among the nations~---}{you are greatly despised.}%%
  
  \pvaa{\verse{1:3} The arrogance\footnotemark\ of your heart has tricked\footnotemark\ you.\footnotemark}%%
  \halot{xxxx}{\textbf{presumption}}%%
  \alt{deceived}%%
  \lit{lifted you up.}%%
  
  \pvbb{O dweller in the clefts in the rock,}{his dwelling-place is lofty.}%%
  
  \pvbb{He says in his heart,}{`Who will bring me down to earth?'}%%
  
  \pvab{\verse{1:4} Although you soar like the eagle,\footnotemark}{and though you place your nest among the stars,}%%
  \alt{vulture}%%
  
  \pvbb{I will bring you down from there,''}{declares the \textsc{Lord}.}%%
  
  \pvbb{\verse{1:5} ``If robbers come in to you~---}{if plunderers come in by night~---}%%
  
  \pvcb{(how you've been cut off!)}{will they not steal enough for themselves?}%%
  
  \pvcb{If gleaners\footnotemark\ come to you,}{won't they leave gleanings?}%%
  \alt{grape harvesters}%%
  
  \pvbb{\verse{1:6} How Esau has been searched!}{his treasures are sought out!}%%
  
  \pvbb{\verse{1:7} You've been sent out to the border}{by all of your allies.\footnotemark}%%
  \lit{the men of your treaty.}%%
  
  \pvcb{They've forgotten you. The men of your peace}{have eaten\footnotemark}%%
  \ed{The stichs have been slightly reworked. It is literally ``They've forgotten you. They have eaten [next stich] the men of your peace\dots.''}
  
  \pvcb{your bread.}{They've placed a trap\footnotemark\ under you.}%%
  \halot{xxxx}{\textbf{(man-)trap}}%%
  
  \pvca{There is no understanding in him.}%%
  
  \pvab{\verse{1:8} On that day, shall I not,''}{declares the \textsc{Lord},}%%
  
  \pvbb{``destroy the wise men out of Edom,}{and understanding from the mountain of Esau?}%%
  
  \pvab{\verse{1:9} Your mighty men, O Teman, shall be struck down}{in order that every man in the mountain of Esau shall be cut off}%%
  
  \pvbb{by being murdered.\footnotemark\ \verse{1:10} Because of violence against your brother Jacob,}{shame shall cover you, and you shall be cut off forever.}%%
  \alt{slaughtered.}%%
  
  \pvab{\verse{1:11} In the day of your standing on the other side,}{in the day when non-Israelites took his property captive,}%%
  
  \pvbb{when\footnotemark\ foreigners entered his gate,}{and they cast lost on Jerusalem,}%%
  \ed{understood}%%
  
  \pvba{even you were one of them!}%%
  
  \pvbb{\verse{1:12} But you looked on the day of your brother,\footnotemark}{on the day of his estrangement;}%%
  \ca{prp \Hebrew{בְאחיך}}{it has been proposed to be ``on your brother''}%%
  
  \pvcb{neither did you rejoice over the children of Judah}{in the day of their going astray,}%%
  
  \pvcb{nor make your mouth great}{in the day of distress,}%%
  
  \pvbb{\verse{1:13} nor come through the gate of My people}{in the day of their disaster,\footnotemark}%%
  \halot{xxxx}{(final) \textbf{disaster}}%%
  
  \pvcb{nor did you, even you, look on its wickedness}{is the day of its disaster,}%%
  
  \pvcb{nor did you send against its outer wall}{in the day of its disaster,}%%
  
  \pvbb{\verse{1:14} nor stand on the crossroads}{to cut off its fugitives,}%%
  
  \pvcb{nor delivered its survivors}{in the day of its distress,}%%
  
  \pvbb{\verse{1:15} because the day of the \textsc{Lord} is near}{on all the nations.}%%
  
  \pvcb{As you've done, it's done to you:}{your reward will return on your own head.}%%
  
  \pvab{\verse{1:16} Because as you've drunk on My holy mountain,}{so shall all the nations drink unceasingly\footnotemark~---}%%
  \ca{mlt Mss \Hebrew{סָבִיב}}{multiple Hebrew manuscripts have ``all around'' [instead of ``unceasingly/continually'']}%%
  
  \pvbb{they shall drink and slurp,\footnotemark}{they shall be like they were note.}%%
  \ca{prp \Hebrew{ונָעו}}{it has been proposed to be ``and stumble/teeter/shake/tremble''}%%
  
  \pvac{\verse{1:17} On mount Zion}{there's deliverance}{and holiness; [why would it be ``there is holiness''?]\footnotemark}%%
  \ca{add?}{[this hemistich] added?}%%
  
  \pvbb{the house of Jacob shall possess}{their possessions.}%%
  
  \pvab{\verse{1:18} The house of Jacob shall be a[?] fire,}{the house of Joseph shall be a flame,}%%
  
  \pvbb{the house of Esau shall be stubble:\footnotemark}{they shall burn among them and consume them.}%%
  \alt{straw}%%
  
  \pvbb{There will be no survivors}{in the house of Esau}%%
  
  \pvba{because the \textsc{Lord} has spoken.}%%
  
  \pvbb{\verse{1:19} The south [subject?] shall possess mount Esau [make consistent with this one] and the lowlands of the Philistines,}{they shall possess}%%
  
  \pvba{the field\footnotemark\ of Ephraim and the field of Samaria and of Benjamin\footnotemark\ and Gilead.}%%
  \ca{\septuagint\ \Greek{τὸ ὄρος}, frt l \Hebrew{הַר}}{the Septuagint has ``the mountain,'' perhaps read ``mount''}%%
  \ca{xxxx}{xxxx}%%
  
  \pvba{\verse{1:20} The captives of this force of the children of Israel (that are of the Canaanites) are to[?] Zarephath,}%%
  
  \pvba{and the captives of Jerusalem who were in Sepharad shall possess the cities of the south.}%%
  
  \pvbb{\verse{1:21} And saviors shall go up on Mount Zion}{to judge the mountain of Esau,}%%
  
  \pvca{and the kingdom shall be the \textsc{Lord}'s.}%%
\end{inparaenum}

    \chap{Zephaniah}{\Hebrew{צפניה}}
    \section{Zephaniah 1}\label{Zephaniah 1}
\heading{xxxx}
\begin{enumerate}
    \verse{Zephaniah^1:1} %
    \verse{Zephaniah^1:2} :::I will totally destroy everything :from off the face of the land :dictates the \textsc{Lord}.%
    \verse{Zephaniah^1:3} :::I will destroy man and beast, :the birds in the sky, :and the fish in the sea.%
    
    :::The stumbling blocks (the wicked) XXXX%
    \verse{Zephaniah^1:4} %
    \verse{Zephaniah^1:5} %
    \verse{Zephaniah^1:6} %
    \verse{Zephaniah^1:7} %
    \verse{Zephaniah^1:8} %
    \verse{Zephaniah^1:9} %
    \verse{Zephaniah^1:10} %
    \verse{Zephaniah^1:11} %
    \verse{Zephaniah^1:12} %
    \verse{Zephaniah^1:13} %
    \verse{Zephaniah^1:14} %
    \verse{Zephaniah^1:15} %
    \verse{Zephaniah^1:16} %
    \verse{Zephaniah^1:17} %
    \verse{Zephaniah^1:18} %
\end{enumerate}

    \chap{Malachi}{\Hebrew{מלאכי}}
    \heading{3}{The Lord will send a messenger to prepare His way~--- His people will be refined and purified~--- men rob God by offering poor sacrifices~---  xxxx}

\begin{inparaenum}
    \pvab{\vn{3:1} ``I will send My messenger}{and He shall prepare the\footnotemark\ way before Me.}%%
    \fntlit{a}%%
    
    \pvbb{The Lord whom you seek}{shall suddenly enter His temple.\footnotemark}%%
    \fnted{reversed the order of these stichs for idiomaticy}%%
    
    \pvca{The angel of the covenant in whom you delight~---He cometh,'' says the \textsc{Lord} of Hosts.\footnotemark}%%
    \fntca{prb add}{[this line] probably added}%%
    
    \pvab{\vn{3:2} ``But who shall endure\footnotemark\ the day of His coming?}{Who shall be standing when He appears?}%%
    \fntalt{abide, live through}%%
    
    \pvbb{For He is like refiner's fire}{and fuller's lye.\footnotemark}%%
    \fntalt{soap; the primary ingredient being lye}%%
    
    % Verse 3 should not be indented.
    \verse{3:3} For He shall sit as a refiner and purifier of silver and He shall purify the children of Levi and purge them like gold and silver. And they shall be an offering unto the \textsc{Lord} in righteousness.%%
    \verse{3:4} The offering of Judah and Jerusalem shall be a sweet savor unto the \textsc{Lord} even as in days of old~--- as in previous years.\smallskip%%
    
    \pvab{\vn{3:5} I will draw near to you for judgment.}{I have been a hasteful\footnotemark\ witness}%%
    \fnthalot{xxxx}%%
    
    \pvbb{against the sorcerers and adulterers,}{against those who swear falsely,}%%
    
    \pvbb{and against those who monetarily extort their employees\footnotemark~---}{widows, orphans, those who turn away the resident aliens,\footnotemark}%%
    \fntlit{oppressors of their hireling's wages}%%
    \fnted{the most exposed (or least protected) groups of people}%%
    
    \pvbb{and those who don't fear Me,''}{says the \textsc{Lord} of Hosts.}%%
    
    \pvab{\vn{3:6} ``For I, the \textsc{Lord}, have not changed.}{And you, children of Jacob, have not been consumed.}%%
    
    \pvab{\vn{3:7} Even from the days of your fathers have you shied away\footnotemark\ from My statutes}{and not taken heed.\footnotemark}%%
    \fntalt{turned away/aside}%%
    \fntca{frt ins \Hebrew{מִשְׁמַרְתִּי} cf 14}{perhaps insert ``to my commands,'' compare verse~14}%%
    
    \pvbb{Turn your back on Me and I will turn My back on you,''}{says the \textsc{Lord} of Hosts.}%%
    
    \pvbb{``Yet you say,}{`How do we turn away?'\footnotemark}%%
    \fntca{add?}{[this line] added?}%%
    
    \pvab{\vn{3:8} Will a man deceive\footnotemark\footnotemark\ God?}{Because you have been deceiving Me.}%%
    \fnthalot{\textbf{deceive} \haref{Ma}{3}{8f} \haref{Pr}{22}{23}; or perhaps \textbf{rob}}%%
    \ed{The critical apparatus gives \Hebrew{עקב} (grasp by the heel, cheat) instead of \Hebrew{קבע} (deceive, rob) for all instances in this verse and verse~9. \textsc{halot} says that this is ``perhaps metathesis of \Hebrew{קבע} to avoid assonance of \Hebrew{יַעֲקֹב}.''}%%
    
    \pvbb{But you say, `How have we been deceiving you?'}{By tithes and offerings.\footnotemark}%%
    \fntca{frt 1 \Hebrew{וּבַתּרומה} \Hebrew{בַּמּעשׂר} cf \peshitta\targum\vulgate, sed cf 9\super{b--b}}{perhaps 1 has ``in tithes and in offerings,'' compare the Peshitta, Targum, and Vulgate, but compare 9\super{b--b} [see following]}%%
    
    \pvac{\vn{3:9} You are cursed with a curse,}{and you rob Me,}{even this whole nation.\footnotemark}%%
    \fntca{add? prb 1 \Hebrew{כלו} \Hebrew{הֲגַם} et tr ad fin 8}{added? probably 1 has ``Will they also all this?'' and transposed to the end of verse~8}%%
    
    \pvac{\vn{3:10} Bring all of the tithes}{into the storehouse:}{there is food in My house.}%%
    
    \pvbb{Please, try Me on this,''}{says the \textsc{Lord} of Hosts,}%%
    
    \pvbb{``If I will not open the windows of heaven for you}{and empty out upon you a blessing \textit{ad infinitum}.}%%
    
    \pvab{\vn{3:11} I will rebuke the devourer for your sake.}{He shall not destroy the fruit of your land,}%%
    
    \pvbb{neither shall the vine of your the field be barren,''}{says the \textsc{Lord} of Hosts.}%%
    
    \pvab{\vn{3:12} ``All of the gentile nations shall call you blessed}{because you shall be a desirable\footnotemark\ land,''}%%
    \fntalt{delightful}%%
    
    \pvaa{says the \textsc{Lord} of Hosts.}%%
    
    \pvab{\vn{3:13} ``Your words have been harsh\footnotemark\ against Me,''}{says the \textsc{Lord}.\footnotemark}%%
    \fntalt{strong}%%
    \ca{prb ins \Hebrew{צְבָאוֺת} cf \septuagint\superit{L}}{probably insert ``of Hosts,'' compare the Septuagint (textus Graecus ex recensione Luciani)}%%
    
    \pvbb{``Yet you say,}{`How have we spoken against You?'}%%
    
    \pvab{\vn{3:14} ``You've said, `It is vain to serve God.}{And what profit is there in keeping His charges}%%
    
    \pvbb{and walking mournfully\footnotemark}{before the \textsc{Lord} of Hosts?}%%
    \fntlit{walking like mourners}%%
    
    \pvab{\vn{3:15} We now declare}{the proud happy.\footnotemark}%%
    \fntalt{arrogant, proud, haughty, insolent}
    
    \pvbb{Moreover, those who work iniquity\footnotemark\ prosper\footnotemark}{and also, those who tempt God escape.'\thinspace''}%%
    \fntalt{do wickedness}%%
    \fntlit{are built up, we build up}%%
    
    \pvac{\vn{3:16} Then those who feared the \textsc{Lord} spoke}{one to another,}{and the \textsc{Lord} observed and heard it.}%%
    
    \pvbb{And a scroll of remembrance was written before Him}{for those who reverenced the \textsc{Lord} and pondered\footnotemark\ on His name.}%%
    \fntalt{consider, reflect on}%%
    
    \pvac{\vn{3:17} ``They shall be Mine,''}{says the \textsc{Lord} of Hosts,}{``in the day I shall make them a peculiar\footnotemark\ treasure.}%%
    \fnted{see further in Appendix~\ref{app:peculiar}}%%
    
    \pvbb{And I shall show mercy on them, just like a man shows mercy}{on his son who serves him.}%%
    
    \pvab{\vn{3:18} You will return and differentiate between those who are righteous and those who are wicked~---}{between those who serve God and those who don't serve Him.''}%%
    
    \pvad{\vn{3:19} \footnotemark Because the day comes,}{burning as a furnace,\footnotemark}{when all the proud}{and all who work wickedness shall be stubble.\footnotemark}%%
    \fnted{In English translations, this is the beginning of chapter~4.}%%
    \fntca{\septuagint* + \Greek{καὶ φλέζει αὺτούς}}{the Septuagint (textus Graecus originalis) adds ``xxxx''}%%
    \fntalt{straw.}%%
    
    \pvbb{``The day is coming when He shall burn them up,''}{says the \textsc{Lord} of Hosts.}%%
    
    \pvbb{``It shall leave them neither}{root nor branch.}%%
    
    \pvab{\vn{3:20} ``Risen to you who fear My name,}{the Sun\footnotemark\footnotemark\ of Righteousness has risen with healing in His wings.}%%
    \fnted{In Egyptian iconography, there is an image of a beneficent sun god with wings. This is likely the same kind of imagery.}%%
    \fnted{We cannot ascribe word play (i.e., sun/son) here because it doesn't exist in Hebrew (i.e., \Hebrew{שֶׁמֶשׁ}/\Hebrew{בן}).}%%
    
    \pvbb{And you shall go forth and grow up}{as calves of the stall.}%%
    
    \pvab{\vn{3:21} You shall tread down the wicked for they shall be ashes}{under the soles of your feet}%%
    
    \pvbb{In the day that I shall do this,''}{says the \textsc{Lord} of Hosts.}%%
    
    \pvab{\vn{3:22} \footnotemark``Remember the Law of Moses, My servant,}{that I commanded him,}%%
    \fntca{22~add; \septuagint* tr~22 post~24}{verse~22 added; the Septuagint (textus Graecus originalis) transposes verse~22 after verse~24}%%
    
    \pvbb{In Horeb, for all Israel~---}{The judgments and statutes.}%%
    
    \pvab{\vn{3:23} \footnotemark I will send you}{Elijah the prophet}%%
    \fntca{23sq add}{verse~23 [and?] following add[ed/ition?]}%%
    
    \pvbb{Before the coming of the day of the \textsc{Lord},}{which is great and dreadful.\footnotemark}%%
    \fntalt{awe-inspiring, terrifying.}%%
    
    \pvab{\vn{3:24} \footnotemark He shall turn the fathers' hearts to the children,}{and the children's hearts to their father,}%%
    \fntca{23sq add}{verse~23 [and?] following add[ed/ition?]}%%
    
    \pvbb{otherwise I will come and smite}{the earth with a curse.''\footnotemark}%%
    \fntalt{destruction.''}%%
\end{inparaenum}

    \section{Malachi 4}\label{Malachi 4}
\heading{xxxx}
\begin{enumerate}[align=center]
    \verse{Malachi^4:1} For behold! the day comes~--- Burning as a furnace: All the proud And all that work wickedness shall be stubble.

:``The day that is coming shall burn them up,'' saith the \textsc{Lord} of Hosts.

:``It shall leave them neither Root nor branch.%%
    \verse{Malachi^4:2} ``The Sun of Righteousness hath risen Unto those\footnote{you} who fear My name with healing in His wings.

:And ye shall go forth. You shall grow up As calves of the stall.%%
    \verse{Malachi^4:3} Ye shall tread down the wicked for they shall be ashes Under the soles of thine feet

:In the day that I shall do this,'' Saith the \textsc{Lord} of Hosts.%%
    \verse{Malachi^4:4} ``Remember the Law of Moses, my servant, That I commanded him

::In Horeb for all Israel: The judgments and statutes.%%
    \verse{Malachi^4:5} Behold! I will send unto you Elijah the prophet

:Before the advent of the Great and dreadful day of the \textsc{Lord}.%%
    \verse{Malachi^4:6} He shall turn the fathers' hearts unto the children And the children's hearts unto their fathers

:Otherwise I will come and smite The earth with a curse.''%%
\end{enumerate}

    \appendix
    % Redefine \chapter for use in \appendix.
    \titleformat{\chapter}[display]{\centering\normalfont\LARGE}{}{0ex}{Appendix \thechapter: #1}
    % Change spacing around \chapter in \appendix.
    \titlespacing*{\chapter}{0em}{-5ex}{5ex}
    % Redefine \section so that it is used instead of \subsection in appendices.
    \titlespacing*{\section}{0em}{0.65em}{0.65em}
    \titleformat{\section}{\large\bfseries}{}{1em}{#1}
    \titlespacing*{\subsection}{0em}{0.65em}{0.65em}
    \titleformat{\subsection}{\bfseries}{}{1.25em}{#1} % Set to 1.25em so \section and \subsection are vertically aligned.
    ~\clearpage\thispagestyle{empty}
    ~\clearpage\thispagestyle{empty}
    \onecolumn
    \chapter{Notes on Psalm 110}\label{app:psalm-110}
Psalm 110 deals with the endowment of a king in ancient Israel. Kings and prophets were in different orders of Priesthood, kings having a higher order. What is interesting about this is that the difference between becoming a king or a priest (the rites, at least) are not very different.

It is assumed that David wrote this psalm before he was crowned (i.e., when Saul was king). As a side note, one remarkable thing about David is that he was consistently obedient to the crown: he served the position, not the person.

The Aaronic, or Levitical, Priesthood was known anciently as \Hebrew{כְּהֻנָּה}.\footnote{from the root \Hebrew{כהנ}, meaning \emph{priest}} This priesthood was for Aaron and his descendants: ``And Aaron and his sons shalt thou appoint that they may attend to their priest's office'' (Numbers 3:10, \textsc{darby}). However, the higher priesthood, \Hebrew{דִּבְרָה},\footnote{As found in Psalm 110:4 as \Hebrew{מַלְכִּי־צֶֽדֶק עַל־דִּבְרָתִי}} was given to prophets and kings. The assumption is that all who reigned in Israel had this latter order of priesthood.

This higher authority allowed those in its possession to enter the Holy Place and the Holy of Holies without particular regard to worthiness (as compared to those of the Aaronic order who had to be ritualistically and ethically clean among other prerequisites). However, an interesting story is found in 2~Chronicles chapter~26 where king Uzziah (the ruler at the time of Isaiah) assumed that he had this authority and walked into the Holy of Holies\footnote{While it is not explicitly stated in 2~Chronicles~26 that he entered the Holy of Holies, it is stated that he went to burn incense before the Lord~--- something that was done in the Holy of Holies.} and got leprosy.\footnote{Most likely some skin disease (\Hebrew{צְרוּעָ}) and not necessarily leprosy} One possible explanation for this is that Uzziah was king of Judah, not Israel, and this may not have been sufficient for him to be of the higher order of priesthood.

    \chapter{Notes on Isaiah}\label{app:isaiah}
Most of the prophetic writings are either in judgment or hope for the people; they are not usually neutral. One theory postulates that there are two Isaiahs: one comprising Isaiah~1--39 and dealing with judgment, the other Isaiah~40--66 and dealing with hope (and possibly a third comprising just 56--66). However, there is plenty of hope in the first section and plenty of judgment in the second section which may controvert this theory.

\section{Isaiah 2:16}\label{app:isa-2-16}
Sailing in antiquity was a dangerous activity~--- it was taking your life in your hands. Sailors would go as close to the coast as possible so that if the ship breaks up (cf. Jonah 1:4) they could still live through the incident by swimming to shore.

The \emph{Hebrew and Aramaic Lexicon of the Old Testament} by Koehler and Baumgartner defines \Hebrew{שְׂכִיָּה} as follows:
\begin{quote}
    Ug. \emph{$\underline{t}$kt} (Gordon \emph{Textbook} \S19:2680; Aistleitner 2862; Driver \emph{Myths}$^2$ 160a; cf. Fisher \emph{Parallels} 2: p. 8 entry 5) < Eg. \emph{\'skty} ship (Erman-G. 4:315), see Lambdin \emph{Loan Words} 154f; Ellenbogen \emph{Foreign Words} 154; cf. also Wildberger BK 10:94: \textbf{ship} Is 2$_{16}$. $\dagger$
\end{quote}

The following is from Ellenbogen's \emph{Foreign Words in the Old Testament}:
\begin{quotation}
    {\noindent\Hebrew{שכיות} -- Ships~~~~~~~~~~~~~~~~~~~~~~~~~~~Egyptian -- $\underline{\emph{\'sk.tj}}$ \includegraphics[scale=1.1]{images/egt-ship} (ship)}
    
    {\noindent Isa. 2:16 -}
    
    The LXX translation of the phrase \Hebrew{החמדה כל־שכיות ועל}, which runs: \Greek{ἐπὶ πᾶσαν θέαν πλοίων κάλλους}, is very remarkable; it incorporates both what seems to be the real meaning of the term \Hebrew{שכיות}, namely, \Greek{πλοιον} ``ship,'' and also what is required by the traditional understanding of the word, namely \Greek{θέᾶ} ``sight, view.'' The Vulgate renders \Hebrew{שכיות} by $\underline{\text{visus}}$,\footnote{Prob. ``views''} and the Peshi\d{t}ta by \Syriac{ܕܘܩܐ} ``view.'' Rashi has a note saying that \Hebrew{שכיות} refers to palaces whose floors are paved with marble mosaics. This seems to be purely conjectural, and possibly based on the Targum's rendering of the term (\Hebrew{בירנתא} ``castle'').
    
    The Verses Isa. 2:13, 14, 15, and 17 contain parallelisms, so it would appear reasonable to assume that Verse 2:16 also contains a parallelism that would match the word \Hebrew{אניות} ``ships,'' so Budde-Begrich ($\underline{\text{ZATW}}$ 49, p.198) are apparently right in seeing in Hebrew \Hebrew{שכיות} the borrowed Egyptian term $\underline{\emph{\'sk.tj}}$, a word which is attested from the days of the New Kingdom. [See Erman-Grapow, $\underline{\text{WB}}$ IV, p.315.] The Egyptian derivation of \Hebrew{שכיות} was also accepted by Albright ($\underline{\text{Bertholet Festschrift}}$, p.5) who mentions that H.L. Ginsberg identifies Ugaritic $\underline{\text{\underline{t}kt}}$ with Hebrew \Hebrew{שכיות}.
\end{quotation}

    \chapter{Aramaic in the Old Testament}\label{app:aramaic-in-the-ot}


    \section{Covenants in Antiquity}\label{app:covenants-in-antiquity}
\subsection{\Hebrew{ידע}~--- to know}
% Split into enumerate
The verb \Hebrew{ידע} has three senses in BH. First, it means ``to know (a fact).''; for instance, ``To know the time.'' Second, it is used with a sexual connotation (to have sexual relations). Lastly, it is used in a covenantal sense~--- to enter a covenant (or treaty) with someone. Examples of this usage include:
\begin{itemize}
    \item ``And there arose a new king over Egypt, who did not \emph{know} Joseph'' (Exodus 1:8, \textsc{darby}, emphasis added). In other words, a king came to succession who had not covenanted with Joseph: ``I didn't know him, so all bets are off.''
    \item ``Before I formed thee in the belly I \emph{knew} thee'' (Jeremiah 1:4, \textsc{darby}, emphasis added).
    \item ``[A]nd then will I avow unto them, I never \emph{knew} you'' (Matthew 7:23, \textsc{darby}, emphasis added).
    \item ``\dots if thou art God, wilt thou make thyself known unto me, and I will give away all my sins to \emph{know} thee'' (Alma 22:18, emphasis added).
\end{itemize}

BH does not have a sense of knowing a person, like the French \textit{conna\^\i tre}. The closest to that sense is \Hebrew{נכר} which means, in the hiphil, ``to be acquainted.''

% This should be in a different section.
In ancient Israelite marriages, covenants were made to God, \emph{not} to the other person. Therefore, the breaker of the covenant must answer to God.

\subsection{Oath taking syntax}
The syntax of oath taking:

``I will not give your grain any longer as food for your enemies'' is literally ``\emph{If} I give your food to your enemies \emph{and} [understood: you will kill me].''

``If I don't do this, may my throat be slit just as the throat of this animal.''

cf.\ Alma 46:22--24. ``Preserved'' is a Muslim, not an Israelite, tradition.

\subsection{Apodictic and casuistic law}
Exodus 20 and on is all part of a covenant making ceremony.

Apodictic law is the ``thou shalt not.'' Casuistic law contains situations in which things happen (e.g., if someone does such and such, then such will happen).

    \chapter{Names of the Lord}\label{app:names-of-the-lord}
\section{The Tetragrammaton}
\subsection{Etymology}
To help prevent people from speaking the ineffable name of God, the Tetragrammaton (\Hebrew{יהוה}) was sometimes voweled with the same vowel points as \Hebrew{אֲדֹנָי} (Adonai). More commonly it is written \Hebrew{יְהוָה}.

The English transliteration is Jehovah, as follows:
\begin{center}
    \Hebrew{יהוה} > hwhy > yhwh > jhwh > jhvh > Jehovah
\end{center}
The \Hebrew{י} was substituted for a \emph{j} because \emph{y} is a weak phoneme. The \Hebrew{ו}, anciently pronounced as \emph{w} was changed to the stronger \emph{v}. Thus did these changes make Jehovah out of the original Yahweh.

\subsection{Sanctity}
Ezra is one of the founders of modern Judaism~--- an ethnic religion practiced by a monotheistic people, the Judahites. It is the only surviving ethnic religion in the world besides Hinduism. This religion observed the sanctity of the Tetragrammaton by not pronouncing it. This practice became so strong that individuals who pronounced it would be put to death. The Romans allowed the Jews to stone those who uttered the ineffable name. Christ may very well have pronounced this when He declared, ``Before Abraham was \textsc{I am}'' (John 8:58).

This practice continued through the 5$^\text{th}$ and 6$^\text{th}$ centuries \textsc{ad}. It is still considered sacrilegious by very observant Jews, so much so that even \Hebrew{אֱלוֺהִים} is pronounced by Hasidic Jews as \Hebrew{אֱלוֺקִים}. In writing, ``G-d'' or ``G\_{}d'' is substituted for ``God.'' However, this practice is not peculiar to Judaism as Parley P.\ Pratt would write ``G-d'' when quoting blasphemers.

When reading, observant Jews will say ``Adonai'' (master, master of the universe, master of an individual) or ``Hashem'' (lit., the name) in place of Yahweh.

Some Orthodox Jews will not write \Hebrew{יהוה} because things containing this cannot be burned, erased, or destroyed. For this reason there is a place in the synagogue known as the \Hebrew{גְּנִיזָה} (genizah, pl.\ genizot) which is used for the disposal of sacred writings. Unfortunately, many \Hebrew{גְּנִיזָוֺת} were destroyed in the Middle Ages by fire (either arson or accidental). Fortunately, the \Hebrew{גְּנִיזָה} in the ben-Asher synagogue in Cairo, Egypt (built in the 10$^\text{th}$ century \textsc{ad}) has never burned. The Damascus scroll, likely part of the Dead Sea Scrolls, was found there because some of the documents belonging to the Qumran community were left there.

Sometimes, especially in the Dead Sea Scrolls, the Tetragrammaton is rendered in the Paleo-Hebrew: \textphnc{hwhy}.

\section{Ahman}
Not much is known on this name. What follows is only preliminary work\footnote{From Professor Stephen D.\ Ricks in Heb~432R (Biblical Hebrew Syntax), Summer term 2013.} and is in no way meant to be authoritative.

Ahman, possibly related to \Hebrew{אֹמֵן}, meaning \emph{to be true}, \emph{to be faithful}, or \emph{to be realized}. May also refer to \emph{faith} or \emph{veracity}.

Could possibly be related to the Book of Mormon name \Hebrew{אֹמְנִי} (Omni) which means either \emph{faith in me} or \emph{my faith}, \Hebrew{אֹמְנִי} being either a subjective or objective genitive.

    \@openrighttrue\makeatother
    \twocolumn
\end{document}
