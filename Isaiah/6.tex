\heading{6}{Throne theophany and divine council~--- seraphs praise the Lord~--- Isaiah proclaims his sins and their cause~--- he is forgiven of his sins~--- the Lord calls Isaiah to proclaim the gospel}

\begin{inparaenum}
    \verse{6:1} In the year King Uzziah died, I saw the Lord sitting on a high and exalted throne, His flowing skirt filled the temple.%%
    \verse{6:2} Seraphs stood right above Him. Each had six wings: they each covered their faces with two, their feet with two, and with two they flew.%%
    \verse{6:3} They proclaimed to each other, and said,\smallskip%%
    
    \pvcb{``Holy, holy, holy is the \textsc{Lord} of Hosts.}{The whole earth is full of His glory.''}%%
    
    {\noindent\verse{6:4} The threshold's foundations moved about at the sound of the crier; the house was filled with smoke.}%%
    \verse{6:5} I said:\smallskip%%
    
    \pvcb{``Woe is me because I am undone}{on account that I'm a man of unclean lips;}%%
    
    \pvca{because I live among people with unclean lips;}%%
    
    \pvca{because my eyes have seen the King, the \textsc{Lord} of Hosts.''}%%
    
    {\noindent\verse{6:6} One of the seraphs flew to me. In his hand he had a glowing coal which he'd taken with tongs from the altar.}%%
    \verse{6:7} He touched me on my mouth,\lit{He caused it to touch my lip}\ed{understood: with it} and said,\smallskip%%
    
    \pvcb{``This has indeed touched your lips:}{your guilt is taken away and your sins expiated.\footnotemark}%%
    \fnted{\Hebrew{רָז}, from Persian and attested in Aramaic, is equivalent to the Hebrew \Hebrew{סוֺד}, \Hebrew{סוֺד} originally meaning ``a divine council,'' ``a council in Heaven,'' or ``the decision of a divine council.'' The Latin came to mean ``secret'' because the decision of the divine council was a secret, cf.\ \vref{Amos}{3}{7}.}%%
    
    \pvac{\vn{6:8} I heard the voice of the Lord, saying,}{``Whom shall I send?}{Who shall go for us?''}%%
    
    \pvac{And I said, ``I'm here.\footnotemark\ Send me.''}{\vn{6:9} And He said,}{``Go. You shall say to this people:}%%
    \fnted{In Abraham's time this could simply be rendered as ``Yes.''}%%
    
    \pvcb{`Pay strict attention, yet you don't understand.}{Watch very carefully, yet you don't know.'\footnotemark}%%
    \fnted{given in the prophetic perfect tense}%%
    
    \pvbb{\vn{6:10} Fatten the heart of this people,}{make their ears heavy, blind their eyes,\footnotemark}%%
    \ed{with pitch}%%
    
    \pvcb{lest they see with their eyes and hear with their ears,}{their hearts\footnotemark\ consider, and they repent and be healed.''\footnotemark}%%
    \fnted{It's wonderful how this language exemplifies that they're united in their rebellion against the Lord.}%%
    \fntlit{they turn back and there is a healing for them.''}%%
    
    \pvac{\vn{6:11} And I said,}{``How long, Lord?''}{and He said,}%%
    
    \pvcb{``Until the cities}{are desolated without inhabitant.\footnotemark}%%
    \fnted{This is literally ``Until wasted / cities without inhabitant.'' It had to be reshuffled so it was idiomatic with SVO~English.}%%
    
    \pvcb{Until\footnotemark\ houses are without men}{and the ground becomes a desolation.}%%
    \fnted{repeated}%%
    
    \pvbb{\vn{6:12} Until\footnotemark\ the \textsc{Lord} has distanced man,}{and the forsakeness is great in the midst\footnotemark\ of the land.}%%
    \fnted{repeated}%%
    \fntlit{heart}%%
    
    \pvbb{\vn{6:13} And yet a tenth}{shall repent\footnotemark\ and shall be burned}%%
    \fntie{a remnant shall return}%%
    
    \pvcb{as the terebinth\footnotemark\ and the oak}{which\footnotemark\ leave a stump\footnotemark\ when they are cut down\footnotemark~---}%%
    \fnted{a small tree that used to be a source of turpentine}%%
    \fntca{prp \Hebrew{אֲשֵׁרָה}}{proposed to be ``Asherah'' [making it ``as the terebinth and the oak~--- Asherah\dots'']}%%
    \fnthalot{xxxx}{earlier ``rootstock'' impossible; either ``bare trunk'' after burning of branches, or ``new growth''}%%
    \fntca{\qumran\super{a} \Hebrew{במָה}}{1QIsa\super{a} has (from \textsc{halot}) ``4.\ (cultic) \textbf{high place}\dots\ associated with pagan worship and cultic prostitution''}%%
    
    \pvca{so shall the holy seed\footnotemark\ be a stump.''\footnotemark\footnotemark}%%
    \fntca{\missing\ \septuagint}{[``as the terebinth\dots\ the holy seed''] is missing in the Septuagint}%%
    \fnted{understood: in the land.''}%%
    \fntca{dl}{[this line] should be deleted}%%
\end{inparaenum}
