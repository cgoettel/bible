\heading{5}{Moses restates the Decalogue and lends historical context to it~--- commands Israel to follow the Lord in all they do}

\begin{inparaenum}
  \verse{5:1} Moses proclaimed to all of Israel, and said to them, ``Listen, Israel, to the statutes and legal decisions that I am speaking into your ears today. Learn them. Observe\ie{to do} them.%%
  \verse{5:2} The \lord\ our God made a covenant with us in Horeb.%%
  \verse{5:3} The \lord\ did not make this covenant with our fathers, but with us: all of us who are here and alive today.%%
  \verse{5:4} On the mountain, the \lord\ spoke face to face with you from the midst of the fire.%%
  \verse{5:5} I stood between you and the \lord\ at that time in order to declare the word of the \lord\ to you because you were afraid of\lit{by reason of} the fire and didn't go up into the mountain, where he said,\lit{saying. This rendering is meant to clarify that it was the Lord speaking, not Moses or Israel.}%%
  
  \verse{5:6} \textsc{Preface.} `I am the \lord\ your God. I brought you out from the land of Egypt~--- from the house of captivity.%%
  \verse{5:7} \textsc{i.} There will be no other gods in preference to me.%%
  \verse{5:8} \textsc{ii.} Never make for yourselves graven images, or any image,\alt{idol, picture, image, or any likeness} in the heavens above, neither that is in the earth nor the waters underneath the earth.%%
  \verse{5:9} Never bow down to them or worship\alt{serve} them because I, the \lord\ your God, am a jealous God: I visit the punishment\alt{iniquity} unto the third and fourth generation of my haters%%
  \verse{5:10} Showing mercy unto thousands who love me and obey my commandments.%%
  
  \verse{5:11} \textsc{iii.} Never take the name of the \lord\ thy God with vain purpose for the \lord\ shall not hold him guiltless who uses His name to vain purpose.%%
  
  \verse{5:12} \textsc{iv.} Keep the Sabbath day holy, just as the \lord\ your God has commanded you.%%
  \verse{5:13} You shall work and perform all your responsibilities in six days,%%
  \verse{5:14} but the seventh day is the Sabbath of the \lord\ your God: you shouldn't work, neither your son, daughter, handmaid, ox, donkey, any of your cattle, nor an outsider who's living in Israel. This so that your manservant and maidservant may rest just like you.%%
  \verse{5:15} Remember when you were slaves in the land of Egypt and the \lord\ your God brought you from there with a strong hand and an outstretched arm.\ed{So that He can show the way; fight our battles} Therefore the \lord\ your God has commanded you to keep the Sabbath day.%%
  
  \verse{5:16} \textsc{v.} Take your mom and dad seriously just as the \lord\ your God has commanded you. This that your days may be lengthened and in order that it may be well for you upon the land which the \lord\ your God is going to give you.%%
  
  \verse{5:17} \textsc{vi.} Don't murder.%%
  
  \verse{5:18} \textsc{vii.} Don't commit adultery.%%
  
  \verse{5:19} \textsc{viii.} And don't steal.%%
  
  \verse{5:20} \textsc{ix.} Never give a vain witness against your neighbor.%%
  
  \verse{5:21} \textsc{x.} Don't desire your neighbor's wife
  
  or his house, field, bondman, handmaid, ox, donkey, or anything that's his.'%%
  
  \verse{5:22} The \lord\ spoke these words on the mountain to the entire congregation. With a great voice from out of the midst of the fire, the cloud, the gloom, and he added no more. He wrote it on two stone slabs and gave them to me.%%
  \verse{5:23} When you heard the voice from out of\alt{the midst of} the darkness and the mountain burned with fire, all of the heads of your tribes and your elders came to me\ie{Moses, not the Lord}%%
  \verse{5:24} and said, `The \lord\ our God showed us His magnificence and greatness. We have heard his voice from the midst of the fire today.\ed{It's ambiguous if ``today'' should go with this sentence or the next. It works idiomatically with either. However, there does remain a theological implication in placing it with the next sentence, viz.: did they not know this before?} We have seen that God speaks with man~--- that He lives.%%
  \verse{5:25} Now, why should we die? Because this great fire consumes us? Additionally, if we again hear the voice of the \lord\ our God, we shall die.%%
  \verse{5:26} Because who among all living\lit{flesh} has heard the \lord's voice speaking from out of the fire as we have and lived?%%
  \verse{5:27} You,\ie{Moses} come here and hear everything that the \lord\ God is going to say. You shall tell\lit{speak to} us everything that the \lord\ our God has told\lit{spoken to} you. We will hear it and obey.'%%
  \verse{5:28} The \lord\ has heard your voice\lit{the sound/voice of your speaking/words} when you spoke to me, and the \lord\ said, `I have heard the sound of this people's words that they've spoken to you. Everything they've spoken has been well put.\lit{spoken}%%
  \verse{5:29} Who shall give?\halot{xxxx}{xxxx} There was a heart in them to fear Me and to keep all of My commandments forever\lit{all of the days} so that it's good for them and their children forever.'%% xxxx this beginning (->in them) needs complete reworking.
  \verse{5:30} Go. Say to them, `Return to your tents.'%%
  \verse{5:31} However, you shall remain\ed{Understood through context. Without, it is literally ``you here,'' but it's the Lord telling Moses to remain where he is.} here and shall certainly stand\ed{Repeated} by Me as I tell you all of the commandments and statutes and legal decisions that you shall teach them; they shall observe these in the land that I shall give them to possess:%%
\end{inparaenum}
  % xxxx work on this formatting. Bulleted lists should be allowed. Issue reported. Verses~32 and~33 should be bulleted.
\begin{enumerate}
  \setcounter{enumi}{31}
  \verse{5:32} Observe to do as the \lord\ your God commands you; don't fall away to the right or to the left.%%
  \verse{5:33} In all His ways that the \lord\ your God commands you to walk in order that you live, these are good for you and shall prolong your\lit{the; understood as \textit{your}.} days in the land which you shall possess.%%
\end{enumerate}
